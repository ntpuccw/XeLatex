%\documentclass[12pt, a4paper]{article} 

\usepackage{fontspec} % Font selection for XeLaTeX; see fontspec.pdf. 
\usepackage{xeCJK}	% 中文使用 XeCJK,利用 \setCJKmainfont 定義中文內文、粗體與斜體的字型
\defaultfontfeatures{Mapping=tex-text} % to support TeX conventions like ``---''
\usepackage{xunicode} % Unicode support for LaTeX character names(accents, European chars, etc)
\usepackage{xltxtra} 				% Extra customizations for XeLaTeX
\usepackage{amsmath, amssymb}
\usepackage{enumerate}
\usepackage{graphicx,subfig,float,wrapfig} % support the \includegraphics command and options
\usepackage[outercaption]{sidecap} %[options]=[outercaption], [innercaption], [leftcaption], [rightcaption]
\usepackage{array, booktabs}
\usepackage{color, xcolor}
\usepackage{longtable}
\usepackage{colortbl}                          				
\usepackage{listings}						% 直接將 latex 碼轉換成顯示文字
\usepackage[parfill]{parskip} 				% 新段落前加一空行,不使用縮排
\usepackage[left=1.5in,right=1in,top=1in,bottom=1in]{geometry} 
\usepackage{url}

%-----------------------------------------------------------------
%  中英文內文字型設定
\setCJKmainfont							% 設定中文內文字型
	[
		BoldFont=Microsoft YaHei	    %定義粗體的字型(Win)
%		BoldFont=蘋果儷中黑	    		%定義粗體的字型(Mac)
	]
	{新細明體}						% 設定中文內文字型(Win)
%	{宋體-繁}							% 設定中文內文字型(Mac)	
\setmainfont{Times New Roman}		% 設定英文內文字型
\setsansfont{Arial}					% 無襯字字型 used with {\sffamily ...}
%\setsansfont[Scale=MatchLowercase,Mapping=tex-text]{Gill Sans}
\setmonofont{Courier New}			% 等寬字型 used with {\ttfamily ...}
%\setmonofont[Scale=MatchLowercase]{Andale Mono}
% 其他字型(隨使用的電腦安裝的字型不同,用註解的方式調整(打開或關閉))
% 英文字型
\newfontfamily{\E}{Calibri}				
\newfontfamily{\A}{Arial}
\newfontfamily{\C}[Scale=0.9]{Arial}
\newfontfamily{\R}{Times New Roman}
\newfontfamily{\TT}[Scale=0.8]{Times New Roman}
% 中文字型
\newCJKfontfamily{\MB}{微軟正黑體}				% 等寬及無襯線字體 Win
%\newCJKfontfamily{\MB}{黑體-繁}				% 等寬及無襯線字體 Mac
\newCJKfontfamily{\SM}[Scale=0.8]{新細明體}	% 縮小版(Win)
%\newCJKfontfamily{\SM}[Scale=0.8]{宋體-繁}	% 縮小版(Mac)
\newCJKfontfamily{\K}{標楷體}                	% Windows下的標楷體
%\newCJKfontfamily{\K}{楷體-繁}               	% Mac下的標楷體
\newCJKfontfamily{\BB}{Microsoft YaHei}		% 粗體 Win
%\newCJKfontfamily{\BB}{蘋果儷中黑}		% 粗體 Mac
% 以下為自行安裝的字型:CwTex 組合
%\newCJKfontfamily{\CF}{cwTeX Q Fangsong Medium}	% CwTex 仿宋體
%\newCJKfontfamily{\CB}{cwTeX Q Hei Bold}			% CwTex 粗黑體
%\newCJKfontfamily{\CK}{cwTeX Q Kai Medium}   	% CwTex 楷體
%\newCJKfontfamily{\CM}{cwTeX Q Ming Medium}		% CwTex 明體
%\newCJKfontfamily{\CR}{cwTeX Q Yuan Medium}		% CwTex 圓體
%-----------------------------------------------------------------------------------------------------------------------
\XeTeXlinebreaklocale "zh"             		%這兩行一定要加,中文才能自動換行
\XeTeXlinebreakskip = 0pt plus 1pt     		%這兩行一定要加,中文才能自動換行
%-----------------------------------------------------------------------------------------------------------------------
\newcommand{\cw}{\texttt{cw}\kern-.6pt\TeX}	% 這是 cwTex 的 logo 文字
\newcommand{\imgdir}{images/}				% 設定圖檔的目錄位置
\renewcommand{\tablename}{表}	% 改變表格標號文字為中文的「表」(預設為 Table)
\renewcommand{\figurename}{圖}% 改變圖片標號文字為中文的「圖」(預設為 Figure)

% 設定顏色 see color Table: http://latexcolor.com
\definecolor{slight}{gray}{0.9}				
\definecolor{airforceblue}{rgb}{0.36, 0.54, 0.66} 
\definecolor{arylideyellow}{rgb}{0.91, 0.84, 0.42}
\definecolor{babyblue}{rgb}{0.54, 0.81, 0.94}
\definecolor{cadmiumred}{rgb}{0.89, 0.0, 0.13}
\definecolor{coolblack}{rgb}{0.0, 0.18, 0.39}
\definecolor{beaublue}{rgb}{0.74, 0.83, 0.9}
\definecolor{beige}{rgb}{0.96, 0.96, 0.86}
\definecolor{bisque}{rgb}{1.0, 0.89, 0.77}
\definecolor{gray(x11gray)}{rgb}{0.75, 0.75, 0.75}
\definecolor{limegreen}{rgb}{0.2, 0.8, 0.2}
\definecolor{splashedwhite}{rgb}{1.0, 0.99, 1.0}

%---------------------------------------------------------------------
% 映出程式碼 \begin{lstlisting} 的內部設定
\lstset
{	language=[LaTeX]TeX,
    breaklines=true,
    %basicstyle=\tt\scriptsize,
    basicstyle=\tt\normalsize,
    keywordstyle=\color{blue},
    identifierstyle=\color{black},
    commentstyle=\color{limegreen}\itshape,
    stringstyle=\rmfamily,
    showstringspaces=false,
    %backgroundcolor=\color{splashedwhite},
    backgroundcolor=\color{slight},
    frame=single,							%default frame=none 
    rulecolor=\color{gray(x11gray)},
    framerule=0.4pt,							%expand outward 
    framesep=3pt,							%expand outward
    xleftmargin=3.4pt,		%to make the frame fits in the text area. 
    xrightmargin=3.4pt,		%to make the frame fits in the text area. 
    tabsize=2				%default :8 only influence the lstlisting and lstinline.
}

% 映出程式碼 \begin{lstlisting} 的內部設定 for Python codes
%\lstset{language=Python}
%\lstset{frame=lines}
%\lstset{basicstyle=\SCP\normalsize}
%\lstset{keywordstyle=\color{blue}}
%\lstset{commentstyle=\color{airforceblue}\itshape}
%\lstset{backgroundcolor=\color{beige}}   % 使用自己維護的定義檔
%-----------------------------------------------------------------------------------------------------------------------
% 文章開始
\title{ \LaTeX 的數學符號與方程式}
\author{{\SM 汪群超}}
\date{{\TT \today}} 	 
\begin{document}
\maketitle
\fontsize{12}{22 pt}\selectfont

本文將常見的數學符號與方程式以 \LaTeX 編排展示,希望降低使用  \LaTeX 編輯數學式的門檻,快速得到  \LaTeX 為人稱頌的優美數學式。不但為初學者提供編輯的概念與方法,也作為未來的文件編輯的參照樣本。本文內容參考吳聰敏老師專書「\cw{} 排版系統」、\footnote{前往 \cw{} 官方網站 \url{http://homepage.ntu.edu.tw/~ntut019/cwtex/cwtex.html} 下載《cwTeX 使用手冊(PDF)》。}學生的作品及作者平日編輯講義時所發現具代表性的數學方程式。

\section{數式環境}
數學式可能以兩種型式出現,一是隨文數式(In-text Formula),是夾在文章段落中的數學式;譬如,當 $\alpha=2$ 時, $\alpha^3=8$ 。另一種是數學式自成一行或一個段落,我們稱之為展示數式(Display Formula),譬如

$$\int_{-2}^{1} f(x)\;dx$$

輸入數學式時,有兩個地方需要特別注意:

\begin{itemize}
\item 隨文數式前後請留一空格,才不會顯得擁擠。
\item 展示數式上下不須多留一空行, \LaTeX\ 會自行調整間距。
\end{itemize}

\section{符號}
數字與普通運算符號可直接由鍵盤上鍵入。譬如,下列符號可以直接由鍵盤鍵入:

        \begin{center}
         $  + \;-\; =\; <\; > \;/ \;:\; !\;\; |\; \;[\;\; ] \;(\; )$\\
        \end{center}
要注意的是, 左右大括號  $\{$ $\}$ 在 \LaTeX 中有特殊用途。欲排版左大括號, 需加上反斜線,指令為 \verb+\{+,右大括號之指令為 \verb+\}+。排版展示數式有以下四種方法可以達到目的:
        \begin{center}
        \verb+\begin{equation} ... \end{equation}+\\
		\verb+\begin{displaymath} ... \end{displaymath}+\\
		\verb+\[ ... \]+\\
		\verb+$$ ... $$+
        \end{center}
除第一種方式外,其餘將不對數學式子進行編號。數式內若要排版文字時,必須置於 \verb+\mbox{...}+ 指令內,否則將被視為數學符號(變為斜體),譬如,

$$f(x)=x^2-3x+1 \mbox{, where}  -2 \leq x \leq 2$$

\section{常見的數學式}
本節列舉一些常見的數學式作為練習與未來使用的參考,每個函數都有其特別之處,請仔細觀察研究。讀者可以依此為基礎 $\displaystyle\int_0^{100} f(x)dx$,在往後的寫作過程中,逐漸累積更多有特殊型態的或符號的數學式,只要這裡出現過的,參照原始檔一定寫得出來。


\subsection{函數}
挑幾個機率分配函數做示範:

\textbf{Binomial}: 
$$f(x)={n\choose x}p^x(1-p)^{1-x}, \;\; x=0,1,2,\cdots,n$$ 

\textbf{Poisson}: 
$$f(x)=\frac{e^{-\lambda}\lambda^x}{x!}, \;\;  x=0,1,2,\cdots$$ 

\textbf{Gamma}: 
$$f(x)=\frac{1}{\Gamma(\alpha)\beta^\alpha}x^{\alpha-1}e^{-\frac{x}{\beta}}, \;\; x\geq 0$$

\textbf{Normal}: 
$$f(x)=\frac{1}{\sigma\sqrt{2\pi}}e^{-\frac{(x-\mu)^2}{2\sigma^2}}, \;\;  -\infty < x < \infty $$

\bigskip
積分式與方程式編號:
  
  \begin{equation}\label{eq:gamma}%.................label後的名稱自訂,代表該方程式
  \int^\infty_0 x^{\alpha-1}e^{-\lambda x} dx = \frac{\Gamma(\alpha)}{\lambda^{\alpha}}
  \end{equation}

方程式  (\ref{eq:gamma}) 是廣義 $\Gamma$ 積分。\footnote{這裡利用方程式標籤(label)來引用方程式,編號將自動更新。請注意這裡的標籤定義為 eg:gamma,是慣用的方式,其中 eq 代表 equation,而 gamma 代表這個方程式的意義。這個做法符合現代程式語言設計的慣例,因為在 \LaTeX 文體裡面,需要賦予標籤的地方很多,譬如表格與圖都是,因此用字首 eq 來區隔,將來引用時,也因同類的標籤排列在一起,方便選取。}\\
  
 開根號(開立方根):
  
  $$f(x)=\sqrt[3]{\frac {\displaystyle 4-x^{3}}{\displaystyle 1+x^{2}}}$$
  
 微分與極限(注意大刮號的使用):
  
  $$f'(x)=\frac{df(x)}{dx}=\lim_{h\rightarrow 0}\left(\frac{f(x+h)-f(x)}{h}\right)$$
  
數學中的括號隨著其涵蓋內容的多寡(層次)與長相(分數),其大小必須調整恰當,如上式的兩種大小不同的括號「$( \cdot)$ 」。外圍較大地括號使用 $\backslash$  left$($ 與 $\backslash$  right$)$ 令編譯器依需求自動調整為適當大小。另外,也可以手動控制括號、的大小,如

 $$ \bigg(\; \big( \;(\;\;\;) \;\big) \;\bigg) \;,\; \bigg[ \;\big[ \;[\;\;\;]\; \big]\; \bigg]\;,\; \bigg\{ \;\big\{ \;\{\;\;\;\} \;\big\} \;\bigg\}$$ 
  
  
  上下限的使用:
  
  $$\int_a^b f(x) dx \approx \lim_{n\rightarrow \infty}\sum_{k=1}^n f(x_k)\triangle x_k$$
  
  最佳化問題(向量使用粗體來表示):
  
  $$\max_{\mathbf{u},\mathbf{u}^T\mathbf{u}=1} \mathbf{u}^T\Sigma_X\mathbf{u}$$
  
  其他符號:
  $$\mathbf{e}=\mathbf{x}-\mathbf{x}_q=(I-P)\mathbf{x} \in V^{\perp}, \mbox{where}\; V\oplus V^{\perp}=\mathbf{R}^p $$

 上述的方程式有許多地方需要做「下標」(subscript)與上標(supscript),不管是在積分的上下界或是 $\Sigma$ 的上下範圍,或只是變數的下標與次方、、、等,做法都是用  $\_$ 做下標,用 $\wedge$ 製作上標。當上下標只有一個符號或字母時,可以不加括號,否則必須以括號涵蓋。
 
\subsection{矩陣與行列式}
矩陣或有規則排列的數學式或組合很常見,以下列舉幾種模式,請特別注意其使用的標籤及一些需要注意的小地方。譬如,\footnote{這裏的項目符號不是預設的 1, 2, ...,改用 a, b...的編號方式。}
\begin{enumerate}[a)]
  \item 矩陣的左右括號需個別加上。
  \item 直行各項之間是以 $\&$ 區隔。
  \item 除最後一列外,每列之末則加上換列指令 $\backslash\backslash$。
  \item 使用 {\A array} 指令時,須加上選項以控制每一直行內各數字或符號要居中排列、靠左或靠右。
\end{enumerate}

範例與注意事項:
\begin{enumerate}
  \item 左右方框括號的使用及各直行的對齊方式:
 
        $$ A = \left[
            \begin{array}{clr}
                a+b & mnop  & xy \\
                a+b & pn    & yz \\
                b+c & mp    & xyz
            \end{array} \right] $$

  \item 左右圓框刮號的使用及各式點狀:
        $$ A=\left(
            \begin{array}{cccc}
                a_{11} 	& a_{12} & \cdots 	& a_{1n}\\
                a_{21} 	& a_{22} & \cdots 	& a_{2n}\\
                \vdots 	& \vdots & \ddots	& \vdots\\
                a_{n1} 	& a_{n2} & \cdots 	& a_{nn}
            \end{array} \right) $$

  \item 排列整齊的符號:
        $$ \begin{array}{clr}\\
            a+b+c   & m+n 	& xy \\
            a+b     & p+n 	& yz \\
            b+c     & m-n 	& xz
        \end{array} $$

    \item 等號對齊的函數組合(不編號)
        \begin{eqnarray*}
          b_1 &=& d_1+c_1 \\
          a_2 &=& c_2+e_2
        \end{eqnarray*}

    \item 等號對齊的函數組合(編號在最後一行),如式 (\ref{eq:eqnarray_last})
        \begin{eqnarray}\label{eq:eqnarray_last}
\nonumber b_1 &=& d_1+c_1 \\
          a_2 &=& c_2+e_2
        \end{eqnarray}

    \item 使用套件 {\A amsmath} 的指令 {\A align}(控制編號在第一行),如式 (\ref{eq:eqnarray_first})
        \begin{align} 
            b_1 &= d_1+c_1 \label{eq:eqnarray_first}\\
            a_2 &= c_2+e_2 \notag
        \end{align}

    \item 兩組數學式分別對齊且同時編號,如式 (\ref{eq:eqnarray_both1})與式(\ref{eq:eqnarray_both2})
    \begin{align}
        \alpha_1 &= \beta_1+\gamma_1+\delta_1, &a_1 &= b_1+c_1 \label{eq:eqnarray_both1}\\
        \alpha_2 &= \beta_2+\gamma_2+\delta_2, &a_2 &= b_2+c_2 \label{eq:eqnarray_both2}
    \end{align}

    \item 編號在中間({\A split} 指令環境),如式 (\ref{eq:split})
        \begin{equation} 
            \begin{split} \label{eq:split}
                \alpha_1 &= \beta_1+\gamma_1\\
                \alpha_2 &= \beta_2+\gamma_2
            \end{split}
        \end{equation}
    \item 只是居中對齊的數學式組(環境指令 {\A gather})
        \begin{gather}
        \alpha_1 + \beta_1\notag\\
        \alpha_2 + \beta_2 + \gamma_2\notag
        \end{gather}

    \item 長數學式的表達(注意第二行加號的位置),如式 (\ref{eq:longmath})
        \begin{align}\label{eq:longmath}
            y  	&= x_1 + x_2 + x_3 \notag\\
                	&\quad + x_4 + x_5
        \end{align}
\end{enumerate}

\subsection{其他}
列出一些表較少見的數學表達式,用 WORD 很不容易做到。
  $$X_{n} \stackrel{d}{\longrightarrow} X$$
  
  $$\overbrace{X_{1} + \ldots + \underbrace{X_{15} + \ldots + X_{30}}}$$\\
  \begin{equation*}
    G = \left\{\begin{array}{l}
          CLASS A \;\;\mbox{if} \;\; \hat{\beta}^T\bf{x} \leq 0 \\
          CLASS B \;\;\mbox{if} \;\; \hat{\beta}^T\bf{x} > 0
        \end{array}\right.
  \end{equation*}\\

以 {\A equation} 或 {\A align} 排版時,數學式會自動編上號碼。文稿其他地方若要引述某數學式,可先在數學式以 $\backslash${\A label} 指令加上標籤,再使用 $\backslash${\A ref} 指令引述。如此一來若排版文稿須反覆修改,使用 $\backslash${\A label} 與$\backslash${\A ref} 指令可以「自動對焦」不會出錯。

\section{練習題}
下列八張內建數學式的圖,涵蓋一些統計領域常見的數學式細節,試著利用本章所學,細心、耐心、一步步地完成(每完成一小部分便立即編譯,才能掌握每一個看不見的錯誤)。


\begin{figure}[h]
    \centering
        \includegraphics[scale=1]{\imgdir formula_1.jpg}
\end{figure}

\begin{figure}[h]
    \centering
        \includegraphics[scale=1]{\imgdir formula_2.jpg}
\end{figure}

\begin{figure}[h]
    \centering
        \includegraphics[scale=1]{\imgdir formula_3.jpg}
\end{figure}

\end{document}
