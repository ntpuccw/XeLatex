% !TEX TS-program = xelatex			% These two lines must be incldued to open file under UTF-8
% !TEX encoding = UTF-8

%--- BOOK 定義檔
\documentclass[12pt, a4paper]{book}
\setlength{\textwidth}{13cm} %設定書面文字寬度

%--- 定義頁眉頁足 -------------------------------
\usepackage{fancyhdr}
\pagestyle{fancy}
\fancyhf{}
\renewcommand{\chaptermark}[1]{\markboth{\thechapter .\ #1}{}} %去除章編號前後的字
\fancyhead[LE]{\thepage \;\; \XeLaTeX 一本書}
\fancyhead[RO]{\leftmark \;\; \thepage }
%\rhead{\ctxff 數學與電腦}
%\cfoot{\thepage}
\renewcommand{\headrulewidth}{0pt} %頁眉下方的橫線

%----- 定義使用的 packages ----------------------
\usepackage[no-math]{fontspec} 		% Font selection for XeLaTeX; see fontspec.pdf for documentation. 
						% The option [no-math] indicates the use of the default(original) math font.
\defaultfontfeatures{Mapping=tex-text} 		% to support TeX conventions like ``---''
\usepackage{xunicode} 		% Unicode support for LaTeX character names (accents, European chars, etc)
\usepackage{xltxtra} 						% Extra customizations for XeLaTeX
\usepackage{amsmath, amssymb}
\usepackage[sf,small]{titlesec}
\usepackage{enumerate}
\usepackage{graphicx, subfig, float} 		% support the \includegraphics command and options
\usepackage{array, booktabs}
\usepackage{color, xcolor}
\usepackage{longtable}
\usepackage{colortbl}                          				%.............................................表格標題註解之巨集套件
\usepackage{natbib}							% for Reference
\usepackage{makeidx}						% for Indexing
\usepackage[parfill]{parskip} 	% Activate to begin paragraphs with an empty line rather than an indent
%\usepackage{geometry} 			% See geometry.pdf to learn the layout options. There are lots.
%\usepackage[left=1.5in,right=1in,top=1in,bottom=1in]{geometry} 

%-----------------------------------------------------------------------------------------------------------------------
%  主字型設定
\setmainfont
	[
	%ItalicFont = Times New Roman,					% 中文一般不使用斜體字。
	BoldFont = 微軟正黑體,  							%需要粗體的地方。譬如 section
	%BoldItalicFont =Times New Roman
	]{新細明體} 
%\setmainfont{華康仿宋體W4}							% 任何在本電腦上的字型都可以使用(for win)
%\setmainfont{CWTEX}								% 選擇自行安裝的字型 CwTex 明體
%\setmainfont{CWTEX-F}
\setsansfont{Arial}									% used with {\sffamily ...}
%\setsansfont[Scale=MatchLowercase,Mapping=tex-text]{Gill Sans}
\setmonofont{Courier New}							% used with {\ttfamily ...}
%\setmonofont[Scale=MatchLowercase]{Andale Mono}
% 設定其他字型(存在電腦內的任何字型)
\newfontfamily{\E}{Cambria}							
\newfontfamily{\A}{Arial}
\newfontfamily{\C}[Scale=0.9]{Cambria}
\newfontfamily{\T}{Times New Roman}
\newfontfamily{\TT}[Scale=0.8]{Times New Roman}
\newfontfamily{\MB}{微軟正黑體}
\newfontfamily{\SM}[Scale=0.8]{新細明體}			% 縮小版
\newfontfamily{\K}{標楷體}                      	% Windows 下的標楷體

%-----------------------------------------------------------------------------------------------------------------------
\XeTeXlinebreaklocale "zh"                  %這兩行一定要加,中文才能自動換行
\XeTeXlinebreakskip = 0pt plus 1pt     %這兩行一定要加,中文才能自動換行
%-----------------------------------------------------------------------------------------------------------------------
%----- 重新定義的指令 ---------------------------
\newcommand{\cw}{\texttt{cw}\kern-.6pt\TeX}	% 這是 cwTex 的 logo 文字
\newcommand{\imgdir}{images/}						% 設定圖檔的位置
\renewcommand{\tablename}{表}						% 改變表格標號文字為中文的「表」(預設為 Table)
\renewcommand{\figurename}{圖}						% 改變圖片標號文字為中文的「圖」(預設為 Figure)
%\renewcommand{\figurename}{圖\hspace*{-.5mm}}
%  for Long Report or Book
\renewcommand{\contentsname}{{\MB 目錄}}
\renewcommand\listfigurename{{\MB 圖目錄}}
\renewcommand\listtablename{{\MB 表目錄}}
\renewcommand{\indexname}{{\MB 索引}}
\renewcommand{\bibname}{{\MB 參考文獻}}
%-----------------------------------------------------------------------------------------------------------------------

%\theoremstyle{plain}
\newtheorem{de}{Definition}[section]		%definition獨立編號
\newtheorem{thm}{{\MB 定理}}[section]		%theorem 獨立編號,取中文名稱並給予不同字型
\newtheorem{lemma}[thm]{Lemma}				%lemma 與 theorem 共用編號
\newtheorem{ex}{{\E Example}}				%example 獨立編號,不編入小節數字,走流水號。也換個字型。
\newtheorem{cor}{Corollary}[section]		%not used here
\newtheorem{exercise}{EXERCISE}				%not used here
\newtheorem{re}{\emph{Result}}[section]		%not used here
\newtheorem{axiom}{AXIOM}					%not used here
%\renewcommand{\proofname}{\bf{Proof}}		%not used here

\newcommand{\loflabel}{圖} % 圖目錄出現 圖 x.x 的「圖」字
\newcommand{\lotlabel}{表}  % 表目錄出現 表 x.x 的「表」字

\parindent=0pt
\setcounter{tocdepth}{0}

%--- 其他定義 ----------------------------------
% 定義章節標題的字型、大小
\titleformat{\chapter}[display]{\raggedleft\LARGE\MB}
 {\MB 第\ \thechapter\ 章}{0.2cm}{}
%\titleformat{\chapter}[hang]{\centering\LARGE\sf}{\MB 第~\thesection~章}{0.2cm}{}%控制章的字體
%\titleformat{\section}[hang]{\Large\sf}{\MB 第~\thesection~節}{0.2cm}{}%控制章的字體
%\titleformat{\subsection}[hang]{\centering\Large\sf}{\MB 第~\thesubsection~節}{0.2cm}{}%控制節的字體
\titleformat*{\section}{\normalfont\Large\bfseries\MB}
\titleformat*{\subsection}{\normalfont\large\bfseries\MB}
\titleformat*{\subsubsection}{\normalfont\large\bfseries\MB}



\definecolor{slight}{gray}{0.9}					% 設定顏色