%%\documentclass[12pt, a4paper]{article} 

\usepackage{fontspec} % Font selection for XeLaTeX; see fontspec.pdf. 
\usepackage{xeCJK}	% 中文使用 XeCJK,利用 \setCJKmainfont 定義中文內文、粗體與斜體的字型
\defaultfontfeatures{Mapping=tex-text} % to support TeX conventions like ``---''
\usepackage{xunicode} % Unicode support for LaTeX character names(accents, European chars, etc)
\usepackage{xltxtra} 				% Extra customizations for XeLaTeX
\usepackage{amsmath, amssymb}
\usepackage{enumerate}
\usepackage{graphicx,subfig,float,wrapfig} % support the \includegraphics command and options
\usepackage[outercaption]{sidecap} %[options]=[outercaption], [innercaption], [leftcaption], [rightcaption]
\usepackage{array, booktabs}
\usepackage{color, xcolor}
\usepackage{longtable}
\usepackage{colortbl}                          				
\usepackage{listings}						% 直接將 latex 碼轉換成顯示文字
\usepackage[parfill]{parskip} 				% 新段落前加一空行,不使用縮排
\usepackage[left=1.5in,right=1in,top=1in,bottom=1in]{geometry} 
\usepackage{url}

%-----------------------------------------------------------------
%  中英文內文字型設定
\setCJKmainfont							% 設定中文內文字型
	[
		BoldFont=Microsoft YaHei	    %定義粗體的字型(Win)
%		BoldFont=蘋果儷中黑	    		%定義粗體的字型(Mac)
	]
	{新細明體}						% 設定中文內文字型(Win)
%	{宋體-繁}							% 設定中文內文字型(Mac)	
\setmainfont{Times New Roman}		% 設定英文內文字型
\setsansfont{Arial}					% 無襯字字型 used with {\sffamily ...}
%\setsansfont[Scale=MatchLowercase,Mapping=tex-text]{Gill Sans}
\setmonofont{Courier New}			% 等寬字型 used with {\ttfamily ...}
%\setmonofont[Scale=MatchLowercase]{Andale Mono}
% 其他字型(隨使用的電腦安裝的字型不同,用註解的方式調整(打開或關閉))
% 英文字型
\newfontfamily{\E}{Calibri}				
\newfontfamily{\A}{Arial}
\newfontfamily{\C}[Scale=0.9]{Arial}
\newfontfamily{\R}{Times New Roman}
\newfontfamily{\TT}[Scale=0.8]{Times New Roman}
% 中文字型
\newCJKfontfamily{\MB}{微軟正黑體}				% 等寬及無襯線字體 Win
%\newCJKfontfamily{\MB}{黑體-繁}				% 等寬及無襯線字體 Mac
\newCJKfontfamily{\SM}[Scale=0.8]{新細明體}	% 縮小版(Win)
%\newCJKfontfamily{\SM}[Scale=0.8]{宋體-繁}	% 縮小版(Mac)
\newCJKfontfamily{\K}{標楷體}                	% Windows下的標楷體
%\newCJKfontfamily{\K}{楷體-繁}               	% Mac下的標楷體
\newCJKfontfamily{\BB}{Microsoft YaHei}		% 粗體 Win
%\newCJKfontfamily{\BB}{蘋果儷中黑}		% 粗體 Mac
% 以下為自行安裝的字型:CwTex 組合
%\newCJKfontfamily{\CF}{cwTeX Q Fangsong Medium}	% CwTex 仿宋體
%\newCJKfontfamily{\CB}{cwTeX Q Hei Bold}			% CwTex 粗黑體
%\newCJKfontfamily{\CK}{cwTeX Q Kai Medium}   	% CwTex 楷體
%\newCJKfontfamily{\CM}{cwTeX Q Ming Medium}		% CwTex 明體
%\newCJKfontfamily{\CR}{cwTeX Q Yuan Medium}		% CwTex 圓體
%-----------------------------------------------------------------------------------------------------------------------
\XeTeXlinebreaklocale "zh"             		%這兩行一定要加,中文才能自動換行
\XeTeXlinebreakskip = 0pt plus 1pt     		%這兩行一定要加,中文才能自動換行
%-----------------------------------------------------------------------------------------------------------------------
\newcommand{\cw}{\texttt{cw}\kern-.6pt\TeX}	% 這是 cwTex 的 logo 文字
\newcommand{\imgdir}{images/}				% 設定圖檔的目錄位置
\renewcommand{\tablename}{表}	% 改變表格標號文字為中文的「表」(預設為 Table)
\renewcommand{\figurename}{圖}% 改變圖片標號文字為中文的「圖」(預設為 Figure)

% 設定顏色 see color Table: http://latexcolor.com
\definecolor{slight}{gray}{0.9}				
\definecolor{airforceblue}{rgb}{0.36, 0.54, 0.66} 
\definecolor{arylideyellow}{rgb}{0.91, 0.84, 0.42}
\definecolor{babyblue}{rgb}{0.54, 0.81, 0.94}
\definecolor{cadmiumred}{rgb}{0.89, 0.0, 0.13}
\definecolor{coolblack}{rgb}{0.0, 0.18, 0.39}
\definecolor{beaublue}{rgb}{0.74, 0.83, 0.9}
\definecolor{beige}{rgb}{0.96, 0.96, 0.86}
\definecolor{bisque}{rgb}{1.0, 0.89, 0.77}
\definecolor{gray(x11gray)}{rgb}{0.75, 0.75, 0.75}
\definecolor{limegreen}{rgb}{0.2, 0.8, 0.2}
\definecolor{splashedwhite}{rgb}{1.0, 0.99, 1.0}

%---------------------------------------------------------------------
% 映出程式碼 \begin{lstlisting} 的內部設定
\lstset
{	language=[LaTeX]TeX,
    breaklines=true,
    %basicstyle=\tt\scriptsize,
    basicstyle=\tt\normalsize,
    keywordstyle=\color{blue},
    identifierstyle=\color{black},
    commentstyle=\color{limegreen}\itshape,
    stringstyle=\rmfamily,
    showstringspaces=false,
    %backgroundcolor=\color{splashedwhite},
    backgroundcolor=\color{slight},
    frame=single,							%default frame=none 
    rulecolor=\color{gray(x11gray)},
    framerule=0.4pt,							%expand outward 
    framesep=3pt,							%expand outward
    xleftmargin=3.4pt,		%to make the frame fits in the text area. 
    xrightmargin=3.4pt,		%to make the frame fits in the text area. 
    tabsize=2				%default :8 only influence the lstlisting and lstinline.
}

% 映出程式碼 \begin{lstlisting} 的內部設定 for Python codes
%\lstset{language=Python}
%\lstset{frame=lines}
%\lstset{basicstyle=\SCP\normalsize}
%\lstset{keywordstyle=\color{blue}}
%\lstset{commentstyle=\color{airforceblue}\itshape}
%\lstset{backgroundcolor=\color{beige}}
%%-----------------------------------------------------------------------------------------------------------------------
%% 文章開始
%\title{ \LaTeX  {\SB 的表格製作}}
%\author{{\MB 汪群超}}
%\date{{\TT \today}} 	% Activate to display a given date or no date (if empty),
%         						% otherwise the current date is printed 
%\begin{document}
%\maketitle
%\fontsize{12}{22pt}\selectfont
\chapter{\LaTeX  的表格製作}
本文根據吳聰敏老師著作 "\cw{} 排版系統"編寫,\footnote{相關文件可在 \cw{} 官方網站  http://homepage.ntu.edu.tw/$\sim$ntut019/cwtex/cwtex.html 下載。}將其中幾種典型的表格陳列說明,方便未來編輯文件時參考。表格的編輯有些法則供遵循,習者宜先參考書上關於表格製作之基本概念後,將來製作表格時才能不失方寸。另外,參考本文時也需隨時與原文檔案互為參照,瞭解各指令的用法,方能在最短時間內掌握表格製作的技巧。下表是最基本的表格型態,不過卻不符合表格的「建構美學」。\\

\begin{tabular}{|l|c|c|}%lcc代表三欄,第一欄靠左,二三欄置中,兩旁的直線代表欄與欄間劃上直線
\hline  %劃上一條橫線
  義大利  & 0.5  & 16.12		\\\hline  % &代表換欄 \\代表換下一列
  英國     & 2     & 12.3		\\\hline
  加拿大  & 3     & 8.1			\\\hline
\end{tabular}\\

就表格的擺設位置來說,一般喜歡置中。就構成表格的線條而言則盡量避免垂直線。如下表眼見為憑,是不是看起來舒服一點呢。
請注意:我們很在意表格呈現出來得樣子,但是在編輯表格的原始檔中,也要多些心思注意。譬如,編輯表格的語法中,欄與欄之間以 \& 符號隔開,習者最好養成習慣在原始檔中將每一列的 \& 符號對齊,方便編輯(請查看原始檔的表格編輯)。\\

\begin{center} %將以下內容置中,直到結束指令出現
\begin{tabular}{lcc}  %取消欄間的直線
\hline
  義大利  & 0.5  & 16.12		\\\hline  % &代表換欄 \\代表換下一列
  英國     & 2     & 12.3		\\\hline
  加拿大  & 3     & 8.1			\\\hline
\end{tabular}
\end{center}
\bigskip
除外觀的感覺外,一般放在文章裡的表格都需要編號(Label)與標題(Caption),方便與內文相呼應,不至於孤伶伶的掛在頁面的一個角落,讓讀者自己去看去猜這表格式要說些什麼?\LaTeX 提供環境指令 {\A table} 達到這個目的。以下的表格都會附上編號與標題。表 \ref{basic_1} 便是為上表附上編號與標題,請從原始檔觀察編號與標題的使用,以及內文參照時的用法。

\begin{table}[h] %加入環境指令 table 以控制表格的位置、編號與標題,[h]代表將表格置於 here,其他位置的標示請參考手冊
    \centering
    \caption{最基本的表格}\label{basic_1}  %加入標題與標號參照的文字
    \bigskip
    \begin{tabular}{lcc}
    \hline
  義大利  & 0.5  & 16.12		\\\hline  % &代表換欄 \\代表換下一列
  英國     & 2     & 12.3		\\\hline
  加拿大  & 3     & 8.1			\\\hline
    \end{tabular}
\end{table}

表格編號與標題一般放置在表格上方,\footnote{一般科學性論文與書籍中,不論圖或表都必須編號並賦予文字說明,習慣上將表格編號與標題放置在表格上方,而圖編號與標題則放置在圖下方。}其中標題前的「表」字,是重新經過定義的。\LaTeX 原來定義的文字是英文 Table,在中文的環境當然不妥,利用 {\A renewcommand} 可以定義作者自己喜歡的字眼, \footnote{使用方式請參考原始檔案的定義區。}當然也不能太另類的自外於文體格式,旁人恐難接受。此外,在文內參照時 \LaTeX 會自動利用使用者給的標籤文字去對照,在 \LaTeX 編譯的階段賦予適當的號碼。但 \LaTeX 需要兩次的編譯才能完成,第一次編譯後將留下兩個問號 ??,直到第二次編譯完成才會出現正確的號碼。

此外,表格的欄寬與列高在製作時由 \LaTeX 自動產生,有時會因中英文的差異,在視覺上感覺不適當,如表 \ref{basic_1} 的列高似乎不夠,感覺擁擠了些。此時可以自行加入適當的指令來彌補,如下表利用 {\A extrarowheight} 指令將行高增加  2pt。


\begin{table}[h]
    \centering
       \caption{改變行高並加入底色的表格}\label{basic_row_color}  %加入標題與標號參照的文字
       \bigskip
    \extrarowheight=2pt   %加高行高2pt
    \colorbox{slight}{\begin{tabular}{lcc}
    \hline
  義大利  & {\C 0.5}  & {\C 16.12}		\\\hline  % &代表換欄 \\代表換下一列
  英國     & {\C 2}     & {\C 12.3}		\\\hline
  加拿大  & {\C 3}     & {\C 8.1}			\\\hline
    \end{tabular}}
\end{table}

表 \ref{basic_row_color} 除了增加列高之外,也趁機加入底色並變更裡面數字的字型。指令不是加上去就可以的,還得先問來源。有很多的指令並非 \LaTeX 內建的,需要外加套件(Package)才有的。譬如,使用 {\A extrarowheight} 指令的同時,必須先在定義區使用

\begin{center}\colorbox{slight}{\begin{tabular}{p{0.9\textwidth}}
	{\A usepackage\{array\}}
\end{tabular}}\end{center}
\bigskip
加入 {\A array} 套件,為表格加上的底色也是靠 {\A xcolor} 套件,並使用指令 {\A colorbox} 與事先定義好的深灰顏色。

表格有許多型態因應不同的資料結構,以下介紹的表格盡量展現不同的特色,方便寫作時參考。如表 \ref{basic_two_table} 展示如何將兩個表格並列。表格並列的關鍵在於第一個表格的結束指令 {\A end\{tabular\}} 與第二個表格的開始指令 {\A begin\{tabular\}} 之間不能空行,但可以加入空格的距離設定,讓表格之間有適當的距離。這兩個並列的表格內容相同,結構稍不同,不過不難從原始檔中看出技巧的不同,無須另言贅述。

\begin{table}[h]
    \centering
    \caption{兩個表格並列的作法}\label{basic_two_table}
    \bigskip
    \extrarowheight=2pt
    \begin{tabular}{lcc}
    \hline
    國家       & 央行獨立性    & 物價上漲率 \\\hline
    義大利    & 0.5              & 16.12 \\
    英國       & 2                 & 12.3 \\
    加拿大    & 3                 & 8.1 \\\hline
    \end{tabular}\hspace{10pt}
    \begin{tabular}{lcc}
    \hline
                & 央行       & 物價 \\[-2pt]
    國家      & 獨立性    & 上漲率 \\\hline
    義大利   & 0.5        & 16.12 \\
    英國      & 2           & 12.3 \\
    加拿大   & 3           & 8.1 \\\hline
    \end{tabular}
\end{table}

表格常需置入跨越多欄的一列,可以使用指令 {\A multicolumn},配合指令 {\A cline} 畫出適當長度的橫線,如表 \ref{basic_multi_col}  所示。

\begin{table}[h]
    \centering
    \caption{表格中跨多欄的表現}\label{basic_multi_col}
    \bigskip
    \extrarowheight=2pt
    \begin{tabular}{lrr}
    \hline
              & \multicolumn{2}{c}{經濟表現}\\\cline{2-3}%指定跨兩欄並畫一條線橫跨第二、三欄
              & 央行      & 物價 \\[-2pt]
    國家    & 獨立性   & 上漲率 \\\hline
    義大利 & 0.5       & 16.12 \\
    英國    & 2          & 12.3 \\
    加拿大 & 3          & 8.1 \\\hline
    \end{tabular}
\end{table}

\LaTeX 的可愛來自許多使用者為它寫了無數的套件加強他的功能,注入源源不斷的活水,更重要的是,一切都是免費。譬如 {\A booktabs} 套件用粗細不等的線條讓表格變得更有氣質,如表 \ref{basic_multi_col_2} 所展示的。

\begin{table}[ht]
    \centering
    \caption{{\C booktabs} 套件的線條表現}\label{basic_multi_col_2}
    \bigskip
    \extrarowheight=2pt
    \begin{tabular}{lrr}
    \toprule
              & \multicolumn{2}{c}{經濟表現}\\\cmidrule(l){2-3}
              & 央行      & 物價 \\[-2pt]
    國家    & 獨立性   & 上漲率 \\\midrule
    義大利 & 0.5        & 16.12 \\
    英國    & 2          & 12.3 \\
    加拿大 & 3          & 8.1 \\
    \bottomrule
    \end{tabular}
\end{table}

套件 {\A booktabs} 還提供欄寬的設定,讓表格的大小更符合版面的要求。表 \ref{booktabs_1} 定義第三欄的寬度,使得長文字自動斷行,另外也在表格結束後加上註解文字。

\begin{table}[h]
    \centering
    \caption{利用 {\C booktabs} 套件定義欄寬}\label{booktabs_1}
    \bigskip
    \begin{tabular}{lcp{2.5cm}}%定義第三欄為2.5cm
    \toprule
    項目      & 分數        & 評述意見 \\\midrule
    方法      & 85          & 本研究的實驗方法是作者發展出來的。 \\[2pt]
    貢獻      & 88          & 從實際應用來看,本研究很有貢獻。 \\[2pt]
    文字      & 85          & 甚佳。 \\
    \bottomrule
    \end{tabular}\par\smallskip %段落結束(par),空出一個小空間(smallskip)
    \parbox{5cm}{以上文字純屬虛構。}%
\end{table}

有許多的套件被設計來加強現有的指令,其實任何套件都可以達到相同的效果,差別只在方便性而已,這也是套件設計的主因。譬如,使用A套件時,指令比較繁複,所需「功力」的要求較高。反之使用B套件時,有些繁複的地方被巧妙組合成新的指令,應用上比較單純,適合初學者。其實,最好的方式就是你最熟練的方式,只要能到目的,使用什麼套件都不打緊的。

本文介紹的表格製作方式,大約已涵蓋一般使用的範圍。只要熟悉這些用法,變化使用,應該夠用。若遇特殊的表格需求,譬如特長的表格,長度超過一頁或太寬,或是需要加底色的,這些都可以在手冊上找到。有時真是技術不及之處,只好更換表格的架構,一樣可以達到目的。以下列舉一些表格套件,或許它們能獲得您得青睞,躍上紙面成為您的文章增添風味。

\begin{table}[h]
\begin{center}
\caption{使用 {\A colortbl} 套件}\label{tab:b}
\extrarowheight=2pt
\begin{tabular}{ll}
\rowcolor[gray]{.9}
函數&說明\\
\toprule
{\C polyval(p,a)}	&	計算多項式 {\C p(x)} 於 {\C x=a} 的值。 {\C a } 可以為一個純量或向量\\
{\C roots(p)}		&	計算多項式 {\C p(x)} 的根\\
\bottomrule
\end{tabular}
\end{center}
\vspace{1cm}
\end{table}

\newpage
表 \ref{basic_4} 是將表格視為圖片做選轉(採用  {\A graphicx} 套件),方便做寬型表格時使用。
\begin{table}[h]
\begin{center}
\caption{旋轉表格}\label{basic_4}
\bigskip
\extrarowheight=2pt
\rotatebox[origin=c]{90}{
\begin{tabular}{|l|ccccc|}
\hline
Source	& Df	& SS			& MS		& F value	& Pr$>$ F \\\hline
model	& 2 	& 543.6 	& 271.8 	& 16.08 	& 0.0004 \\
Error		& 12 & 202.8 	& 16.9 		&{}  			&{} \\\hline
Total		& 14 & 746.4 	&{}  			&{}  			&{} \\
\hline
\end{tabular}}\hspace{10pt}
\extrarowheight=2pt
\rotatebox[origin=c]{270}{
\begin{tabular}{|l|ccccc|}
\hline
Source	& Df	& SS			& MS		& F value	& Pr$>$ F \\\hline
model	& 2 	& 543.6 	& 271.8 	& 16.08 	& 0.0004 \\
Error		& 12 & 202.8 	& 16.9 		&{}  			&{} \\\hline
Total		& 14 & 746.4 	&{}  			&{}  			&{} \\
\hline
\end{tabular}}
\end{center}
\end{table}


如果表格長度超過版面高度,可以使用longtable巨集套件,原來之表格自動拆為兩部分以上,分別排版於兩頁或是多頁之中,如表   \ref{longtable} 所示。請注意不同頁面在表格斷續處的文字處理。
\newpage%......................................................................希望表格從下一頁開始排版
\begin{longtable}{@{}lrrrrr@{}}
\caption{長型表格}
\label{longtable}\\
\toprule
年度& 勝率& 平均得分 & 平均失分 & 平均籃板 &平均助攻\\
\midrule
\endfirsthead
\multicolumn{6}{l}{承接上頁}\\[2pt]
\toprule
年度& 勝率& 平均得分 & 平均失分 & 平均籃板 &平均助攻\\
\midrule
\endhead
\midrule
\multicolumn{6}{r}{續接下頁}
\endfoot
\endlastfoot
1896&--~&--~&--~&60.31&59.16\\
1896&--~&--~&--~&60.31&59.16\\
1896&--~&--~&--~&60.31&59.16\\
1896&--~&--~&--~&60.31&59.16\\
1896&--~&--~&--~&60.31&59.16\\
1896&--~&--~&--~&60.31&59.16\\
1896&--~&--~&--~&60.31&59.16\\
1896&--~&--~&--~&60.31&59.16\\
1896&--~&--~&--~&60.31&59.16\\
1896&--~&--~&--~&60.31&59.16\\
1896&--~&--~&--~&60.31&59.16\\
1896&--~&--~&--~&60.31&59.16\\
1896&--~&--~&--~&60.31&59.16\\
1985&135.98&187&131.49&97.84&432.91\\
1985&135.98&187&131.49&97.84&432.91\\
1985&135.98&187&131.49&97.84&432.91\\
1985&135.98&187&131.49&97.84&432.91\\
1985&135.98&187&131.49&97.84&432.91\\
1985&135.98&187&131.49&97.84&432.91\\
1985&135.98&187&131.49&97.84&432.91\\
1985&135.98&187&131.49&97.84&432.91\\
1985&135.98&187&131.49&97.84&432.91\\
1985&135.98&187&131.49&97.84&432.91\\
1985&135.98&187&131.49&97.84&432.91\\
1985&135.98&187&131.49&97.84&432.91\\
1985&135.98&187&131.49&97.84&432.91\\
1985&135.98&187&131.49&97.84&432.91\\
1985&135.98&187&131.49&97.84&432.91\\
1985&135.98&187&131.49&97.84&432.91\\
1985&135.98&187&131.49&97.84&432.91\\
1985&135.98&187&131.49&97.84&432.91\\

\bottomrule
\end{longtable}

%\end{document}
