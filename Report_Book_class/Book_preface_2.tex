\addcontentsline{toc}{chapter}{再序:不可變與不可不變:我的程式寫作觀}
\chapter*{再序\\ 不可變與不可不變:我的程式寫作觀}
教學七年了,這本講義也用了三年。其間經過多次的修改,不管擴編還是刪減,多半是依據上課時學生的反應而來。
這本講義其實很多地方寫得不夠詳細,本想進一步將所有細節完整呈現,成為一本書。但幾經思量,仍維持原貌,
原因是太詳細的內容會養成學生的依賴心,喪失原先期望學生自己去補足不清楚、不詳細的部分。
希望學生藉著這門課拾回過去學得不清不楚的微積分、統計學與線性代數。\\

這本講義企圖將數學原理以電腦數據圖表的方式呈現出來,再要求同學以文字圖案呈現出其間的條理,
這樣的訓練是現今大學生十分欠缺的。說穿了就是「表達的能力」的培養。這可不是說、學、逗、唱之類的表達,
而是一種試圖將不易說清楚或難以理解的東西,透過文字、圖表或語言將它交代清楚。這樣的能力絕對需要長時間的訓練,
有了這項「絕技,」大學畢業生不必急著說自己學非所用。有太多的事實證明,擁有絕佳的表達能力,放諸四海都餓不著肚子。\\

表達能力的養成必須按部就班,一點都急不得。可惜的是,莘莘學子不是自作聰明,便是固執己見,
往往喜歡憑自己過去的經驗來解決未知的問題,缺乏耐心去熟練不熟悉的工具,不願將專注力用在問題的觀察。
學習過程像極矇著雙眼亂砍亂殺,到頭來學不到東西還怪老師出太多怪怪的功課,既對升學沒有幫助,也無助以後做生意賺大錢,
不多久便放棄了,殊是可惜。告訴他這是未來升官發財的利器,他當你在三娘教子,在家裡聽多了。\\

以寫作程式為例,每一種程式語言都有其語法規範,該怎麼寫怎麼用,一點也馬虎不得,連錯一點點都不行,
沒得商量的。初學者往往輕忽之,不喜歡被「規範」束縛,不顧老師一再地提醒,愛怎麼寫就怎麼寫,
天才般的自己編撰起語法來了,結果當然是錯誤百出,急得老師在一旁乾著急。更有甚之,錯了還不認帳,
直呼語法太不人性化,不能隨意更動,學它何用,便率性的打起電動或MSN來了。\\

寫程式首要遵守語法教條,待熟悉語法規範之後,才能漸漸懂得運用,透過寫一些不痛不癢的小程式,
一方面熟悉語法,一方面體驗其威力。漸熟,才慢慢從觀察別人寫的「模範程式」中,瞭解死的語言原來也能玩出活把戲,
這才一步步進入寫作程式的精髓,進一步玩出樂趣。這道理亙古不變,古今達人不管學習琴棋書畫,還是打拿摔跌等武藝,
無不遵循這樣的哲理\footnote{謫自五絕奇人鄭曼青先生名著「曼髯三論。」}\\

{\MB 能力未至不可變也、學識未敷不得變也、功侯未到不能變也。\\
學於師已窮其法,不可不變也、友古人已悉其意,不得不變也、\\
師造化已盡其理,不能不變也。}\\

從「不可變、」「不得變、」「不能變」,到「不可不變、」「不得不變、」「不能不變,」可以作為寫作程式的養成過程。
學習之初應謹慎遵循所有的規範,一絲不苟,不能濫用自己的小聰明亂抄捷徑,要聽話、要服從,
將老師的交代與叮嚀當作聖旨般遵循,務必做到。如此這般一段時日之後,犯錯愈少,進步愈多,
自然而然當變則變,逐漸形成自己的風格。\\

不能急,成就總在不知不覺中「赫然」被別人發現,絕非刻意營造而能得。別人眼中看到的成就,
對自己而言永遠都是平常事而已,只不過在許多小地方比別人好一點點罷了。但別小看這一點點,
許許多多的一點點累積起來,那可有多少啊!

\begin{flushright}
    汪群超
    \par\vspace*{-2pt}\hfill 2005年2月於台北大學
\end{flushright}
