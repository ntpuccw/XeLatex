%\input{preamble_Regular}
%% 以下的設定在測試成功後,可以轉入前置檔( def_for_TexMaker.tex )一起管理。
%\usepackage{amsthm}									% theroemstyle 需要使用的套件
%\theoremstyle{plain}
%
%\newtheorem{de}{Definition}[section]				%definition獨立編號
%\newtheorem{thm}{{\MB 定理}}[section]			%theorem 獨立編號,取中文名稱並給予不同字型
%\newtheorem{lemma}[thm]{Lemma}				%lemma 與 theorem 共用編號
%\newtheorem{ex}{{\E Example}}						%example 獨立編號,不編入小節數字,走流水號。也換個字型。
%
%\newtheorem{cor}{Corollary}[section]				%not used here
%\newtheorem{exercise}{EXERCISE}					%not used here
%\newtheorem{re}{\emph{Result}}[section]		%not used here
%\newtheorem{axiom}{AXIOM}							%not used here
%\renewcommand{\proofname}{\bf{Proof}}		%not used here
%
%%-----------------------------------------------------------------------------------------------------------------------
%% 文章開始
%\title{ \LaTeX {\MB 定理計數器的使用}}
%\author{{\MB 汪群超}}
%\date{{\TT \today}} 	% Activate to display a given date or no date (if empty),
%         						% otherwise the current date is printed 
%\begin{document}
%\maketitle
%\fontsize{12}{22pt}\selectfont
\chapter{定理計數器的使用}
數理方面的文章或書籍常會使用到定理定義,或類似需要給予編號的一段文字。這些編號的管理有些是順序排列,有些則隨章節排列,有些一起編號,有些分開。這類的編號都由指令 {\C $\backslash$newtheorem} 處理。  

\section{{\MB 語法}}
{\C $\backslash$newtheorem} 的語法如下

 \begin{center}{\begin{tabular}{l}
 {\C  $\backslash$newtheorem\{env\_name\}\{caption\}[within]}\\
 {\C $\backslash$newtheorem\{env\_name\}[numbered\_like]\{caption\}}\\
 \end{tabular}}\end{center}
其中
\begin{itemize}
\item {\C env\_name}:新計數器名稱,通常以簡短文字代表將呈現的文字。譬如, {\C thm} 代表 {\C Theorem} 字樣。
\item {\C caption}:表示將呈現的文字,一般如 {\C Theorem, Definition, Lemma...} 等或使用中文的「定理」「定義」等。
\item {\C within}:代表一個已經存在的計數器,譬如,章 ({\C chapter}) 或節 ({\C section}),表示目前的計數器將以該存在的計數器為計數範圍。以章為例,在第二章出現的第一個編號將是 {\C 2.1},以節為例,第三章第二節出現的第三個編號將是 {\C 3.2.3}。
\item {\C number\_like}:一個已經被定義過的計數器名稱,譬如,{\C thm}。代表目前定義的計數器將共用相同的計數器。沒有這項參數定義的都是為獨立編號,不予其他計數器共用。
\end{itemize}
以下範例舉 {\C Definition, Example, Theorem, Lemma} 為例,其中 {\C Definition, Example} 獨立編號,而 {\C Theorem} 與 {\C Lemma} 共同編號。定義方式如下:

 \begin{center}{\begin{tabular}{l}
 {\C $\backslash$newtheorem\{de\}\{Definition\}[section]}\\
 {\C  $\backslash$newtheorem\{ex\}\{$\backslash$emph\{Example\}\}[section]}\\
 {\C $\backslash$newtheorem\{th\}\{Theorem\}[section]}\\
 {\C $\backslash$newtheorem\{lemma\}[th]\{Lemma\}}\\
\end{tabular}}\end{center}


\section{{\MB 隨機變數的定義}}
 \rule{\textwidth}{0.2pt}
 \begin{de}\footnote{摘自 {\C Casella and Berger 2002, Definition 1.1.1}}  %def 1.1.1(Casella & Berger (2002))
{\C The set, $\mathcal{S}$, of all possible outcomes of a particular experiment is called the \textbf{sample space} for the experiment.}\\
 \rule{\textwidth}{0.2pt}
\end{de}
\noindent 通常定義定理會用特別的方式呈現出來,讓讀者容易一眼看到。本章將定義前後個加上一條橫線來突顯它的位置。另外一種常見的方式請參考本章最後的「定理」與「{\E lemma}」。這是用表格加上底色做出明顯的框架,裡面採用的表格與底色技術請參考講義「表格製作參考」。

\noindent \rule{\textwidth}{0.2pt}
\begin{de}\footnote{摘自 {\C Casella and Berger 2002, Definition 1.5.1}} %def 1.1.2
{\C The \textbf{cumulative distribution function}  or CDF of a random variable X, denoted by $F_X(x)$,
is defined by}
\[F_X(x)=P_X(X \leq x). \mbox{ for all x}, x\in \mathcal{S}.\]
 \rule{\textwidth}{0.2pt}
\end{de}
\noindent 以下的 {\E Example} 編號不隨小節計數,係按流水號順序。
\begin{ex}[指數分配隨機變數的呈現] %
假設 {\C X} 服從指數分配,其 {\C CDF} 為 $y=F_X(x)=1-e^{-x}, \forall x>0$,記為 $X\sim F_X(x)=1-e^{-x}$。
\end{ex}

\begin{ex}[幾何分配隨機變數的呈現] %
假設 {\C X}服從幾何分配,其 {\C CDF} 為 $y=F_X(x)=1-(1-p)^k$,其中 $k=[x]\in \mathcal{N}$,記為 $X\sim F_X(x)=1-(1-p)^x$。此函數又稱為階梯函數({\C step function})。
\end{ex}


\section{{\MB 離散型隨機變數}}
\noindent \rule{\textwidth}{0.2pt}
\begin{de}\footnote{摘自 {\C Casella and Berger 2002, Definition 1.6.1}} %def 1.6.1
{\C The \textbf{probability massfunction} (\textbf{pmf}) of a discrete random variable $X$ is given by}
\[f_X(x)=P(X=x) \mbox{ for all } x.\]
\noindent \rule{\textwidth}{0.2pt}
\end{de}
\bigskip
 
\begin{ex}[{\C Geometric probabilities}] %ex 1.6.2
{\C For the \textbf{geometric distributio}n of Example 1.1.2, we have the \textbf{pmf}}
\[f_X(x)=P(X=x)=\left\{\begin{array}{ll} p(1-p)^{x-1}  & \mbox{ for } x=1, 2, \cdots \\
                                              0        & \mbox{ otherwise } \\ \end{array}\right.\]
\end{ex}
\bigskip
\noindent \rule{\textwidth}{0.2pt}
\begin{de}\footnote{摘自 {\C Casella and Berger 2002, Definition 1.6.1}}  %def 1.6.3
{\C The \textbf{probability density function} or PDF, $f_X(x)$, of a continuous random variable $X$ is the function that
satisfies}
\[F_X(x)=\int_{-\infty}^x f_X(t)dt \mbox{ for all } x.\]
\noindent \rule{\textwidth}{0.2pt}
\end{de}
\bigskip

\begin{ex}[{\C Exponential probabilities}] %ex 1.6.4
{\C For the exponential distribution of the previous Example we have}
\[F_X(x)=1-e^{-x}\]
{\C and, hence,}
\[f_X(x)=\frac{d}{dx}F_X(x)=e^{-x}.\]
\end{ex}
\bigskip

\noindent \rule{\textwidth}{0.2pt}
\begin{de}\footnote{摘自 {\C Casella and Berger 2002, Definition 4.5.10}}   %def 4.5.10
{\C Let $-\infty<\mu_X<\infty$, $-\infty<\mu_Y<\infty$, $0<\sigma_X$, $0<\sigma_Y$, and
$-1<\rho<1$ be five real numbers.  The bivariate normal pdf with means $\mu_X$ and $\mu_Y$,
variances $\sigma_X^2$ and $\sigma_Y^2$, and correlation $\rho$ is the bivariate pdf given by}
\begin{eqnarray*}
f(x,y)&=&\left( 2\pi\sigma_X\sigma_Y\sqrt{1-\rho^2}\right)^{-1}\\
      && \times\exp\left(-\frac{1}{2(1-\rho^2)}\left(\left(\frac{x-\mu_x}{\sigma_X}\right)^2\right.\right.\\
      && \left.\left.-2\rho\left(\frac{x-\mu_x}{\sigma_X}\right)\left(\frac{y-\mu_y}{\sigma_Y}\right)
      +\left(\frac{y-\mu_y}{\sigma_Y}\right)^2 \right)\right)
\end{eqnarray*}
for $-\infty<x<\infty$ and $-\infty<y<\infty$.\\
\noindent \rule{\textwidth}{0.2pt}
\end{de}

\begin{ex}[{\C Bivariate Normal}] %ex 1.6.2
二維常態參數 $\mu_X=10, \mu_Y=20, \sigma_X=1, \sigma_Y=2, \rho=0.6$ 其分配函數為
\begin{equation*}
f(x,y)=\frac{1}{3.2\pi}\exp\left[-\frac{1}{1.28}\left((\frac{x-10}{1})^2
       -1.2(\frac{x-10}{1})(\frac{y-20}{2})+(\frac{y-20}{2})^2 \right)\right]
\end{equation*}
\end{ex}
\bigskip
\begin{center}\colorbox{slight}{\begin{tabular}{p{0.9\textwidth}}
\begin{thm}\label{demo_ref}\footnote{摘自 {\C Casella and Berger 2002, Theorem 5.3.1}} %theorem 5.3.1
{\C Let $X_1, \cdots, X_n$ be a random sample from a $N(\mu, \sigma^2)$ distribution, and let
$\bar{X}=\frac{1}{n}\sum_{i=1}^n X_i$ and $S^2=\frac{1}{n-1}\sum_{i=1}^n (X_i-\bar{X})^2$.  Then}
\begin{itemize}
\item[a.] {\C $\bar{X}$ and $S^2$ are independent random variables,}
\item[b.] {\C $\bar{X}$ has a $N(\mu, \sigma^2/n)$ distribution,}
\item[c.] {\C $(n-1)S^2/\sigma^2$ has a chi squared distribution with $n-1$ degrees of freedom.}
\end{itemize}
\end{thm}
 \end{tabular}}\end{center}
\bigskip

定理 \ref{demo_ref} 展示兩件式,其一是加入標號的引用($\backslash$label)與此處的參照對應,其二是自訂的項目符號。接著是個 {\E lemma} ,其編號隨著定理續編。
\begin{center}\colorbox{slight}{\begin{tabular}{p{0.9\textwidth}}
\begin{lemma}. {\C Let $a_1,a_2,\cdots$ be a sequence of numbers converging to $a$, that is, $\lim_{n\rightarrow \infty} a_n=a$. Then}
$$\lim_{n\rightarrow \infty} (1+\frac{a_n}{n})^n=e^n.$$
\end{lemma}
 \end{tabular}}\end{center}



%\end{document}
