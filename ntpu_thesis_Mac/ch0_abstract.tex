%\documentclass[12pt]{article}
%\usepackage{fancyhdr}
%\usepackage{indentfirst}

%\begin{document}
%\fontsize{12}{20pt}\selectfont
\fancyhf{}
\begin{center}
{\large 國立台北大學一$\bigcirc$二學年度第二學期碩士學位論文提要}
\fboxrule=1pt
\fboxsep=20pt
\fbox{\begin{minipage}{12cm}
\noindent 論文題目:~\underline{~~建構多變量二元聯合機率分配並應用於長期追蹤資料模型~~}

\noindent 論文頁數:~\underline{~~~85~~~}

\noindent 所~組~別:~\underline{~~統計學系碩士班~~}~~系(所)~~\underline{~~~~~~~~}~組~(學號:~\underline{~~710133121}~~)

\noindent 研~究~生:~\underline{~~~~~~葉麗芬~~~~~}~~指導教授:~\underline{~~~~黃怡婷~~~~}

\bigskip
\noindent 論文提要內容:

\qquad 現今許多科學研究常藉由觀察相同群體多個時間點的狀況來了解所關心事件對此群體所產生的長期平均影響,
這類型研究需要使用長期追蹤資料分析方法來瞭解影響平均反應趨勢的變數,以供後續決策或研究參考。

\qquad 在一階馬可夫鏈 (First-Order Markov Chains) 的假設下,本論文利用 Biswas 和 Hwang (2002) 提出之二元二項分配 (Bivariate Binomial Distribution) 建構出多變量聯合二元機率
分配,並討論該分配的特性,推導參數的最大概似估計式,及其大樣本性質。藉由此多變量分配,本論文提出利用最大概似估計法來估計長期追蹤資料之廣義線性模型中參數,
最後運用統計模擬來探討新的多變量二元聯合機率分配參數與長期追蹤資料之廣義線性模型參數的最大概似估計式表現,再與現行研究者常採用的廣義估計方程式的參數估計方法進行比較。

\vspace*{2cm}
\noindent 關鍵詞: 二元長期追蹤資料、多變量二元分配、長期追蹤資料的廣義線性模型、最大概似估計法
\medskip
\end{minipage}}
\end{center}

%%%%%%%%%%%%%%%%%%%%%%%%%%%% ABSTRACT %%%%%%%%%%%%%%%%%%%%%%%%%%%%%%%%%%%%%%%%%%%%%
\newpage
\thispagestyle{empty}
\fontsize{12}{18pt}\selectfont

\begin{center}{\Large \bf ABSTRACT}\\[20pt]
    {\large Building the Multivariate Joint Distribution for Binary Data and its Application in Longitudinal Marginal Model}\\[10pt]
        by\\[10pt]  YE,\,LI-FEN\\[10pt] July 2014
\end{center}
{\small ADVISOR: Dr. HWANG, YI-TING \\[5pt]
        DEPARTMENT: DEPARTMENT OF STATISTICS\\[5pt]
        MAJOR: STATISTICS\\[5pt]
        DEGREE: MASTER OF SCIENCE}\\[10pt]
\noindent
Many recent studies often observe the response variables repeatedly to understand the influence of certain conditions longitudinally.
The general linear model and generalized linear model for longitudinal data are used to make inference of this kind of data.
Since the response variable is observed repeatedly, the model settings and estimations would need the multivariate distribution.
Many continuous multivariate distributions have been proposed in the literatures.
However, owing to the complexity of describing the association among the multivariate discrete random variables, it is lack of the well-known distribution.
To estimate the parameter in the generalized linear model for longitudinal data, the generalized estimating equation (GEE) proposed by Liang and Zeger (1986) is a commonly used estimating method.

\vspace*{2cm}
\noindent {\scshape KEY WORDS}: Binary Longitudinal Data, Multivariate Binomial Distribution, Marginal Model, Maximum Likelihood Estimation.
%\end{document}

