%% !TEX TS-program = xelatex								
% !TEX encoding = UTF-8

\documentclass[12pt, a4paper]{article} 		
\usepackage{fontspec} 				% Font selection for XeLaTeX; see fontspec.pdf for documentation. 
%\usepackage[BoldFont, SlantFont]{xeCJK}% 中文使用 XeCJK,並模擬粗體與斜體(即可以用 \textbf{ } \textit{ })
\usepackage{xeCJK}							% 中文使用 XeCJK,但利用 \setCJKmainfont 定義粗體與斜體的字型
\defaultfontfeatures{Mapping=tex-text} 		% to support TeX conventions like ``---''
\usepackage{xunicode} 						% Unicode support for LaTeX character names (accents, European chars, etc)
\usepackage{xltxtra} 						% Extra customizations for XeLaTeX


\usepackage[parfill]{parskip} % Activate to begin paragraphs with an empty line rather than an indent
%\usepackage{geometry} % See geometry.pdf to learn the layout options. There are lots.
\usepackage[left=1.5in,right=1in,top=1in,bottom=1in]{geometry} %設定頁面的四個邊界
%---------------------------------------------------
 %防止Package reports “command already defined”出現
%\usepackage{amsmath, amssymb}
\usepackage{savesym}    
\usepackage{amsmath}
\savesymbol{iint}
\usepackage{txfonts}
\restoresymbol{TXF}{iint}  
%--------------------------------------------------
\usepackage{enumerate}

\usepackage{bm} %對希臘字母加粗

%-----------------------------------------------------------------------------------------------------------------------
%  主字型設定
\setCJKmainfont								% 設定中文內文字型
	[
		BoldFont=cwTeX Q Hei Bold			% 定義粗體的字型(依使用的電腦安裝的字型而定)
	]
%	{cwTeX Q Ming Medium} 					% 設定中文內文字型
	{新細明體}	
\setmainfont{Times New Roman}				% 設定英文內文字型
\setsansfont{Arial}							% used with {\sffamily ...}
%\setsansfont[Scale=MatchLowercase,Mapping=tex-text]{Gill Sans}
\setmonofont{Courier New}					% used with {\ttfamily ...}
%\setmonofont[Scale=MatchLowercase]{Andale Mono}
% 其他字型(隨使用的電腦安裝的字型不同,用註解的方式調整(打開或關閉))
% 英文字型
\newfontfamily{\C}{Cambria}					% 套用在內文中所有的英文字母
\newfontfamily{\A}{Arial}
\newfontfamily{\sC}[Scale=0.9]{Cambria}
\newfontfamily{\TNR}{Times New Roman}
\newfontfamily{\TN}[Scale=0.8]{Times New Roman}
\newfontfamily{\F}{Forte}
\newfontfamily{\JF}{JackeyFont}
\newfontfamily{\BR}{Bunny Rabbits}
\newfontfamily{\SF}{SanafonMaru}
% 中文字型
\newCJKfontfamily{\MJH}{微軟正黑體}			
\newCJKfontfamily{\sMLU}[Scale=0.8]{新細明體}	  
\newCJKfontfamily{\BK}{標楷體}    
\newCJKfontfamily{\UD}{UD Digi Kyokasho NP-B} 
\newCJKfontfamily{\BM}{BugMaruGothic} 
\newCJKfontfamily{\pig}{pigmo-00}
\newCJKfontfamily{\HC}{華康兒風體W4}
\newCJKfontfamily{\NC}{Nagurigaki Crayon}
\newCJKfontfamily{\OY}{onryou}
% 以下為自行安裝的字型:CwTex 組合
\newCJKfontfamily{\CF}{cwTeX Q Fangsong Medium}		% CwTex 仿宋體
\newCJKfontfamily{\BCF}[Scale=2.0]{cwTeX Q Fangsong Medium}
\newCJKfontfamily{\CB}{cwTeX Q Hei Bold}			% CwTex 粗黑體
\newCJKfontfamily{\CK}{cwTeX Q Kai Medium}   		% CwTex 楷體
\newCJKfontfamily{\CM}{cwTeX Q Ming Medium}			% CwTex 明體
\newCJKfontfamily{\CR}{cwTeX Q Yuan Medium}			% CwTex 圓體
%-----------------------------------------------------------------------------------------------------------------------
\XeTeXlinebreaklocale "zh"                  		%這兩行一定要加,中文才能自動換行
\XeTeXlinebreakskip = 0pt plus 1pt     				%這兩行一定要加,中文才能自動換行
%-----------------------------------------------------------------------------------------------------------------------
\newcommand{\cw}{\texttt{cw}\kern-.6pt\TeX}			% 這是 cwTex 的 logo 文字
\renewcommand{\tablename}{表}						% 改變表格標號文字為中文的「表」(預設為 Table)
\renewcommand{\figurename}{圖}						% 改變圖片標號文字為中文的「圖」(預設為 Figure)



%計數器---------------------------------------------------------------------------------------------------------
\let\openbox\relax   %避免! LaTeX Error: Command \openbox already defined. 
\usepackage{amsthm} % theroemstyle 需要使用的套件

\newtheorem{Def}{Definition}[section]		%definition獨立編號
\newtheorem{thm}{{\HC 定理}}[section]		%theorem 獨立編號,取中文名稱並給予不同字型
\newtheorem{lemma}[thm]{Lemma}				%lemma 與 theorem 共用編號
\newtheorem{ex}{{\F Example}}				%example 獨立編號,不編入小節數字,走流水號。也換個字型。
\newtheorem{EX}[ex]{{\HC 範例}} 				%定義與example共用編號的範例


%設定表格-------------------------------------------------------------------------------------------------------
\usepackage{array} 
\usepackage{booktabs}
\usepackage{multirow}
\usepackage{longtable}
\usepackage{dcolumn}   %用以對齊小數點 
\usepackage{graphicx}  %用以旋轉
\usepackage{diagbox} %製作斜線表頭 
%縮放
\newcommand{\bpara}[4]{ % #1 x; #2 y; #3 angle; #4 height
\begin{picture}(0,0)%
\setlength{\unitlength}{1pt}%
\put(#1,#2){\rotatebox{#3}{\raisebox{0mm}[0mm][0mm]{%
\makebox[0mm]{$\left.\rule{0mm}{#4pt}\right\}$}}}}%
\end{picture}}		

%表格內折行
\newcommand{\tabincell}[2]{\begin{tabular}{@{}#1@{}}#2\end{tabular}}  
		%用法:\tabincell{clr}{第一行\\第二行} 

%設定圖片-------------------------------------------------------------------------------------------------------
\usepackage{graphicx}		 	%插入圖片的套件
\usepackage{float} 				%設置圖片浮動位置
\usepackage{subfig} 			%插入多圖時用子圖顯示
\usepackage{wrapfig}			%文繞圖
%\usepackage{subfigure}
%\usepackage{graphicx, subfig, float} 		% support the \includegraphics command and options

	
\newcommand{\imgdir}{images/}		%設定圖形所在子目錄

%圖形樣式設計
\usepackage{picins}
%要把圖標號與圖形一起旋轉
\usepackage{blindtext}
\usepackage{adjustbox}



%設定顏色-------------------------------------------------------------------------------------------------------
\usepackage{color, xcolor}
\usepackage{colortbl}
\definecolor{slight}{gray}{0.6}			
\definecolor{lightpink}{rgb}{1.0, 0.71, 0.76}
\definecolor{lightskyblue}{rgb}{0.53, 0.81, 0.98}
\definecolor{lightsalmon}{rgb}{1.0, 0.63, 0.48}
\definecolor{champagne}{rgb}{0.97, 0.91, 0.81}
\definecolor{paleblue}{rgb}{0.69, 0.93, 0.93}
\definecolor{bananamania}{rgb}{0.98, 0.91, 0.71}
\definecolor{lavendergray}{rgb}{0.77, 0.76, 0.82}
\definecolor{lightyellow}{rgb}{1.0, 1.0, 0.88}
%-----------------------------------------------------------------------------------------------------------------------
%設定縮格(section或subsection底下默認不縮排)
\usepackage{indentfirst}
\setlength{\parindent}{2em}

%置入網頁連結---------------------------------------------------------------------
\usepackage[colorlinks,linkcolor=black, urlcolor=blue]{hyperref}
		%用法:\href{網頁連結},若連結為電子郵件:\href{mailto;電子郵件}
		
%參考資料樣式----------------------------------------------------------------------------------------------------
%\usepackage{natbib}
%\usepackage[sort&compress,square,comma,authoryear]{natbib}		
		
%-----------------------------------------------------------------------------------------------------------------------
%更改章節編號字體
%\usepackage{titlesec}
%\newfontfamily\sectionNC{Nagurigaki Crayon}
%\newfontfamily\subsectionLBP{Local BaseBall Park}
%\titleformat*{\section}{sectionNC}
%\titleformat*{\subsection}{sectionLBP}
%\title{第3章 \\ 二元長期追蹤資料之平均反應模型}
%\date{}
%\author{}
%\begin{document}
%\maketitle

\chapter{二元長期追蹤資料之平均反應模型}\label{ch:model}

\section{邊際模型與 MLE}\label{sec:mle}
\noindent 在已知隨機變數分配的情況下,邊際模型將可藉由最大概似估計法來估計參數。假設二元隨機變數服從第 \ref{ch:dist} 章所提的多變量二元分配(\ref{equ:jpmf}),
故在GLM之邏吉斯模型架構下,已知時間點 $t$ 試驗結果,時間點 $t+1$ 之平均反應模型為
\begin{align}\label{equ:regf}
    \mbox{logit}\left[\mbox{E}(Y_{i,t+1}=1 \,|\,Y_{it}=y_{it})\right] &=\bm{x}'_{i,t+1}\bm{\beta}+\gamma_t\, y_{it},\;\;t=1,\cdots,k-1,
\end{align}
其中, $\gamma_t$ 係配合聯合機率密度函數中的一階馬可夫鏈假設,用以解釋時間 $t$ 與 $t+1$ 間關係之參數。
此時,條件機率可表示如下:
\begin{align}\label{equ:wcovf2}
    f(y_{i,t+1}\,|\, y_{it} \,;\bm{x}_{i,t+1}) &= \frac{\exp[y_{i,t+1}(\bm{x}'_{i,t+1}\bm{\beta} +\gamma_t\, y_{it})]}{1+\exp(\bm{x}'_{i,t+1}\bm{\beta} +\gamma_t \, y_{it})}, \,\;y_{it},\;y_{i,t+1}=0,1,
\end{align}
而第 $i$ 個試驗對象 $k$ 次試驗結果的聯合機率模型可表示如下:
\begin{align*}%\label{equ:wcovf}
    &f(y_{i1},\ldots,y_{ik} \,;\bm{x}) \notag\\
        &\quad = f(y_{i1};\bm{x}_{i1})f(y_{i2} \,|\, y_{i1}\,;\bm{x}_{i2})f(y_{i3} \,|\, y_{i2}\,;\bm{x}_{i3}) \cdots f(y_{ik} \,|\, y_{i,k-1}\,;\bm{x}_{ik})\notag\\
        &\quad = f(y_{i1};\bm{x}_{i1}) \;\prod_{t=1}^{k-1}\; f(y_{i,t+1} \,|\, y_{it}\,;\bm{x}_{i,t+1}) \notag\\
        &\quad =\frac{\exp[y_1(\bm{x}'_{i1} \bm{\beta})]}{1+\exp(\bm{x}'_{i1}\bm{\beta})} \;\prod_{t=1}^{k-1}\; \frac{\exp[y_{i,t+1}(\bm{x}'_{i,t+1}\bm{\beta} +\gamma_t \, y_{it})]}{1+\exp(\bm{x}'_{i,t+1}\bm{\beta} +\gamma_t \, y_{it})},\;\forall t=1,\cdots,k-1\,\mbox{。}
\end{align*}
以 $\bm{\theta}_l$ 代表概似函數中所有參數, $\bm{\theta}_l=(\beta_1,\cdots,\beta_p,\gamma_1,\cdots,\gamma_{k-1})'$, 則此隨機樣本之對數概似函數為
\begin{align}
 \ell &=\log L(\bm{\theta}_l\,|\,\bm{y}) = \sum_{i=1}^n\, \log L_i(\bm{\theta}_l\,|\,\bm{y}_i)    \notag\\
      &=\sum_{i=1}^n\, \log \left[f(y_{i1};\bm{x}_{i1}) \;\prod_{t=1}^{k-1}\; f(y_{i,t+1} \,|\, y_{it}\,;\bm{x}_{i,t+1}) \right] \notag\\
      &=\sum_{i=1}^n\, \left\{ \log f(y_{i1};\bm{x}_{i1}) + \,\sum_{t=1}^{k-1} \, \log f(y_{i,t+1} \,|\, y_{it}\,;\bm{x}_{i,t+1}) \right\} \notag\\
      &=\sum_{i=1}^n \, \left\{ \log \left(\frac{\exp[y_{i1}(\bm{x}'_{i1}\bm{\beta})]}{1+\exp(\bm{x}'_{i1}\bm{\beta})} \right) + \sum_{t=1}^{k-1} \, \log \left( \frac{\exp[y_{i,t+1}(\bm{x}'_{i,t+1}\bm{\beta} +\gamma_t \, y_{it})]}{1+\exp(\bm{x}'_{i,t+1}\bm{\beta} +\gamma_t \, y_{it})} \right) \right\},
\end{align}

最後,藉由最大概似估計法估計模型中參數,及費雪訊息矩陣之反矩陣作為參數的共變異數矩陣,並分別以 $\hat{\bm{\theta}}_l$ 及 $\mbox{Cov}(\hat{\bm{\theta}}_l)$ 表示。
其數學推導過程同前面所述,在此僅列示對數概似函數中各參數一階偏微分結果如下:

\begin{align*}
    \frac{\partial \ell}{\partial \beta_j} &=\sum_{i=1}^n \,\Bigg [ \frac{y_{i1}\, x_{i1j} + x_{i1j}\,(y_{i1}-1)\exp(\bm{x}'_{i1}\bm{\beta})}{1+\exp(\bm{x}'_{i1}\bm{\beta})} \\
                                           &\qquad + \sum_{t=1}^{k-1} \frac{y_{i,t+1} \, x_{itj} + x_{itj}\,(y_{i,t+1}-1)\, \exp(\bm{x}'_{i,t+1}\bm{\beta} +\gamma_t \, y_{it})}{1+\exp(\bm{x}'_{i,t+1}\bm{\beta} +\gamma_t \, y_{it})} \Bigg],\;\; j=1,\ldots,p, \\
    \frac{\partial \ell}{\partial \gamma_s} &= \frac{\partial}{\partial \gamma_s} \left\{ \sum_{i=1}^n \, \log  \frac{\exp[\,y_{i,s+1}(\bm{x}'_{i,s+1}\bm{\beta} +\gamma_s \, y_{is})\,]}{1+\exp(\bm{x}'_{i,s+1}\bm{\beta} +\gamma_s \, y_{is})} \right\} \\
                                            &=\sum_{i=1}^n \, \frac{y_{is}\, y_{i,s+1}+ y_{is}\,(y_{i,s+1}-1)\,\exp (\bm{x}'_{i,s+1} \bm{\beta} +\gamma_s \, y_{is})}{1+\exp(\bm{x}'_{i,s+1} \bm{\beta} +\gamma_s \, y_{is})},\;\; s=1,\ldots,k-1,\\
\end{align*}

\noindent 二階偏微分結果如下:
\begin{align*}
    \frac{\partial}{\partial \beta_l}\frac{\partial}{\partial \beta_j} \ell &= - \sum_{i=1}^n \, \left\{ \frac{x_{i1l}\, x_{i1j} \exp(\bm{x}'_{i1}\bm{\beta})}{[1+\exp(\bm{x}'_{i1}\bm{\beta})]^2} + \sum_{t=1}^{k-1} \, \frac{ x_{itl} \, x_{itj} \exp(\bm{x}'_{i,t+1}\bm{\beta} +\gamma_t \, y_{it})}{[1+\exp(\bm{x}'_{i,t+1}\bm{\beta} +\gamma_t \, y_{it})]^2} \right\},\\
             &\qquad\qquad\qquad\qquad\qquad\qquad\qquad\qquad\qquad\qquad\qquad\qquad l, j=1,\ldots,p, \\
    \frac{\partial}{\partial \beta_j}\frac{\partial}{\partial \gamma_s} \ell &= - \sum_{i=1}^n \,  \frac{x_{i,s+1,j} \, y_{is} \exp (\bm{x}'_{i,s+1}\bm{\beta} +\gamma_s \, y_{is})}{[1+ \exp (\bm{x}'_{i,s+1}\bm{\beta} +\gamma_s \, y_{is})]^2},\; j=1,\ldots,p; s=1,\ldots,k-1, \\
    \frac{\partial^2 \ell}{\partial \gamma^2_s} &= - \sum_{i=1}^n \, \frac{y^2_{is} \exp (\bm{x}'_{i,s+1}\bm{\beta} +\gamma_s \, y_{is})}{[1+ \exp (\bm{x}'_{i,s+1} \bm{\beta} +\gamma_s \, y_{is})]^2},\;\; s=1,\ldots,k-1\mbox{。}
\end{align*}

\noindent 在不失一般性假設下, MLE 具有一致性(Consistency)。此外,當 $\mbox{Cov}(\hat{\bm{\theta}}_l)$ 係藉由費雪訊息矩陣的反矩陣估計時,此共變異數估計值為估計誤差的下界(即 Cram\'{e}r-Rao Lower Bound),
故根據 MLE 近似有效性(Asymptotic Efficiency)性質(Casella $\&$ Berger, 1990),本論文在多變量二元分配假設下所建構之邊際模型,
其MLE在大樣本假設下近似多變量常態分配,亦即
    $$\sqrt{n}(\hat{\bm{\theta}}_l - \bm{\theta}_l) \rightarrow \mbox{MVN}[\bm{0},\mbox{Cov}(\hat{\bm{\theta}}_l)]\,\mbox{。} $$


%\end{document}







