%\documentclass[12pt, oneside, a4paper]{book}
%\documentclass[12pt, a4paper]{book}
%----- 定義使用的 packages ----------------------

\usepackage{fontspec} 										% Font selection for XeLaTeX; see fontspec.pdf for documentation. 
\usepackage{xeCJK}											% 中文使用 XeCJK,但利用 \setCJKmainfont 定義粗體與斜體的字型
\defaultfontfeatures{Mapping=tex-text} 				% to support TeX conventions like ``---''
\usepackage{xunicode} 										% Unicode support for LaTeX character names (accents, European chars, etc)
\usepackage{xltxtra} 											% Extra customizations for XeLaTeX
\usepackage[sf,small]{titlesec}
\usepackage{amsmath, amssymb}
\usepackage{amsthm}										% theroemstyle 需要使用的套件
\usepackage{bm}                                                 % 排版粗體數學符號
\usepackage{enumerate}
\usepackage{graphicx, subfig, float} 					% support the \includegraphics command and options
\usepackage{array}
\usepackage{color, xcolor}
\usepackage{longtable, lscape}                                   % 跨頁的超長表格;lscape是旋轉此類表格的
\usepackage{threeparttable}                                     % 巨集,使表格加註解更容易(手冊p169)
\usepackage{multirow, booktabs}                                   % 讓表格編起來更美的套件(手冊p166),編輯跨列標題重覆的表格(手冊p182)
\usepackage{colortbl}                          				%.............................................表格標題註解之巨集套件
\usepackage{natbib}											% for Reference
\usepackage{makeidx}										% for Indexing
\usepackage[parfill]{parskip} % Activate to begin paragraphs with an empty line rather than an indent
%\usepackage{geometry} % See geometry.pdf to learn the layout options. There are lots.
%\usepackage[left=1.5in,right=1in,top=1in,bottom=1in]{geometry} 
\usepackage{url}                                                % 文稿內徵引網址
    \def\UrlFont{\rm}                                           % 網頁
\usepackage{fancyhdr}
	\pagestyle{fancy}
	\fancyhf{}                                     % 清除所有頁眉頁足
	\renewcommand{\headrulewidth}{0pt}                              % 頁眉下方的橫線    
%-----------------------------------------------------------------------------------------------------------------------
%  主字型設定
\setCJKmainfont
	[
		BoldFont=Heiti TC Medium								% 定義粗體的字型(依使用的電腦安裝的字型而定)
	]
	{cwTeX Q Ming Medium} 										% 設定中文內文字型
%	{新細明體}	
\setmainfont{Times New Roman}								% 設定英文內文字型
\setsansfont{Arial}														% used with {\sffamily ...}
%\setsansfont[Scale=MatchLowercase,Mapping=tex-text]{Gill Sans}
\setmonofont{Courier New}										% used with {\ttfamily ...}
%\setmonofont[Scale=MatchLowercase]{Andale Mono}
% 其他字型(隨使用的電腦安裝的字型不同,用註解的方式調整(打開或關閉))
% 英文字型
\newfontfamily{\E}{Cambria}										% 套用在內文中所有的英文字母
\newfontfamily{\A}{Arial}
\newfontfamily{\C}[Scale=0.9]{Cambria}
\newfontfamily{\T}{Times New Roman}
\newfontfamily{\TT}[Scale=0.8]{Times New Roman}
% 中文字型
\newCJKfontfamily{\MB}{微軟正黑體}							% 適用在 Mac 與 Win
\newCJKfontfamily{\SM}[Scale=0.8]{新細明體}				% 縮小版
%\newCJKfontfamily{\K}{標楷體}                        			% Windows 下的標楷體
\newCJKfontfamily{\K}{Kaiti TC Regular}         			% Mac OS 下的標楷體
\newCJKfontfamily{\BM}{Heiti TC Medium}					% Mac OS 下的黑體(粗體)
\newCJKfontfamily{\SR}{Songti TC Regular}				% Mac OS 下的宋體
\newCJKfontfamily{\SB}{Songti TC Bold}					% Mac OS 下的宋體(粗體)
\newCJKfontfamily{\CF}{cwTeX Q Fangsong Medium}	% CwTex 仿宋體
\newCJKfontfamily{\CB}{cwTeX Q Hei Bold}				% CwTex 粗黑體
\newCJKfontfamily{\CK}{cwTeX Q Kai Medium}   		% CwTex 楷體
\newCJKfontfamily{\CM}{cwTeX Q Ming Medium}		% CwTex 明體
\newCJKfontfamily{\CR}{cwTeX Q Yuan Medium}		% CwTex 圓體
%-----------------------------------------------------------------------------------------------------------------------
\XeTeXlinebreaklocale "zh"                  				%這兩行一定要加,中文才能自動換行
\XeTeXlinebreakskip = 0pt plus 1pt     %這兩行一定要加,中文才能自動換行
%-----------------------------------------------------------------------------------------------------------------------
%----- 重新定義的指令 ---------------------------
\newcommand{\cw}{\texttt{cw}\kern-.6pt\TeX}	% 這是 cwTex 的 logo 文字
\newcommand{\imgdir}{graph/}							% 設定圖檔的位置
\renewcommand{\tablename}{表}						% 改變表格標號文字為中文的「表」(預設為 Table)
\renewcommand{\figurename}{圖}						% 改變圖片標號文字為中文的「圖」(預設為 Figure)
\renewcommand{\contentsname}{目~錄}
\renewcommand\listfigurename{圖目錄}
\renewcommand\listtablename{表目錄}
\renewcommand{\appendixname}{附~錄}                  
\renewcommand{\indexname}{索引}
\renewcommand{\bibname}{參考文獻}
%-----------------------------------------------------------------------------------------------------------------------

\theoremstyle{plain}
\newtheorem{de}{Definition}[section]				%definition獨立編號
\newtheorem{thm}{定理}[section]			%theorem 獨立編號,取中文名稱並給予不同字型
\newtheorem{lemma}[thm]{引理}				%lemma 與 theorem 共用編號
\newtheorem{ex}{{\E Example}}						%example 獨立編號,不編入小節數字,走流水號。也換個字型。
\newtheorem{cor}{Corollary}[section]				%not used here
\newtheorem{exercise}{EXERCISE}					%not used here
\newtheorem{re}{\emph{Result}}[section]		%not used here
\newtheorem{axiom}{AXIOM}							%not used here
\renewcommand{\proofname}{\textbf{Proof}}		%not used here

\newcommand{\loflabel}{圖} % 圖目錄出現 圖 x.x 的「圖」字
\newcommand{\lotlabel}{表}  % 表目錄出現 表 x.x 的「表」字

\parindent=0pt

%--- 其他定義 ----------------------------------
% 定義章節標題的字型、大小
\titleformat{\chapter}[display]{\raggedleft\LARGE\bfseries\CF}		% 定義章抬頭靠右(\reggedleft)
 { 第\ \thechapter\ 章}{0.2cm}{}
%\titleformat{\chapter}[hang]{\centering\LARGE\sf}{\MB 第~\thesection~章}{0.2cm}{}%控制章的字體
%\titleformat{\section}[hang]{\Large\sf}{\MB 第~\thesection~節}{0.2cm}{}%控制章的字體
%\titleformat{\subsection}[hang]{\centering\Large\sf}{\MB 第~\thesubsection~節}{0.2cm}{}%控制節的字體
%\titleformat*{\section}{\normalfont\Large\bfseries\MB}
%\titleformat*{\subsection}{\normalfont\large\bfseries\MB}
%\titleformat*{\subsubsection}{\normalfont\large\bfseries\MB}


% 顏色定義
\definecolor{heavy}{gray}{.9}								% 0.9深淺度之灰色
\definecolor{light}{gray}{.8}
\definecolor{pink}{rgb}{0.99,0.91,0.95}               % 定義pink顏色

%\title{第2章 \\ 多變量二元聯合機率分配}
%\date{}
%\author{}
%\begin{document}
%\maketitle

\chapter{多變量二元聯合機率分配}\label{ch:dist}
\noindent 本論文以指標 $i$ 代表某研究對象,指標 $t$ 代表某觀察或試驗時間點;並假設重複試驗次數為 $k$ 次,各時間點參與試驗之人數皆為 $n$。
令二元隨機變數 $Y_{it},\,i=1,\ldots,n,\;t=1,\ldots,k,$ 代表第 $i$ 個對象在時間點 $t$ 的試驗結果,
並假設 $Y_{it}$ 之邊際分配為參數 $p_t$ 的白努利分配, $t=1,\ldots,k$, 且不同研究對象間的試驗結果為獨立,但同一研究對象各次試驗結果間可能存在關係。
此外,令 $y_{it}$ 為其所對應的樣本值, $y_{it}=$ 0, 1, 並令
$$\bm{y}_i =\left(
            \begin{array}{c}
             y_{i1} \\
             y_{i2} \\
             \vdots \\
             y_{ik}
            \end{array} \right),$$
為一個 $k \times 1$ 的向量,表示第 $i$ 個試驗對象 $k$ 次試驗結果。
於論文中若未重新進行符號定義,在不造成閱讀困擾的前提下,將以 $Y_t$ 表示第 $i$ 個對象在時間點 $t$ 的試驗結果,以簡化符號。
本章將先說明成對資料之聯合機率函數的建構概念,之後再將此方法一般化,提出新的多變量二元資料的聯合機率函數。

\section{多變量聯合機率質量函數}
\noindent 本論文假設 $\{Y_t,1\leq t \leq k\}$ 為一階馬可夫鏈(First-Order Markov Chains),亦即時間 $t+1$ 的狀態僅與時間 $t$ 有關;
利用 Biswas 和 Hwang (2002) 提出的條件機率公式定義時間 $t$ 與 $t+1$ 間診斷試驗結果的條件機率,提出隨機變數 $Y_1,\ldots,Y_k$ 之JPMF如下:
\begin{align}\label{equ:jpmf}
    f(y_1,\cdots,y_k) &=f(y_1)f(y_2|y_1)f(y_3|y_1,y_2) \cdots f(y_k|y_1,\ldots y_{k-1}) \notag\\
                      &=f(y_1)f(y_2|y_1)f(y_3|y_2) \cdots f(y_k|y_{k-1}) \notag\\
                      &=p_1^{y_1} (1-p_1)^{1-y_1} (\mbox{P}^{12}_{y_1,1})^{y_2} (1-\mbox{P}^{12}_{y_1,1})^{1-y_2} (\mbox{P}^{23}_{y_2,1})^{y_3} (1-\mbox{P}^{23}_{y_2,1})^{1-y_3} \notag\\
                      &\quad \times \cdots \times (\mbox{P}^{k-1,k}_{y_{k-1},1})^{y_k} (1-\mbox{P}^{k-1,k}_{y_{k-1},1})^{1-y_k} \notag\\
                      &= p_1^{y_1} (1-p_1)^{1-y_1} \, \prod_{t=1}^{k-1} \, (\mbox{P}^{t,t+1}_{y_t,1})^{y_{t+1}} \, (1-\mbox{P}^{t,t+1}_{y_t,1})^{1-y_{t+1}}
\end{align}
在此函數假設下,任意兩時間點 $Y_t=1$ 與 $Y_u=1$ 的聯合機率跟時間點 $t$ 的變異數有關,關係式如引理~ \ref{lem:lem1} ,此證明請參考附錄 A; 而由引理~ \ref{lem:lem1},可證明出任意兩時間點診斷試驗結果的共變異數如定理~ \ref{thm:cov}。
\begin{lemma}\label{lem:lem1}
假設 $\sigma_t^2$ 為診斷試驗結果 $Y_t$ 的變異數,則任意兩時間點 $Y_t$ 與 $Y_u$ 事件皆發生的機率為
$$\mbox{\textup{Pr}}(Y_t=1,Y_u=1)=p_t p_u + \sigma_t^2 \, \prod_{m=t}^{u-1} \frac{\alpha_{m,m+1}}{1+\alpha_{m,m+1}},\;\;\forall\; 1\leq t<u \leq k$$
\end{lemma}

\begin{thm}\label{thm:cov}
假設 $\sigma_t^2$ 為診斷試驗結果 $Y_t$ 的變異數,則任意兩時間點 $Y_t$ 與 $Y_u$ 的共變異數為
$$\mbox{\textup{Cov}}(Y_t,Y_u)=\sigma_t^2 \, \prod_{m=t}^{u-1} \frac{\alpha_{m,m+1}}{1+\alpha_{m,m+1}} , \;\;\;\;\forall\; 1\leq t<u \leq k$$
\end{thm}

\begin{proof}

由共變異數之定義可得
\begin{align*}
  \mbox{\textup{Cov}}(Y_t,Y_u) &= \mbox{\textup{E}}(Y_t Y_u)-\mbox{\textup{E}}(Y_t)\mbox{\textup{E}}(Y_u) \\
                      &= \sum_{y_t}\sum_{y_u}\; y_t y_u f(y_t,y_u)-p_t p_u \\
                      &= \mbox{Pr}(Y_t=1,Y_u=1) - p_t p_u \\
                      &= p_t p_u + \sigma_t^2 \, \prod_{m=t}^{u-1} \frac{\alpha_{m,m+1}}{1+\alpha_{m,m+1}}  - p_t p_u \\
                      &= \sigma_t^2 \, \prod_{m=t}^{u-1} \frac{\alpha_{m,m+1}}{1+\alpha_{m,m+1}}
\end{align*}
\end{proof}


%\end{document}

