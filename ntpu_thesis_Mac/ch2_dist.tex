%% !TEX TS-program = xelatex								
% !TEX encoding = UTF-8

\documentclass[12pt, a4paper]{article} 		
\usepackage{fontspec} 				% Font selection for XeLaTeX; see fontspec.pdf for documentation. 
%\usepackage[BoldFont, SlantFont]{xeCJK}% 中文使用 XeCJK,並模擬粗體與斜體(即可以用 \textbf{ } \textit{ })
\usepackage{xeCJK}							% 中文使用 XeCJK,但利用 \setCJKmainfont 定義粗體與斜體的字型
\defaultfontfeatures{Mapping=tex-text} 		% to support TeX conventions like ``---''
\usepackage{xunicode} 						% Unicode support for LaTeX character names (accents, European chars, etc)
\usepackage{xltxtra} 						% Extra customizations for XeLaTeX


\usepackage[parfill]{parskip} % Activate to begin paragraphs with an empty line rather than an indent
%\usepackage{geometry} % See geometry.pdf to learn the layout options. There are lots.
\usepackage[left=1.5in,right=1in,top=1in,bottom=1in]{geometry} %設定頁面的四個邊界
%---------------------------------------------------
 %防止Package reports “command already defined”出現
%\usepackage{amsmath, amssymb}
\usepackage{savesym}    
\usepackage{amsmath}
\savesymbol{iint}
\usepackage{txfonts}
\restoresymbol{TXF}{iint}  
%--------------------------------------------------
\usepackage{enumerate}

\usepackage{bm} %對希臘字母加粗

%-----------------------------------------------------------------------------------------------------------------------
%  主字型設定
\setCJKmainfont								% 設定中文內文字型
	[
		BoldFont=cwTeX Q Hei Bold			% 定義粗體的字型(依使用的電腦安裝的字型而定)
	]
%	{cwTeX Q Ming Medium} 					% 設定中文內文字型
	{新細明體}	
\setmainfont{Times New Roman}				% 設定英文內文字型
\setsansfont{Arial}							% used with {\sffamily ...}
%\setsansfont[Scale=MatchLowercase,Mapping=tex-text]{Gill Sans}
\setmonofont{Courier New}					% used with {\ttfamily ...}
%\setmonofont[Scale=MatchLowercase]{Andale Mono}
% 其他字型(隨使用的電腦安裝的字型不同,用註解的方式調整(打開或關閉))
% 英文字型
\newfontfamily{\C}{Cambria}					% 套用在內文中所有的英文字母
\newfontfamily{\A}{Arial}
\newfontfamily{\sC}[Scale=0.9]{Cambria}
\newfontfamily{\TNR}{Times New Roman}
\newfontfamily{\TN}[Scale=0.8]{Times New Roman}
\newfontfamily{\F}{Forte}
\newfontfamily{\JF}{JackeyFont}
\newfontfamily{\BR}{Bunny Rabbits}
\newfontfamily{\SF}{SanafonMaru}
% 中文字型
\newCJKfontfamily{\MJH}{微軟正黑體}			
\newCJKfontfamily{\sMLU}[Scale=0.8]{新細明體}	  
\newCJKfontfamily{\BK}{標楷體}    
\newCJKfontfamily{\UD}{UD Digi Kyokasho NP-B} 
\newCJKfontfamily{\BM}{BugMaruGothic} 
\newCJKfontfamily{\pig}{pigmo-00}
\newCJKfontfamily{\HC}{華康兒風體W4}
\newCJKfontfamily{\NC}{Nagurigaki Crayon}
\newCJKfontfamily{\OY}{onryou}
% 以下為自行安裝的字型:CwTex 組合
\newCJKfontfamily{\CF}{cwTeX Q Fangsong Medium}		% CwTex 仿宋體
\newCJKfontfamily{\BCF}[Scale=2.0]{cwTeX Q Fangsong Medium}
\newCJKfontfamily{\CB}{cwTeX Q Hei Bold}			% CwTex 粗黑體
\newCJKfontfamily{\CK}{cwTeX Q Kai Medium}   		% CwTex 楷體
\newCJKfontfamily{\CM}{cwTeX Q Ming Medium}			% CwTex 明體
\newCJKfontfamily{\CR}{cwTeX Q Yuan Medium}			% CwTex 圓體
%-----------------------------------------------------------------------------------------------------------------------
\XeTeXlinebreaklocale "zh"                  		%這兩行一定要加,中文才能自動換行
\XeTeXlinebreakskip = 0pt plus 1pt     				%這兩行一定要加,中文才能自動換行
%-----------------------------------------------------------------------------------------------------------------------
\newcommand{\cw}{\texttt{cw}\kern-.6pt\TeX}			% 這是 cwTex 的 logo 文字
\renewcommand{\tablename}{表}						% 改變表格標號文字為中文的「表」(預設為 Table)
\renewcommand{\figurename}{圖}						% 改變圖片標號文字為中文的「圖」(預設為 Figure)



%計數器---------------------------------------------------------------------------------------------------------
\let\openbox\relax   %避免! LaTeX Error: Command \openbox already defined. 
\usepackage{amsthm} % theroemstyle 需要使用的套件

\newtheorem{Def}{Definition}[section]		%definition獨立編號
\newtheorem{thm}{{\HC 定理}}[section]		%theorem 獨立編號,取中文名稱並給予不同字型
\newtheorem{lemma}[thm]{Lemma}				%lemma 與 theorem 共用編號
\newtheorem{ex}{{\F Example}}				%example 獨立編號,不編入小節數字,走流水號。也換個字型。
\newtheorem{EX}[ex]{{\HC 範例}} 				%定義與example共用編號的範例


%設定表格-------------------------------------------------------------------------------------------------------
\usepackage{array} 
\usepackage{booktabs}
\usepackage{multirow}
\usepackage{longtable}
\usepackage{dcolumn}   %用以對齊小數點 
\usepackage{graphicx}  %用以旋轉
\usepackage{diagbox} %製作斜線表頭 
%縮放
\newcommand{\bpara}[4]{ % #1 x; #2 y; #3 angle; #4 height
\begin{picture}(0,0)%
\setlength{\unitlength}{1pt}%
\put(#1,#2){\rotatebox{#3}{\raisebox{0mm}[0mm][0mm]{%
\makebox[0mm]{$\left.\rule{0mm}{#4pt}\right\}$}}}}%
\end{picture}}		

%表格內折行
\newcommand{\tabincell}[2]{\begin{tabular}{@{}#1@{}}#2\end{tabular}}  
		%用法:\tabincell{clr}{第一行\\第二行} 

%設定圖片-------------------------------------------------------------------------------------------------------
\usepackage{graphicx}		 	%插入圖片的套件
\usepackage{float} 				%設置圖片浮動位置
\usepackage{subfig} 			%插入多圖時用子圖顯示
\usepackage{wrapfig}			%文繞圖
%\usepackage{subfigure}
%\usepackage{graphicx, subfig, float} 		% support the \includegraphics command and options

	
\newcommand{\imgdir}{images/}		%設定圖形所在子目錄

%圖形樣式設計
\usepackage{picins}
%要把圖標號與圖形一起旋轉
\usepackage{blindtext}
\usepackage{adjustbox}



%設定顏色-------------------------------------------------------------------------------------------------------
\usepackage{color, xcolor}
\usepackage{colortbl}
\definecolor{slight}{gray}{0.6}			
\definecolor{lightpink}{rgb}{1.0, 0.71, 0.76}
\definecolor{lightskyblue}{rgb}{0.53, 0.81, 0.98}
\definecolor{lightsalmon}{rgb}{1.0, 0.63, 0.48}
\definecolor{champagne}{rgb}{0.97, 0.91, 0.81}
\definecolor{paleblue}{rgb}{0.69, 0.93, 0.93}
\definecolor{bananamania}{rgb}{0.98, 0.91, 0.71}
\definecolor{lavendergray}{rgb}{0.77, 0.76, 0.82}
\definecolor{lightyellow}{rgb}{1.0, 1.0, 0.88}
%-----------------------------------------------------------------------------------------------------------------------
%設定縮格(section或subsection底下默認不縮排)
\usepackage{indentfirst}
\setlength{\parindent}{2em}

%置入網頁連結---------------------------------------------------------------------
\usepackage[colorlinks,linkcolor=black, urlcolor=blue]{hyperref}
		%用法:\href{網頁連結},若連結為電子郵件:\href{mailto;電子郵件}
		
%參考資料樣式----------------------------------------------------------------------------------------------------
%\usepackage{natbib}
%\usepackage[sort&compress,square,comma,authoryear]{natbib}		
		
%-----------------------------------------------------------------------------------------------------------------------
%更改章節編號字體
%\usepackage{titlesec}
%\newfontfamily\sectionNC{Nagurigaki Crayon}
%\newfontfamily\subsectionLBP{Local BaseBall Park}
%\titleformat*{\section}{sectionNC}
%\titleformat*{\subsection}{sectionLBP}
%\title{第2章 \\ 多變量二元聯合機率分配}
%\date{}
%\author{}
%\begin{document}
%\maketitle

\chapter{多變量二元聯合機率分配}\label{ch:dist}
\noindent 本論文以指標 $i$ 代表某研究對象,指標 $t$ 代表某觀察或試驗時間點;並假設重複試驗次數為 $k$ 次,各時間點參與試驗之人數皆為 $n$。
令二元隨機變數 $Y_{it},\,i=1,\ldots,n,\;t=1,\ldots,k,$ 代表第 $i$ 個對象在時間點 $t$ 的試驗結果,
並假設 $Y_{it}$ 之邊際分配為參數 $p_t$ 的白努利分配, $t=1,\ldots,k$, 且不同研究對象間的試驗結果為獨立,但同一研究對象各次試驗結果間可能存在關係。
此外,令 $y_{it}$ 為其所對應的樣本值, $y_{it}=$ 0, 1, 並令
$$\bm{y}_i =\left(
            \begin{array}{c}
             y_{i1} \\
             y_{i2} \\
             \vdots \\
             y_{ik}
            \end{array} \right),$$
為一個 $k \times 1$ 的向量,表示第 $i$ 個試驗對象 $k$ 次試驗結果。
於論文中若未重新進行符號定義,在不造成閱讀困擾的前提下,將以 $Y_t$ 表示第 $i$ 個對象在時間點 $t$ 的試驗結果,以簡化符號。
本章將先說明成對資料之聯合機率函數的建構概念,之後再將此方法一般化,提出新的多變量二元資料的聯合機率函數。

\section{多變量聯合機率質量函數}
\noindent 本論文假設 $\{Y_t,1\leq t \leq k\}$ 為一階馬可夫鏈(First-Order Markov Chains),亦即時間 $t+1$ 的狀態僅與時間 $t$ 有關;
利用 Biswas 和 Hwang (2002) 提出的條件機率公式定義時間 $t$ 與 $t+1$ 間診斷試驗結果的條件機率,提出隨機變數 $Y_1,\ldots,Y_k$ 之JPMF如下:
\begin{align}\label{equ:jpmf}
    f(y_1,\cdots,y_k) &=f(y_1)f(y_2|y_1)f(y_3|y_1,y_2) \cdots f(y_k|y_1,\ldots y_{k-1}) \notag\\
                      &=f(y_1)f(y_2|y_1)f(y_3|y_2) \cdots f(y_k|y_{k-1}) \notag\\
                      &=p_1^{y_1} (1-p_1)^{1-y_1} (\mbox{P}^{12}_{y_1,1})^{y_2} (1-\mbox{P}^{12}_{y_1,1})^{1-y_2} (\mbox{P}^{23}_{y_2,1})^{y_3} (1-\mbox{P}^{23}_{y_2,1})^{1-y_3} \notag\\
                      &\quad \times \cdots \times (\mbox{P}^{k-1,k}_{y_{k-1},1})^{y_k} (1-\mbox{P}^{k-1,k}_{y_{k-1},1})^{1-y_k} \notag\\
                      &= p_1^{y_1} (1-p_1)^{1-y_1} \, \prod_{t=1}^{k-1} \, (\mbox{P}^{t,t+1}_{y_t,1})^{y_{t+1}} \, (1-\mbox{P}^{t,t+1}_{y_t,1})^{1-y_{t+1}}
\end{align}
在此函數假設下,任意兩時間點 $Y_t=1$ 與 $Y_u=1$ 的聯合機率跟時間點 $t$ 的變異數有關,關係式如引理~ \ref{lem:lem1} ,此證明請參考附錄 A; 而由引理~ \ref{lem:lem1},可證明出任意兩時間點診斷試驗結果的共變異數如定理~ \ref{thm:cov}。
\begin{lemma}\label{lem:lem1}
假設 $\sigma_t^2$ 為診斷試驗結果 $Y_t$ 的變異數,則任意兩時間點 $Y_t$ 與 $Y_u$ 事件皆發生的機率為
$$\mbox{\textup{Pr}}(Y_t=1,Y_u=1)=p_t p_u + \sigma_t^2 \, \prod_{m=t}^{u-1} \frac{\alpha_{m,m+1}}{1+\alpha_{m,m+1}},\;\;\forall\; 1\leq t<u \leq k$$
\end{lemma}

\begin{thm}\label{thm:cov}
假設 $\sigma_t^2$ 為診斷試驗結果 $Y_t$ 的變異數,則任意兩時間點 $Y_t$ 與 $Y_u$ 的共變異數為
$$\mbox{\textup{Cov}}(Y_t,Y_u)=\sigma_t^2 \, \prod_{m=t}^{u-1} \frac{\alpha_{m,m+1}}{1+\alpha_{m,m+1}} , \;\;\;\;\forall\; 1\leq t<u \leq k$$
\end{thm}

\begin{proof}

由共變異數之定義可得
\begin{align*}
  \mbox{\textup{Cov}}(Y_t,Y_u) &= \mbox{\textup{E}}(Y_t Y_u)-\mbox{\textup{E}}(Y_t)\mbox{\textup{E}}(Y_u) \\
                      &= \sum_{y_t}\sum_{y_u}\; y_t y_u f(y_t,y_u)-p_t p_u \\
                      &= \mbox{Pr}(Y_t=1,Y_u=1) - p_t p_u \\
                      &= p_t p_u + \sigma_t^2 \, \prod_{m=t}^{u-1} \frac{\alpha_{m,m+1}}{1+\alpha_{m,m+1}}  - p_t p_u \\
                      &= \sigma_t^2 \, \prod_{m=t}^{u-1} \frac{\alpha_{m,m+1}}{1+\alpha_{m,m+1}}
\end{align*}
\end{proof}


%\end{document}

