\documentclass[12pt, a4paper]{article} 	
\usepackage{xeCJK}					% 中文使用 XeCJK,但利用 \setCJKmainfont 定義字型
\usepackage{xltxtra} 				% Extra customizations for XeLaTeX
\setCJKmainfont{新細明體}				% 設定中文內文字型(Win)
% \setCJKmainfont{宋體-繁}			% 設定中文內文字型(Mac)
\setmainfont{Times New Roman}		% 設定英文內文字型(Win & Mac)
% 設定其他字型:中英文不同  ------------------------------
\newfontfamily{\A}{Arial}									
\newfontfamily{\C}[Scale=0.9]{Arial} % Try other fonts
\newCJKfontfamily{\MB}{微軟正黑體}			% 適用在 Win
%\newCJKfontfamily{\MB}{黑體-繁}				% 適用在 Mac
\newCJKfontfamily{\SM}[Scale=0.8]{新細明體}	% 縮小版新細明體(Win)
%\newCJKfontfamily{\SM}[Scale=0.8]{宋體-繁}	% 縮小版新細明體(Mac)
%-------這兩行一定要加,中文才能自動換行-------------------
\XeTeXlinebreaklocale "zh"                  
\XeTeXlinebreakskip = 0pt plus 1pt     
%-----------------------------------------------------
\title{{\MB 中文字型與套件}}			% 使用設定的字型
\author{{\SM 汪群超}}					% 使用設定的小字體
\date{{\C \today} } 				
         									
\begin{document}
\maketitle
\fontsize{12}{22pt}\selectfont % 設定在本行之後的字型大小與行距。此設定與 WORD 上的 pt 大小不一致

示範中文:暑假即將來臨,同學和家長都開始計劃如何有效地利用這一段寶貴的假期。也一定有不少家長希望能在這一段時間為孩子加強英文的能力。尚未讓孩子開始學習英文的父母,則想在這段期間內為孩子奠定良好的基礎。本人也曾經面臨小孩學習英文的一些問題。目前問題似乎已經不復存在,故想把小孩習得英文的經驗提供給大家參考。

\section{其他語系}
以下小節示範英文、簡體中文及日本。代表 \XeLaTeX 支援 unicode 不同語系字型的能力。
\subsection{English}
The test statistics of assessing multivariate normality based on Roy’s union-intersection principle (Roy, Some Aspects of Multivariate Analysis, Wiley, New York, 1953) are generalizations of univariate normality, and are formed as the optimal value of a nonlinear multivariate function. Due to the difficulty of solving multivariate optimization problems, researchers have proposed various approximations.


\subsection{简体中文}
每个人生来平等,享有相同的地位和权利。\footnote{資料來自維基百科 https://zh.wikipedia.org/zh-tw/XeTeX}

\subsection{日本語}
すべての人間は自由であり、かつ、尊厳と権利とについて平等である。

\section{其他中英文字型示範}
\subsection{其他中文字型}
{\MB 這是微軟正黑體。}
\subsection{其他英文字型}
{\A Arial:}\\
{\A The test statistics of assessing multivariate normality based on Roy's union-intersection principle.}\\

英文粗體\\
\textbf{ The test statistics of assessing multivariate normality based on Roy's union-intersection principle.}\\
\end{document}
