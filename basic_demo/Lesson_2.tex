\documentclass[12pt, a4paper]{article} 	
\usepackage{xltxtra} 			% Extra customizations for XeLaTeX
\setmainfont[Mapping=tex-text]{細明體}		% Win 內文預設字型
%\setmainfont[Mapping=tex-text]{宋體-繁}	% Mac 內文預設字型 

% 設定其他字型 ---------------------------------
\newfontfamily{\A}{Arial}									
\newfontfamily{\T}{Times New Roman} 
\newfontfamily{\MB}{微軟正黑體}  	% 設定新字型(Win)
%\newfontfamily{\MB}{黑體-繁}  	% 設定新字型(Mac)

%-------這兩行一定要加,中文才能自動換行-----------
\XeTeXlinebreaklocale "zh"                  
\XeTeXlinebreakskip = 0pt plus 1pt    
 
%---------------------------------------------
\title{ Lesson 2: \XeLaTeX 的多國語系與多種字型}		
\author{汪群超}						 	
\date{\today } 
         										 
\begin{document}
\maketitle
\fontsize{12}{22pt}\selectfont % 設定在本行之後的字型大小與行距。此設定與 WORD 上的 pt 大小不一致。

示範中文:暑假即將來臨,同學和家長都開始計劃如何有效地利用這一段寶貴的假期。也一定有不少家長希望能在這一段時間為孩子加強英文的能力。尚未讓孩子開始學習英文的父母,則想在這段期間內為孩子奠定良好的基礎。本人也曾經面臨小孩學習英文的一些問題。目前問題似乎已經不復存在,故想把小孩習得英文的經驗提供給大家參考。

\section{其他語系}
以下小節示範英文、簡體中文及日本。代表 \XeTeX 支援 unicode 不同語系字型的能力。
\subsection{English}
The test statistics of assessing multivariate normality based on Roy's union-intersection principle (Roy,Some Aspects of Multivariate Analysis, Wiley, New York, 1953) are generalizations of univariate normality, and areformed as the optimal value of a nonlinear multivariate function. Due to the difficulty of solving multivariate optimization problems, researchers have proposed various approximations.

\subsection{简体中文}
每个人生来平等,享有相同的地位和权利。\footnote{資料來自維基百科 https://zh.wikipedia.org/zh-tw/XeTeX}

\subsection{日本語}
すべての人間は自由であり、かつ、尊厳と権利とについて平等である。

\section{其他中英文字型示範}
\subsection{其他中文字型}
{\MB 這是微軟正黑體(Windows)或 蘋果專用黑體(Mac)}

\subsection{其他英文字型}
{\A Arial:}\\
{\A The test statistics of assessing multivariate normality based on Roy's union-intersection principle.}\\

\noindent {\T Times New Roman:\\}
{\T The test statistics of assessing multivariate normality based on Roy's union-intersection principle.}

\end{document}
