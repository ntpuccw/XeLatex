%\documentclass[14pt, handout]{beamer} 		% for print 
\documentclass[12pt]{beamer} 					% for projector
%- begin of Beamer setting--------------------------------------
%\setbeamercovered{transparent}
\usetheme{Malmoe}
%\usetheme{Darmstadt}
%\usetheme[secheader]{Boadilla}
\usecolortheme{dolphin}
\usefonttheme{structurebold}
%\usefonttheme{professionalfonts}
%\useoutertheme{miniframes}					% 外觀上層增加點選的連結
%\useoutertheme{infolines}
\usepackage{xmpmulti}
\linespread{1.2}
\newenvironment{num}							% 新定義的項目符號
 {\leftmargini=6mm\leftmarginii=8mm
  \begin{itemize}}{\end{itemize}}
%- end of Beamer setting------------------------------------------

\usepackage{fontspec} 						% Font selection for XeLaTeX; 
%\usepackage[BoldFont, SlantFont]{xeCJK}		% 中文使用 XeCJK,並模擬粗體與斜體)
\usepackage{xeCJK}							% 利用 \setCJKmainfont 定義粗體與斜體的字型
\defaultfontfeatures{Mapping=tex-text} 		% to support TeX conventions like ``---''
\usepackage{xunicode} 						% Unicode support for LaTeX character names (accents, European chars, etc)
\usepackage{xltxtra} 						% Extra customizations for XeLaTeX
\usepackage{amsmath, amssymb}
\usepackage{enumerate}
\usepackage{graphicx, subfig, float} 			% support the \includegraphics command and options
\usepackage{array, booktabs}
\usepackage{color, xcolor}
\usepackage{longtable}
\usepackage{colortbl}                          				
%-----------------------------------------------------------------------------------------------------------------------
%  主字型設定
\setCJKmainfont
	[
		BoldFont=微軟正黑體			% 定義粗體的字型(依使用的電腦安裝的字型而定)
	]
	{新細明體}						% 設定中文內文字型

\setmainfont{Times New Roman}			% 設定英文內文字型
\setsansfont{Arial}					% used with {\sffamily ...}
%\setsansfont[Scale=MatchLowercase,Mapping=tex-text]{Gill Sans}
\setmonofont{Courier New}			% used with {\ttfamily ...}
%\setmonofont[Scale=MatchLowercase]{Andale Mono}
% 其他字型(隨使用的電腦安裝的字型不同,用註解的方式調整(打開或關閉))
% 英文字型
\newfontfamily{\E}{Cambria}				% 套用在內文中所有的英文字母
\newfontfamily{\A}{Arial}
%\newfontfamily{\C}[Scale=0.9]{Cambria}
\newfontfamily{\R}{Times New Roman}
\newfontfamily{\TT}[Scale=0.8]{Times New Roman}
% 中文字型
\newCJKfontfamily{\MB}{微軟正黑體}				% 適用在 Mac 與 Win
\newCJKfontfamily{\SM}[Scale=0.8]{新細明體}	% 縮小版
%\newCJKfontfamily{\K}{標楷體}                	% Windows 下的標楷體
% 自行安裝的字體
%\newCJKfontfamily{\CF}{cwTeX Q Fangsong Medium}	% CwTex 仿宋體
%\newCJKfontfamily{\CB}{cwTeX Q Hei Bold}			% CwTex 粗黑體
%\newCJKfontfamily{\CK}{cwTeX Q Kai Medium}   	% CwTex 楷體
%\newCJKfontfamily{\CM}{cwTeX Q Ming Medium}		% CwTex 明體
%\newCJKfontfamily{\CR}{cwTeX Q Yuan Medium}		% CwTex 圓體
%-----------------------------------------------------------------------------------------------------------------------
\XeTeXlinebreaklocale "zh"                  		%這兩行一定要加,中文才能自動換行
\XeTeXlinebreakskip = 0pt plus 1pt     			%這兩行一定要加,中文才能自動換行
%-----------------------------------------------------------------------------------------------------------------------
\newcommand{\cw}{\texttt{cw}\kern-.6pt\TeX}	% 這是 cwTex 的 logo 文字
\newcommand{\imgdir}{images/}					% 設定圖檔的位置
\renewcommand{\tablename}{表}					% 改變表格標號文字為中文的「表」(預設為 Table)
\renewcommand{\figurename}{圖}				% 改變圖片標號文字為中文的「圖」(預設為 Figure)
\definecolor{slight}{gray}{0.9}				% 設定顏色