%% These two lines must be incldued to open file under UTF-8
% !TEX TS-program = xelatex								
% !TEX encoding = UTF-8

\documentclass[12pt, a4paper]{article} 		% use larger type; default would be 10pt
\usepackage{fontspec} 				% Font selection for XeLaTeX; see fontspec.pdf for documentation. 
%\usepackage[BoldFont, SlantFont]{xeCJK}% 中文使用 XeCJK,並模擬粗體與斜體(即可以用 \textbf{ } \textit{ })
\usepackage{xeCJK}							% 中文使用 XeCJK,但利用 \setCJKmainfont 定義粗體與斜體的字型
\defaultfontfeatures{Mapping=tex-text} 		% to support TeX conventions like ``---''
\usepackage{xunicode} 		% Unicode support for LaTeX character names (accents, European chars, etc)
\usepackage{xltxtra} 						% Extra customizations for XeLaTeX
\usepackage{amsmath, amssymb}
\usepackage{enumerate}
\usepackage{graphicx, subfig, float} 		% support the \includegraphics command and options
\usepackage{array, booktabs}
\usepackage{color, xcolor}
\usepackage{longtable}
\usepackage{colortbl}   
\usepackage{multirow} 
\usepackage{multicol} 
\usepackage{arydshln}  
\usepackage{dcolumn} 
\usepackage{rotating} 
\usepackage{diagbox} 
\usepackage{wrapfig} %文繞圖
\usepackage{overpic}%圖片上加文字
%\usepackage{natbib}
\usepackage[sort&compress,square,comma,authoryear]{natbib}
%.............................................表格標題註解之巨集套件
\usepackage[parfill]{parskip} % Activate to begin paragraphs with an empty line rather than an indent
%\usepackage{geometry} % See geometry.pdf to learn the layout options. There are lots.
\usepackage[left=1.5in,right=1in,top=1in,bottom=1in]{geometry} 

%-----------------------------------------------------------------------------------------------------------------------
%  主字型設定
\setCJKmainfont							% 設定中文內文字型
	[
		BoldFont=微軟正黑體				% 定義粗體的字型(依使用的電腦安裝的字型而定)
	]
	{新細明體}							% 設定中文內文字型
\setmainfont{Times New Roman}			% 設定英文內文字型
\setsansfont{Arial}						% used with {\sffamily ...}
%\setsansfont[Scale=MatchLowercase,Mapping=tex-text]{Gill Sans}
\setmonofont{Courier New}				% used with {\ttfamily ...}
%\setmonofont[Scale=MatchLowercase]{Andale Mono}
% 其他字型(隨使用的電腦安裝的字型不同,用註解的方式調整(打開或關閉))
% 英文字型
\newfontfamily{\E}{Cambria}				
\newfontfamily{\A}{Arial}
\newfontfamily{\C}[Scale=0.9]{Cambria}
%\newfontfamily{\T}{Times New Roman}
\newfontfamily{\TT}[Scale=0.8]{Times New Roman}

% 中文字型
\newCJKfontfamily{\MB}{微軟正黑體}			% 適用在 Mac 與 Win
\newCJKfontfamily{\SM}[Scale=0.8]{新細明體}	% 縮小版新細明體
\newCJKfontfamily{\K}{標楷體} 
               % Windows 下的標楷體
% 以下為自行安裝的字型:CwTex 組合
\newCJKfontfamily{\CF}{cwTeX Q Fangsong Medium}	% CwTex 仿宋體
\newCJKfontfamily{\CB}{cwTeX Q Hei Bold}			% CwTex 粗黑體
\newCJKfontfamily{\CK}{cwTeX Q Kai Medium}   		% CwTex 楷體
\newCJKfontfamily{\CM}{cwTeX Q Ming Medium}		% CwTex 明體
\newCJKfontfamily{\CR}{cwTeX Q Yuan Medium}		% CwTex 圓體
%-----------------------------------------------------------------------------------------------------------------------
\XeTeXlinebreaklocale "zh"             %這兩行一定要加,中文才能自動換行
\XeTeXlinebreakskip = 0pt plus 1pt     %這兩行一定要加,中文才能自動換行
%-----------------------------------------------------------------------------------------------------------------------
\newcommand{\cw}{\texttt{cw}\kern-.6pt\TeX}	% 這是 cwTex 的 logo 文字
\newcommand{\imgdir}{../images/}				% 設定圖檔的位置
\renewcommand{\tablename}{表}				% 改變表格標號文字為中文的「表」(預設為 Table)
\renewcommand{\figurename}{圖}				% 改變圖片標號文字為中文的「圖」(預設為 Figure)


%------------------------------------------------------------------------------------------------------
%  計數器
\usepackage{amsthm}							% theroemstyle 需要使用的套件
\theoremstyle{plain}                       %樣式
\newtheorem{de}{Definition定義}[section]    %definition獨立編號 %跟section走憶起編號
\newtheorem{thm}{{\MB 定理}}[section]		%theorem 獨立編號,取中文名稱並給予不同字型
\newtheorem{lemma}[thm]{Lemma}				%lemma 與 theorem 共用編號 %跟thm走憶起編號
\newtheorem{ex}{{\E Example}}			 %example 獨立編號,不編入小節數字,走流水號。也換個字型。
\newcounter{e}
\newtheorem{cor}{Corollary}[section]		%not used here
\newtheorem{exercise}{EXERCISE}				%not used here
\newtheorem{re}{\emph{Result}}[section]	%not used here
\newtheorem{axiom}{AXIOM}					%not used here
\renewcommand{\proofname}{\bf{Proof}}		%not used here
\bibliographystyle{plain}
%-----------------------------------------------------------------------------------------------------------------------
% 設定顏色
\definecolor{slight}{gray}{0.9}				
\definecolor{airforceblue}{rgb}{0.36, 0.54, 0.66} % color Table: http://latexcolor.com
\definecolor{arylideyellow}{rgb}{0.91, 0.84, 0.42}
\definecolor{babyblue}{rgb}{0.54, 0.81, 0.94}
\definecolor{cadmiumred}{rgb}{0.89, 0.0, 0.13}
\definecolor{coolblack}{rgb}{0.0, 0.18, 0.39}
\definecolor{cottoncandy}{rgb}{1.0, 0.74, 0.85}
\definecolor{desertsand}{rgb}{0.93, 0.79, 0.69}
\definecolor{electriclavender}{rgb}{0.96, 0.73, 1.0}
\definecolor{lightsalmonpink}{rgb}{1.0, 0.6, 0.6}
\definecolor{amaranth}{rgb}{0.9, 0.17, 0.31}
\definecolor{amethyst}{rgb}{0.6, 0.4, 0.8}
\definecolor{atomictangerine}{rgb}{1.0, 0.6, 0.4}
\definecolor{babyblueeyes}{rgb}{0.63, 0.79, 0.95}
\definecolor{gray(x11gray)}{rgb}{0.75, 0.75, 0.75}
\definecolor{bananamania}{rgb}{0.98, 0.91, 0.71}	
\definecolor{ballblue}{rgb}{0.13, 0.67, 0.8}
\definecolor{azure(colorwheel)}{rgb}{0.0, 0.5, 1.0}	
\definecolor{ceruleanblue}{rgb}{0.16, 0.32, 0.75}
\definecolor{lightsalmonpink}{rgb}{1.0, 0.6, 0.6}
\definecolor{palepink}{rgb}{0.98, 0.85, 0.87}
\definecolor{lightcarminepink}{rgb}{0.9, 0.4, 0.38}
\definecolor{lightcornflowerblue}{rgb}{0.6, 0.81, 0.93}
\definecolor{carolinablue}{rgb}{0.6, 0.73, 0.89}
   % 使用自己維護的定義檔
%-----------------------------------------------------------------------------------------------------------------------
% 文章開始
%\title{ \LaTeX{\CB 的數學符號與方程式}}
%\author{{\SM 游筑鈞}}
%\date{{\TT \today}} 	% Activate to display a given date or no date (if empty),
         				% otherwise the current date is printed 
%\begin{document}       %document有預設的字型
%\maketitle %顯示標題
%\fontsize{12}{22 pt}\selectfont   %\selectfont-讓前面的設定生效
\chapter{\CB{數學符號與式子}}
本文將常見的數學符號與方程式以 \LaTeX 編排,並以各式各樣數學方程式展現出 \LaTeX 在編輯數學式時的強大功能。相信使用過word排版過數學式的人都知道要排出整齊優美的方程式是一件很難的事,透過 \LaTeX 可以輕鬆地完成。文章中會將不同類型的方程式大致分類,以便以後在使用時可以快速的找到。這裡希望透過蒐集不同的方程式,並以 \LaTeX 進行編排作為後續參考內容。本文內容參考汪群超教授網站 \footnote{相關文件可在https://ntpuccw.blog/supplements/xetex-tutorial/ 下載。}\\
%FOOTNOTE放在本句話最後,標點符號後\\\

\section{\MB{數式環境}}
我們平常寫文章的模式無法正確處理數學式子間的空間位置。因此,所有的數學式子都得進入數學模式來處理。在數學模式下,不僅大部份文字、符號會採用斜體字,而且空間會另做安排,額外的空白會被 \LaTeX  忽略。\smallskip

LaTeX 的數學模式有兩種,一種是和內文排列在一起的隨文數式(math inline mode),他是和一般正常文字混在一起排版的;另外一種是獨立的展式數式(math display mode),他會單獨成一行,而且上下會和正常文字有一定的空間來區隔。

\subsection{隨文數式(math inline mode)}
用兩個錢字號前後包圍這樣會進入隨文的數學模式,在一般文字段落內要使用到一些數學式子的話,這是最方便的方法。
隨文數式的應用很多,例如:因為 $\lvert t^{*}\rvert=10.2 < 2.069$ 所以拒絕 $H_0$,也就是說$\beta_0 \neq 0$。等等 $\cdots$

\subsection{展式數式(math display mode)}
通常獨立的數學式子,通常會單獨成一行,需要的話也可以加入編號,以方便在文章中引用。展示數式會適當的選用較大的數學符號及字體,尤其是較複雜的數學式子的時候。例如:
$$\beta_0+\beta_1\pm{W\sqrt{MSE}}\bigg[\frac{1}{n}+\frac{{(X-\bar{X}})^2}{\sum{(X-\bar{X})^2}}\bigg]^{1/2}$$
展式數式的幾種做法:
\begin{enumerate}
\item 在數學式子前面跟後面各加2個錢字號,讓 \LaTeX 知道要進入數式環境,並讓數學式置中。
\item 使用 begin equation 和 end equation 來做,利用這種做法可以幫數學式子標號。
\end{enumerate}


\section{\MB{符號}}
數字與普通運算符號可直接由鍵盤上鍵入。譬如,下列符號可以直接由鍵盤鍵入:

        \begin{center}
         $  + \;-\; =\; <\; > \;/ \;:\; !\;\; |\; \;[\;\; ] \;(\; )$\\
        \end{center}

\subsection{特殊符號}

\begin{itemize}
\item 希臘字符:$ \;\alpha \;\theta \;\tau \;\beta \;\pi  \;\upsilon \;\gamma  \;\iota \;\varpi \;\phi \;\delta  \;\kappa \;\epsilon  \;\lambda $ 
\item 分隔符號:$ ( )  \uparrow  \Uparrow [  ] \downarrow  \Downarrow \{  \}  \updownarrow \lfloor  \rfloor  \lceil \rceil \langle \rangle \backslash$
\item 重音符號:$\hat{a}  \;\acute{a}  \;\bar{a}  \;\dot{a}  \;\breve{a} \;\check{a}  \;\grave{a} \;\vec{a}  \;\ddot{a} \;\tilde{a}$
\item 大型運算符號:$\;\sum \;\bigcap \;\bigodot \;\prod \;\bigcup \;\bigotimes \;\coprod \;\bigsqcup  \;\bigoplus \;\int \;\bigvee$
\item 運算結構:$\;\widetilde{abc} \;\widehat{abc} \;\overleftarrow{abc} \;\overrightarrow{abc} \;\overline{abc} \;\underline{abc}\;\overbrace{abc} \;\underbrace{abc} \;\sqrt{abc} \;\sqrt[n]{abc} \;\frac{abc}{xyz}$
\item 字符間空格:
\begin{center} 
\begin{tabular}{|l|c|l|c|}%
\hline  %劃上一條橫線
  2个quad空格  & $\alpha\qquad\beta$  & quad空格	&$\alpha\quad\beta$	\\\hline  % &代表換欄 \\代表換下一列
  大空格	      & $\alpha\ \beta$	      & 中等空格 & $\alpha\;\beta	$		\\\hline
  小空格       & $\alpha\,\beta	$     & 没有空格	& $\alpha\beta$  \\\hline
  緊貼         & $\alpha\!\beta	$     &         &                  \\\hline
\end{tabular}\\
\end{center}
\end{itemize}


\section{\MB{常見的數學式}}
本節列舉一些常見的數學式作為練習與未來使用的參考。相信做過了越多的數學式子練習後,以後就可以輕而易舉地寫出複雜的公式了。

\subsection{分式與根式}
範例一 :
$$
\frac{\frac{\displaystyle a}{\displaystyle x-y}+
\frac{\displaystyle b}{\displaystyle x+y}}
{\frac{\displaystyle x-y}{\displaystyle x+y}+
\frac{\displaystyle a-b}{\displaystyle a+b}}
$$


\bigskip
如果覺得字符太小可以調整設定:
\begin{itemize}

\item$\backslash${\A displaystyle}:	展示數式的標準字體大小
\item$\backslash${\A textstyle}	:隨文數式的標準字體大小
\item$\backslash${\A scriptstyle}:	第一層上下標字體大小
\item$\backslash${\A scriptscriptstyle}:	第二層上下標字體大小
\end{itemize}
\bigskip
範例二 :傳說中的拉馬努金公式
\bigskip

拉馬努金,每天廢寢忘食,只研究數學,就在這時神奇的事情又發生了,他每天晚上睡覺的時候,都會夢到自己所信宗教的女神。 拉馬努金醒來以後,腦子裏充滿了各種各樣的公式,那以後,女神每天都出現在他的夢裏,告訴他一些新公式。拉馬努金每天清晨都要趕快拿出筆記本,把夢中得到的公式記在本上,由於筆記本的費用對他來說很高昂,所以每次女神告訴他的時候,他只把最終得出的,最簡化的公式抄到本上。幾年下來,他得到了3,900個複雜的公式! 

\bigskip
$$
\nonumber\frac{1}{\pi}=\frac{2\sqrt{2}}{9801}\sum_{k=0}^\infty{\frac{(4k)!(1103+26390k)}{(4k)!3962^{4k}}}
$$

$$
\nonumber\sqrt{\frac{1+\sqrt{5}}{2}+2}-\frac{1+\sqrt{5}}{2}=\frac{\displaystyle e^{\frac{-2\pi}{5}}}{1+\frac{\displaystyle e^{-2\pi}}{1+\frac{\displaystyle e^{-4\pi}}{1+\frac{\displaystyle e^{-6\pi}}{1+\ldots}}}}
$$

拉馬努金恆等式
\begin{eqnarray}
% 等號對齊
        \nonumber 3&=&\sqrt{1+2+4} \\
        \nonumber &=& \sqrt{1+2\sqrt{1+3*5}} \\
        \nonumber &=& \sqrt{1+2\sqrt{1+3\sqrt{1+4*6}}} \\
        \nonumber  &=& \sqrt{1+2\sqrt{1+3\sqrt{1+4\sqrt{1+5*7}}}} \\
        \nonumber &=& \ldots
\end{eqnarray}

對齊的兩種方法:
\begin{enumerate}
\item align可以用來讓公式對齊,在公式中,加 $\backslash$ $\backslash$ 表示换行、加 $\&$ 表示要對齊的地方。
\item ennarray 也是其中一種方法,在$ = $前後各加一個 $\&$ 讓等號對齊。
\end{enumerate}


\subsection{函數}

  \textbf{Likelihood function of Normal distribution}: 
  
  $$
  L(\mu,\sigma^{2},x_1,\ldots,x_n)=\prod_{j=1}^{n}f_x(x_j;\mu_i\sigma^{2})=(2\pi\sigma^{2})^{\frac{-n}{2}}\exp\left(\frac{-1}{2\sigma^{2}}\sum_{j=1}^{n}(x_i-\mu)^{2}\right)
  $$
\bigskip

括號的使用:數學中的括號隨著其內容的多寡,其大小必須調整恰當,如上式的兩種大小不同的括號「$( \cdot)$ 」。外圍較大地括號使用 $\backslash$  left$($ 與 $\backslash$  right$($ 令編譯器依需求自動調整為適當大小。另外,也可以手動控制括號、的大小,如
$$ \bigg(\; \big( \;(\;\;\;) \;\big) \;\bigg) \;,\; \bigg[ \;\big[ \;[\;\;\;]\; \big]\; \bigg]\;,\; \bigg\{ \;\big\{ \;\{\;\;\;\} \;\big\} \;\bigg\}$$ 
 
 \bigskip
 
  \textbf{Multivariate normal distribution}: 
  
$$
\begin{aligned}
f(y_1,y_2)&= \frac{1}{2\pi\sigma_1\sigma_2}\exp\left[-\frac{(y_1-\mu_1)^{2}}{2\sigma_1^{2}}-\frac{(y_2-\mu_2)^{2}}{\sigma_2^{2}}\right]\\[4mm]
&= \frac{1}{\sqrt{2\pi}\sigma_1}\exp^{\displaystyle-\frac{(y_1-\mu_1)^{2}}{2\sigma_1^{2}}}\frac{1}{\sqrt{2\pi}\sigma_2}\exp^{\displaystyle-\frac{(y_2-\mu_2)^{2}}{\sigma_2^{2}}}
\end{aligned}
$$
\textbf{Beta distribution}: 
$$
\begin{aligned}
f(x;\alpha,\beta)&=\frac{x^{\alpha-1}(1-x)^{\beta-1}}{\int^1_0u^{\alpha-1}(1-u)^{\beta-1}du}\\[4mm]
&=\frac{\Gamma(\alpha+\beta)}{\Gamma(\alpha)\Gamma(\beta)}x^{\alpha-1}(1-x)^{\beta-1}\\[4mm]
&=\frac{1}{B(\alpha,\beta)}x^{\alpha-1}(1-x)^{\beta-1}
\end{aligned}
$$

公式間行距微調:數學式中常有複雜的計算過程造成公式之間看起來很擠,所以可以在$\backslash$ $\backslash$ 换行符號後面加上[4mm],增加適當的行距。
 

\subsection{積分與微分式}
積分式:
\begin{equation}\label{inter01}
 \int^{2\pi}_0(\cos\alpha)^{m}(\sin\alpha)^{n}\exp^{-\alpha(J\sin\alpha)+K\cos\alpha}d\alpha
\end{equation}

方程式(\ref{inter01}) 是三角函數的積分。可以在begin equation 後面加上label(命名)為數學式加上編號並命名方便在內文中做交互參照。欲做交互參照可以在文中插入ref。 

微分式:
\begin{equation}\label{diff01}
\frac{\partial}{\partial a}\left(\int^b_a f(x)dx\right)=\lim_{\Delta a \rightarrow 0}\frac{1}{\Delta a}\left[\int^b_{a+\Delta a} f(x)dx-\int^b_a f(x)dx\right]
\end{equation}



\begin{equation}\label{diff02}
\nabla\cdot\mathbf{A}=\frac{1}{r^{2}\sin\theta}\left[\sin\theta\frac{\partial }{\partial r}(r^{2}\mathbf{A_r})+r\frac{\partial}{\partial\theta}(\sin\theta \mathbf{A_\theta})+r\frac{\partial \mathbf{A_\phi}}{\partial\phi^{2}}\right]
 \end{equation}
方程式(\ref{diff02})是偏微分的公式。數學式子中難免會有向量符號,在 \LaTeX 數式環境中可以用\A mathbf 來表示,呈現粗體字型。

向量數學式:
$$
\overrightarrow{a}=(a_1,a_2) \qquad \overrightarrow{b}=(b_1,b_2)
$$
$$
\overrightarrow{a}\centerdot \overrightarrow{b}=a_1b_1+a_2+b_2= \arrowvert \overrightarrow{a} \arrowvert  \arrowvert \overrightarrow{b} \arrowvert\cos\theta
$$
\subsection{矩陣與行列式}
矩陣或有規則排列的數學式或組合很常見,以下列舉幾種模式,請特別注意其使用的標籤及一些需要注意的小地方。

\begin{enumerate} 
  \item 矩陣的左右括號需各別加上。
  \item 橫行各項之間是以 $\&$ 區隔。
  \item 除最後一行外,每行之末則加上換行指令 $\backslash\backslash$。
  \item 使用 {\A array} 指令時,須加上選項以控制每一直欄內各數字或符號要居中排列、靠左或靠右。
\end{enumerate}

聯立式:
\begin{equation}\label{01}
\vec{y} =
\left( \begin{array}{cc}
y_{1} =
\left| \begin{array}{cc}
x_{11} & x_{12} \\
x_{21} & x_{22}
\end{array} \right| \\
y_{2} \\
y_{3}
\end{array} \right)
\end{equation}


其中 \A left 以及 \A right 兩指令必須同時使用, 且 \A left] 後面接 \A right( 是會被 \LaTeX 允許的。

\begin{equation}\label{02}
g(x,y) = \left\{\begin{array}{ll}
                 f(x,y), & \mbox{if $x<y$} \\  
                 f(y,x), & \mbox{if $x>y$} \\  
                 0,      & \mbox{otherwise.}
                \end{array} \right.
\end{equation}

\LaTeX 表示聯立方程的方法亦使用陣列, 其中大括號只會有一個, 此時必須輸入 \A left. 或 \A right. 對應才會被 \LaTeX 接受。
有時候數學式中的文字不需要斜體如(\ref{02}),可以用mbox來控制字體不要斜體。


\begin{equation}
\begin{cases}
 \ u_{tt}(x,t)= b(t)\triangle u(x,t-4)&\\
\ \hspace{42pt}- q(x,t)f[u(x,t-3)]+te^{-t}\sin^2 x,  &  t \neq t_k; \\
 \ u(x,t_k^+) - u(x,t_k^-) = c_k u(x,t_k), & k=1,2,3\ldots ;\\
 \ u_{t}(x,t_k^+) - u_{t}(x,t_k^-) =c_k u_{t}(x,t_k), &
 k=1,2,3\ldots\ .
\end{cases}
\end{equation}

這裡用 begin cases 和 end cases 來代替上面(\ref{02})中的大括號,hspace{距离}可以插入任意空格。
\bigskip
矩陣:
\begin{equation}
A =
\begin{pmatrix}              
  t_{11} & t_{12} & t_{13} \\
  t_{21} & t_{22} & t_{23} \\
  t_{31} & t_{32} & t_{33}
\end{pmatrix}
\end{equation}

距陣的類型:

\begin{center} 
\begin{tabular}{c c c c}%
  
   \extrarowheight=8pt 
  matrix & 
  $ \begin{matrix} 1&0&0\\ 0&1&0\\ 0&0&1\\ \end{matrix}
  $  
  &bmatrix	& 
  $\begin{bmatrix} 1&0&0\\ 0&1&0\\ 0&0&1\\ \end{bmatrix}
  $	\\  % &代表換欄 \\代表換下一列
  \\
  vmatrix      & $\begin{vmatrix} 1&0&0\\ 0&1&0\\ 0&0&1\\ \end{vmatrix}$ & pmatrix & $\begin{pmatrix} 1&0&0\\ 0&1&0\\ 0&0&1\\ \end{pmatrix}$	\\
  \\
 Bmatrix       & $\begin{Bmatrix} 1&0&0\\ 0&1&0\\ 0&0&1\\ \end{Bmatrix}$    & Vmatrix	& $\begin{Vmatrix} 1&0&0\\ 0&1&0\\ 0&0&1\\ \end{Vmatrix}$ \\

\end{tabular}\\

\end{center}
\bigskip
距陣並排:
\begin{equation}
H_x=\frac{1}{3}\times{
\left[ \begin{array}{ccc}
1 & 0 & -1\\
1 & 0 & -1\\
1 & 0 & -1
\end{array} 
\right ]}
\qquad 
H_y=\frac{1}{3}\times{
\left[ \begin{array}{ccc}
1 & 1 & 1\\
0& 0 & 0\\
1 & 1 & 1
\end{array}
\right ]}
\end{equation}

對角矩陣:
\begin{equation}
\begin{pmatrix}
\alpha_1                                    &           &\multicolumn{2}{c}{\multirow{2}*{{\Huge0}}}\\
                                            &\alpha_2\\
\multicolumn{2}{c}{\multirow{2}*{{\Huge0}}}             &\ddots\\
                                            &           &       &\alpha_n
\end{pmatrix}
\end{equation}
左下角的0需要放在第三列的第一位置,因為他要合併的是第三行和第四行, 當然他還合併了第一第二列,斜點其實位於第三列儘管寫法上看是第二列的元素。這裡需要使用multirow的package。

矩陣中分區塊:

 $$\left[ \begin{array}{@{\,}c|c c c c@{\,}}
 
   s & y & u & o &c\\
   p & c & h & f & x\\
   u & v & k & s & b    \\\hline
   w & x & y &a & f \\
\end{array}\right]$$ 

其實矩陣中的分區塊,和在做表格的方法很相似先把位置規畫好,再放數字。指定直線和橫線要擺放的位置。\\
矩陣上下註記:
\begin{gather}
\underbrace{\begin{bmatrix}
y_{1}&1&1\\[4pt]
\frac{1}{\sqrt{2}} & y_{2} & 1 \\[4pt]
1 & 1 & y_{3}
\end{bmatrix}}_{Y(S)}
\underbrace{\begin{bmatrix}
v_{1} \\[4pt] v_{2} \\[4pt] v_{3}
\end{bmatrix}}_{V(s)}=0
\end{gather}

最前面用 begin underbrace 和 end underbrace 在矩陣的底部加上說明。
\subsection{\MB{其他}}

練習一些表較少見的數學表達式,用 WORD 很不容易做到。\\
課堂練習:\\
練習一: 

$$
SR=n\left(\frac{2}{n}\sum_{k=1}^n E\parallel y_i-Z\parallel-2\frac{\Gamma((p+1)/2)}{\Gamma(p/2)}-\frac{1}{n^2} \sum_{j,k=1}^n\parallel y_j-y_k \parallel\right)
$$

練習二:

$$
P_m,i=\sum_{j=i}^{m-1}\begin{pmatrix} m\\ j\\ \end{pmatrix}\begin{pmatrix} m-i-1\\ j-1\\ \end{pmatrix}p^jq^{m-j}\left(\frac{\eta_1}{\eta_1+\eta_2}\right)^{j-1}\left(\frac{\eta_2}{\eta_1+\eta_2}\right)^{m-j},1\leq i\leq m-1
$$

練習三:

$$
\Lambda(t)=\mbox{exp}\left(\int^t_0 \xi(u)\cdot dW(t)-\frac{1}{2}\int^t_0\parallel \xi(u)\parallel^2 du+(\lambda-\tilde{\lambda})t \right)\prod_{i=1}^{N(t)}\frac{\tilde{\lambda}\tilde{f}(Y_i)}{\lambda f(Y_i)}
$$

練習四:
$$
W_MA=\frac{(\sum_{j=1}^na_j U_{(j)})^2}{(\mathbf{X_0}-\bar{\mathbf{X})}'A^{-1}(\mathbf{X_0}-\bar{\mathbf{X})}}
$$
\section*{\MB{結論與心得}}
經過了這份作品的練習之後,漸漸地熟悉了數學式子的語法也熟悉了 \LaTeX 的使用環境。其實使用 \LaTeX 來排版數學式子時需要很細心,一個符號、細節都不能忽略不然都會出現紅字錯誤,但是為了\LaTeX 強大的數式環境,我們需要透過大量的練習掌握技巧。雖然有些數學式子看起來很複雜只要先把它分區塊,一部份一部份來輸入語法,就沒有想像中的那麼難掌握了。

%\end{document}