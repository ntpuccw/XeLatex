% !TEX TS-program = xelatex								% These two lines must be incldued to open file under UTF-8
% !TEX encoding = UTF-8

%--- BOOK 定義檔
\documentclass[12pt, a4paper]{book}
\setlength{\textwidth}{13cm} %設定書面文字寬度

%--- 定義頁眉頁足 -------------------------------
\usepackage{fancyhdr}
\pagestyle{fancy}
\fancyhf{}
\renewcommand{\chaptermark}[1]{\markboth{Chapter\thechapter .\ #1}{}} %去除章編號前後的字
\fancyhead[LE]{ \XeLaTeX }
\fancyhead[RO]{  \leftmark  }
\fancyfoot[RO,LE]{第~\thepage ~頁}
\renewcommand{\headrulewidth}{0.2pt} %頁眉下方的橫線
\renewcommand{\footrulewidth}{0.2pt} %頁眉下方的橫線
%----- 定義使用的 packages ----------------------
\usepackage{fontspec} 										% Font selection for XeLaTeX; see fontspec.pdf for documentation. 
%\usepackage[BoldFont, SlantFont]{xeCJK}		% 中文使用 XeCJK,並模擬粗體與斜體(即可以用 \textbf{ } \textit{ })
\usepackage{xeCJK}											% 中文使用 XeCJK,但利用 \setCJKmainfont 定義粗體與斜體的字型
\defaultfontfeatures{Mapping=tex-text} 				% to support TeX conventions like ``---''
\usepackage{xunicode} 										% Unicode support for LaTeX character names (accents, European chars, etc)
\usepackage{xltxtra} 											% Extra customizations for XeLaTeX
\usepackage{amsmath, amssymb}
\usepackage[sf,small]{titlesec}
\usepackage{enumerate}
\usepackage{graphicx, subfig, float} 					% support the \includegraphics command and options
\usepackage{array, booktabs}
\usepackage{color, xcolor}
\usepackage{longtable}
\usepackage{colortbl}     
\usepackage{multirow} 
\usepackage{multicol} 
\usepackage{arydshln}  
\usepackage{dcolumn} 
\usepackage{rotating} 
\usepackage{diagbox} 
\usepackage{wrapfig} %文繞圖
\usepackage{overpic}%圖片上加文字
%\usepackage{natbib}
%\usepackage[sort&compress,square,comma,authoryear]{natbib}
%.............................................表格標題註解之巨集套件
% for Reference
\usepackage{makeidx}										% for Indexing
\usepackage[parfill]{parskip} 							% Activate to begin paragraphs with an empty line rather than an indent
%\usepackage{geometry} 									% See geometry.pdf to learn the layout options. There are lots.
%\usepackage[left=1.5in,right=1in,top=1in,bottom=1in]{geometry} 
%-----------------------------------------------------------------------------------------------------------------------
\setCJKmainfont
	[
	%	BoldFont=cwTeX Q Hei Bold								% 定義粗體的字型(依使用的電腦安裝的字型而定)
	]
%	{cwTeX Q Ming Medium} 										% 設定中文內文字型
	{新細明體}	
\setmainfont{Times New Roman}								% 設定英文內文字型
\setsansfont{Arial}														% used with {\sffamily ...}
%\setsansfont[Scale=MatchLowercase,Mapping=tex-text]{Gill Sans}
\setmonofont{Courier New}										% used with {\ttfamily ...}
%\setmonofont[Scale=MatchLowercase]{Andale Mono}
% 其他字型(隨使用的電腦安裝的字型不同,用註解的方式調整(打開或關閉))
% 英文字型
\newfontfamily{\ST}{StencilStd}	
\newfontfamily{\E}{Cambria}										% 套用在內文中所有的英文字母
\newfontfamily{\A}{Arial}
\newfontfamily{\C}[Scale=0.9]{Cambria}
\newfontfamily{\TT}[Scale=0.8]{Times New Roman}
% 中文字型
\newCJKfontfamily{\MB}{微軟正黑體}							% 適用在 Mac 與 Win
\newCJKfontfamily{\SM}[Scale=0.8]{新細明體}				% 縮小版新細明體
\newCJKfontfamily{\K}{標楷體}  

% 以下為自行安裝的字型:CwTex 組合
\newCJKfontfamily{\CF}{cwTeX Q Fangsong Medium}	% CwTex 仿宋體
\newCJKfontfamily{\CB}{cwTeX Q Hei Bold}			% CwTex 粗黑體
\newCJKfontfamily{\CK}{cwTeX Q Kai Medium}   		% CwTex 楷體
\newCJKfontfamily{\CM}{cwTeX Q Ming Medium}		% CwTex 明體
\newCJKfontfamily{\CR}{cwTeX Q Yuan Medium}		% CwTex 圓體                      			% Windows 下的標楷體
%-----------------------------------------------------------------------------------------------------------------------
\XeTeXlinebreaklocale "zh"                  %這兩行一定要加,中文才能自動換行
\XeTeXlinebreakskip = 0pt plus 1pt     %這兩行一定要加,中文才能自動換行
%-----------------------------------------------------------------------------------------------------------------------
%----- 重新定義的指令 ---------------------------
\newcommand{\cw}{\texttt{cw}\kern-.6pt\TeX}	% 這是 cwTex 的 logo 文字
\newcommand{\imgdir}{images/}						% 設定圖檔的位置
\renewcommand{\tablename}{表}						% 改變表格標號文字為中文的「表」(預設為 Table)
\renewcommand{\figurename}{圖}						% 改變圖片標號文字為中文的「圖」(預設為 Figure)
%\renewcommand{\figurename}{圖\hspace*{-.5mm}}
%  for Long Report or Book
\renewcommand{\contentsname}{{\CB 目錄}}
\renewcommand\listfigurename{{\CB 圖目錄}}
\renewcommand\listtablename{{\CB 表目錄}}
\renewcommand{\indexname}{{\CB 索引}}
\renewcommand{\bibname}{{\CB 參考文獻}}
%-----------------------------------------------------------------------------------------------------------------------

%\theoremstyle{plain}
\newtheorem{de}{Definition}[section]				%definition獨立編號
\newtheorem{thm}{{\MB 定理}}[section]			%theorem 獨立編號,取中文名稱並給予不同字型
\newtheorem{lemma}[thm]{Lemma}				%lemma 與 theorem 共用編號
\newtheorem{ex}{{\E Example}}						%example 獨立編號,不編入小節數字,走流水號。也換個字型。
\newtheorem{cor}{Corollary}[section]				%not used here
\newtheorem{exercise}{EXERCISE}					%not used here
\newtheorem{re}{\emph{Result}}[section]		%not used here
\newtheorem{axiom}{AXIOM}							%not used here
%\renewcommand{\proofname}{\bf{Proof}}		%not used here

\newcommand{\loflabel}{圖} % 圖目錄出現 圖 x.x 的「圖」字
\newcommand{\lotlabel}{表}  % 表目錄出現 表 x.x 的「表」字

\parindent=0pt
\setcounter{tocdepth}{0}

%--- 其他定義 ----------------------------------
% 定義章節標題的字型、大小
\titleformat{\chapter}[display]{\centering\LARGE\MB}
 {\CB 第\ \thechapter\ 章}{0.2cm}{}
\titlespacing{\chapter}{0cm}{-1.3cm}{1em}
%\titleformat{\chapter}[hang]{\centering\LARGE\sf}{\MB 第~\thesection~章}{0.2cm}{}%控制章的字體
\titleformat{\section}[hang]{\Large\sf}{\MB 第~\thesection~節}{0.2cm}{}%控制章的字體
%\titleformat{\subsection}[hang]{\centering\Large\sf}{\MB 第~\thesubsection~節}{0.2cm}{}%控制節的字體
%\titleformat*{\section}{\normalfont\Large\bfseries\MB}
\titleformat*{\subsection}{\normalfont\large\bfseries\MB}
%\titleformat*{\subsubsection}{\normalfont\large\bfseries\MB}
\usepackage{titlesec}
\usepackage{titletoc}

%目錄裡的 "章"文字
\titlecontents{chapter}[1em]{}{\makebox[4.1em][l]
{\CB{第}\ST{\thecontentslabel}\CB{章}}}{}{~\titlerule*[0.7pc]{.}~\contentspage}



\definecolor{slight2}{gray}{0.5}	
\definecolor{slight}{gray}{0.8}								% 設定顏色
\definecolor{airforceblue}{rgb}{0.36, 0.54, 0.66} % color Table: http://latexcolor.com
\definecolor{arylideyellow}{rgb}{0.91, 0.84, 0.42}
\definecolor{babyblue}{rgb}{0.54, 0.81, 0.94}
\definecolor{cadmiumred}{rgb}{0.89, 0.0, 0.13}
\definecolor{coolblack}{rgb}{0.0, 0.18, 0.39}
\definecolor{cottoncandy}{rgb}{1.0, 0.74, 0.85}
\definecolor{desertsand}{rgb}{0.93, 0.79, 0.69}
\definecolor{electriclavender}{rgb}{0.96, 0.73, 1.0}
\definecolor{lightsalmonpink}{rgb}{1.0, 0.6, 0.6}
\definecolor{amaranth}{rgb}{0.9, 0.17, 0.31}
\definecolor{amethyst}{rgb}{0.6, 0.4, 0.8}
\definecolor{atomictangerine}{rgb}{1.0, 0.6, 0.4}
\definecolor{babyblueeyes}{rgb}{0.63, 0.79, 0.95}
\definecolor{gray(x11gray)}{rgb}{0.75, 0.75, 0.75}
\definecolor{bananamania}{rgb}{0.98, 0.91, 0.71}	
\definecolor{ballblue}{rgb}{0.13, 0.67, 0.8}
\definecolor{azure(colorwheel)}{rgb}{0.0, 0.5, 1.0}	
\definecolor{ceruleanblue}{rgb}{0.16, 0.32, 0.75}
\definecolor{lightsalmonpink}{rgb}{1.0, 0.6, 0.6}
\definecolor{palepink}{rgb}{0.98, 0.85, 0.87}
\definecolor{lightcarminepink}{rgb}{0.9, 0.4, 0.38}
\definecolor{lightcornflowerblue}{rgb}{0.6, 0.81, 0.93}
\definecolor{carolinablue}{rgb}{0.6, 0.73, 0.89}