%---置入定義檔 ----------------------------------
% !TEX TS-program = xelatex								% These two lines must be incldued to open file under UTF-8
% !TEX encoding = UTF-8

%--- BOOK 定義檔
\documentclass[12pt, a4paper]{book}
\setlength{\textwidth}{13cm} %設定書面文字寬度

%--- 定義頁眉頁足 -------------------------------
\usepackage{fancyhdr}
\pagestyle{fancy}
\fancyhf{}
\renewcommand{\chaptermark}[1]{\markboth{Chapter\thechapter .\ #1}{}} %去除章編號前後的字
\fancyhead[LE]{ \XeLaTeX }
\fancyhead[RO]{  \leftmark  }
\fancyfoot[RO,LE]{第~\thepage ~頁}
\renewcommand{\headrulewidth}{0.2pt} %頁眉下方的橫線
\renewcommand{\footrulewidth}{0.2pt} %頁眉下方的橫線
%----- 定義使用的 packages ----------------------
\usepackage{fontspec} 										% Font selection for XeLaTeX; see fontspec.pdf for documentation. 
%\usepackage[BoldFont, SlantFont]{xeCJK}		% 中文使用 XeCJK,並模擬粗體與斜體(即可以用 \textbf{ } \textit{ })
\usepackage{xeCJK}											% 中文使用 XeCJK,但利用 \setCJKmainfont 定義粗體與斜體的字型
\defaultfontfeatures{Mapping=tex-text} 				% to support TeX conventions like ``---''
\usepackage{xunicode} 										% Unicode support for LaTeX character names (accents, European chars, etc)
\usepackage{xltxtra} 											% Extra customizations for XeLaTeX
\usepackage{amsmath, amssymb}
\usepackage[sf,small]{titlesec}
\usepackage{enumerate}
\usepackage{graphicx, subfig, float} 					% support the \includegraphics command and options
\usepackage{array, booktabs}
\usepackage{color, xcolor}
\usepackage{longtable}
\usepackage{colortbl}     
\usepackage{multirow} 
\usepackage{multicol} 
\usepackage{arydshln}  
\usepackage{dcolumn} 
\usepackage{rotating} 
\usepackage{diagbox} 
\usepackage{wrapfig} %文繞圖
\usepackage{overpic}%圖片上加文字
%\usepackage{natbib}
%\usepackage[sort&compress,square,comma,authoryear]{natbib}
%.............................................表格標題註解之巨集套件
% for Reference
\usepackage{makeidx}										% for Indexing
\usepackage[parfill]{parskip} 							% Activate to begin paragraphs with an empty line rather than an indent
%\usepackage{geometry} 									% See geometry.pdf to learn the layout options. There are lots.
%\usepackage[left=1.5in,right=1in,top=1in,bottom=1in]{geometry} 
%-----------------------------------------------------------------------------------------------------------------------
\setCJKmainfont
	[
	%	BoldFont=cwTeX Q Hei Bold								% 定義粗體的字型(依使用的電腦安裝的字型而定)
	]
%	{cwTeX Q Ming Medium} 										% 設定中文內文字型
	{新細明體}	
\setmainfont{Times New Roman}								% 設定英文內文字型
\setsansfont{Arial}														% used with {\sffamily ...}
%\setsansfont[Scale=MatchLowercase,Mapping=tex-text]{Gill Sans}
\setmonofont{Courier New}										% used with {\ttfamily ...}
%\setmonofont[Scale=MatchLowercase]{Andale Mono}
% 其他字型(隨使用的電腦安裝的字型不同,用註解的方式調整(打開或關閉))
% 英文字型
\newfontfamily{\ST}{StencilStd}	
\newfontfamily{\E}{Cambria}										% 套用在內文中所有的英文字母
\newfontfamily{\A}{Arial}
\newfontfamily{\C}[Scale=0.9]{Cambria}
\newfontfamily{\TT}[Scale=0.8]{Times New Roman}
% 中文字型
\newCJKfontfamily{\MB}{微軟正黑體}							% 適用在 Mac 與 Win
\newCJKfontfamily{\SM}[Scale=0.8]{新細明體}				% 縮小版新細明體
\newCJKfontfamily{\K}{標楷體}  

% 以下為自行安裝的字型:CwTex 組合
\newCJKfontfamily{\CF}{cwTeX Q Fangsong Medium}	% CwTex 仿宋體
\newCJKfontfamily{\CB}{cwTeX Q Hei Bold}			% CwTex 粗黑體
\newCJKfontfamily{\CK}{cwTeX Q Kai Medium}   		% CwTex 楷體
\newCJKfontfamily{\CM}{cwTeX Q Ming Medium}		% CwTex 明體
\newCJKfontfamily{\CR}{cwTeX Q Yuan Medium}		% CwTex 圓體                      			% Windows 下的標楷體
%-----------------------------------------------------------------------------------------------------------------------
\XeTeXlinebreaklocale "zh"                  %這兩行一定要加,中文才能自動換行
\XeTeXlinebreakskip = 0pt plus 1pt     %這兩行一定要加,中文才能自動換行
%-----------------------------------------------------------------------------------------------------------------------
%----- 重新定義的指令 ---------------------------
\newcommand{\cw}{\texttt{cw}\kern-.6pt\TeX}	% 這是 cwTex 的 logo 文字
\newcommand{\imgdir}{images/}						% 設定圖檔的位置
\renewcommand{\tablename}{表}						% 改變表格標號文字為中文的「表」(預設為 Table)
\renewcommand{\figurename}{圖}						% 改變圖片標號文字為中文的「圖」(預設為 Figure)
%\renewcommand{\figurename}{圖\hspace*{-.5mm}}
%  for Long Report or Book
\renewcommand{\contentsname}{{\CB 目錄}}
\renewcommand\listfigurename{{\CB 圖目錄}}
\renewcommand\listtablename{{\CB 表目錄}}
\renewcommand{\indexname}{{\CB 索引}}
\renewcommand{\bibname}{{\CB 參考文獻}}
%-----------------------------------------------------------------------------------------------------------------------

%\theoremstyle{plain}
\newtheorem{de}{Definition}[section]				%definition獨立編號
\newtheorem{thm}{{\MB 定理}}[section]			%theorem 獨立編號,取中文名稱並給予不同字型
\newtheorem{lemma}[thm]{Lemma}				%lemma 與 theorem 共用編號
\newtheorem{ex}{{\E Example}}						%example 獨立編號,不編入小節數字,走流水號。也換個字型。
\newtheorem{cor}{Corollary}[section]				%not used here
\newtheorem{exercise}{EXERCISE}					%not used here
\newtheorem{re}{\emph{Result}}[section]		%not used here
\newtheorem{axiom}{AXIOM}							%not used here
%\renewcommand{\proofname}{\bf{Proof}}		%not used here

\newcommand{\loflabel}{圖} % 圖目錄出現 圖 x.x 的「圖」字
\newcommand{\lotlabel}{表}  % 表目錄出現 表 x.x 的「表」字

\parindent=0pt
\setcounter{tocdepth}{0}

%--- 其他定義 ----------------------------------
% 定義章節標題的字型、大小
\titleformat{\chapter}[display]{\centering\LARGE\MB}
 {\CB 第\ \thechapter\ 章}{0.2cm}{}
\titlespacing{\chapter}{0cm}{-1.3cm}{1em}
%\titleformat{\chapter}[hang]{\centering\LARGE\sf}{\MB 第~\thesection~章}{0.2cm}{}%控制章的字體
\titleformat{\section}[hang]{\Large\sf}{\MB 第~\thesection~節}{0.2cm}{}%控制章的字體
%\titleformat{\subsection}[hang]{\centering\Large\sf}{\MB 第~\thesubsection~節}{0.2cm}{}%控制節的字體
%\titleformat*{\section}{\normalfont\Large\bfseries\MB}
\titleformat*{\subsection}{\normalfont\large\bfseries\MB}
%\titleformat*{\subsubsection}{\normalfont\large\bfseries\MB}
\usepackage{titlesec}
\usepackage{titletoc}

%目錄裡的 "章"文字
\titlecontents{chapter}[1em]{}{\makebox[4.1em][l]
{\CB{第}\ST{\thecontentslabel}\CB{章}}}{}{~\titlerule*[0.7pc]{.}~\contentspage}



\definecolor{slight2}{gray}{0.5}	
\definecolor{slight}{gray}{0.8}								% 設定顏色
\definecolor{airforceblue}{rgb}{0.36, 0.54, 0.66} % color Table: http://latexcolor.com
\definecolor{arylideyellow}{rgb}{0.91, 0.84, 0.42}
\definecolor{babyblue}{rgb}{0.54, 0.81, 0.94}
\definecolor{cadmiumred}{rgb}{0.89, 0.0, 0.13}
\definecolor{coolblack}{rgb}{0.0, 0.18, 0.39}
\definecolor{cottoncandy}{rgb}{1.0, 0.74, 0.85}
\definecolor{desertsand}{rgb}{0.93, 0.79, 0.69}
\definecolor{electriclavender}{rgb}{0.96, 0.73, 1.0}
\definecolor{lightsalmonpink}{rgb}{1.0, 0.6, 0.6}
\definecolor{amaranth}{rgb}{0.9, 0.17, 0.31}
\definecolor{amethyst}{rgb}{0.6, 0.4, 0.8}
\definecolor{atomictangerine}{rgb}{1.0, 0.6, 0.4}
\definecolor{babyblueeyes}{rgb}{0.63, 0.79, 0.95}
\definecolor{gray(x11gray)}{rgb}{0.75, 0.75, 0.75}
\definecolor{bananamania}{rgb}{0.98, 0.91, 0.71}	
\definecolor{ballblue}{rgb}{0.13, 0.67, 0.8}
\definecolor{azure(colorwheel)}{rgb}{0.0, 0.5, 1.0}	
\definecolor{ceruleanblue}{rgb}{0.16, 0.32, 0.75}
\definecolor{lightsalmonpink}{rgb}{1.0, 0.6, 0.6}
\definecolor{palepink}{rgb}{0.98, 0.85, 0.87}
\definecolor{lightcarminepink}{rgb}{0.9, 0.4, 0.38}
\definecolor{lightcornflowerblue}{rgb}{0.6, 0.81, 0.93}
\definecolor{carolinablue}{rgb}{0.6, 0.73, 0.89}  									
\usepackage{wallpaper} %浮水印
\URCornerWallPaper{0.25}{\imgdir{圖片2.jpg}}   %來自package wallpaper %0.8圖片大小
% 啟動索引製作,(1) 配合 \usepackage{makeidx} 使用,
%                      (2) 在文件 \begin{document} 前加入 \makeindex
%                      (3) 在文件最後要印出索引的地方加入 \printondex
%                      (4) 在文章中欲編入索引的文字後加上 \index{該文字}
%                      (5) 編譯時加入 makeindex 一至兩次
				
%----文件開始-------------------------------------
\begin{document}
%---封面製作 ------------------------------------
%\ThisURCornerWallPaper{-1}{\imgdir{ntpu.eps}}
\thispagestyle{empty} %不要其他東西 ex浮水印
\fontsize{40}{30pt}\selectfont
\begin{flushleft}
    \bigskip\bigskip\bigskip\bigskip\bigskip\bigskip
    {\CB \XeLaTeX }\\    
    \smallskip{\color{slight2}{\CB 使用介紹}}\\
    \rule{3in}{0.2pt}\\
    \bigskip\LARGE {\MB 游筑鈞}\;\;410578049\rule{8mm}{0pt}\\
    \bigskip\par\small\today\rule{8mm}{0pt}
\end{flushleft}
\bigskip\bigskip\bigskip\bigskip\bigskip\bigskip\bigskip\bigskip\bigskip\bigskip\bigskip
\bigskip\bigskip\bigskip\bigskip\bigskip\bigskip\bigskip\bigskip\bigskip


\begin{overpic}[scale=0.75]{\imgdir{圖片1.jpg}}
    \centering
    %\put(20,52){\color{slight}{\bf L}}
    %\put(30,40){\color{slight}{\bf A}}
    %\put(40,30){\color{slight}{\bf T}}
    %\put(40,40){\color{slight}{\bf E}}
\end{overpic}



    
\fontsize{12}{22pt}\selectfont

%--- 製作目錄 -----------------------------------
\newpage
\pagenumbering{roman}  % 羅馬文頁碼
\cleardoublepage %機樹葉 開始 偶數業 空
\setcounter{tocdepth}{2} % 目錄層次

\tableofcontents

\newpage
\renewcommand{\numberline}[1]{\loflabel~#1\hspace*{1em}}% 圖目錄出現 圖 x.x 的「圖」字
\listoffigures %圖目錄
\newpage
\renewcommand{\numberline}[1]{\lotlabel~#1\hspace*{1em}}% 表目錄出現 表 x.x 的「表」字
\listoftables % 表目錄
\newpage
%--- 放入各章 -----------------------------------
%\addcontentsline{toc}{chapter}{\CB{序文}}
\chapter*{\CB{序文}}

\LaTeX 在排版軟體中扮演著重要的角色,尤其是在撰寫論文時,更是大家推崇的排版軟體。由於\LaTeX 的排版方法和平常大家熟悉的其他軟體使用滑鼠點選不一樣。\LaTeX 需要輸入程式碼,而且特別需要注意小細節才能成功的排版出文件,因此要接觸、熟悉\LaTeX 並不是一件容易的事。本學期藉由統計排版軟體課程中,使用了\LaTeX 排版軟體,因此有了這份作品。這裡將會介紹\LaTeX 當中一些常用的部分。數學環境、製作表格、圖片的插入、以及大型文件的管理是排版中最基本的功能,也是身為學生最常會用到的部分。接著本書將會逐一介紹這些功能可以使用的套件和需要注意的事項,最後說明如何將數個小型文件合併成大型的文件或書籍,並且善加管理目錄等等。希望透過本書的分享,讓使用\LaTeX 排版變成一件輕鬆的事。
 % 序言
%\addcontentsline{toc}{chapter}{再序:不可變與不可不變:我的程式寫作觀}
\chapter*{再序\\ 不可變與不可不變:我的程式寫作觀}
教學七年了,這本講義也用了三年。其間經過多次的修改,不管擴編還是刪減,多半是依據上課時學生的反應而來。
這本講義其實很多地方寫得不夠詳細,本想進一步將所有細節完整呈現,成為一本書。但幾經思量,仍維持原貌,
原因是太詳細的內容會養成學生的依賴心,喪失原先期望學生自己去補足不清楚、不詳細的部分。
希望學生藉著這門課拾回過去學得不清不楚的微積分、統計學與線性代數。\\

這本講義企圖將數學原理以電腦數據圖表的方式呈現出來,再要求同學以文字圖案呈現出其間的條理,
這樣的訓練是現今大學生十分欠缺的。說穿了就是「表達的能力」的培養。這可不是說、學、逗、唱之類的表達,
而是一種試圖將不易說清楚或難以理解的東西,透過文字、圖表或語言將它交代清楚。這樣的能力絕對需要長時間的訓練,
有了這項「絕技,」大學畢業生不必急著說自己學非所用。有太多的事實證明,擁有絕佳的表達能力,放諸四海都餓不著肚子。\\

表達能力的養成必須按部就班,一點都急不得。可惜的是,莘莘學子不是自作聰明,便是固執己見,
往往喜歡憑自己過去的經驗來解決未知的問題,缺乏耐心去熟練不熟悉的工具,不願將專注力用在問題的觀察。
學習過程像極矇著雙眼亂砍亂殺,到頭來學不到東西還怪老師出太多怪怪的功課,既對升學沒有幫助,也無助以後做生意賺大錢,
不多久便放棄了,殊是可惜。告訴他這是未來升官發財的利器,他當你在三娘教子,在家裡聽多了。\\

以寫作程式為例,每一種程式語言都有其語法規範,該怎麼寫怎麼用,一點也馬虎不得,連錯一點點都不行,
沒得商量的。初學者往往輕忽之,不喜歡被「規範」束縛,不顧老師一再地提醒,愛怎麼寫就怎麼寫,
天才般的自己編撰起語法來了,結果當然是錯誤百出,急得老師在一旁乾著急。更有甚之,錯了還不認帳,
直呼語法太不人性化,不能隨意更動,學它何用,便率性的打起電動或MSN來了。\\

寫程式首要遵守語法教條,待熟悉語法規範之後,才能漸漸懂得運用,透過寫一些不痛不癢的小程式,
一方面熟悉語法,一方面體驗其威力。漸熟,才慢慢從觀察別人寫的「模範程式」中,瞭解死的語言原來也能玩出活把戲,
這才一步步進入寫作程式的精髓,進一步玩出樂趣。這道理亙古不變,古今達人不管學習琴棋書畫,還是打拿摔跌等武藝,
無不遵循這樣的哲理\footnote{謫自五絕奇人鄭曼青先生名著「曼髯三論。」}\\

{\MB 能力未至不可變也、學識未敷不得變也、功侯未到不能變也。\\
學於師已窮其法,不可不變也、友古人已悉其意,不得不變也、\\
師造化已盡其理,不能不變也。}\\

從「不可變、」「不得變、」「不能變」,到「不可不變、」「不得不變、」「不能不變,」可以作為寫作程式的養成過程。
學習之初應謹慎遵循所有的規範,一絲不苟,不能濫用自己的小聰明亂抄捷徑,要聽話、要服從,
將老師的交代與叮嚀當作聖旨般遵循,務必做到。如此這般一段時日之後,犯錯愈少,進步愈多,
自然而然當變則變,逐漸形成自己的風格。\\

不能急,成就總在不知不覺中「赫然」被別人發現,絕非刻意營造而能得。別人眼中看到的成就,
對自己而言永遠都是平常事而已,只不過在許多小地方比別人好一點點罷了。但別小看這一點點,
許許多多的一點點累積起來,那可有多少啊!

\begin{flushright}
    汪群超
    \par\vspace*{-2pt}\hfill 2005年2月於台北大學
\end{flushright}
% 序言
\ifodd\count0 \else \thispagestyle{empty}\mbox{}\clearpage\fi % 如果是偶數頁就呈現空白頁
\addcontentsline{toc}{chapter}{\CB{序文}}
\chapter*{\CB{序文}}

\LaTeX 在排版軟體中扮演著重要的角色,尤其是在撰寫論文時,更是大家推崇的排版軟體。由於\LaTeX 的排版方法和平常大家熟悉的其他軟體使用滑鼠點選不一樣。\LaTeX 需要輸入程式碼,而且特別需要注意小細節才能成功的排版出文件,因此要接觸、熟悉\LaTeX 並不是一件容易的事。本學期藉由統計排版軟體課程中,使用了\LaTeX 排版軟體,因此有了這份作品。這裡將會介紹\LaTeX 當中一些常用的部分。數學環境、製作表格、圖片的插入、以及大型文件的管理是排版中最基本的功能,也是身為學生最常會用到的部分。接著本書將會逐一介紹這些功能可以使用的套件和需要注意的事項,最後說明如何將數個小型文件合併成大型的文件或書籍,並且善加管理目錄等等。希望透過本書的分享,讓使用\LaTeX 排版變成一件輕鬆的事。

\pagenumbering{arabic} % 阿拉伯文頁碼
\setcounter{page}{1} % 第一章 從第 1 頁開始

%% These two lines must be incldued to open file under UTF-8
% !TEX TS-program = xelatex								
% !TEX encoding = UTF-8

\documentclass[12pt, a4paper]{article} 		% use larger type; default would be 10pt
\usepackage{fontspec} 				% Font selection for XeLaTeX; see fontspec.pdf for documentation. 
%\usepackage[BoldFont, SlantFont]{xeCJK}% 中文使用 XeCJK,並模擬粗體與斜體(即可以用 \textbf{ } \textit{ })
\usepackage{xeCJK}							% 中文使用 XeCJK,但利用 \setCJKmainfont 定義粗體與斜體的字型
\defaultfontfeatures{Mapping=tex-text} 		% to support TeX conventions like ``---''
\usepackage{xunicode} 		% Unicode support for LaTeX character names (accents, European chars, etc)
\usepackage{xltxtra} 						% Extra customizations for XeLaTeX
\usepackage{amsmath, amssymb}
\usepackage{enumerate}
\usepackage{graphicx, subfig, float} 		% support the \includegraphics command and options
\usepackage{array, booktabs}
\usepackage{color, xcolor}
\usepackage{longtable}
\usepackage{colortbl}   
\usepackage{multirow} 
\usepackage{multicol} 
\usepackage{arydshln}  
\usepackage{dcolumn} 
\usepackage{rotating} 
\usepackage{diagbox} 
\usepackage{wrapfig} %文繞圖
\usepackage{overpic}%圖片上加文字
%\usepackage{natbib}
\usepackage[sort&compress,square,comma,authoryear]{natbib}
%.............................................表格標題註解之巨集套件
\usepackage[parfill]{parskip} % Activate to begin paragraphs with an empty line rather than an indent
%\usepackage{geometry} % See geometry.pdf to learn the layout options. There are lots.
\usepackage[left=1.5in,right=1in,top=1in,bottom=1in]{geometry} 

%-----------------------------------------------------------------------------------------------------------------------
%  主字型設定
\setCJKmainfont							% 設定中文內文字型
	[
		BoldFont=微軟正黑體				% 定義粗體的字型(依使用的電腦安裝的字型而定)
	]
	{新細明體}							% 設定中文內文字型
\setmainfont{Times New Roman}			% 設定英文內文字型
\setsansfont{Arial}						% used with {\sffamily ...}
%\setsansfont[Scale=MatchLowercase,Mapping=tex-text]{Gill Sans}
\setmonofont{Courier New}				% used with {\ttfamily ...}
%\setmonofont[Scale=MatchLowercase]{Andale Mono}
% 其他字型(隨使用的電腦安裝的字型不同,用註解的方式調整(打開或關閉))
% 英文字型
\newfontfamily{\E}{Cambria}				
\newfontfamily{\A}{Arial}
\newfontfamily{\C}[Scale=0.9]{Cambria}
%\newfontfamily{\T}{Times New Roman}
\newfontfamily{\TT}[Scale=0.8]{Times New Roman}

% 中文字型
\newCJKfontfamily{\MB}{微軟正黑體}			% 適用在 Mac 與 Win
\newCJKfontfamily{\SM}[Scale=0.8]{新細明體}	% 縮小版新細明體
\newCJKfontfamily{\K}{標楷體} 
               % Windows 下的標楷體
% 以下為自行安裝的字型:CwTex 組合
\newCJKfontfamily{\CF}{cwTeX Q Fangsong Medium}	% CwTex 仿宋體
\newCJKfontfamily{\CB}{cwTeX Q Hei Bold}			% CwTex 粗黑體
\newCJKfontfamily{\CK}{cwTeX Q Kai Medium}   		% CwTex 楷體
\newCJKfontfamily{\CM}{cwTeX Q Ming Medium}		% CwTex 明體
\newCJKfontfamily{\CR}{cwTeX Q Yuan Medium}		% CwTex 圓體
%-----------------------------------------------------------------------------------------------------------------------
\XeTeXlinebreaklocale "zh"             %這兩行一定要加,中文才能自動換行
\XeTeXlinebreakskip = 0pt plus 1pt     %這兩行一定要加,中文才能自動換行
%-----------------------------------------------------------------------------------------------------------------------
\newcommand{\cw}{\texttt{cw}\kern-.6pt\TeX}	% 這是 cwTex 的 logo 文字
\newcommand{\imgdir}{../images/}				% 設定圖檔的位置
\renewcommand{\tablename}{表}				% 改變表格標號文字為中文的「表」(預設為 Table)
\renewcommand{\figurename}{圖}				% 改變圖片標號文字為中文的「圖」(預設為 Figure)


%------------------------------------------------------------------------------------------------------
%  計數器
\usepackage{amsthm}							% theroemstyle 需要使用的套件
\theoremstyle{plain}                       %樣式
\newtheorem{de}{Definition定義}[section]    %definition獨立編號 %跟section走憶起編號
\newtheorem{thm}{{\MB 定理}}[section]		%theorem 獨立編號,取中文名稱並給予不同字型
\newtheorem{lemma}[thm]{Lemma}				%lemma 與 theorem 共用編號 %跟thm走憶起編號
\newtheorem{ex}{{\E Example}}			 %example 獨立編號,不編入小節數字,走流水號。也換個字型。
\newcounter{e}
\newtheorem{cor}{Corollary}[section]		%not used here
\newtheorem{exercise}{EXERCISE}				%not used here
\newtheorem{re}{\emph{Result}}[section]	%not used here
\newtheorem{axiom}{AXIOM}					%not used here
\renewcommand{\proofname}{\bf{Proof}}		%not used here
\bibliographystyle{plain}
%-----------------------------------------------------------------------------------------------------------------------
% 設定顏色
\definecolor{slight}{gray}{0.9}				
\definecolor{airforceblue}{rgb}{0.36, 0.54, 0.66} % color Table: http://latexcolor.com
\definecolor{arylideyellow}{rgb}{0.91, 0.84, 0.42}
\definecolor{babyblue}{rgb}{0.54, 0.81, 0.94}
\definecolor{cadmiumred}{rgb}{0.89, 0.0, 0.13}
\definecolor{coolblack}{rgb}{0.0, 0.18, 0.39}
\definecolor{cottoncandy}{rgb}{1.0, 0.74, 0.85}
\definecolor{desertsand}{rgb}{0.93, 0.79, 0.69}
\definecolor{electriclavender}{rgb}{0.96, 0.73, 1.0}
\definecolor{lightsalmonpink}{rgb}{1.0, 0.6, 0.6}
\definecolor{amaranth}{rgb}{0.9, 0.17, 0.31}
\definecolor{amethyst}{rgb}{0.6, 0.4, 0.8}
\definecolor{atomictangerine}{rgb}{1.0, 0.6, 0.4}
\definecolor{babyblueeyes}{rgb}{0.63, 0.79, 0.95}
\definecolor{gray(x11gray)}{rgb}{0.75, 0.75, 0.75}
\definecolor{bananamania}{rgb}{0.98, 0.91, 0.71}	
\definecolor{ballblue}{rgb}{0.13, 0.67, 0.8}
\definecolor{azure(colorwheel)}{rgb}{0.0, 0.5, 1.0}	
\definecolor{ceruleanblue}{rgb}{0.16, 0.32, 0.75}
\definecolor{lightsalmonpink}{rgb}{1.0, 0.6, 0.6}
\definecolor{palepink}{rgb}{0.98, 0.85, 0.87}
\definecolor{lightcarminepink}{rgb}{0.9, 0.4, 0.38}
\definecolor{lightcornflowerblue}{rgb}{0.6, 0.81, 0.93}
\definecolor{carolinablue}{rgb}{0.6, 0.73, 0.89}
   % 使用自己維護的定義檔
%-----------------------------------------------------------------------------------------------------------------------
% 文章開始
%\title{ \LaTeX{\CB 的數學符號與方程式}}
%\author{{\SM 游筑鈞}}
%\date{{\TT \today}} 	% Activate to display a given date or no date (if empty),
         				% otherwise the current date is printed 
%\begin{document}       %document有預設的字型
%\maketitle %顯示標題
%\fontsize{12}{22 pt}\selectfont   %\selectfont-讓前面的設定生效
\chapter{\CB{數學符號與式子}}
本文將常見的數學符號與方程式以 \LaTeX 編排,並以各式各樣數學方程式展現出 \LaTeX 在編輯數學式時的強大功能。相信使用過word排版過數學式的人都知道要排出整齊優美的方程式是一件很難的事,透過 \LaTeX 可以輕鬆地完成。文章中會將不同類型的方程式大致分類,以便以後在使用時可以快速的找到。這裡希望透過蒐集不同的方程式,並以 \LaTeX 進行編排作為後續參考內容。本文內容參考汪群超教授網站 \footnote{相關文件可在https://ntpuccw.blog/supplements/xetex-tutorial/ 下載。}\\
%FOOTNOTE放在本句話最後,標點符號後\\\

\section{\MB{數式環境}}
我們平常寫文章的模式無法正確處理數學式子間的空間位置。因此,所有的數學式子都得進入數學模式來處理。在數學模式下,不僅大部份文字、符號會採用斜體字,而且空間會另做安排,額外的空白會被 \LaTeX  忽略。\smallskip

LaTeX 的數學模式有兩種,一種是和內文排列在一起的隨文數式(math inline mode),他是和一般正常文字混在一起排版的;另外一種是獨立的展式數式(math display mode),他會單獨成一行,而且上下會和正常文字有一定的空間來區隔。

\subsection{隨文數式(math inline mode)}
用兩個錢字號前後包圍這樣會進入隨文的數學模式,在一般文字段落內要使用到一些數學式子的話,這是最方便的方法。
隨文數式的應用很多,例如:因為 $\lvert t^{*}\rvert=10.2 < 2.069$ 所以拒絕 $H_0$,也就是說$\beta_0 \neq 0$。等等 $\cdots$

\subsection{展式數式(math display mode)}
通常獨立的數學式子,通常會單獨成一行,需要的話也可以加入編號,以方便在文章中引用。展示數式會適當的選用較大的數學符號及字體,尤其是較複雜的數學式子的時候。例如:
$$\beta_0+\beta_1\pm{W\sqrt{MSE}}\bigg[\frac{1}{n}+\frac{{(X-\bar{X}})^2}{\sum{(X-\bar{X})^2}}\bigg]^{1/2}$$
展式數式的幾種做法:
\begin{enumerate}
\item 在數學式子前面跟後面各加2個錢字號,讓 \LaTeX 知道要進入數式環境,並讓數學式置中。
\item 使用 begin equation 和 end equation 來做,利用這種做法可以幫數學式子標號。
\end{enumerate}


\section{\MB{符號}}
數字與普通運算符號可直接由鍵盤上鍵入。譬如,下列符號可以直接由鍵盤鍵入:

        \begin{center}
         $  + \;-\; =\; <\; > \;/ \;:\; !\;\; |\; \;[\;\; ] \;(\; )$\\
        \end{center}

\subsection{特殊符號}

\begin{itemize}
\item 希臘字符:$ \;\alpha \;\theta \;\tau \;\beta \;\pi  \;\upsilon \;\gamma  \;\iota \;\varpi \;\phi \;\delta  \;\kappa \;\epsilon  \;\lambda $ 
\item 分隔符號:$ ( )  \uparrow  \Uparrow [  ] \downarrow  \Downarrow \{  \}  \updownarrow \lfloor  \rfloor  \lceil \rceil \langle \rangle \backslash$
\item 重音符號:$\hat{a}  \;\acute{a}  \;\bar{a}  \;\dot{a}  \;\breve{a} \;\check{a}  \;\grave{a} \;\vec{a}  \;\ddot{a} \;\tilde{a}$
\item 大型運算符號:$\;\sum \;\bigcap \;\bigodot \;\prod \;\bigcup \;\bigotimes \;\coprod \;\bigsqcup  \;\bigoplus \;\int \;\bigvee$
\item 運算結構:$\;\widetilde{abc} \;\widehat{abc} \;\overleftarrow{abc} \;\overrightarrow{abc} \;\overline{abc} \;\underline{abc}\;\overbrace{abc} \;\underbrace{abc} \;\sqrt{abc} \;\sqrt[n]{abc} \;\frac{abc}{xyz}$
\item 字符間空格:
\begin{center} 
\begin{tabular}{|l|c|l|c|}%
\hline  %劃上一條橫線
  2个quad空格  & $\alpha\qquad\beta$  & quad空格	&$\alpha\quad\beta$	\\\hline  % &代表換欄 \\代表換下一列
  大空格	      & $\alpha\ \beta$	      & 中等空格 & $\alpha\;\beta	$		\\\hline
  小空格       & $\alpha\,\beta	$     & 没有空格	& $\alpha\beta$  \\\hline
  緊貼         & $\alpha\!\beta	$     &         &                  \\\hline
\end{tabular}\\
\end{center}
\end{itemize}


\section{\MB{常見的數學式}}
本節列舉一些常見的數學式作為練習與未來使用的參考。相信做過了越多的數學式子練習後,以後就可以輕而易舉地寫出複雜的公式了。

\subsection{分式與根式}
範例一 :
$$
\frac{\frac{\displaystyle a}{\displaystyle x-y}+
\frac{\displaystyle b}{\displaystyle x+y}}
{\frac{\displaystyle x-y}{\displaystyle x+y}+
\frac{\displaystyle a-b}{\displaystyle a+b}}
$$


\bigskip
如果覺得字符太小可以調整設定:
\begin{itemize}

\item$\backslash${\A displaystyle}:	展示數式的標準字體大小
\item$\backslash${\A textstyle}	:隨文數式的標準字體大小
\item$\backslash${\A scriptstyle}:	第一層上下標字體大小
\item$\backslash${\A scriptscriptstyle}:	第二層上下標字體大小
\end{itemize}
\bigskip
範例二 :傳說中的拉馬努金公式
\bigskip

拉馬努金,每天廢寢忘食,只研究數學,就在這時神奇的事情又發生了,他每天晚上睡覺的時候,都會夢到自己所信宗教的女神。 拉馬努金醒來以後,腦子裏充滿了各種各樣的公式,那以後,女神每天都出現在他的夢裏,告訴他一些新公式。拉馬努金每天清晨都要趕快拿出筆記本,把夢中得到的公式記在本上,由於筆記本的費用對他來說很高昂,所以每次女神告訴他的時候,他只把最終得出的,最簡化的公式抄到本上。幾年下來,他得到了3,900個複雜的公式! 

\bigskip
$$
\nonumber\frac{1}{\pi}=\frac{2\sqrt{2}}{9801}\sum_{k=0}^\infty{\frac{(4k)!(1103+26390k)}{(4k)!3962^{4k}}}
$$

$$
\nonumber\sqrt{\frac{1+\sqrt{5}}{2}+2}-\frac{1+\sqrt{5}}{2}=\frac{\displaystyle e^{\frac{-2\pi}{5}}}{1+\frac{\displaystyle e^{-2\pi}}{1+\frac{\displaystyle e^{-4\pi}}{1+\frac{\displaystyle e^{-6\pi}}{1+\ldots}}}}
$$

拉馬努金恆等式
\begin{eqnarray}
% 等號對齊
        \nonumber 3&=&\sqrt{1+2+4} \\
        \nonumber &=& \sqrt{1+2\sqrt{1+3*5}} \\
        \nonumber &=& \sqrt{1+2\sqrt{1+3\sqrt{1+4*6}}} \\
        \nonumber  &=& \sqrt{1+2\sqrt{1+3\sqrt{1+4\sqrt{1+5*7}}}} \\
        \nonumber &=& \ldots
\end{eqnarray}

對齊的兩種方法:
\begin{enumerate}
\item align可以用來讓公式對齊,在公式中,加 $\backslash$ $\backslash$ 表示换行、加 $\&$ 表示要對齊的地方。
\item ennarray 也是其中一種方法,在$ = $前後各加一個 $\&$ 讓等號對齊。
\end{enumerate}


\subsection{函數}

  \textbf{Likelihood function of Normal distribution}: 
  
  $$
  L(\mu,\sigma^{2},x_1,\ldots,x_n)=\prod_{j=1}^{n}f_x(x_j;\mu_i\sigma^{2})=(2\pi\sigma^{2})^{\frac{-n}{2}}\exp\left(\frac{-1}{2\sigma^{2}}\sum_{j=1}^{n}(x_i-\mu)^{2}\right)
  $$
\bigskip

括號的使用:數學中的括號隨著其內容的多寡,其大小必須調整恰當,如上式的兩種大小不同的括號「$( \cdot)$ 」。外圍較大地括號使用 $\backslash$  left$($ 與 $\backslash$  right$($ 令編譯器依需求自動調整為適當大小。另外,也可以手動控制括號、的大小,如
$$ \bigg(\; \big( \;(\;\;\;) \;\big) \;\bigg) \;,\; \bigg[ \;\big[ \;[\;\;\;]\; \big]\; \bigg]\;,\; \bigg\{ \;\big\{ \;\{\;\;\;\} \;\big\} \;\bigg\}$$ 
 
 \bigskip
 
  \textbf{Multivariate normal distribution}: 
  
$$
\begin{aligned}
f(y_1,y_2)&= \frac{1}{2\pi\sigma_1\sigma_2}\exp\left[-\frac{(y_1-\mu_1)^{2}}{2\sigma_1^{2}}-\frac{(y_2-\mu_2)^{2}}{\sigma_2^{2}}\right]\\[4mm]
&= \frac{1}{\sqrt{2\pi}\sigma_1}\exp^{\displaystyle-\frac{(y_1-\mu_1)^{2}}{2\sigma_1^{2}}}\frac{1}{\sqrt{2\pi}\sigma_2}\exp^{\displaystyle-\frac{(y_2-\mu_2)^{2}}{\sigma_2^{2}}}
\end{aligned}
$$
\textbf{Beta distribution}: 
$$
\begin{aligned}
f(x;\alpha,\beta)&=\frac{x^{\alpha-1}(1-x)^{\beta-1}}{\int^1_0u^{\alpha-1}(1-u)^{\beta-1}du}\\[4mm]
&=\frac{\Gamma(\alpha+\beta)}{\Gamma(\alpha)\Gamma(\beta)}x^{\alpha-1}(1-x)^{\beta-1}\\[4mm]
&=\frac{1}{B(\alpha,\beta)}x^{\alpha-1}(1-x)^{\beta-1}
\end{aligned}
$$

公式間行距微調:數學式中常有複雜的計算過程造成公式之間看起來很擠,所以可以在$\backslash$ $\backslash$ 换行符號後面加上[4mm],增加適當的行距。
 

\subsection{積分與微分式}
積分式:
\begin{equation}\label{inter01}
 \int^{2\pi}_0(\cos\alpha)^{m}(\sin\alpha)^{n}\exp^{-\alpha(J\sin\alpha)+K\cos\alpha}d\alpha
\end{equation}

方程式(\ref{inter01}) 是三角函數的積分。可以在begin equation 後面加上label(命名)為數學式加上編號並命名方便在內文中做交互參照。欲做交互參照可以在文中插入ref。 

微分式:
\begin{equation}\label{diff01}
\frac{\partial}{\partial a}\left(\int^b_a f(x)dx\right)=\lim_{\Delta a \rightarrow 0}\frac{1}{\Delta a}\left[\int^b_{a+\Delta a} f(x)dx-\int^b_a f(x)dx\right]
\end{equation}



\begin{equation}\label{diff02}
\nabla\cdot\mathbf{A}=\frac{1}{r^{2}\sin\theta}\left[\sin\theta\frac{\partial }{\partial r}(r^{2}\mathbf{A_r})+r\frac{\partial}{\partial\theta}(\sin\theta \mathbf{A_\theta})+r\frac{\partial \mathbf{A_\phi}}{\partial\phi^{2}}\right]
 \end{equation}
方程式(\ref{diff02})是偏微分的公式。數學式子中難免會有向量符號,在 \LaTeX 數式環境中可以用\A mathbf 來表示,呈現粗體字型。

向量數學式:
$$
\overrightarrow{a}=(a_1,a_2) \qquad \overrightarrow{b}=(b_1,b_2)
$$
$$
\overrightarrow{a}\centerdot \overrightarrow{b}=a_1b_1+a_2+b_2= \arrowvert \overrightarrow{a} \arrowvert  \arrowvert \overrightarrow{b} \arrowvert\cos\theta
$$
\subsection{矩陣與行列式}
矩陣或有規則排列的數學式或組合很常見,以下列舉幾種模式,請特別注意其使用的標籤及一些需要注意的小地方。

\begin{enumerate} 
  \item 矩陣的左右括號需各別加上。
  \item 橫行各項之間是以 $\&$ 區隔。
  \item 除最後一行外,每行之末則加上換行指令 $\backslash\backslash$。
  \item 使用 {\A array} 指令時,須加上選項以控制每一直欄內各數字或符號要居中排列、靠左或靠右。
\end{enumerate}

聯立式:
\begin{equation}\label{01}
\vec{y} =
\left( \begin{array}{cc}
y_{1} =
\left| \begin{array}{cc}
x_{11} & x_{12} \\
x_{21} & x_{22}
\end{array} \right| \\
y_{2} \\
y_{3}
\end{array} \right)
\end{equation}


其中 \A left 以及 \A right 兩指令必須同時使用, 且 \A left] 後面接 \A right( 是會被 \LaTeX 允許的。

\begin{equation}\label{02}
g(x,y) = \left\{\begin{array}{ll}
                 f(x,y), & \mbox{if $x<y$} \\  
                 f(y,x), & \mbox{if $x>y$} \\  
                 0,      & \mbox{otherwise.}
                \end{array} \right.
\end{equation}

\LaTeX 表示聯立方程的方法亦使用陣列, 其中大括號只會有一個, 此時必須輸入 \A left. 或 \A right. 對應才會被 \LaTeX 接受。
有時候數學式中的文字不需要斜體如(\ref{02}),可以用mbox來控制字體不要斜體。


\begin{equation}
\begin{cases}
 \ u_{tt}(x,t)= b(t)\triangle u(x,t-4)&\\
\ \hspace{42pt}- q(x,t)f[u(x,t-3)]+te^{-t}\sin^2 x,  &  t \neq t_k; \\
 \ u(x,t_k^+) - u(x,t_k^-) = c_k u(x,t_k), & k=1,2,3\ldots ;\\
 \ u_{t}(x,t_k^+) - u_{t}(x,t_k^-) =c_k u_{t}(x,t_k), &
 k=1,2,3\ldots\ .
\end{cases}
\end{equation}

這裡用 begin cases 和 end cases 來代替上面(\ref{02})中的大括號,hspace{距离}可以插入任意空格。
\bigskip
矩陣:
\begin{equation}
A =
\begin{pmatrix}              
  t_{11} & t_{12} & t_{13} \\
  t_{21} & t_{22} & t_{23} \\
  t_{31} & t_{32} & t_{33}
\end{pmatrix}
\end{equation}

距陣的類型:

\begin{center} 
\begin{tabular}{c c c c}%
  
   \extrarowheight=8pt 
  matrix & 
  $ \begin{matrix} 1&0&0\\ 0&1&0\\ 0&0&1\\ \end{matrix}
  $  
  &bmatrix	& 
  $\begin{bmatrix} 1&0&0\\ 0&1&0\\ 0&0&1\\ \end{bmatrix}
  $	\\  % &代表換欄 \\代表換下一列
  \\
  vmatrix      & $\begin{vmatrix} 1&0&0\\ 0&1&0\\ 0&0&1\\ \end{vmatrix}$ & pmatrix & $\begin{pmatrix} 1&0&0\\ 0&1&0\\ 0&0&1\\ \end{pmatrix}$	\\
  \\
 Bmatrix       & $\begin{Bmatrix} 1&0&0\\ 0&1&0\\ 0&0&1\\ \end{Bmatrix}$    & Vmatrix	& $\begin{Vmatrix} 1&0&0\\ 0&1&0\\ 0&0&1\\ \end{Vmatrix}$ \\

\end{tabular}\\

\end{center}
\bigskip
距陣並排:
\begin{equation}
H_x=\frac{1}{3}\times{
\left[ \begin{array}{ccc}
1 & 0 & -1\\
1 & 0 & -1\\
1 & 0 & -1
\end{array} 
\right ]}
\qquad 
H_y=\frac{1}{3}\times{
\left[ \begin{array}{ccc}
1 & 1 & 1\\
0& 0 & 0\\
1 & 1 & 1
\end{array}
\right ]}
\end{equation}

對角矩陣:
\begin{equation}
\begin{pmatrix}
\alpha_1                                    &           &\multicolumn{2}{c}{\multirow{2}*{{\Huge0}}}\\
                                            &\alpha_2\\
\multicolumn{2}{c}{\multirow{2}*{{\Huge0}}}             &\ddots\\
                                            &           &       &\alpha_n
\end{pmatrix}
\end{equation}
左下角的0需要放在第三列的第一位置,因為他要合併的是第三行和第四行, 當然他還合併了第一第二列,斜點其實位於第三列儘管寫法上看是第二列的元素。這裡需要使用multirow的package。

矩陣中分區塊:

 $$\left[ \begin{array}{@{\,}c|c c c c@{\,}}
 
   s & y & u & o &c\\
   p & c & h & f & x\\
   u & v & k & s & b    \\\hline
   w & x & y &a & f \\
\end{array}\right]$$ 

其實矩陣中的分區塊,和在做表格的方法很相似先把位置規畫好,再放數字。指定直線和橫線要擺放的位置。\\
矩陣上下註記:
\begin{gather}
\underbrace{\begin{bmatrix}
y_{1}&1&1\\[4pt]
\frac{1}{\sqrt{2}} & y_{2} & 1 \\[4pt]
1 & 1 & y_{3}
\end{bmatrix}}_{Y(S)}
\underbrace{\begin{bmatrix}
v_{1} \\[4pt] v_{2} \\[4pt] v_{3}
\end{bmatrix}}_{V(s)}=0
\end{gather}

最前面用 begin underbrace 和 end underbrace 在矩陣的底部加上說明。
\subsection{\MB{其他}}

練習一些表較少見的數學表達式,用 WORD 很不容易做到。\\
課堂練習:\\
練習一: 

$$
SR=n\left(\frac{2}{n}\sum_{k=1}^n E\parallel y_i-Z\parallel-2\frac{\Gamma((p+1)/2)}{\Gamma(p/2)}-\frac{1}{n^2} \sum_{j,k=1}^n\parallel y_j-y_k \parallel\right)
$$

練習二:

$$
P_m,i=\sum_{j=i}^{m-1}\begin{pmatrix} m\\ j\\ \end{pmatrix}\begin{pmatrix} m-i-1\\ j-1\\ \end{pmatrix}p^jq^{m-j}\left(\frac{\eta_1}{\eta_1+\eta_2}\right)^{j-1}\left(\frac{\eta_2}{\eta_1+\eta_2}\right)^{m-j},1\leq i\leq m-1
$$

練習三:

$$
\Lambda(t)=\mbox{exp}\left(\int^t_0 \xi(u)\cdot dW(t)-\frac{1}{2}\int^t_0\parallel \xi(u)\parallel^2 du+(\lambda-\tilde{\lambda})t \right)\prod_{i=1}^{N(t)}\frac{\tilde{\lambda}\tilde{f}(Y_i)}{\lambda f(Y_i)}
$$

練習四:
$$
W_MA=\frac{(\sum_{j=1}^na_j U_{(j)})^2}{(\mathbf{X_0}-\bar{\mathbf{X})}'A^{-1}(\mathbf{X_0}-\bar{\mathbf{X})}}
$$
\section*{\MB{結論與心得}}
經過了這份作品的練習之後,漸漸地熟悉了數學式子的語法也熟悉了 \LaTeX 的使用環境。其實使用 \LaTeX 來排版數學式子時需要很細心,一個符號、細節都不能忽略不然都會出現紅字錯誤,但是為了\LaTeX 強大的數式環境,我們需要透過大量的練習掌握技巧。雖然有些數學式子看起來很複雜只要先把它分區塊,一部份一部份來輸入語法,就沒有想像中的那麼難掌握了。

%\end{document}
%% These two lines must be incldued to open file under UTF-8
% !TEX TS-program = xelatex								
% !TEX encoding = UTF-8

\documentclass[12pt, a4paper]{article} 		% use larger type; default would be 10pt
\usepackage{fontspec} 				% Font selection for XeLaTeX; see fontspec.pdf for documentation. 
%\usepackage[BoldFont, SlantFont]{xeCJK}% 中文使用 XeCJK,並模擬粗體與斜體(即可以用 \textbf{ } \textit{ })
\usepackage{xeCJK}							% 中文使用 XeCJK,但利用 \setCJKmainfont 定義粗體與斜體的字型
\defaultfontfeatures{Mapping=tex-text} 		% to support TeX conventions like ``---''
\usepackage{xunicode} 		% Unicode support for LaTeX character names (accents, European chars, etc)
\usepackage{xltxtra} 						% Extra customizations for XeLaTeX
\usepackage{amsmath, amssymb}
\usepackage{enumerate}
\usepackage{graphicx, subfig, float} 		% support the \includegraphics command and options
\usepackage{array, booktabs}
\usepackage{color, xcolor}
\usepackage{longtable}
\usepackage{colortbl}   
\usepackage{multirow} 
\usepackage{multicol} 
\usepackage{arydshln}  
\usepackage{dcolumn} 
\usepackage{rotating} 
\usepackage{diagbox} 
\usepackage{wrapfig} %文繞圖
\usepackage{overpic}%圖片上加文字
%\usepackage{natbib}
\usepackage[sort&compress,square,comma,authoryear]{natbib}
%.............................................表格標題註解之巨集套件
\usepackage[parfill]{parskip} % Activate to begin paragraphs with an empty line rather than an indent
%\usepackage{geometry} % See geometry.pdf to learn the layout options. There are lots.
\usepackage[left=1.5in,right=1in,top=1in,bottom=1in]{geometry} 

%-----------------------------------------------------------------------------------------------------------------------
%  主字型設定
\setCJKmainfont							% 設定中文內文字型
	[
		BoldFont=微軟正黑體				% 定義粗體的字型(依使用的電腦安裝的字型而定)
	]
	{新細明體}							% 設定中文內文字型
\setmainfont{Times New Roman}			% 設定英文內文字型
\setsansfont{Arial}						% used with {\sffamily ...}
%\setsansfont[Scale=MatchLowercase,Mapping=tex-text]{Gill Sans}
\setmonofont{Courier New}				% used with {\ttfamily ...}
%\setmonofont[Scale=MatchLowercase]{Andale Mono}
% 其他字型(隨使用的電腦安裝的字型不同,用註解的方式調整(打開或關閉))
% 英文字型
\newfontfamily{\E}{Cambria}				
\newfontfamily{\A}{Arial}
\newfontfamily{\C}[Scale=0.9]{Cambria}
%\newfontfamily{\T}{Times New Roman}
\newfontfamily{\TT}[Scale=0.8]{Times New Roman}

% 中文字型
\newCJKfontfamily{\MB}{微軟正黑體}			% 適用在 Mac 與 Win
\newCJKfontfamily{\SM}[Scale=0.8]{新細明體}	% 縮小版新細明體
\newCJKfontfamily{\K}{標楷體} 
               % Windows 下的標楷體
% 以下為自行安裝的字型:CwTex 組合
\newCJKfontfamily{\CF}{cwTeX Q Fangsong Medium}	% CwTex 仿宋體
\newCJKfontfamily{\CB}{cwTeX Q Hei Bold}			% CwTex 粗黑體
\newCJKfontfamily{\CK}{cwTeX Q Kai Medium}   		% CwTex 楷體
\newCJKfontfamily{\CM}{cwTeX Q Ming Medium}		% CwTex 明體
\newCJKfontfamily{\CR}{cwTeX Q Yuan Medium}		% CwTex 圓體
%-----------------------------------------------------------------------------------------------------------------------
\XeTeXlinebreaklocale "zh"             %這兩行一定要加,中文才能自動換行
\XeTeXlinebreakskip = 0pt plus 1pt     %這兩行一定要加,中文才能自動換行
%-----------------------------------------------------------------------------------------------------------------------
\newcommand{\cw}{\texttt{cw}\kern-.6pt\TeX}	% 這是 cwTex 的 logo 文字
\newcommand{\imgdir}{../images/}				% 設定圖檔的位置
\renewcommand{\tablename}{表}				% 改變表格標號文字為中文的「表」(預設為 Table)
\renewcommand{\figurename}{圖}				% 改變圖片標號文字為中文的「圖」(預設為 Figure)


%------------------------------------------------------------------------------------------------------
%  計數器
\usepackage{amsthm}							% theroemstyle 需要使用的套件
\theoremstyle{plain}                       %樣式
\newtheorem{de}{Definition定義}[section]    %definition獨立編號 %跟section走憶起編號
\newtheorem{thm}{{\MB 定理}}[section]		%theorem 獨立編號,取中文名稱並給予不同字型
\newtheorem{lemma}[thm]{Lemma}				%lemma 與 theorem 共用編號 %跟thm走憶起編號
\newtheorem{ex}{{\E Example}}			 %example 獨立編號,不編入小節數字,走流水號。也換個字型。
\newcounter{e}
\newtheorem{cor}{Corollary}[section]		%not used here
\newtheorem{exercise}{EXERCISE}				%not used here
\newtheorem{re}{\emph{Result}}[section]	%not used here
\newtheorem{axiom}{AXIOM}					%not used here
\renewcommand{\proofname}{\bf{Proof}}		%not used here
\bibliographystyle{plain}
%-----------------------------------------------------------------------------------------------------------------------
% 設定顏色
\definecolor{slight}{gray}{0.9}				
\definecolor{airforceblue}{rgb}{0.36, 0.54, 0.66} % color Table: http://latexcolor.com
\definecolor{arylideyellow}{rgb}{0.91, 0.84, 0.42}
\definecolor{babyblue}{rgb}{0.54, 0.81, 0.94}
\definecolor{cadmiumred}{rgb}{0.89, 0.0, 0.13}
\definecolor{coolblack}{rgb}{0.0, 0.18, 0.39}
\definecolor{cottoncandy}{rgb}{1.0, 0.74, 0.85}
\definecolor{desertsand}{rgb}{0.93, 0.79, 0.69}
\definecolor{electriclavender}{rgb}{0.96, 0.73, 1.0}
\definecolor{lightsalmonpink}{rgb}{1.0, 0.6, 0.6}
\definecolor{amaranth}{rgb}{0.9, 0.17, 0.31}
\definecolor{amethyst}{rgb}{0.6, 0.4, 0.8}
\definecolor{atomictangerine}{rgb}{1.0, 0.6, 0.4}
\definecolor{babyblueeyes}{rgb}{0.63, 0.79, 0.95}
\definecolor{gray(x11gray)}{rgb}{0.75, 0.75, 0.75}
\definecolor{bananamania}{rgb}{0.98, 0.91, 0.71}	
\definecolor{ballblue}{rgb}{0.13, 0.67, 0.8}
\definecolor{azure(colorwheel)}{rgb}{0.0, 0.5, 1.0}	
\definecolor{ceruleanblue}{rgb}{0.16, 0.32, 0.75}
\definecolor{lightsalmonpink}{rgb}{1.0, 0.6, 0.6}
\definecolor{palepink}{rgb}{0.98, 0.85, 0.87}
\definecolor{lightcarminepink}{rgb}{0.9, 0.4, 0.38}
\definecolor{lightcornflowerblue}{rgb}{0.6, 0.81, 0.93}
\definecolor{carolinablue}{rgb}{0.6, 0.73, 0.89}
   % 使用自己維護的定義檔
%-----------------------------------------------------------------------------------------------------------------------
% 文章開始
%\title{ \LaTeX  {\MB 的表格製作}}
%\author{{\SM 游筑鈞}}
%\date{{\TT \today}} 	


%\begin{document}
%\maketitle
%\fontsize{12}{22pt}\selectfont
\chapter{\CB{表格製作}}
表格是一般人覺得比較困難,但卻是很重要的部份,需要我們多花點時間研究。\LaTeX 的表格,因為是抽象邏輯的思考方式來製作表格,對一般使用者而言比較不容易轉換成直觀印象。和 WORD 相比確實比較複雜,但是畫出來比較整齊,尤其是在做學術性表格的時候,\LaTeX 比較專業。這章節就來探討一些關於製作表格的小技巧,以及常用的套件介紹,以此來熟悉\LaTeX 的表格環境。
\\
\section{\MB{表格環境}}

\subsection{tabbing 環境}
這是 \LaTeX 裡頭最基本的表格形式,除非自行另外定義、繪製,他並沒有方便可用的線條指令來做區隔。完全使用空間、位置的配置來顯示表格內容。\\
在 tabbing 環境中,第一個列(row)是以 $\backslash =$ 來標示 Tab 寬度來區隔欄位,這個寬度是由欄位裡頭的字串寬度所決定的。後續的每個欄位是由 $\backslash > $這個符號來區隔,每列尾要自行加上 $\backslash \backslash$ 來換行,最後一行可以不必使用 $\backslash \backslash$ 換行。如表\ref{t1}\\

\begin{table}[h]
    \caption{tabbing 環境中製作表格}\label{t1}
    \smallskip
\begin{tabbing}
column1 \= column2 \= column3 \\
item1   \> item2   \> item3   \\
itemA   \> itemB   \> itemC
\end{tabbing}
\end{table}

對於欄位內文字的控制,tabbing 較不完備,因此表格主要還是以 tabular 環境較為常用。但 tabbing 環境的好處是,他不見得一定要用於表格的排版,例如他也可以表現如條列環境般的另一種表現方式,而且他可以跨頁排版。

\subsection{tabular 環境}
這大概是最常使用的表格形式,可以很方便的畫線框。這種表格, \LaTeX 是把整個表格當成一個單位來處理,就像字母一樣,因此他在版面的安排上是和一般的字母一般的處理,所以這種表格不經特殊處理,無法被分割成兩個部份來跨頁。\\
和 tabbing 環境的不同,除了可以有線條之外(tabular 環境,分隔欄位的符號是 $\&$,而且一定要指定欄內文字的置放位置,欄內文字超出指定的寬度時,會自動折行,還有許多其他更細節的調整。因此接下來的介紹將以tabular 環境為主。

\section{\MB{基本表格介紹(Tabular環境)}}
\subsection{表格基本架構}
\begin{center}
\begin{figure}[htb] 
    \includegraphics{\imgdir{01.PNG}}
\end{figure}
\begin{tabular}[b]{|l|r|c|}
\hline
    column1 & column2 & column3 \\\hline
    item1   & item2   & item3 \\\hline
    itemA   & itemB   & itemC \\\hline
\end{tabular}

\end{center}


其中 t 表示 top,也可以是 b 表示 bottom,或 c 代表 center,這要在前後有文字相並排的時候才會顯現作用,因為\LaTeX 會把整個 tabular 表格當成一個字母單位,所以可以和其他文字、圖表並排排版。這些參數的意思是和同行文字的對齊方式,top 是表格頂端和前後文字對齊,bottom 則是表格底部和前後文字對齊,center 則是和表格中央對齊。

換行的方式和 tabbing 環境一樣,其中的 $\backslash$hline 是畫一條橫線的意思,連續兩個 $\backslash$hline$\backslash$hline 會畫雙橫線,他本身會自動換行,因此不必加上換行符號。其中$\backslash$begin $\lbrace$tabular$\rbrace$ $[$t$]$ $\lbrace$| l | r | c |$\rbrace$  的 lll 是在指定各欄位內容在小方框內的置放位置,l 表示靠左(left),r 表示靠右(right),c 表示置中(center)。在 $\lbrace$| l | r | c |$\rbrace$ 中加上 bar(|)會畫縱線而兩個 bar 就會畫雙縱線。

\subsection{表格欄位調整}


\begin{table}[h]
    \centering
    \caption{欄位調整示範}\label{e1}
    \smallskip
\begin{center}
\extrarowheight=2pt
\doublerulesep=3pt
\renewcommand{\arraystretch}{1.2} % 將表格行間距加大為原來的 1.2 倍
\tabcolsep=12pt                   % 調整欄間距為 24pt               
\begin{tabular}[b]{|lcc|}   \hline
\rowcolor[gray]{0.8}
Specific Heats & $c$ (J/kg$\cdot$K) & $C$ (J/mol$\cdot$K) \\\hline\hline
Aluminum     & 900  & 24.3 \\\cline{2-3} 
Copper       & 385  & 24.4 \\\cline{2-3} 
Gold         & 130  & 25.6 \\\cline{2-3} 
Steel/Iron   & 450  & 25.0 \\\cline{2-3} 
Lead         & 130  & 26.8 \\\cline{2-3} 
Mercury      & 140  & 28.0 \\\cline{2-3} 
Water        & 4190 & 75.4 \\\hline
\end{tabular}
\end{center}
\end{table}

\fbox{\color{lightcarminepink}array 巨集套件指令功能} 
\begin{itemize}
\item $\backslash$doublerulesep=單位長度 連線兩直線 (||) 或兩橫線 ($\backslash$hline$\backslash$hline) 之間距,預設為2pt。
\item $\backslash$extrarowheight=單位長度 將行高增加幾pt
\item $\backslash$cline$\lbrace$a-b$\rbrace$ 畫某部份欄位的橫線,其中的 a-b 指的就是要畫線的欄位數。
\item $\backslash$arrayrulewidth=單位長度 調整表格線條的粗細,預設值是 0.4pt。但要注意的是要在進入 tabular 環境之前設定好。
\item $\backslash$tabcolsep=單位長度 調整兩欄位的左右間距,預設是 6pt。請注意這個值是實際兩欄位間距值的一半。在進入 tabular 環境之前設定好。
\item $\backslash$doublerulesep=單位長度 調整畫雙線時,這兩線間的間距,預設值是 2pt。在進入 tabular 環境之前設定好。
\item $\backslash$arraystretch 調整表格的上下行距。請注意,這要由$\backslash$renewcommand 來重設,因為在 \LaTeX 定義出的一個常數值,而這個$\backslash$arraystretch 只是這些常數值的倍數,我們要重新改變他才能改變預設倍數。
\end{itemize}
\bigskip

以上之指令同時更動所有欄位之間距。 如果我們只想要更動某兩欄位之間距也有其他方式,如\ref{e2}。\\

\begin{table}[h]
    \centering
    \caption{特定欄位調整示範}\label{e2}
    \smallskip
\begin{tabular}{p{1cm}p{3cm}p{2.5cm}p{1.5cm}p{0.5cm}}
\hline
\rowcolor[gray]{0.8}
Gene name   & GeneNo. & length &    size  (cm) \\
\hline
        001  & 01g009860.2   & 819             & 272   \\
        002  & 01g021730.2   & 798             & 265    \\
        003  & 01g094490.2   & 630             & 209    \\
        004  & 01g102740.2   & 1242            & 413     \\
        005  & 01g104900.2   & 597             & 198      \\
        006  & 02g036430.1   & 1698            & 565       \\
        \hline       
\end{tabular}
\end{table}

\bigskip
\begin{itemize}
\item p$\lbrace$寬度$\rbrace$ 
這裡的 p 指的是段落(paragraph)。通常用於一個小段落的文字,指定了寬度後裡頭的文字會自動折行,而且這個段落的頂端會和其他欄位的頂端對齊。
\item @$\lbrace$文字、符號或指令 $\rbrace$ 
這可以作用在本欄的各個列,讓他們都出現某個文字、符號或都在某個指令的作用下。這個指令另外會同時將欄位間距縮成 0,置於首尾的話,會有讓橫線和文字切齊的作用(預設不會切齊,橫線兩端會多出欄位間距的部份)。事實上, @$\lbrace$...$\rbrace$ 指令大括號內除了設定欄位間距外, 也可以鍵入任何文字或指令。 排版時, 括號內之文字或指令即填入表格中對應的欄位間隔處, 而且原有之空白自動消除。 
\end{itemize}

\subsection{表格線條變化}
\bigskip


\begin{table}[h]
    \centering
    \caption{線條樣式示範}\label{e3}
    \extrarowheight=3pt
    \smallskip
\begin{tabular}{cc|c;{2pt/2pt}c|c;{4pt/0.5pt}c|c!{\vrule width 1pt}}
\toprule[1pt]

          & P(\%) & R(\%) & F1(\%) & P(\%) & R(\%) & F1(\%) \\\hline 
Baseline1 & 76.84 & 76.84 & 76.84 & 76.84 & 76.84 & 76.84 \\
\cdashline{1-7}[0.5pt/2pt]
Baseline2  & 76.84 & 76.84 & 76.84 & 76.84 & 76.84 & 76.84 \\
\hdashline[4pt/0.5pt]
Baseline3  & 76.84 & 76.84 & 76.84 & 76.84 & 76.84 & 76.84 \\
\hline
{\bf Our approach}  & {\bf 76.84} & {\bf 76.84} & {\bf 76.84} & {\bf 76.84} & {\bf 76.84} & {\bf 76.84} \\
\bottomrule[1pt]\\
\end{tabular}
\end{table}
\bigskip
我們前面曾提到 $\backslash$arraryrulewidth 指令,可以調整線條的粗細,但是這無法各別調整線條,每個在 tabular 表格環境內的線條會調整成一樣的粗細。booktabs 巨集套件可以很方便的去控制每一欄位的線條粗細。arydshln巨集套件甚至還可以變化線條的樣式。如表\ref{e3}\\
\\
\fbox{\color{lightcarminepink}booktabs 巨集套件指令功能} 
\bigskip

使用方法和 tabular 環境差不多但可在指令後加個方括號來指定線條的粗細,不指定的話,toprule 及 bottomrule 都會比中間的其他線條粗一點。
\begin{itemize}
\item $\backslash$toprule[線條粗細]	畫表格頂端的橫線
\item$\backslash$midrule[線條粗細]	畫表格裡頭的橫線
\item$\backslash$bottomrule[線條粗細]	畫表格底部的橫線
\item$\backslash$cmidrule[線條粗細](左右是否去邊){畫線欄位}
\end{itemize}

tabular 指令環境中, 利用 | 指令可畫出垂直線, 但其粗細無法調整。利用 array 巨集套件所提供之 !$\lbrace$...$\rbrace$ 指令, 我們可以畫出任意粗細之垂直線。 垂直線指令 | 或 !$\lbrace$ $\backslash$ vrule width 2pt$\rbrace$ 所畫出之直線由上至下貫穿整個表格。 如果只要在某一橫列中間畫短直線, 可使用 $\backslash$ vline 指令。
\bigskip

\fbox{\color{lightcarminepink}arydshln 巨集套件指令功能} 
\begin{itemize}
\item ;$\lbrace$(dash)pt/(gap)pt$\rbrace$ :在tabular 環境中加入,可以控制直線樣式。
\item $\backslash$hdashline[(dash)pt/(gap)pt]:設定橫線的樣式
\item $\backslash$cdashline$\lbrace$cola-colb$\rbrace$[(dash)pt/(gap)pt]:設定橫線的樣式
\end{itemize}


\subsection{表格顏色設計}

$\backslash$arrayrulecolor$\lbrace$顏色$\rbrace$:指定線條顏色,如表 \ref{e4}

\begin{table}[h]
\centering
\caption{指定線條顏色}\label{e4}
\extrarowheight=3pt
\smallskip
\setlength{\arrayrulewidth}{2pt}
\arrayrulecolor{babyblueeyes}
\begin{tabular}{|l|c|r|}
\arrayrulecolor{electriclavender}\hline
United Kingdom & London & Thames\\
\arrayrulecolor{amethyst}\hline
France & Paris & Seine \\
\arrayrulecolor{atomictangerine}\cline{1-1}
\arrayrulecolor{lightsalmonpink}\cline{2-3}
Russia & Moscow & Moskva \\ \hline
\end{tabular}
\end{table}
\bigskip

$\backslash$rowcolor[色彩模型]$\lbrace$顏色$\rbrace$[左緣突出長度][右緣突出長度]:讓整個橫列著色,如表\ref{e5}\\
\begin{table}[h]
\centering
\caption{讓整個橫列著色}\label{e5}
\extrarowheight=3pt
\smallskip
\setlength{\extrarowheight}{2mm}
\begin{tabular}{|l|c|c|c|c|c|c|c|}
\hline
Sydney & OG4G &Thu Oct 10 &Mon Oct 21 or 28 &11 or 18 days &999\\
\rowcolor[gray]{0.8}
& &Thu Oct 17 &Mon Oct 21 or 28 & 4 or 11 days &999\\
&OG7A &Sun Oct 13 &Mon Oct 21 or 28 & 8 or 15 days &999\\
\rowcolor[gray]{0.8}
& &Sun Oct 20 &Mon Oct 28 & 8 days &999\\
\hline
\end{tabular}
\end{table}
\bigskip

$\backslash$columncolor[色彩模型]$\lbrace$顏色$\rbrace$[左緣突出長度][右緣突出長度]:讓整個欄位著色,如表\ref{e6}\\
\begin{table}[h]
\centering
\caption{讓整個欄位著色}\label{e6}
\extrarowheight=3pt
\smallskip
\setlength{\extrarowheight}{2mm}
\setlength{\extrarowheight}{2mm}
\begin{tabular}{|>{\columncolor{babyblueeyes}}l|c|>{\columncolor{electriclavender}}c|c|>{\columncolor{bananamania}}c|c|c|c|}
\hline
Sydney & OG4G &Thu Oct 10 &Mon Oct 21 or 28 &11 or 18 days &999\\\hline
& &Thu Oct 17 &Mon Oct 21 or 28 & 4 or 11 days &999\\\hline
&OG7A &Sun Oct 13 &Mon Oct 21 or 28 & 8 or 15 days &999\\\hline
& &Sun Oct 20 &Mon Oct 28 & 8 days &999\\\hline
\end{tabular}
\end{table}
\bigskip


$\backslash$doublerulesepcolor$\lbrace$顏色$\rbrace$:指定雙並線內間隔的顏色,如表\ref{e7}\\
\begin{table}[h]
\centering
\caption{指定雙並線內間隔的顏色}\label{e7}
\extrarowheight=3pt
\smallskip
\setlength{\extrarowheight}{2mm}
\setlength{\extrarowheight}{2mm}
\doublerulesep=4pt
\doublerulesepcolor{ballblue}
\begin{tabular}{|l|c|c|c|c|c|c|c|}
\hline\hline
Sydney & OG4G &Thu Oct 10 &Mon Oct 21 or 28 &11 or 18 days &999\\\hline
& &Thu Oct 17 &Mon Oct 21 or 28 & 4 or 11 days &999\\\hline
&OG7A &Sun Oct 13 &Mon Oct 21 or 28 & 8 or 15 days &999\\\hline
& &Sun Oct 20 &Mon Oct 28 & 8 days &999\\
\hline\hline
\end{tabular}
\end{table}
\bigskip

注意此一指令之使用須利用 array 巨集套件所提供之 >$\lbrace$...$\rbrace$ 指令之功能。並在 tabular 指令環境中使用。


\subsection{表格分割、合併}
1. $\backslash$multirow$\lbrace$2$\rbrace$ $\lbrace$*$\rbrace$ $\lbrace$Multi-Row$\rbrace$:跨列功能\\
第一個參數2,表示跨兩列, 第二個指令為合併後水平對齊方式, c 為置中, r 為置右,*表示系統自動調整文字
最後一個參數即是要填入的文字
另外,跨列需注意的是,使用$\backslash$multirow指令的那一列表格,到了要撰寫下一列表格時,被跨列的該爛位,直接留空,不可填字。

2. $\backslash$multicolumn$\lbrace$2$\rbrace$ $\lbrace$c|$\rbrace$ $\lbrace$Multi-Column$\rbrace$:跨行功能\\
第一個參數2,表示跨兩行,第二個參數c|,表示文字置中,並在欄位右邊畫一條直線框,最後一個參數即是要填入的文字
\bigskip
\begin{table}[h]
\centering
\caption{表格跨行、跨列示範1}\label{e8}
\extrarowheight=3pt
\smallskip
\setlength{\extrarowheight}{2mm}
\begin{tabular}{|c|c|c|c|c|}
\hline
\multirow{2}{*}{Multi-Row} &
\multicolumn{2}{c|}{Multi-Column} &
\multicolumn{2}{c|}{\multirow{2}{*}{Multi-Row and Col}} \\
\cline{2-3}
  & column-1 & column-2 & \multicolumn{2}{c|}{} \\
\hline
label-1 & label-2 & label-3 & label-4 & label-5 \\
\hline
\end{tabular}
\end{table}
\bigskip

\begin{table}[h]
\centering
\caption{表格跨行、跨列示範2}\label{e9}
\extrarowheight=3pt
\smallskip
\setlength{\extrarowheight}{2mm}
 \begin{tabular}{ccccccc}
    \toprule
    \multirow{2}{*}{Method}&
    \multicolumn{3}{>{\columncolor{carolinablue}}c}{ A}&\multicolumn{3}{>{\columncolor{lightsalmonpink}}c}{ G}\cr
    \cmidrule(lr){2-4} \cmidrule(lr){5-7}
    &Precision&Recall&F1-Measure&Precision&Recall&F1-Measure\cr
    \midrule
    kNN&0.7324&0.7388&0.7301&0.6371&0.6462&0.6568\cr
    F&0.7321&0.7385&0.7323&0.6363&0.6462&0.6559\cr
    E&0.7321&0.7222&0.7311&0.6243&0.6227&0.6570\cr
    D&0.7654&0.7716&0.7699&0.6695&0.6684&0.6642\cr
    C&0.7435&0.7317&0.7343&0.6386&0.6488&0.6435\cr
    B&0.7667&0.7644&0.7646&0.6609&0.6687&0.6574\cr
   \rowcolor[gray]{0.8}
    A&{\bf 0.8189}&{\bf 0.8139}&{\bf 0.8146}&{\bf 0.6971}&{\bf 0.6904}&{\bf 0.6935}\cr
    \bottomrule
    \end{tabular}
     \end{table}
\subsection{小數點對齊}
原來的 tabular 環境的作法是去增加一個欄位,那個欄位使用 @$\lbrace$.$\rbrace$ 來專門排小數點,這樣一來兩欄的間距會消掉,看起來就像連在一起的數字了,但是如果使用 dcolumn 巨集的話,就可以很有規律的去對齊小數點或逗點。dcolumn 的用法,主要是去取代 tabular 參數中的 lrc 這些參數。\\
如表\ref{ex02}我們再怎麼去排,小數點總是無法對齊。我們只要使用dcolumn 巨集把 tabular 的後面參數改掉就可以讓小數點對齊。如表\ref{ex03}
\begin{table}[h]
    \centering
    \caption{小數點無法對齊}\label{ex02}
    \smallskip
\begin{tabular}{lllll}
\toprule
      & headA & headB & headC & headD \\
\midrule
test1 & 7.879  & 921.661 & 1382.81 & 998.98 \\
test2 & 1.97   & 35.21   & 321.3   & 4791112.11 \\
test3 & 211.97 & 5.2     & 213.629 & 748261594.106 \\
\bottomrule
\end{tabular}
\end{table}\\

\bigskip
\begin{table}[h]
    \centering
    \caption{使用 dcolumn 巨集讓小數點對齊}\label{ex03}
    \smallskip
\begin{tabular}{lD{.}{.}{3}D{.}{.}{3}D{.}{.}{3}D{.}{.}{3}}
\toprule
      & headA & headB & headC & headD \\
\midrule
test1 & 7.879  & 921.661 & 1382.81 & 998.98 \\
test2 & 1.97   & 35.21   & 321.3   & 4791112.11 \\
test3 & 211.97 & 5.2     & 213.629 & 748261594.106 \\
\bottomrule
\end{tabular}
\end{table}

這裡要特別注意的是,在 dcolumn 的效力範圍裡頭,他會自動進入數學模式,裡頭要表現數學式的話,前後不必再加 \$,否則會跳出數學模式。例如 headA 會變成斜體字,這是因為進入了數學模式,要讓他正常的話,就要寫成 \$headA\$ 這樣來跳出數學模式。

  
\subsection{斜線表頭} 
可以使用$\backslash$usepackage$\lbrace$diagbox$\rbrace$套件來製做表頭上的斜線。以便清楚的呈現各個欄位的意思。如表\ref{ex8}

\begin{table}[!htbp]
\centering
\caption{示範斜線表頭}\label{ex8}
\smallskip
\begin{tabular}{|c|c|c|c|}
\hline
\diagbox{甲}{$\alpha_{i,j}$}{乙}&$\beta_1$&$\beta_2$&$\beta_3$\\ 
\hline
$\alpha_1$&-4&0&-8\\
\hline
$\alpha_2$&3&2&4\\
\hline
$\alpha_3$&16&1&-9\\
\hline
$\alpha_4$&-1&1&7\\
\hline
\end{tabular}
\end{table}

  
  
  
\subsection{表格編號與標題}

一般放在文章裡的表格都需要編號(Label)與標題(Caption),方便與內文相呼應。為了讓表格能夠移動,常常會把就是把表格置於 table 環境當中。在裡頭有 $\backslash$caption$\lbrace$ $\rbrace$ 指令可以指定表格的標頭,$\backslash $label$\lbrace$ $\rbrace$指令可以標號。
\bigskip

一般國際上較正式的文件,caption 置放的位置慣例是「表上圖下」,也就是說表格的標題是置於表格上方,圖形則在下方。其中標題前的「表」字,是重新經過定義的。\LaTeX 原來定義的文字是英文 Table,在中文的環境當然不妥,利用$\backslash$renewcommand$\lbrace$ $\backslash$tablename$\rbrace$ $\lbrace$表 $\rbrace$ 可以定義作者自己喜歡的字眼。

\bigskip
\fbox{\color{lightcarminepink}LaTeX 的浮動環境(table環境)} 
\begin{itemize}
\item h(here)	置於下指令處位置
\item t(top)	置於一頁的頂端
\item b(bottom)	置於本頁底部,如空間不夠會置於次頁
\item p(page)	單獨佔一頁,此頁沒有內文的部份
\item $\backslash$suppressfloats[位置]抑制浮動物件置放於本頁的某處,他會出現在次頁
\item ! 置於以上選頁之前,會更強烈要求達到此選項的作用。但對 p 則無作用
\end{itemize}

\begin{table}[H]
    \centering
    \caption{表格編號示範}\label{dou}
    \smallskip
    \extrarowheight=2pt
    \begin{tabular}{lcc}
    \rowcolor{babyblue}
    \hline
    姓名       & 座號    & 成績 \\\hline
    A 同學    & 15       & 80 \\
    B 同學      & 27     & 72 \\
    C 同學     & 35      & 81 \\\hline
    \end{tabular}
\end{table}
\bigskip
表 \ref{dou}中的標題部分也可以加上顏色,甚至是調整行高。

\subsection{表格並排}

常常我們會需要有2個表格並排的情況,所以在\LaTeX 中使用$\backslash$begin $\lbrace$ minipage$\rbrace$的指令來完成,如表\ref{ex5}和表\ref{ex6}。其實這個方法除了能讓表格並排之外還能讓圖片和表格並排。

\bigskip
\begin{minipage}{\textwidth}
        \begin{minipage}[t]{0.45\textwidth}
            \centering
            \makeatletter\def\@captype{table}\makeatother\caption{表格並排示範1}\label{ex5}
            \begin{tabular}{|ccc|}
\hline
\rowcolor{lightcornflowerblue}
n  & L  & $L+n$ \\ \hline
0  & 1  & 1 \\
1  & 3  & 4 \\
2  & 5  & 7 \\
3  & 7  & 10\\
4  & 9  & 13 \\
5  & 11 & 16 \\
\hline
\end{tabular}
        \end{minipage}
        \begin{minipage}[t]{0.45\textwidth}
        \centering
        \makeatletter\def\@captype{table}\makeatother\caption{表格並排示範2}\label{ex6}
            \begin{tabular}{|ccc|}
\hline
\rowcolor{lightcornflowerblue}
n  & L  & $L+n$ \\ \hline
0  & 1  & 1 \\
1  & 3  & 4 \\
2  & 5  & 7 \\
3  & 7  & 10\\
4  & 9  & 13 \\
5  & 11 & 16 \\
\hline
\end{tabular}
\end{minipage}
\end{minipage}


\subsection{大型表格(longtable)}

這可能有兩種情形。一種是很寬的表格,另一種是很長的表格。太寬的表格可考慮旋轉一下,讓他橫放,至於長的表格可以使用 longtable巨集 讓他可以跨頁連續。

\bigskip

用\LaTeX 排版,如果要旋轉文字,圖片,表格等,可以使用rotating巨集來完成。\\

\bigskip
\fbox{\color{lightcarminepink}rotating 巨集套件指令功能} 
\begin{itemize}
\item $\backslash$begin$\lbrace$sideways$\rbrace$將内容旋轉90度$\backslash$end$\lbrace$sideways$\rbrace$
\item $\backslash$begin$\lbrace$turn$\rbrace$ $\lbrace$45$\rbrace$將内容旋轉自定義角度$\backslash$end$\lbrace$turn$\rbrace$
\item $\backslash$begin$\lbrace$rotate$\lbrace$120$\rbrace$將内容旋轉自定義角度,但是旋轉結果並不能保證所需要的空間$\backslash$end$\lbrace$rotate$\rbrace$\\
\end{itemize}

表 \ref{ex07} 是將表格旋轉(採用 rotating 套件),方便做寬型表格時使用。\\
表 \ref{basic_4} 是將表格視為圖片做旋轉(採用  {\A graphicx} 套件),方便做寬型表格時使用。\\
\begin{minipage}{\textwidth}
        \begin{minipage}[t]{0.45\textwidth}
\begin{table}[H]
 \centering
 \caption{寬表格旋轉}\label{ex07}
 \smallskip
 \extrarowheight=2pt
\begin{sideways}
\begin{tabular}{cccccc} \toprule
Models  &  $\hat c$  &  $\hat\alpha$  &  $\hat\beta_0$  &  $\hat\beta_1$  &  $\hat\beta_2$  \\ \hline
model  & 30.6302  & 0.4127  & 9.4257  & - & -  \\
model  & 12.4089  & 0.5169  & 18.6986  & -6.6157  & - \\
model  & 14.8586  & 0.4991  & 19.5421  & -7.0717  & 0.2183 \\
model  & 3.06302  & 0.41266  & 0.11725  & - & - \\
model  & 1.24089  & 0.51691  & 0.83605  & -0.66157  & -  \\
\bottomrule
\end{tabular}
\end{sideways}
\end{table}
\end{minipage}
\begin{minipage}[t]{0.45\textwidth}
\begin{table}[H]
\begin{center}
\caption{旋轉表格}\label{basic_4}
\bigskip
\extrarowheight=2pt
\rotatebox[origin=c]{90}{
\begin{tabular}{|l|ccccc|}
\hline
Source	& Df	& SS			& MS		& F value	& Pr$>$ F \\\hline
model	& 2 	& 543.6 	& 271.8 	& 16.08 	& 0.0004 \\
Error		& 12 & 202.8 	& 16.9 		&{}  			&{} \\\hline
Total		& 14 & 746.4 	&{}  			&{}  			&{} \\
\hline
\end{tabular}}\hspace{10pt}
\end{center}
\end{table}
\end{minipage}
\end{minipage}


\bigskip
如果表格長度超過版面高度,可以使用longtable巨集套件,原來之表格自動拆為兩部分以上,分別排版於兩頁或是多頁之中,如表   \ref{grid_mlmmh} 所示。請注意不同頁面在表格斷續處的文字處理。
\newpage


\begin{center}
\begin{longtable}{|l|l|l|}
\caption[Feasible triples for a highly variable Grid]{Feasible triples for
highly variable Grid, MLMMH .} \label{grid_mlmmh} \\
\hline \multicolumn{1}{|c|}{\textbf{Time (s)}} & \multicolumn{1}{c|}{\textbf{Triple chosen}} & \multicolumn{1}{c|}{\textbf{Other feasible triples}} \\ \hline
\endfirsthead
\multicolumn{3}{c}%
{{\bfseries \tablename\ \thetable{} -- continued from previous page}} \\
\hline \multicolumn{1}{|c|}{\textbf{Time (s)}} &
\multicolumn{1}{c|}{\textbf{Triple chosen}} &
\multicolumn{1}{c|}{\textbf{Other feasible triples}} \\ \hline
\endhead
\hline \multicolumn{3}{|r|}{{Continued on next page}} \\ \hline
\endfoot
\hline \hline
\endlastfoot
0 & (1, 11, 13725) & (1, 12, 10980), (1, 13, 8235), (2, 2, 0), (3, 1, 0) \\
2745 & (1, 12, 10980) & (1, 13, 8235), (2, 2, 0), (2, 3, 0), (3, 1, 0) \\
5490 & (1, 12, 13725) & (2, 2, 2745), (2, 3, 0), (3, 1, 0) \\
8235 & (1, 12, 16470) & (1, 13, 13725), (2, 2, 2745), (2, 3, 0), (3, 1, 0) \\
10980 & (1,12, 16470) & (1, 13, 13725), (2, 2, 2745), (2, 3, 0), (3, 1, 0) \\
13725 & (1, 12, 16470) & (1, 13, 13725), (2, 2, 2745), (2, 3, 0), (3, 1, 0) \\
16470 & (1, 13, 16470) & (2, 2, 2745), (2, 3, 0), (3, 1, 0) \\
19215 & (1, 12, 16470) & (1, 13, 13725), (2, 2, 2745), (2, 3, 0), (3, 1, 0) \\
21960 & (1, 12, 16470) & (1, 13, 13725), (2, 2, 2745), (2, 3, 0), (3, 1, 0) \\
24705 & (1, 12, 16470) & (1, 13, 13725), (2, 2, 2745), (2, 3, 0), (3, 1, 0) \\
27450 & (1, 12, 16470) & (1, 13, 13725), (2, 2, 2745), (2, 3, 0), (3, 1, 0) \\
30195 & (2, 2, 2745) & (2, 3, 0), (3, 1, 0) \\
32940 & (1, 13, 16470) & (2, 2, 2745), (2, 3, 0), (3, 1, 0) \\
35685 & (1, 13, 13725) & (2, 2, 2745), (2, 3, 0), (3, 1, 0) \\
38430 & (1, 13, 10980) & (2, 2, 2745), (2, 3, 0), (3, 1, 0) \\
41175 & (1, 12, 13725) & (1, 13, 10980), (2, 2, 2745), (2, 3, 0), (3, 1, 0) \\
43920 & (1, 13, 10980) & (2, 2, 2745), (2, 3, 0), (3, 1, 0) \\
46665 & (2, 2, 2745) & (2, 3, 0), (3, 1, 0) \\
49410 & (2, 2, 2745) & (2, 3, 0), (3, 1, 0) \\
52155 & (1, 12, 16470) & (1, 13, 13725), (2, 2, 2745), (2, 3, 0), (3, 1, 0) \\
54900 & (1, 13, 13725) & (2, 2, 2745), (2, 3, 0), (3, 1, 0) \\
57645 & (1, 13, 13725) & (2, 2, 2745), (2, 3, 0), (3, 1, 0) \\
60390 & (1, 12, 13725) & (2, 2, 2745), (2, 3, 0), (3, 1, 0) \\
63135 & (1, 13, 16470) & (2, 2, 2745), (2, 3, 0), (3, 1, 0) \\
65880 & (1, 13, 16470) & (2, 2, 2745), (2, 3, 0), (3, 1, 0) \\
68625 & (2, 2, 2745) & (2, 3, 0), (3, 1, 0) \\
71370 & (1, 13, 13725) & (2, 2, 2745), (2, 3, 0), (3, 1, 0) \\
74115 & (1, 12, 13725) & (2, 2, 2745), (2, 3, 0), (3, 1, 0) \\
76860 & (1, 13, 13725) & (2, 2, 2745), (2, 3, 0), (3, 1, 0) \\
79605 & (1, 13, 13725)& (2, 2, 2745), (2, 3, 0), (3, 1, 0) \\
82350 & (1, 12, 13725) & (2, 2, 2745), (2, 3, 0), (3, 1, 0) \\
104310 & (1, 13, 16470) & (2, 2, 2745), (2, 3, 0), (3, 1, 0) \\
107055 & (1, 13, 13725) & (2, 2, 2745), (2, 3, 0), (3, 1, 0) \\
109800 & (1, 13, 13725) & (2, 2, 2745), (2, 3, 0), (3, 1, 0) \\
112545 & (1, 12, 16470) & (1, 13, 13725), (2, 2, 2745), (2, 3, 0), (3, 1, 0) \\
115290 & (1, 13, 16470) & (2, 2, 2745), (2, 3, 0), (3, 1, 0) \\
118035 & (1, 13, 13725) & (2, 2, 2745), (2, 3, 0), (3, 1, 0) \\
120780 & (1, 13, 16470) & (2, 2, 2745), (2, 3, 0), (3, 1, 0) \\
123525 & (1, 13, 13725) & (2, 2, 2745), (2, 3, 0), (3, 1, 0) \\
126270 & (1, 12, 16470) & (1, 13, 13725), (2, 2, 2745), (2, 3, 0), (3, 1, 0) \\
129015 & (2, 2, 2745) & (2, 3, 0), (3, 1, 0) \\
131760 & (2, 2, 2745) & (2, 3, 0), (3, 1, 0) \\
134505 & (1, 13, 16470) & (2, 2, 2745), (2, 3, 0), (3, 1, 0) \\
137250 & (1, 13, 13725) & (2, 2, 2745), (2, 3, 0), (3, 1, 0) \\
139995 & (2, 2, 2745) & (2, 3, 0), (3, 1, 0) \\
142740 & (2, 2, 2745) & (2, 3, 0), (3, 1, 0) \\
145485 & (1, 12, 16470) & (1, 13, 13725), (2, 2, 2745), (2, 3, 0), (3, 1, 0) \\
148230 & (2, 2, 2745) & (2, 3, 0), (3, 1, 0) \\
150975 & (1, 13, 16470) & (2, 2, 2745), (2, 3, 0), (3, 1, 0) \\
153720 & (1, 12, 13725) & (2, 2, 2745), (2, 3, 0), (3, 1, 0) \\
156465 & (1, 13, 13725) & (2, 2, 2745), (2, 3, 0), (3, 1, 0) \\
159210 & (1, 13, 13725) & (2, 2, 2745), (2, 3, 0), (3, 1, 0) \\
161955 & (1, 13, 16470) & (2, 2, 2745), (2, 3, 0), (3, 1, 0) \\
164700 & (1, 13, 13725) & (2, 2, 2745), (2, 3, 0), (3, 1, 0) \\
\end{longtable}
\end{center}
\section*{\MB{結論}}
藉由這次練習之後,已經大致熟悉製作表格的過程和概念,其實如何做出一個美觀又易懂的表格真的不容易。需要多次的練習與嘗試才能掌握技巧。

%\end{document}
%% These two lines must be incldued to open file under UTF-8
% !TEX TS-program = xelatex								
% !TEX encoding = UTF-8

\documentclass[12pt, a4paper]{article} 		% use larger type; default would be 10pt
\usepackage{fontspec} 				% Font selection for XeLaTeX; see fontspec.pdf for documentation. 
%\usepackage[BoldFont, SlantFont]{xeCJK}% 中文使用 XeCJK,並模擬粗體與斜體(即可以用 \textbf{ } \textit{ })
\usepackage{xeCJK}							% 中文使用 XeCJK,但利用 \setCJKmainfont 定義粗體與斜體的字型
\defaultfontfeatures{Mapping=tex-text} 		% to support TeX conventions like ``---''
\usepackage{xunicode} 		% Unicode support for LaTeX character names (accents, European chars, etc)
\usepackage{xltxtra} 						% Extra customizations for XeLaTeX
\usepackage{amsmath, amssymb}
\usepackage{enumerate}
\usepackage{graphicx, subfig, float} 		% support the \includegraphics command and options
\usepackage{array, booktabs}
\usepackage{color, xcolor}
\usepackage{longtable}
\usepackage{colortbl}   
\usepackage{multirow} 
\usepackage{multicol} 
\usepackage{arydshln}  
\usepackage{dcolumn} 
\usepackage{rotating} 
\usepackage{diagbox} 
\usepackage{wrapfig} %文繞圖
\usepackage{overpic}%圖片上加文字
%\usepackage{natbib}
\usepackage[sort&compress,square,comma,authoryear]{natbib}
%.............................................表格標題註解之巨集套件
\usepackage[parfill]{parskip} % Activate to begin paragraphs with an empty line rather than an indent
%\usepackage{geometry} % See geometry.pdf to learn the layout options. There are lots.
\usepackage[left=1.5in,right=1in,top=1in,bottom=1in]{geometry} 

%-----------------------------------------------------------------------------------------------------------------------
%  主字型設定
\setCJKmainfont							% 設定中文內文字型
	[
		BoldFont=微軟正黑體				% 定義粗體的字型(依使用的電腦安裝的字型而定)
	]
	{新細明體}							% 設定中文內文字型
\setmainfont{Times New Roman}			% 設定英文內文字型
\setsansfont{Arial}						% used with {\sffamily ...}
%\setsansfont[Scale=MatchLowercase,Mapping=tex-text]{Gill Sans}
\setmonofont{Courier New}				% used with {\ttfamily ...}
%\setmonofont[Scale=MatchLowercase]{Andale Mono}
% 其他字型(隨使用的電腦安裝的字型不同,用註解的方式調整(打開或關閉))
% 英文字型
\newfontfamily{\E}{Cambria}				
\newfontfamily{\A}{Arial}
\newfontfamily{\C}[Scale=0.9]{Cambria}
%\newfontfamily{\T}{Times New Roman}
\newfontfamily{\TT}[Scale=0.8]{Times New Roman}

% 中文字型
\newCJKfontfamily{\MB}{微軟正黑體}			% 適用在 Mac 與 Win
\newCJKfontfamily{\SM}[Scale=0.8]{新細明體}	% 縮小版新細明體
\newCJKfontfamily{\K}{標楷體} 
               % Windows 下的標楷體
% 以下為自行安裝的字型:CwTex 組合
\newCJKfontfamily{\CF}{cwTeX Q Fangsong Medium}	% CwTex 仿宋體
\newCJKfontfamily{\CB}{cwTeX Q Hei Bold}			% CwTex 粗黑體
\newCJKfontfamily{\CK}{cwTeX Q Kai Medium}   		% CwTex 楷體
\newCJKfontfamily{\CM}{cwTeX Q Ming Medium}		% CwTex 明體
\newCJKfontfamily{\CR}{cwTeX Q Yuan Medium}		% CwTex 圓體
%-----------------------------------------------------------------------------------------------------------------------
\XeTeXlinebreaklocale "zh"             %這兩行一定要加,中文才能自動換行
\XeTeXlinebreakskip = 0pt plus 1pt     %這兩行一定要加,中文才能自動換行
%-----------------------------------------------------------------------------------------------------------------------
\newcommand{\cw}{\texttt{cw}\kern-.6pt\TeX}	% 這是 cwTex 的 logo 文字
\newcommand{\imgdir}{../images/}				% 設定圖檔的位置
\renewcommand{\tablename}{表}				% 改變表格標號文字為中文的「表」(預設為 Table)
\renewcommand{\figurename}{圖}				% 改變圖片標號文字為中文的「圖」(預設為 Figure)


%------------------------------------------------------------------------------------------------------
%  計數器
\usepackage{amsthm}							% theroemstyle 需要使用的套件
\theoremstyle{plain}                       %樣式
\newtheorem{de}{Definition定義}[section]    %definition獨立編號 %跟section走憶起編號
\newtheorem{thm}{{\MB 定理}}[section]		%theorem 獨立編號,取中文名稱並給予不同字型
\newtheorem{lemma}[thm]{Lemma}				%lemma 與 theorem 共用編號 %跟thm走憶起編號
\newtheorem{ex}{{\E Example}}			 %example 獨立編號,不編入小節數字,走流水號。也換個字型。
\newcounter{e}
\newtheorem{cor}{Corollary}[section]		%not used here
\newtheorem{exercise}{EXERCISE}				%not used here
\newtheorem{re}{\emph{Result}}[section]	%not used here
\newtheorem{axiom}{AXIOM}					%not used here
\renewcommand{\proofname}{\bf{Proof}}		%not used here
\bibliographystyle{plain}
%-----------------------------------------------------------------------------------------------------------------------
% 設定顏色
\definecolor{slight}{gray}{0.9}				
\definecolor{airforceblue}{rgb}{0.36, 0.54, 0.66} % color Table: http://latexcolor.com
\definecolor{arylideyellow}{rgb}{0.91, 0.84, 0.42}
\definecolor{babyblue}{rgb}{0.54, 0.81, 0.94}
\definecolor{cadmiumred}{rgb}{0.89, 0.0, 0.13}
\definecolor{coolblack}{rgb}{0.0, 0.18, 0.39}
\definecolor{cottoncandy}{rgb}{1.0, 0.74, 0.85}
\definecolor{desertsand}{rgb}{0.93, 0.79, 0.69}
\definecolor{electriclavender}{rgb}{0.96, 0.73, 1.0}
\definecolor{lightsalmonpink}{rgb}{1.0, 0.6, 0.6}
\definecolor{amaranth}{rgb}{0.9, 0.17, 0.31}
\definecolor{amethyst}{rgb}{0.6, 0.4, 0.8}
\definecolor{atomictangerine}{rgb}{1.0, 0.6, 0.4}
\definecolor{babyblueeyes}{rgb}{0.63, 0.79, 0.95}
\definecolor{gray(x11gray)}{rgb}{0.75, 0.75, 0.75}
\definecolor{bananamania}{rgb}{0.98, 0.91, 0.71}	
\definecolor{ballblue}{rgb}{0.13, 0.67, 0.8}
\definecolor{azure(colorwheel)}{rgb}{0.0, 0.5, 1.0}	
\definecolor{ceruleanblue}{rgb}{0.16, 0.32, 0.75}
\definecolor{lightsalmonpink}{rgb}{1.0, 0.6, 0.6}
\definecolor{palepink}{rgb}{0.98, 0.85, 0.87}
\definecolor{lightcarminepink}{rgb}{0.9, 0.4, 0.38}
\definecolor{lightcornflowerblue}{rgb}{0.6, 0.81, 0.93}
\definecolor{carolinablue}{rgb}{0.6, 0.73, 0.89}
   % 使用自己維護的定義檔
%-----------------------------------------------------------------------------------------------------------------------
% 文章開始
%\title{ \LaTeX {\MB 外製圖形({\C EPS/JPG/PNG/PDF})的引入與計數器之應用}}
%\author{{\SM 游筑鈞}}
%\date{{\TT \today}} 	        				
%\begin{document}
%\maketitle
%\fontsize{12}{22pt}\selectfont
\chapter{\CB{圖片呈現}}
圖片經常是文件中最生動、最精采的部分,但是用過 Word排 版文件的人都知道,圖片是最不聽話的物件,常常無法依你的要求擺放在你要的位置。因此這裡將會介紹如何用 \LaTeX 來插入圖片並且指定圖片的擺放位置。除了基本的圖片排版之外, \LaTeX 還可以指定幾張圖並排、圖片旋轉等等。圖形檔的型態最常見的有 {\C EPS/JPG/PNG/PDF} 等格式,在這裡將會分別介紹如何插入不同型態的圖檔。在介紹圖片使用的同時搭配 \LaTeX 的計數器和專業的參考文獻功能,一步一步的介紹如何排版一份專業的正式的報告。\LaTeX 提供了許多不同的套件可以使用,接著我們就來一起熟悉如何排版圖片吧 !

\section{\MB{插入圖片}}
插入圖片的方式有好幾種,目前最常用{\A graphicx} 套件。

\subsection{圖片路徑設定}
圖形環境指令$\backslash$begin$\lbrace$figure$\rbrace$可以用來控制圖形開始與結束的位置。預設的圖形檔路徑與文章相同,圖形若不是放置於此,必須指定完整的路徑。

\begin{center}\colorbox{slight}{\begin{tabular}{p{0.9\textwidth}}
	{\A $\backslash$includegraphics\{檔案位置/檔案名稱.jpg\}}
\end{tabular}}\end{center}
\bigskip
如果圖片檔案和此份資料在同一個目錄位置的話,路徑可以省略。
\begin{center}\colorbox{slight}{\begin{tabular}{p{0.9\textwidth}}
	{\A $\backslash$includegraphics\{檔案名稱.jpg\}}
\end{tabular}}\end{center}
當然如果一份文件中引入許多分散在不同目錄的圖檔,相當麻煩,因此將所有檔案都集中到預設的目錄。為避免在指令中放在冗長得完整路徑,一般會在定義區設定一個路徑命令,用來縮短指令所需的長度。本文在定義區設定以下的新命令:
\smallskip
\begin{center}\colorbox{slight}{\begin{tabular}{p{0.9\textwidth}}
	{\A $\backslash$newcommand\{$\backslash$imgdir\}\{images/\}}
\end{tabular}}\end{center}
\smallskip

這個新命令自訂為 {\A $\backslash$imgdir} ,定義了一個與編譯文章路徑相同的子目錄:images,也就是所有圖形檔案放置的目錄。譬如:

\smallskip
\begin{center}\colorbox{slight}{\begin{tabular}{p{0.9\textwidth}}
	{\A $\backslash$includegraphics\{$\backslash$imgdir\{圖片名稱.eps\}\}}
\smallskip\end{tabular}}\end{center}

\subsection{圖片位置與大小}

\textbf{$\backslash$begin$\lbrace$figure$\rbrace$ $[$ H $]$}\\
位置選項變數:h$($here$)$置於現在的位置,t$($top$)$置於本頁上端,b$($bottom$)$置於本頁下端,p$($page$)$自成一頁。H強制要在現在的位置。如果不加選項, 內定值為 $[$tbp$]$。下指令時, 位置選項變數之順序無關緊要, 永遠依照 htbp 之順序尋找適當位置。 \\
\textbf{$\backslash$centering} \\
圖片置中\\
\textbf{$\backslash$includegraphics$[$width=0.7$\backslash$textwidth,angle$=$270$]$ $\lbrace$檔案名稱$\rbrace$}\\
調整圖形之大小: height 圖形高度,totalheight 圖形全高,width 圖形的寬度,angle 圖形旋轉 $($反時鐘方向$)$ 角度,scale 圖形放大 $($或縮小$)$ 之倍數。其中$\backslash$textwidth是以整頁的寬度為調整基準。寬度調整時, 高度也同比例調整。\\
\textbf{$\backslash$ caption$\lbrace$給圖一個名字$\rbrace$} \\
\textbf{$\backslash$label$\lbrace$標籤$\rbrace$} \\
常常在寫文件,應該都需要在載入圖片,除了載入圖片以外,還需要在底下加入標號,好在閱讀的時候可以在文章中指定要用來解釋得圖片,也可以在圖片下方加入簡單的敘述。\\
\textbf{$\backslash$vskip 10pt} \\
指定圖片與下方文字的距離\\
\textbf{$\backslash$end$\lbrace$figure$\rbrace$}

\subsection{插入不同類型的圖檔}
圖(\ref{fig:type1})和圖(\ref{fig:type2})分別是四種不同類型的圖檔,其中可以發現eps檔圖片比較清晰,適合數學圖形的表現。jpg和png檔比較模糊。

\begin{figure}[H]
    \centering
        \subfloat[png圖檔]{
        \includegraphics[width=0.5\textwidth]{\imgdir{p2.PNG}}}
        \subfloat[jpg圖檔]{
        \includegraphics[width=0.5\textwidth]{\imgdir{p2.jpg}}}
    \caption{不同類型的圖檔}
    \label{fig:type1}
\end{figure}
\begin{figure}[H]
    \centering
        \subfloat[eps圖檔]{
        \includegraphics[width=0.5\textwidth]{\imgdir{p2.eps}}}
        \subfloat[pdf圖檔]{
        \includegraphics[width=0.5\textwidth]{\imgdir{p2.pdf}}}
    \caption{不同類型的圖檔}
    \label{fig:type2}
\end{figure}

\subsection{文繞圖}
用wrapfigure套件可以呈現文繞圖的排版方式。\\
\textbf{$\backslash$begin$\lbrace$wrapfigure$\rbrace$ $\lbrace$ l $\rbrace$ $\lbrace$6cm$\rbrace$}\\
設定圖片的位置靠左,並和文字距離6cm,如下圖(\ref{ex01}):
\begin{wrapfigure}{l}{6cm}
\centering
\includegraphics[width=0.4\textwidth]{\imgdir{XeTex.png}}
\label{ex01}
\caption{文繞圖示範}
\end{wrapfigure}

\LaTeX 遵循呈現與內容分離的設計理念,以便作者可以專注於他們正在編寫的內容,而不必同時注視其外觀。在準備\LaTeX 文件時,作者使用章、節、表、圖等簡單的概念指定文件的邏輯結構,並讓LaTeX系統負責這些結構的格式和布局。因此,它鼓勵從內容中分離布局,同時仍然允許在需要時進行手動排版調整。這個概念類似於許多文書處理器允許全域定義整個文件的樣式的機制,或使用層疊樣式表來規定HTML的樣式。\LaTeX 系統是一種可以處理排版和彩現的標記式語言。
\subsection{多圖並排}
圖片並排是常用於排版的方法,方便讀者將多張圖一起比較參照。這裡介紹不同的多圖並排方式。通常會使用subfloat套件\\
兩張圖並排:\\
\begin{figure}[H]
    \centering
        \subfloat[立體圖]{
        \includegraphics[width=0.5\textwidth]{\imgdir{p3.eps}}}
        \subfloat[立體圖]{
        \includegraphics[width=0.5\textwidth]{\imgdir{p4.eps}}}
    \caption{兩張圖並排}
    \label{fig:parallel}
\end{figure}

三張圖並排:\\
\begin{figure}[H]
    \centering
        \subfloat[$\alpha$ $>$ $\beta$]{
        \includegraphics[width=0.33\textwidth]{\imgdir{p5.eps}}}
        \subfloat[$\alpha$ $<$ $\beta$]{
        \includegraphics[width=0.33\textwidth]{\imgdir{p6.eps}}}
        \subfloat[$\alpha$ $=$ $\beta$]{
        \includegraphics[width=0.33\textwidth]{\imgdir{P7.eps}}}
    \caption{三張圖並排}
    \label{fig:parallel}
\end{figure}

兩列兩行:\\
\begin{figure}[H]
    \centering
        \subfloat[$\alpha$ $>$ $\beta$]{
        \includegraphics[width=0.5\textwidth]{\imgdir{p5.eps}}}
        \subfloat[$\alpha$ $<$ $\beta$]{
        \includegraphics[width=0.5\textwidth]{\imgdir{p6.eps}}}
        \quad
        \subfloat[$\alpha$ $=$ $\beta$]{
        \includegraphics[width=0.5\textwidth]{\imgdir{P7.eps}}}
        \subfloat[all]{
        \includegraphics[width=0.5\textwidth]{\imgdir{p8.eps}}}
    \caption{四張圖並排}
    \label{fig:parallel}
\end{figure}
\subsection{圖片上加文字}
用 overpic套件可以在圖片上加入文字或公式。\\
\begin{overpic}[scale=0.8]{\imgdir{p8.eps}}
    \centering
    %\caption{圖片加入文字}
    \label{addtext}
    \put(40,52){\huge \color{slight}{\bf \LaTeX}}
    \put(40,42){\huge \color{slight}{\bf Graphics}}
\end{overpic}
\chapter{\CB{計數器}}
\section{ 計數器的應用}
數理方面的文章或書籍常會使用到定理定義,或類似需要給予編號的一段文字。這些編號的管理有些是順序排列,有些則隨章節排列,有些一起編號,有些分開。這類的編號都由指令 {\C $\backslash$newtheorem} 處理。接著我們就由數統課本裡的例子來示範計數器的使用。\\  

$\backslash$newtheorem 的語法如下:\\
\begin{center}\colorbox{slight}{\begin{tabular}{l}
$\backslash$newtheorem\{env\_name\}\{caption\}[within]\\
$\backslash$newtheorem\{env\_name\}[numbered\_like]\{caption\}\\
\end{tabular}}\end{center}
以下示範說明:\\
\begin{center}\colorbox{slight}{\begin{tabular}{l}

\textbf{$\backslash$theoremstyle\{plain\}} 選擇樣式\\
\textbf{$\backslash$newtheorem\{de\}\{Definition定義\}[section]}\\
definition隨section走,一起編號\\
\textbf{$\backslash$newtheorem\{thm\}\{\{$\backslash$MB theorem定理\}\}[Definition定義]}\\
theorem定理隨section走,一起編號\\
\textbf{$\backslash$newtheorem\{ex\}\{\{$\backslash$E Example\}\}}\\
獨立編號,不編入小節數字,走流水號。\\
\end{tabular}}\end{center}
\section{ \MB{圖片與計數器之實際應用--Random Variables}}
藉由上學期MATLAB作業中製作的圖片搭配數統課本,進行以下得練習。

\subsection{Introduction}

{\color{slight}\rule{\textwidth}{0.2pt}}
\begin{de} 
A \textbf{random variable} is a mappimg\\
 \begin{center}
 X:$\Omega$ $\longleftrightarrow$ $\Re$\\
that assigns a real number X($\omega$)to each outcome.\\
\end{center} 
{\color{slight}\rule{\textwidth}{0.2pt}}
\end{de}

\begin{ex} %
Flip a coin ten times. Let X($\omega$)be the number of heads in the sequence $\omega$.For example, if $\omega$=HHTHHTHHTT,then X($\omega$)=6.
\end{ex}

{\color{slight}\rule{\textwidth}{0.2pt}}
\begin{de} 
The \textbf{cumulative distribution function},or CDF,is the function \\
$F_X$:$\Re$ $\longleftrightarrow$[0,1]defined by\\
 \begin{center}
 $F_X(x)$= : P(X$\leqslant$x)\\
\end{center} 
{\color{slight}\rule{\textwidth}{0.2pt}}
\end{de}

\begin{thm}
Let X have CDF F and Y have CDF G.If F(x)=G(x) for all x,then P(X$\in$A)=P(Y$\in$A) for all A.
\end{thm}

\begin{thm}
A function F mapping the real line to [0,1] is a CDF for some probability P if and only if F satisfies the following three conditions:
\begin{enumerate}
\item F is non-decreasing : $x_1$ < $x_2$ implies that F($x_1$)<F($x_2$).
\item F is normalized:\\
$$\lim_{x \rightarrow -\infty}F(x)=0 \qquad \lim_{x \rightarrow \infty}F(x)=1$$
\item F is right-continuous: F(x) = F($x^+$) for all x.
\end{enumerate}
\end{thm}


{\color{slight}\rule{\textwidth}{0.2pt}}
\begin{de} 
X is \textbf{discrete} if it takes countably many values $\lbrace$ $x_1$,$x_2$,...$\rbrace$. We define the \textbf{probability function} or \textbf{probability mass function} for X by $f_X(x)$=P(X=x) \\
{\color{slight}\rule{\textwidth}{0.2pt}}
\end{de}

\begin{de} 
A random variable X is \textbf{continous} if there exists a function $f_X$ such that $f_X(x)$ $\ge$ 0  for all x,$\int_{-\infty}^{\infty}$ $f_X(x)$dx=1 and for every a $\le$b ,
\begin{center}
$$ P(a<X<b)=\int_{a}^{b}f_X(x)dx$$
\end{center}
The function $f_X$ is called the \textbf{probability density function}(PDF). We have that 
\begin{center}
$$ F_X(x)=\int_{-\infty}^{x}f_X(t)dt$$
\end{center}
and  $f_X(x)$=$F_X(x)$ at all points x and which $F_X$ is differentiable.\\
{\color{slight}\rule{\textwidth}{0.2pt}}
\end{de}


\begin{ex} %
Suppose that X has PDF\\
$$
f_X(x) = \left\{\begin{array}{ll}
                 1, & \mbox{for 0 $\le$ x $\le$ 1} \\  
                 0, & \mbox{otherwise.} \\  
                \end{array} \right.
$$
The CDF is given by
$$
F_X(x) = \left\{\begin{array}{ll}
                 0, & \mbox{  $x<$ 0} \\  
                 x, & \mbox{ 0 $\le$ x $\le$ 1} \\ 
                 1, & \mbox{  $x>$ 1}\\  
                \end{array} \right.
$$
\end{ex}

\subsection{Some Important Discete Random Variables}
\setcounter{ex}{1}
{\color{slight}\rule{\textwidth}{0.2pt}}
\theex\;\; THE DISCRETE UNIFORM DISTRIBUTION.\\
Let k>1 be a given integer.Suppose that X has probability mass function given by\\
$$
f_X(x) = \left\{\begin{array}{ll}
                 1/k & \mbox{for $x$=1,...,k} \\  
                 0 & \mbox{otherwise.} \\  
                \end{array} \right.
$$
We say that X has a uniform distribution on $\lbrace$1,2,.....,k$\rbrace$.\\
{\color{slight}\rule{\textwidth}{0.2pt}}
\addtocounter{ex}{1}
\theex\;\; THE BERNOULLI DISTRIBUTION.\\
Let X represent a binary coin flip. Then P(X=1)=p and P(X=0)=1-p for some p $\in$ [0,1].We say that X has a Bernoulli distribution written X$\sim$Bernoulli(p).The probability function is f(x)=$p^{x}$ ${1-p}^{1-x}$ for x$\in$ $\lbrace$ 0,1 $\rbrace$.
\bigskip
\begin{figure}[H]
    \centering
    \includegraphics[width=0.8\textwidth]{\imgdir{ber01.png}}
    \caption{BERNOULLI DISTRIBUTION}
    \label{ber}
\end{figure}

{\color{slight}\rule{\textwidth}{0.2pt}}
\addtocounter{ex}{1}
\theex\;\; THE BINOMIAL DISTRIBUTION.\\
Suppose we have a coin which falls heads up with probability p for some 0$\le$p$\le$1.Flip the cion n times and let X be the number of heads.Assume that the tosses are independent. Let f(x)=P(X=x)be the mass function.It can be shown that\\
$$
f_X(x) = \left\{\begin{array}{ll}
                 \tbinom{n}{x}p^{x}(1-p)^{n-x} & \mbox{for $x$=0,...,n} \\  
                 0 & \mbox{otherwise.} \\  
                \end{array} \right.
$$

\bigskip
\begin{figure}[H]
    \centering
    \includegraphics[width=0.8\textwidth]{\imgdir{bin01.png}}
    \caption{BINOMIAL DISTRIBUTION}
    \label{bin}
\end{figure}

\begin{thm}
A random variable with this mass function is called a Binomial random variable and we write X $\sim$ Binomial(n , p). If $X_1$ $\sim$ Binomial($n_1$ , p) and $X_2$ $\sim$ Binomial($n_2$ , p) then $X_1$+$X_2$ $\sim$ Binomial($n_1+n_2$ , p).
\end{thm}

\bigskip
\begin{figure}[H]
    \centering
    \includegraphics[width=0.8\textwidth]{\imgdir{ber02.png}}
    \caption{BINOMIAL MASS DISTRIBUTION}
    \label{bin2}
\end{figure}


%{\color{slight}\rule{\textwidth}{0.2pt}}
\addtocounter{ex}{1}
\newpage
\theex\;\; THE GEOMETRIC DISTRIBUTION.\\
X has a geometric distribution with parameter p $\in$(0,1),written X $\sim$ Geom(p),if
$$ 
P(X=k)=p(1-p)^{k-1},k \geqslant 1.
$$
We have that 
$$
\sum_{k=1}^{\infty}P(X=k)=p\sum_{k=1}^{\infty}=\frac{p}{1-(1-p)}=1
$$
Think of X as the number of flips needed until the first head when flipping a coin.

\begin{figure}[H]
    \centering
        \subfloat[PDF]{
        \includegraphics[width=0.5\textwidth]{\imgdir{p13.eps}}}
        \subfloat[CDF]{
        \includegraphics[width=0.5\textwidth]{\imgdir{p14.eps}}}
    \caption{GEOMETRIC DISTRIBUTION}
    \label{fig:geo}
\end{figure}

%1112 exp 1516poi 910 chi2
{\color{slight}\rule{\textwidth}{0.2pt}}
\addtocounter{ex}{1}
\theex\;\; THE POISSON DISTRIBUTION.\\
X has a Poisson distribution with parameter $\lambda$,written X$\sim$Poisson($\lambda$)if
$$
f(x)=e^{-\lambda}\frac{\lambda^{x}}{x!}\qquad x \ge 0.
$$
Note that
$$
\sum_{x=0}^{\infty}=e^{-\lambda}\sum_{x=0}^{\infty}\frac{\lambda^{x}}{x!}=1
$$
\begin{figure}[H]
    \centering
        \subfloat[PDF]{
        \includegraphics[width=0.5\textwidth]{\imgdir{p15.eps}}}
        \subfloat[CDF]{
        \includegraphics[width=0.5\textwidth]{\imgdir{p16.eps}}}
    \caption{Poisson distribution}
    \label{fig:poi}
\end{figure}

\begin{thm}
The Poisson is often used as a model for counts of rare events like radioactive decay and traffic accidents.If $X_1$ $\sim$ Poisson($\lambda_1$) and $X_2$ $\sim$ Poisson($\lambda_2$)then $X_1+X_2$ $\sim$ Poisson($\lambda_1+\lambda_2$)
\end{thm}

\bigskip
\begin{figure}[H]
    \centering
    \includegraphics[width=0.8\textwidth]{\imgdir{POISS02.png}}
    \caption{POISSON DISTRIBUTION}
    \label{poi2}
\end{figure}
\newpage
\subsection{Some Important Continuous Random Variables}

{\color{slight}\rule{\textwidth}{0.2pt}}
\addtocounter{ex}{1}
\theex\;\; THE UNIFORM DISTRIBUTION.\\
X has a Uniform(a,b) distribution, written X $\sim$ Uniform(a,b),if

$$
f_X(x) = \left\{\begin{array}{ll}
                 \frac{1}{b-a} & \mbox{for $x$=1,...,k} \\  
                 0 & \mbox{otherwise.} \\  
                \end{array} \right.
$$

Where a < b. The distribution function if

$$
F_X(x) = \left\{\begin{array}{ll}
                 0, & \mbox{  $x<$ a} \\  
   \frac{x-a}{b-a}, & \mbox{ $x$ $\in$ [a,b]} \\ 
                 1, & \mbox{  $x>$ b}\\  
                \end{array} \right.
$$

 
\begin{figure}[H]
    \centering
        \subfloat[PDF]{
        \includegraphics[width=0.5\textwidth]{\imgdir{p17.eps}}}
        \subfloat[CDF]{
        \includegraphics[width=0.5\textwidth]{\imgdir{p18.eps}}}
    \caption{uniform distribution}
    \label{fig:UNI}
\end{figure}

%{\color{slight}\rule{\textwidth}{0.2pt}}
\addtocounter{ex}{1}
\theex\;\; THE NORMAL DISTRIBUTION.\\
X has a Normal(or Gaussian) distribution with parameters $\mu$ and $\sigma$,denoted by X $\sim$ N($\mu$,$\sigma$),if
$$
f(x)=\frac{1}{\sigma\sqrt{2\pi}}e^{-\frac{(x-\mu)^2}{2\sigma^2}}, \;\;  -\infty < x < \infty 
$$

\bigskip
\begin{figure}[H]
    \centering
    \includegraphics[width=0.8\textwidth]{\imgdir{norm01.png}}
    \caption{NORMAL DISTRIBUTION}
    \label{NORM}
\end{figure}

{\color{slight}\rule{\textwidth}{0.2pt}}
\addtocounter{ex}{1}
\theex\;\; THE EXPONENTIAL DISTRIBUTION.\\
X has an Exponential distribution with parameter $\beta$, denoted by X $\sim$ Exp($\beta$),if
$$
f(x)=\frac{1}{\beta}e^{\frac{-x}{\beta}},\qquad x > 0
$$
Where $\beta$ >0. The exponential distribution is used to model the lifetimes of electronic components and the waiting times between rare events.

\begin{figure}[H]
    \centering
        \subfloat[PDF]{
        \includegraphics[width=0.5\textwidth]{\imgdir{p11.eps}}}
        \subfloat[CDF]{
        \includegraphics[width=0.5\textwidth]{\imgdir{p12.eps}}}
    \caption{Exponential distribution}
    \label{fig:exp}
\end{figure}
{\color{slight}\rule{\textwidth}{0.2pt}}
\addtocounter{ex}{1}
\theex\;\; THE GAMMA DISTRIBUTION.\\
For $\alpha$ > 0 , the \textbf{Gamma function} is defined by $\Gamma(\alpha)$=$\int_{0}^{\infty}$ $y^{\alpha-1}$ $e^{-y}$dy.X has a Gamma distribution with parameters $\alpha$ and $\beta$, denoted by X $\sim $ Gamma($\alpha$,$\beta$),if
$$f(x)=\frac{1}{\Gamma(\alpha)\beta^\alpha}x^{\alpha-1}e^{-\frac{x}{\beta}}, \;\; x\geq 0
$$
where $\alpha$,$\beta$ > 0.The exponential distribution is just a Gamma(1,$\beta$) distribution. If $X_i$ $\sim$ Gamma($\alpha_i$,$\beta$)are independent,then $\sum_{i-1}^{n}$ $X_i$ $\sim$ Gamma($\sum_{i=1}^{n}$,$\beta$).

\begin{figure}[H]
    \centering
        \subfloat[$\alpha$ $>$ $\beta$]{
        \includegraphics[width=0.5\textwidth]{\imgdir{p5.eps}}}
        \subfloat[$\alpha$ $<$ $\beta$]{
        \includegraphics[width=0.5\textwidth]{\imgdir{p6.eps}}}
        \quad
        \subfloat[$\alpha$ $=$ $\beta$]{
        \includegraphics[width=0.5\textwidth]{\imgdir{P7.eps}}}
        \subfloat[all]{
        \includegraphics[width=0.5\textwidth]{\imgdir{p8.eps}}}
    \caption{GAMMA DISTRIBUTION}
    \label{fig:GAM}
\end{figure}
\newpage
%{\color{slight}\rule{\textwidth}{0.2pt}}
\addtocounter{ex}{1}
\theex\;\; THE BETA DISTRIBUTION.\\
X has a Beta distribution with parameters $\alpha$ > 0 and $\beta$ > 0,denoted by X $\sim$ Beta($\alpha$,$\beta$),if
$$
\begin{aligned}
f(x;\alpha,\beta)&=\frac{x^{\alpha-1}(1-x)^{\beta-1}}{\int^1_0u^{\alpha-1}(1-u)^{\beta-1}du}\\[4mm]
&=\frac{\Gamma(\alpha+\beta)}{\Gamma(\alpha)\Gamma(\beta)}x^{\alpha-1}(1-x)^{\beta-1}\\[4mm]
&=\frac{1}{B(\alpha,\beta)}x^{\alpha-1}(1-x)^{\beta-1}
\end{aligned}
$$

\bigskip
\begin{figure}[H]
    \centering
    \includegraphics[width=0.8\textwidth]{\imgdir{beta01.png}}
    \caption{BETA DISTRIBUTION}
    \label{beta}
\end{figure}
\newpage
%{\color{slight}\rule{\textwidth}{0.2pt}}
\addtocounter{ex}{1}
\theex\;\; THE t AND CAUCHY DISTRIBUTION.\\
X has a t distribution with $\nu$ degrees of freedom-written X $\sim$ $t_{\nu}$-if
$$
f(x)=\frac{\Gamma(\frac{\nu+1}{2})}{\Gamma(\frac{\nu}{2})}\frac{1}{(1+\frac{x^{2}}{\nu})^{(\nu+1)/2}}
$$

\bigskip
\begin{figure}[H]
    \centering
    \includegraphics[width=0.8\textwidth]{\imgdir{t2.png}}
    \caption{t DISTRIBUTION}
    \label{t}
\end{figure}

{\color{slight}\rule{\textwidth}{0.2pt}}
\addtocounter{ex}{1}
\theex\;\; THE $\chi^{2}$ DISTRIBUTION.\\
X has a $\chi^{2}$ distribution with p degrees of freedom--written X $\sim$ $\chi_{p}^{2}$-if

$$
f(x)=\frac{1}{\Gamma(p/2)2^{p/2}}x^{(p/2)^{-1}e^{-x/2}},\qquad x>0
$$

\begin{figure}[H]
    \centering
        \subfloat[PDF]{
        \includegraphics[width=0.5\textwidth]{\imgdir{p9.eps}}}
        \subfloat[CDF]{
        \includegraphics[width=0.5\textwidth]{\imgdir{p10.eps}}}
    \caption{$\chi^{2}$ distribution}
    \label{fig:chi}
\end{figure}

\chapter{\CB{參考文獻(Bibliography)}}


\section{\MB{thebibliography 環境}}
在進入 thebibliography,編譯後他會自成一個獨立的章節,如果是 article 類別的文稿,他會自動印出 Referrences 的字樣為標題,如果是 report 或 book 類別的文稿,他會印出 Bibliography 的字樣為標題。 

\section{\MB{BibTeX 簡介}}
如果常常有寫論文的機會,整理出自己的一份參考文獻資料庫可以節省許多時間,正常情況下,使用 bibtex 來處理外部文獻檔案的情形,只有引用到的文獻才會印出來,這樣也就不必擔心印出一堆不相關的文獻了。另外一個好處是,這個參考文獻資料庫可以另外獨立維護,所有的文章都用這一份資料庫,這在維護上會很方便,也減少錯誤的機會。BibTeX 本身提供一個外部的 bibtex 工具程式,在 latex 編譯過文稿後,再利用 bibtex 編譯一次文稿,最後再使用 latex 重編譯過。而參考文獻資料庫是按一定的格式寫於bib 檔案裡頭,在文稿中則以bibliogrphy 指令來引入,編譯過程中自然會去參考這個外部考文獻資料庫。

%\section*{section{參考文獻} }
%本文參考自數統課本\cite{LW} 和\cite{CW}
\subsection{製作方式}

由於使用了另一個檔案( bib 檔),編輯的過程要經過幾道程序。編譯前先準備好 bib 檔。對本檔案編譯四次,程序如下:

\begin{enumerate}
\item XeLatex
\item Bib Tex
\item XeLatex
\item XeLatex
\end{enumerate}
%\bibliography{CHUN_ref}
\section*{\MB{結論}}
在使用latex 的過程中,需要花一些時間摸索,雖然在撰寫時較為複雜,但在經過多次的操作後,便會逐漸熟悉其各個功能的使用。無論是在表格的製作、圖片的安排、計數器、參考文獻等等功能,都有更正式更專業的表現。因此多花一些時間來練習latex排版是值得的。最近第一次嘗試將所學到的latex排版技巧用在競賽的資料分析報告書上,可以發現真的和word排版很不一樣。藉由幾次的練習與應用盡量減少錯誤的出現,讓未來在進行文書處理的工作時, 能更加的得心應手。 

%\end{document}

 


%%\documentclass[12pt, a4paper]{article} 
\usepackage{fontspec} % Font selection for XeLaTeX; see fontspec.pdf. 

\usepackage{xeCJK}	% 中文使用 XeCJK,利用 \setCJKmainfont 定義中文內文、粗體與斜體的字型
%\usepackage[BoldFont, SlantFont]{xeCJK}% 中文使用 XeCJK,並模擬粗體與斜體(\textbf{ } \textit{ })
\defaultfontfeatures{Mapping=tex-text} % to support TeX conventions like ``---''
\usepackage{xunicode} % Unicode support for LaTeX character names(accents, European chars, etc)
\usepackage{xltxtra} 						% Extra customizations for XeLaTeX
\usepackage{amsmath, amssymb}
\usepackage{enumerate}
\usepackage{graphicx, subfig, float, wrapfig} % support the \includegraphics command and options
\usepackage[outercaption]{sidecap} %[options]=[outercaption], [innercaption], [leftcaption], [rightcaption]
\usepackage{array, booktabs}
\usepackage{color, xcolor}
\usepackage{longtable}
\usepackage{colortbl}                          				
\usepackage{listings}						%直接將 latex 碼轉換成顯示文字
\usepackage[parfill]{parskip} 				% 新段落前加一空行,不使用縮排
%\usepackage{geometry} % See geometry.pdf to learn the layout options. There are lots.
\usepackage[left=1.5in,right=1in,top=1in,bottom=1in]{geometry} 

%-----------------------------------------------------------------
%  中英文內文字型設定
\setCJKmainfont							% 設定中文內文字型
	[
		BoldFont=Microsoft YaHei	% 定義粗體的字型(依使用的電腦安裝的字型而定)
	]
	{新細明體}							% 設定中文內文字型
\setmainfont{Times New Roman}				% 設定英文內文字型
\setsansfont{Arial}						% 無襯字字型 used with {\sffamily ...}
%\setsansfont[Scale=MatchLowercase,Mapping=tex-text]{Gill Sans}
\setmonofont{Courier New}				% 等寬字型 used with {\ttfamily ...}
%\setmonofont[Scale=MatchLowercase]{Andale Mono}
% 其他字型(隨使用的電腦安裝的字型不同,用註解的方式調整(打開或關閉))
% 英文字型
\newfontfamily{\E}{Calibri}				
\newfontfamily{\A}{Arial}
\newfontfamily{\C}[Scale=0.9]{Calibri}
\newfontfamily{\R}{Times New Roman}
\newfontfamily{\TT}[Scale=0.8]{Times New Roman}
% 中文字型
\newCJKfontfamily{\MB}{Microsoft YaHei}				% 適用在 Mac 與 Win
\newCJKfontfamily{\SM}[Scale=0.8]{新細明體}	% 縮小版新細明體
\newCJKfontfamily{\K}{標楷體}                	% Windows下的標楷體
% 以下為自行安裝的字型:CwTex 組合
%\newCJKfontfamily{\CF}{cwTeX Q Fangsong Medium}	% CwTex 仿宋體
%\newCJKfontfamily{\CB}{cwTeX Q Hei Bold}			% CwTex 粗黑體
%\newCJKfontfamily{\CK}{cwTeX Q Kai Medium}   		% CwTex 楷體
%\newCJKfontfamily{\CM}{cwTeX Q Ming Medium}		% CwTex 明體
%\newCJKfontfamily{\CR}{cwTeX Q Yuan Medium}		% CwTex 圓體
%-----------------------------------------------------------------------------------------------------------------------
\XeTeXlinebreaklocale "zh"             		%這兩行一定要加,中文才能自動換行
\XeTeXlinebreakskip = 0pt plus 1pt     		%這兩行一定要加,中文才能自動換行
%-----------------------------------------------------------------------------------------------------------------------
\newcommand{\cw}{\texttt{cw}\kern-.6pt\TeX}	% 這是 cwTex 的 logo 文字
\newcommand{\imgdir}{../images/}				% 設定圖檔的目錄位置
\renewcommand{\tablename}{表}					% 改變表格標號文字為中文的「表」(預設為 Table)
\renewcommand{\figurename}{圖}				% 改變圖片標號文字為中文的「圖」(預設為 Figure)

% 設定顏色 see color Table: http://latexcolor.com
\definecolor{slight}{gray}{0.9}				
\definecolor{airforceblue}{rgb}{0.36, 0.54, 0.66} 
\definecolor{arylideyellow}{rgb}{0.91, 0.84, 0.42}
\definecolor{babyblue}{rgb}{0.54, 0.81, 0.94}
\definecolor{cadmiumred}{rgb}{0.89, 0.0, 0.13}
\definecolor{coolblack}{rgb}{0.0, 0.18, 0.39}
\definecolor{beaublue}{rgb}{0.74, 0.83, 0.9}
\definecolor{beige}{rgb}{0.96, 0.96, 0.86}
\definecolor{bisque}{rgb}{1.0, 0.89, 0.77}
\definecolor{gray(x11gray)}{rgb}{0.75, 0.75, 0.75}
\definecolor{limegreen}{rgb}{0.2, 0.8, 0.2}
\definecolor{splashedwhite}{rgb}{1.0, 0.99, 1.0}

%---------------------------------------------------------------------
% 映出程式碼 \begin{lstlisting} 的內部設定
\lstset
{	language=[LaTeX]TeX,
    breaklines=true,
    %basicstyle=\tt\scriptsize,
    basicstyle=\tt\normalsize,
    keywordstyle=\color{blue},
    identifierstyle=\color{black},
    commentstyle=\color{limegreen}\itshape,
    stringstyle=\rmfamily,
    showstringspaces=false,
    %backgroundcolor=\color{splashedwhite},
    backgroundcolor=\color{slight},
    frame=single,							%default frame=none 
    rulecolor=\color{gray(x11gray)},
    framerule=0.4pt,							%expand outward 
    framesep=3pt,							%expand outward
    xleftmargin=3.4pt,						%to make the frame fits in the text area. 
    xrightmargin=3.4pt,						%to make the frame fits in the text area. 
    tabsize=2								%default :8 only influence the lstlisting and lstinline.
}
   % 使用自己維護的定義檔
%
%% 以下的設定在測試成功後,可以轉入前置檔(preamble_CJK.tex )一起管理。
%\usepackage{amsthm}						% theroemstyle 需要使用的套件
%\theoremstyle{plain}
%
%\newtheorem{de}{Definition}[section]		%definition獨立編號
%\newtheorem{thm}{{\MB 定理}}[section]		%theorem 獨立編號,取中文名稱並給予不同字型
%\newtheorem{lemma}[thm]{Lemma}			%lemma 與 theorem 共用編號
%\newtheorem{ex}{{\E Example}}				%example 獨立編號,不編入小節數字,走流水號。也換個字型。
%
%\newtheorem{cor}{Corollary}[section]		%not used here
%\newtheorem{exercise}{EXERCISE}			%not used here
%\newtheorem{re}{\emph{Result}}[section]	%not used here
%\newtheorem{axiom}{AXIOM}				%not used here
%\renewcommand{\proofname}{證明}			%not used here
%
%\newcounter{quiz}						% start a simple and new counter
%\setcounter{quiz}{1}						% start to count from 1
%
%%---------------------------------------------------------
%% 文章開始
%\title{ \LaTeX {\MB 定理計數器的使用}}
%\author{{\MB 汪群超}}
%\date{{\TT \today}} 			 
%\begin{document}
%\maketitle
%\fontsize{12}{22pt}\selectfont
\chapter{ \LaTeX {\MB 定理計數器的使用}}
數理方面的文章或書籍常會使用到定理、定義,或類似需要給予編號的一段文字。這些編號的管理有些是順序排列,有些則隨章節排列,有些一起編號,有些分開。這類的編號都由指令 {\A $\backslash$newtheorem} 處理。  

\section{{\MB 語法}}
$\backslash$newtheorem 的語法如下

 \begin{center}{\begin{tabular}{l}
 $\backslash$newtheorem\{env\_name\}\{caption\}[within]\\
 $\backslash$newtheorem\{env\_name\}[numbered\_like]\{caption\}\\
 \end{tabular}}\end{center}
其中
\begin{itemize}
\item env\_name:新計數器名稱,通常以簡短文字代表將呈現的文字。譬如, thm 代表 Theorem 字樣。
\item caption:表示將呈現的文字,一般如 Theorem, Definition, Lemma... 等或使用中文的「定理」「定義」等。
\item within:代表一個已經存在的計數器,譬如,章 (chapter) 或節 (section),表示目前的計數器將以該存在的計數器為計數範圍。以章為例,在第二章出現的第一個編號將是 2.1,以節為例,第三章第二節出現的第三個編號將是 3.2.3。
\item number\_like:一個已經被定義過的計數器名稱,譬如,thm。代表目前定義的計數器將共用相同的計數器。沒有這項參數定義的都是為獨立編號,不予其他計數器共用。
\end{itemize}
以下範例舉 Definition, Example, Theorem, Lemma 為例,其中 Definition, Example 獨立編號,而 Theorem 與 Lemma 共同編號。定義方式如下:

 \begin{center}{\begin{tabular}{l}
 $\backslash$newtheorem\{de\}\{Definition\}[section]\\
 $\backslash$newtheorem\{ex\}\{$\backslash$emph\{Example\}\}[section]\\
 $\backslash$newtheorem\{th\}\{Theorem\}[section]\\
 $\backslash$newtheorem\{lemma\}[th]\{Lemma\}\\
\end{tabular}}\end{center}


\section{{\MB 隨機變數的定義}}
 \rule{\textwidth}{0.2pt}
 \begin{de}\footnote{摘自 Casella and Berger 2002, Definition 1.1.1}  %def 1.1.1(Casella & Berger (2002))
The set, $\mathcal{S}$, of all possible outcomes of a particular experiment is called the \textbf{sample space} for the experiment.\\
 \rule{\textwidth}{0.2pt}
\end{de}
\noindent 通常定義、定理會用特別的方式呈現出來,讓讀者容易一眼看到。本章將定義前後個加上一條橫線來突顯它的位置。另外一種常見的方式請參考本章最後的「定理」與「 lemma」,用表格加上底色做出明顯的框架,裡面採用的表格與底色技術請參考講義「表格製作參考」。

\noindent \rule{\textwidth}{0.2pt}
\begin{de}\footnote{摘自 Casella and Berger 2002, Definition 1.5.1} %def 1.1.2
The \textbf{cumulative distribution function}  or CDF of a random variable X, denoted by $F_X(x)$,
is defined by
\[F_X(x)=P_X(X \leq x). \mbox{ for all x}, x\in \mathcal{S}.\]
 \rule{\textwidth}{0.2pt}
\end{de}
\noindent 以下的 Example 編號不隨小節計數,係按流水號順序。
\begin{ex}[指數分配隨機變數的呈現] %
假設  X 服從指數分配,其 CDF 為 $y=F_X(x)=1-e^{-x}, \forall x>0$,記為 $X\sim F_X(x)=1-e^{-x}$。
\end{ex}

\begin{ex}[幾何分配隨機變數的呈現] %
假設 X 服從幾何分配,其 CDF 為 $y=F_X(x)=1-(1-p)^k$,其中 $k=[x]\in \mathcal{N}$,記為 $X\sim F_X(x)=1-(1-p)^x$。此函數又稱為階梯函數(step function)。
\end{ex}


\section{{\MB 離散型隨機變數}}
\noindent \rule{\textwidth}{0.2pt}
\begin{de}\footnote{摘自 Casella and Berger 2002, Definition 1.6.1} %def 1.6.1
The \textbf{probability massfunction} (\textbf{pmf}) of a discrete random variable $X$ is given by
\[f_X(x)=P(X=x) \mbox{ for all } x.\]
\noindent \rule{\textwidth}{0.2pt}
\end{de}
\bigskip
 
\begin{ex}[Geometric probabilities] %ex 1.6.2
For the \textbf{geometric distributio}n of Example 1.1.2, we have the \textbf{pmf}
\[f_X(x)=P(X=x)=\left\{\begin{array}{ll} p(1-p)^{x-1}  & \mbox{ for } x=1, 2, \cdots \\
                                              0        & \mbox{ otherwise } \\ \end{array}\right.\]
\end{ex}
\bigskip
\noindent \rule{\textwidth}{0.2pt}
\begin{de}\footnote{摘自 Casella and Berger 2002, Definition 1.6.1}  %def 1.6.3
The \textbf{probability density function} or PDF, $f_X(x)$, of a continuous random variable $X$ is the function that
satisfies
\[F_X(x)=\int_{-\infty}^x f_X(t)dt \mbox{ for all } x.\]
\noindent \rule{\textwidth}{0.2pt}
\end{de}
\bigskip

\begin{ex}[Exponential probabilities] %ex 1.6.4
For the exponential distribution of the previous Example we have
\[F_X(x)=1-e^{-x}\]
and, hence,
\[f_X(x)=\frac{d}{dx}F_X(x)=e^{-x}.\]
\end{ex}
\bigskip

\noindent \rule{\textwidth}{0.2pt}
\begin{de}\footnote{摘自 Casella and Berger 2002, Definition 4.5.10}   %def 4.5.10
Let $-\infty<\mu_X<\infty$, $-\infty<\mu_Y<\infty$, $0<\sigma_X$, $0<\sigma_Y$, and
$-1<\rho<1$ be five real numbers.  The bivariate normal pdf with means $\mu_X$ and $\mu_Y$,
variances $\sigma_X^2$ and $\sigma_Y^2$, and correlation $\rho$ is the bivariate pdf given by
\begin{eqnarray*}
f(x,y)&=&\left( 2\pi\sigma_X\sigma_Y\sqrt{1-\rho^2}\right)^{-1}\\
      && \times\exp\left(-\frac{1}{2(1-\rho^2)}\left(\left(\frac{x-\mu_x}{\sigma_X}\right)^2\right.\right.\\
      && \left.\left.-2\rho\left(\frac{x-\mu_x}{\sigma_X}\right)\left(\frac{y-\mu_y}{\sigma_Y}\right)
      +\left(\frac{y-\mu_y}{\sigma_Y}\right)^2 \right)\right)
\end{eqnarray*}
for $-\infty<x<\infty$ and $-\infty<y<\infty$.\\
\noindent \rule{\textwidth}{0.2pt}
\end{de}

\begin{ex}[{\C Bivariate Normal}] %ex 1.6.2
二維常態參數 $\mu_X=10, \mu_Y=20, \sigma_X=1, \sigma_Y=2, \rho=0.6$ 其分配函數為
\begin{equation*}
f(x,y)=\frac{1}{3.2\pi}\exp\left[-\frac{1}{1.28}\left((\frac{x-10}{1})^2
       -1.2(\frac{x-10}{1})(\frac{y-20}{2})+(\frac{y-20}{2})^2 \right)\right]
\end{equation*}
\end{ex}
\bigskip
\begin{center}\colorbox{slight}{\begin{tabular}{p{0.9\textwidth}}
\begin{thm}\label{demo_ref}\footnote{摘自 Casella and Berger 2002, Theorem 5.3.1} %theorem 5.3.1
Let $X_1, \cdots, X_n$ be a random sample from a $N(\mu, \sigma^2)$ distribution, and let
$\bar{X}=\frac{1}{n}\sum_{i=1}^n X_i$ and $S^2=\frac{1}{n-1}\sum_{i=1}^n (X_i-\bar{X})^2$.  Then
\begin{itemize}
\item[a.] $\bar{X}$ and $S^2$ are independent random variables,
\item[b.] $\bar{X}$ has a $N(\mu, \sigma^2/n)$ distribution,
\item[c.] $(n-1)S^2/\sigma^2$ has a chi squared distribution with $n-1$ degrees of freedom.
\end{itemize}
\end{thm}
 \end{tabular}}\end{center}
\bigskip

定理 \ref{demo_ref} 展示兩件事,其一是加入標號的引用($\backslash$label)與此處的參照對應,其二是自訂的項目符號(\textit{a. b. c.})。接著是個 lemma ,其編號隨著定理續編。
\begin{center}\colorbox{slight}{\begin{tabular}{p{0.9\textwidth}}
\begin{lemma}. Let $a_1,a_2,\cdots$ be a sequence of numbers converging to $a$, that is, $\lim_{n\rightarrow \infty} a_n=a$. Then
$$\lim_{n\rightarrow \infty} (1+\frac{a_n}{n})^n=e^n.$$
\end{lemma}
 \end{tabular}}\end{center}

\section{練習題}
利用上課時間完成以下兩個練習,順便展示簡易型計數器(newcounter)的使用。

\rule{\textwidth}{0.2pt}
\thequiz \;\;對本文所使用到的計數器,分別再多加一個(或以上)範例,方能確定可以掌握這個技術。\\
\rule{\textwidth}{0.2pt}
\addtocounter{quiz}{1} 		% 指定計數器 quiz 加 1
\thequiz \;\;新增一個計數器,並在適當位置加入計數的範例。\\
\rule{\textwidth}{0.2pt}
\addtocounter{quiz}{1} 		% 指定計數器 quiz 加 1
\thequiz \;\;為前一節的定理與 Lemma 的方框換顏色。\\
\rule{\textwidth}{0.2pt}
%\end{document}

%%\input{preamble_REGULAR}
%\usepackage{natbib}
%%\usepackage[sort&compress,square,comma,authoryear]{natbib}
%%-----------------------------------------------------------------------------------------------------------------------
%% 文章開始
%\title{ {\MB 參考文獻的使用與引用(三)}}		% 使用設定的字型
%\author{{\SM 汪群超}}									% 使用設定的小字體
%\date{{\TT \today }} 									% Activate to display a given date or no date (if empty),
%         															% otherwise the current date is printed 
%\begin{document}
%\maketitle
%\fontsize{12}{22pt}\selectfont % 設定在本行之後的字型大小與行距。此設定與編輯器有關(與 WORD 上的 pt 大小也不一致)。
\chapter{參考文獻的使用與引用}
\section{初步觀念}
參考文獻\index{參考文獻}的引用分兩部分:一、內文的引用方式與呈現,二、參考文獻的排序呈現。不管是哪一部分都沒有統一的標準,隨期刊書籍自訂規範。在   \LaTeX 裡,這些規範表現在 bibliography style\index{bibliography style} 所引用的 bst 檔。這些檔案有些是公開的,可以直接引用,譬如,美國數學學會的 amsplain.bst、abbrvnat.bst 或 unsrtnat.bst。有些需要下載,如統計計算與模擬期刊(Journal of Statistical Computation and Simulation)的 gSCS.bst 檔(如附檔)、統計軟體期刊(Journal of Statistical Software)的 jss.bst 檔(如附檔)。文獻規範檔\index{文獻規範檔}(bibliography style)一方面用來呈現不同刊物的需求與特色,一方面也能減輕寫作者的負擔,無需為符合不同刊物的規定,撰寫不同格式的參考文獻。

本文以 bibtex\index{bibtex} 的文獻資料庫\index{文獻資料庫}方式呈現文獻的引用。除了運用 bibliography style 檔外,也增加一個知名的 package: natbib,可以選用文獻引用的呈現方式。讀者可以從本文原始檔下方的 $\backslash$bibliographystyle  試用不同的 bst 檔,看看結果有何不同。

\section{參考文獻的引用:作者與年份}
{\T The second class of MVN tests in this package examine the skewness and kurtosis of the data. Two approaches are adopted. One uses the combination of the univariate skewness and kurtosis for all marginals, as proposed by \cite{SMALL:1980}, and \cite{DOORNIK:2008}. The other approach considers multivariate skewness and kurtosis proposed by \cite{MARDIA:1970}.   \cite{FOSTER:1981} and \cite{HORSWELL:1990} consider the MVN test statistics by Small as "among the most powerful" and "of practical importance,"  while \cite{MM} consider Mardia's procedures, based on multivariate kurtosis, as among the commonly used tests of MVN.  Mardia's procedures are considered as a competitor  in many related studies.
In particular, the omnibus test by \cite{DOORNIK:2008} is widely cited in economics and business journals. Section 3 introduces these procedures, and explain how they are implemented in the \textbf{TWVN} software package \cite{WH}. Comprehensive  comparisons between these two types of tests were conducted by \cite{HORSWELL:1992}.}

\section{文獻引用方式}
所謂 bibtex 的文獻資料庫是一個檔案(副檔名為 bib),將所有文獻依固定格式輸入(請參考附檔 WANG\_ref.bib 檔),譬如典型的幾個欄位「author」、「title」、「journal」、「year」、「volumn」及「pages」。作者可以將所有曾經引用過文獻都放在這個檔案一起維護。需要引用時,只要 $\backslash$cite 標號(label)即可。最大的優點是不需要為每篇文章重複的文獻資料再輸入或複製一份。也因為格式固定的關係,維護與管理都很方便,當需要以不同方式呈現時,只要呼叫適當的 bibliographystyle 即可。

上節的陳述方式所配合的 bibliographystyle 是 {\E plainnat}\index{plainnat}。這是個常見的格式,有很多選項(options),預設為「作者[年份]」,也就是上文看到的樣子。與前兩篇文章不同之處在引用時不需要再輸入作者姓氏,直接用 $\backslash$cite 引用就會出現在作者資料庫檔的作者名字與年份。讀者可以試著採用不同的 bibliograph ystyle,\footnote{請查看本文原始檔引用 bibliography style 的地方,旁邊有註解好幾個不同的格式。}看看有什麼不同。

\section{製作方式}
由於使用了另一個檔案( bib 檔),編輯的過程要經過幾道程序。編譯前先準備好 bib 檔(格式如附件的 {\T WANG\_ref.bib}),及本檔(檔名:{\T template\_ref.tex},引用參考文獻的方式見前段的示範)。對本檔案編譯四次,程序如下:
{\E 
\begin{enumerate}
\item XeLatex
\item Bib Tex
\item XeLatex
\item XeLatex
\end{enumerate}
}

第一次引用 bib 檔或是更新 bib 檔時才需要四道程序,如果只是修改文章內容或更動引用,只需進行一般的編譯。但如果出現異常狀況,可以試著先清除所有編譯過程的附屬檔,再重新執行上述程序。

%\bibliographystyle{gSCS}		% not compatible with \usepackage{natbib}
%\bibliographystyle{jss}
%\bibliographystyle{amsplain}
%\bibliographystyle{plain}
\bibliographystyle{plainnat}
%\bibliographystyle{abbrvnat}
%\bibliographystyle{unsrtnat}

%{\T
%\bibliography{WANG_ref}
%}
%\end{document}

%-------Reference--------------------------------------------------------------------------------------------
%\bibliographystyle{amsplain}
%\bibliographystyle{plain}

%\bibliographystyle{abbrvnat}
%\bibliographystyle{unsrtnat}
%\bibliographystyle{apacite}   
%\renewcommand\bibname{}
\addcontentsline{toc}{chapter}{\CB{參考文獻}} %加入目錄
\chapter*{\CB{參考文獻引用}} %製作標題

本書中關於\LaTeX 的排版技巧,參考\cite{CW},內容關於數學公式及統計定理的部分參考\cite{LW}。希望透過課堂所學,真實的用\LaTeX 來呈現。
\begin{thebibliography}{99} % 99 代表最多 99 個項目
\bibitem{LW}
Larry Wasserman,A Concise Course in Statistical Inference,New York:Srring,2004.
\bibitem{CW}
吳聰敏.吳聰慧,cwTEX排版系統,翰蘆圖書出版有限公司,2005.
\end{thebibliography}

\end{document}