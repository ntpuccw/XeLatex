%% These two lines must be incldued to open file under UTF-8
% !TEX TS-program = xelatex								
% !TEX encoding = UTF-8

\documentclass[12pt, a4paper]{article} 		% use larger type; default would be 10pt
\usepackage{fontspec} 				% Font selection for XeLaTeX; see fontspec.pdf for documentation. 
%\usepackage[BoldFont, SlantFont]{xeCJK}% 中文使用 XeCJK,並模擬粗體與斜體(即可以用 \textbf{ } \textit{ })
\usepackage{xeCJK}							% 中文使用 XeCJK,但利用 \setCJKmainfont 定義粗體與斜體的字型
\defaultfontfeatures{Mapping=tex-text} 		% to support TeX conventions like ``---''
\usepackage{xunicode} 		% Unicode support for LaTeX character names (accents, European chars, etc)
\usepackage{xltxtra} 						% Extra customizations for XeLaTeX
\usepackage{amsmath, amssymb}
\usepackage{enumerate}
\usepackage{graphicx, subfig, float} 		% support the \includegraphics command and options
\usepackage{array, booktabs}
\usepackage{color, xcolor}
\usepackage{longtable}
\usepackage{colortbl}   
\usepackage{multirow} 
\usepackage{multicol} 
\usepackage{arydshln}  
\usepackage{dcolumn} 
\usepackage{rotating} 
\usepackage{diagbox} 
\usepackage{wrapfig} %文繞圖
\usepackage{overpic}%圖片上加文字
%\usepackage{natbib}
\usepackage[sort&compress,square,comma,authoryear]{natbib}
%.............................................表格標題註解之巨集套件
\usepackage[parfill]{parskip} % Activate to begin paragraphs with an empty line rather than an indent
%\usepackage{geometry} % See geometry.pdf to learn the layout options. There are lots.
\usepackage[left=1.5in,right=1in,top=1in,bottom=1in]{geometry} 

%-----------------------------------------------------------------------------------------------------------------------
%  主字型設定
\setCJKmainfont							% 設定中文內文字型
	[
		BoldFont=微軟正黑體				% 定義粗體的字型(依使用的電腦安裝的字型而定)
	]
	{新細明體}							% 設定中文內文字型
\setmainfont{Times New Roman}			% 設定英文內文字型
\setsansfont{Arial}						% used with {\sffamily ...}
%\setsansfont[Scale=MatchLowercase,Mapping=tex-text]{Gill Sans}
\setmonofont{Courier New}				% used with {\ttfamily ...}
%\setmonofont[Scale=MatchLowercase]{Andale Mono}
% 其他字型(隨使用的電腦安裝的字型不同,用註解的方式調整(打開或關閉))
% 英文字型
\newfontfamily{\E}{Cambria}				
\newfontfamily{\A}{Arial}
\newfontfamily{\C}[Scale=0.9]{Cambria}
%\newfontfamily{\T}{Times New Roman}
\newfontfamily{\TT}[Scale=0.8]{Times New Roman}

% 中文字型
\newCJKfontfamily{\MB}{微軟正黑體}			% 適用在 Mac 與 Win
\newCJKfontfamily{\SM}[Scale=0.8]{新細明體}	% 縮小版新細明體
\newCJKfontfamily{\K}{標楷體} 
               % Windows 下的標楷體
% 以下為自行安裝的字型:CwTex 組合
\newCJKfontfamily{\CF}{cwTeX Q Fangsong Medium}	% CwTex 仿宋體
\newCJKfontfamily{\CB}{cwTeX Q Hei Bold}			% CwTex 粗黑體
\newCJKfontfamily{\CK}{cwTeX Q Kai Medium}   		% CwTex 楷體
\newCJKfontfamily{\CM}{cwTeX Q Ming Medium}		% CwTex 明體
\newCJKfontfamily{\CR}{cwTeX Q Yuan Medium}		% CwTex 圓體
%-----------------------------------------------------------------------------------------------------------------------
\XeTeXlinebreaklocale "zh"             %這兩行一定要加,中文才能自動換行
\XeTeXlinebreakskip = 0pt plus 1pt     %這兩行一定要加,中文才能自動換行
%-----------------------------------------------------------------------------------------------------------------------
\newcommand{\cw}{\texttt{cw}\kern-.6pt\TeX}	% 這是 cwTex 的 logo 文字
\newcommand{\imgdir}{../images/}				% 設定圖檔的位置
\renewcommand{\tablename}{表}				% 改變表格標號文字為中文的「表」(預設為 Table)
\renewcommand{\figurename}{圖}				% 改變圖片標號文字為中文的「圖」(預設為 Figure)


%------------------------------------------------------------------------------------------------------
%  計數器
\usepackage{amsthm}							% theroemstyle 需要使用的套件
\theoremstyle{plain}                       %樣式
\newtheorem{de}{Definition定義}[section]    %definition獨立編號 %跟section走憶起編號
\newtheorem{thm}{{\MB 定理}}[section]		%theorem 獨立編號,取中文名稱並給予不同字型
\newtheorem{lemma}[thm]{Lemma}				%lemma 與 theorem 共用編號 %跟thm走憶起編號
\newtheorem{ex}{{\E Example}}			 %example 獨立編號,不編入小節數字,走流水號。也換個字型。
\newcounter{e}
\newtheorem{cor}{Corollary}[section]		%not used here
\newtheorem{exercise}{EXERCISE}				%not used here
\newtheorem{re}{\emph{Result}}[section]	%not used here
\newtheorem{axiom}{AXIOM}					%not used here
\renewcommand{\proofname}{\bf{Proof}}		%not used here
\bibliographystyle{plain}
%-----------------------------------------------------------------------------------------------------------------------
% 設定顏色
\definecolor{slight}{gray}{0.9}				
\definecolor{airforceblue}{rgb}{0.36, 0.54, 0.66} % color Table: http://latexcolor.com
\definecolor{arylideyellow}{rgb}{0.91, 0.84, 0.42}
\definecolor{babyblue}{rgb}{0.54, 0.81, 0.94}
\definecolor{cadmiumred}{rgb}{0.89, 0.0, 0.13}
\definecolor{coolblack}{rgb}{0.0, 0.18, 0.39}
\definecolor{cottoncandy}{rgb}{1.0, 0.74, 0.85}
\definecolor{desertsand}{rgb}{0.93, 0.79, 0.69}
\definecolor{electriclavender}{rgb}{0.96, 0.73, 1.0}
\definecolor{lightsalmonpink}{rgb}{1.0, 0.6, 0.6}
\definecolor{amaranth}{rgb}{0.9, 0.17, 0.31}
\definecolor{amethyst}{rgb}{0.6, 0.4, 0.8}
\definecolor{atomictangerine}{rgb}{1.0, 0.6, 0.4}
\definecolor{babyblueeyes}{rgb}{0.63, 0.79, 0.95}
\definecolor{gray(x11gray)}{rgb}{0.75, 0.75, 0.75}
\definecolor{bananamania}{rgb}{0.98, 0.91, 0.71}	
\definecolor{ballblue}{rgb}{0.13, 0.67, 0.8}
\definecolor{azure(colorwheel)}{rgb}{0.0, 0.5, 1.0}	
\definecolor{ceruleanblue}{rgb}{0.16, 0.32, 0.75}
\definecolor{lightsalmonpink}{rgb}{1.0, 0.6, 0.6}
\definecolor{palepink}{rgb}{0.98, 0.85, 0.87}
\definecolor{lightcarminepink}{rgb}{0.9, 0.4, 0.38}
\definecolor{lightcornflowerblue}{rgb}{0.6, 0.81, 0.93}
\definecolor{carolinablue}{rgb}{0.6, 0.73, 0.89}
   % 使用自己維護的定義檔
%-----------------------------------------------------------------------------------------------------------------------
% 文章開始
%\title{ \LaTeX  {\MB 的表格製作}}
%\author{{\SM 游筑鈞}}
%\date{{\TT \today}} 	


%\begin{document}
%\maketitle
%\fontsize{12}{22pt}\selectfont
\chapter{\CB{表格製作}}
表格是一般人覺得比較困難,但卻是很重要的部份,需要我們多花點時間研究。\LaTeX 的表格,因為是抽象邏輯的思考方式來製作表格,對一般使用者而言比較不容易轉換成直觀印象。和 WORD 相比確實比較複雜,但是畫出來比較整齊,尤其是在做學術性表格的時候,\LaTeX 比較專業。這章節就來探討一些關於製作表格的小技巧,以及常用的套件介紹,以此來熟悉\LaTeX 的表格環境。
\\
\section{\MB{表格環境}}

\subsection{tabbing 環境}
這是 \LaTeX 裡頭最基本的表格形式,除非自行另外定義、繪製,他並沒有方便可用的線條指令來做區隔。完全使用空間、位置的配置來顯示表格內容。\\
在 tabbing 環境中,第一個列(row)是以 $\backslash =$ 來標示 Tab 寬度來區隔欄位,這個寬度是由欄位裡頭的字串寬度所決定的。後續的每個欄位是由 $\backslash > $這個符號來區隔,每列尾要自行加上 $\backslash \backslash$ 來換行,最後一行可以不必使用 $\backslash \backslash$ 換行。如表\ref{t1}\\

\begin{table}[h]
    \caption{tabbing 環境中製作表格}\label{t1}
    \smallskip
\begin{tabbing}
column1 \= column2 \= column3 \\
item1   \> item2   \> item3   \\
itemA   \> itemB   \> itemC
\end{tabbing}
\end{table}

對於欄位內文字的控制,tabbing 較不完備,因此表格主要還是以 tabular 環境較為常用。但 tabbing 環境的好處是,他不見得一定要用於表格的排版,例如他也可以表現如條列環境般的另一種表現方式,而且他可以跨頁排版。

\subsection{tabular 環境}
這大概是最常使用的表格形式,可以很方便的畫線框。這種表格, \LaTeX 是把整個表格當成一個單位來處理,就像字母一樣,因此他在版面的安排上是和一般的字母一般的處理,所以這種表格不經特殊處理,無法被分割成兩個部份來跨頁。\\
和 tabbing 環境的不同,除了可以有線條之外(tabular 環境,分隔欄位的符號是 $\&$,而且一定要指定欄內文字的置放位置,欄內文字超出指定的寬度時,會自動折行,還有許多其他更細節的調整。因此接下來的介紹將以tabular 環境為主。

\section{\MB{基本表格介紹(Tabular環境)}}
\subsection{表格基本架構}
\begin{center}
\begin{figure}[htb] 
    \includegraphics{\imgdir{01.PNG}}
\end{figure}
\begin{tabular}[b]{|l|r|c|}
\hline
    column1 & column2 & column3 \\\hline
    item1   & item2   & item3 \\\hline
    itemA   & itemB   & itemC \\\hline
\end{tabular}

\end{center}


其中 t 表示 top,也可以是 b 表示 bottom,或 c 代表 center,這要在前後有文字相並排的時候才會顯現作用,因為\LaTeX 會把整個 tabular 表格當成一個字母單位,所以可以和其他文字、圖表並排排版。這些參數的意思是和同行文字的對齊方式,top 是表格頂端和前後文字對齊,bottom 則是表格底部和前後文字對齊,center 則是和表格中央對齊。

換行的方式和 tabbing 環境一樣,其中的 $\backslash$hline 是畫一條橫線的意思,連續兩個 $\backslash$hline$\backslash$hline 會畫雙橫線,他本身會自動換行,因此不必加上換行符號。其中$\backslash$begin $\lbrace$tabular$\rbrace$ $[$t$]$ $\lbrace$| l | r | c |$\rbrace$  的 lll 是在指定各欄位內容在小方框內的置放位置,l 表示靠左(left),r 表示靠右(right),c 表示置中(center)。在 $\lbrace$| l | r | c |$\rbrace$ 中加上 bar(|)會畫縱線而兩個 bar 就會畫雙縱線。

\subsection{表格欄位調整}


\begin{table}[h]
    \centering
    \caption{欄位調整示範}\label{e1}
    \smallskip
\begin{center}
\extrarowheight=2pt
\doublerulesep=3pt
\renewcommand{\arraystretch}{1.2} % 將表格行間距加大為原來的 1.2 倍
\tabcolsep=12pt                   % 調整欄間距為 24pt               
\begin{tabular}[b]{|lcc|}   \hline
\rowcolor[gray]{0.8}
Specific Heats & $c$ (J/kg$\cdot$K) & $C$ (J/mol$\cdot$K) \\\hline\hline
Aluminum     & 900  & 24.3 \\\cline{2-3} 
Copper       & 385  & 24.4 \\\cline{2-3} 
Gold         & 130  & 25.6 \\\cline{2-3} 
Steel/Iron   & 450  & 25.0 \\\cline{2-3} 
Lead         & 130  & 26.8 \\\cline{2-3} 
Mercury      & 140  & 28.0 \\\cline{2-3} 
Water        & 4190 & 75.4 \\\hline
\end{tabular}
\end{center}
\end{table}

\fbox{\color{lightcarminepink}array 巨集套件指令功能} 
\begin{itemize}
\item $\backslash$doublerulesep=單位長度 連線兩直線 (||) 或兩橫線 ($\backslash$hline$\backslash$hline) 之間距,預設為2pt。
\item $\backslash$extrarowheight=單位長度 將行高增加幾pt
\item $\backslash$cline$\lbrace$a-b$\rbrace$ 畫某部份欄位的橫線,其中的 a-b 指的就是要畫線的欄位數。
\item $\backslash$arrayrulewidth=單位長度 調整表格線條的粗細,預設值是 0.4pt。但要注意的是要在進入 tabular 環境之前設定好。
\item $\backslash$tabcolsep=單位長度 調整兩欄位的左右間距,預設是 6pt。請注意這個值是實際兩欄位間距值的一半。在進入 tabular 環境之前設定好。
\item $\backslash$doublerulesep=單位長度 調整畫雙線時,這兩線間的間距,預設值是 2pt。在進入 tabular 環境之前設定好。
\item $\backslash$arraystretch 調整表格的上下行距。請注意,這要由$\backslash$renewcommand 來重設,因為在 \LaTeX 定義出的一個常數值,而這個$\backslash$arraystretch 只是這些常數值的倍數,我們要重新改變他才能改變預設倍數。
\end{itemize}
\bigskip

以上之指令同時更動所有欄位之間距。 如果我們只想要更動某兩欄位之間距也有其他方式,如\ref{e2}。\\

\begin{table}[h]
    \centering
    \caption{特定欄位調整示範}\label{e2}
    \smallskip
\begin{tabular}{p{1cm}p{3cm}p{2.5cm}p{1.5cm}p{0.5cm}}
\hline
\rowcolor[gray]{0.8}
Gene name   & GeneNo. & length &    size  (cm) \\
\hline
        001  & 01g009860.2   & 819             & 272   \\
        002  & 01g021730.2   & 798             & 265    \\
        003  & 01g094490.2   & 630             & 209    \\
        004  & 01g102740.2   & 1242            & 413     \\
        005  & 01g104900.2   & 597             & 198      \\
        006  & 02g036430.1   & 1698            & 565       \\
        \hline       
\end{tabular}
\end{table}

\bigskip
\begin{itemize}
\item p$\lbrace$寬度$\rbrace$ 
這裡的 p 指的是段落(paragraph)。通常用於一個小段落的文字,指定了寬度後裡頭的文字會自動折行,而且這個段落的頂端會和其他欄位的頂端對齊。
\item @$\lbrace$文字、符號或指令 $\rbrace$ 
這可以作用在本欄的各個列,讓他們都出現某個文字、符號或都在某個指令的作用下。這個指令另外會同時將欄位間距縮成 0,置於首尾的話,會有讓橫線和文字切齊的作用(預設不會切齊,橫線兩端會多出欄位間距的部份)。事實上, @$\lbrace$...$\rbrace$ 指令大括號內除了設定欄位間距外, 也可以鍵入任何文字或指令。 排版時, 括號內之文字或指令即填入表格中對應的欄位間隔處, 而且原有之空白自動消除。 
\end{itemize}

\subsection{表格線條變化}
\bigskip


\begin{table}[h]
    \centering
    \caption{線條樣式示範}\label{e3}
    \extrarowheight=3pt
    \smallskip
\begin{tabular}{cc|c;{2pt/2pt}c|c;{4pt/0.5pt}c|c!{\vrule width 1pt}}
\toprule[1pt]

          & P(\%) & R(\%) & F1(\%) & P(\%) & R(\%) & F1(\%) \\\hline 
Baseline1 & 76.84 & 76.84 & 76.84 & 76.84 & 76.84 & 76.84 \\
\cdashline{1-7}[0.5pt/2pt]
Baseline2  & 76.84 & 76.84 & 76.84 & 76.84 & 76.84 & 76.84 \\
\hdashline[4pt/0.5pt]
Baseline3  & 76.84 & 76.84 & 76.84 & 76.84 & 76.84 & 76.84 \\
\hline
{\bf Our approach}  & {\bf 76.84} & {\bf 76.84} & {\bf 76.84} & {\bf 76.84} & {\bf 76.84} & {\bf 76.84} \\
\bottomrule[1pt]\\
\end{tabular}
\end{table}
\bigskip
我們前面曾提到 $\backslash$arraryrulewidth 指令,可以調整線條的粗細,但是這無法各別調整線條,每個在 tabular 表格環境內的線條會調整成一樣的粗細。booktabs 巨集套件可以很方便的去控制每一欄位的線條粗細。arydshln巨集套件甚至還可以變化線條的樣式。如表\ref{e3}\\
\\
\fbox{\color{lightcarminepink}booktabs 巨集套件指令功能} 
\bigskip

使用方法和 tabular 環境差不多但可在指令後加個方括號來指定線條的粗細,不指定的話,toprule 及 bottomrule 都會比中間的其他線條粗一點。
\begin{itemize}
\item $\backslash$toprule[線條粗細]	畫表格頂端的橫線
\item$\backslash$midrule[線條粗細]	畫表格裡頭的橫線
\item$\backslash$bottomrule[線條粗細]	畫表格底部的橫線
\item$\backslash$cmidrule[線條粗細](左右是否去邊){畫線欄位}
\end{itemize}

tabular 指令環境中, 利用 | 指令可畫出垂直線, 但其粗細無法調整。利用 array 巨集套件所提供之 !$\lbrace$...$\rbrace$ 指令, 我們可以畫出任意粗細之垂直線。 垂直線指令 | 或 !$\lbrace$ $\backslash$ vrule width 2pt$\rbrace$ 所畫出之直線由上至下貫穿整個表格。 如果只要在某一橫列中間畫短直線, 可使用 $\backslash$ vline 指令。
\bigskip

\fbox{\color{lightcarminepink}arydshln 巨集套件指令功能} 
\begin{itemize}
\item ;$\lbrace$(dash)pt/(gap)pt$\rbrace$ :在tabular 環境中加入,可以控制直線樣式。
\item $\backslash$hdashline[(dash)pt/(gap)pt]:設定橫線的樣式
\item $\backslash$cdashline$\lbrace$cola-colb$\rbrace$[(dash)pt/(gap)pt]:設定橫線的樣式
\end{itemize}


\subsection{表格顏色設計}

$\backslash$arrayrulecolor$\lbrace$顏色$\rbrace$:指定線條顏色,如表 \ref{e4}

\begin{table}[h]
\centering
\caption{指定線條顏色}\label{e4}
\extrarowheight=3pt
\smallskip
\setlength{\arrayrulewidth}{2pt}
\arrayrulecolor{babyblueeyes}
\begin{tabular}{|l|c|r|}
\arrayrulecolor{electriclavender}\hline
United Kingdom & London & Thames\\
\arrayrulecolor{amethyst}\hline
France & Paris & Seine \\
\arrayrulecolor{atomictangerine}\cline{1-1}
\arrayrulecolor{lightsalmonpink}\cline{2-3}
Russia & Moscow & Moskva \\ \hline
\end{tabular}
\end{table}
\bigskip

$\backslash$rowcolor[色彩模型]$\lbrace$顏色$\rbrace$[左緣突出長度][右緣突出長度]:讓整個橫列著色,如表\ref{e5}\\
\begin{table}[h]
\centering
\caption{讓整個橫列著色}\label{e5}
\extrarowheight=3pt
\smallskip
\setlength{\extrarowheight}{2mm}
\begin{tabular}{|l|c|c|c|c|c|c|c|}
\hline
Sydney & OG4G &Thu Oct 10 &Mon Oct 21 or 28 &11 or 18 days &999\\
\rowcolor[gray]{0.8}
& &Thu Oct 17 &Mon Oct 21 or 28 & 4 or 11 days &999\\
&OG7A &Sun Oct 13 &Mon Oct 21 or 28 & 8 or 15 days &999\\
\rowcolor[gray]{0.8}
& &Sun Oct 20 &Mon Oct 28 & 8 days &999\\
\hline
\end{tabular}
\end{table}
\bigskip

$\backslash$columncolor[色彩模型]$\lbrace$顏色$\rbrace$[左緣突出長度][右緣突出長度]:讓整個欄位著色,如表\ref{e6}\\
\begin{table}[h]
\centering
\caption{讓整個欄位著色}\label{e6}
\extrarowheight=3pt
\smallskip
\setlength{\extrarowheight}{2mm}
\setlength{\extrarowheight}{2mm}
\begin{tabular}{|>{\columncolor{babyblueeyes}}l|c|>{\columncolor{electriclavender}}c|c|>{\columncolor{bananamania}}c|c|c|c|}
\hline
Sydney & OG4G &Thu Oct 10 &Mon Oct 21 or 28 &11 or 18 days &999\\\hline
& &Thu Oct 17 &Mon Oct 21 or 28 & 4 or 11 days &999\\\hline
&OG7A &Sun Oct 13 &Mon Oct 21 or 28 & 8 or 15 days &999\\\hline
& &Sun Oct 20 &Mon Oct 28 & 8 days &999\\\hline
\end{tabular}
\end{table}
\bigskip


$\backslash$doublerulesepcolor$\lbrace$顏色$\rbrace$:指定雙並線內間隔的顏色,如表\ref{e7}\\
\begin{table}[h]
\centering
\caption{指定雙並線內間隔的顏色}\label{e7}
\extrarowheight=3pt
\smallskip
\setlength{\extrarowheight}{2mm}
\setlength{\extrarowheight}{2mm}
\doublerulesep=4pt
\doublerulesepcolor{ballblue}
\begin{tabular}{|l|c|c|c|c|c|c|c|}
\hline\hline
Sydney & OG4G &Thu Oct 10 &Mon Oct 21 or 28 &11 or 18 days &999\\\hline
& &Thu Oct 17 &Mon Oct 21 or 28 & 4 or 11 days &999\\\hline
&OG7A &Sun Oct 13 &Mon Oct 21 or 28 & 8 or 15 days &999\\\hline
& &Sun Oct 20 &Mon Oct 28 & 8 days &999\\
\hline\hline
\end{tabular}
\end{table}
\bigskip

注意此一指令之使用須利用 array 巨集套件所提供之 >$\lbrace$...$\rbrace$ 指令之功能。並在 tabular 指令環境中使用。


\subsection{表格分割、合併}
1. $\backslash$multirow$\lbrace$2$\rbrace$ $\lbrace$*$\rbrace$ $\lbrace$Multi-Row$\rbrace$:跨列功能\\
第一個參數2,表示跨兩列, 第二個指令為合併後水平對齊方式, c 為置中, r 為置右,*表示系統自動調整文字
最後一個參數即是要填入的文字
另外,跨列需注意的是,使用$\backslash$multirow指令的那一列表格,到了要撰寫下一列表格時,被跨列的該爛位,直接留空,不可填字。

2. $\backslash$multicolumn$\lbrace$2$\rbrace$ $\lbrace$c|$\rbrace$ $\lbrace$Multi-Column$\rbrace$:跨行功能\\
第一個參數2,表示跨兩行,第二個參數c|,表示文字置中,並在欄位右邊畫一條直線框,最後一個參數即是要填入的文字
\bigskip
\begin{table}[h]
\centering
\caption{表格跨行、跨列示範1}\label{e8}
\extrarowheight=3pt
\smallskip
\setlength{\extrarowheight}{2mm}
\begin{tabular}{|c|c|c|c|c|}
\hline
\multirow{2}{*}{Multi-Row} &
\multicolumn{2}{c|}{Multi-Column} &
\multicolumn{2}{c|}{\multirow{2}{*}{Multi-Row and Col}} \\
\cline{2-3}
  & column-1 & column-2 & \multicolumn{2}{c|}{} \\
\hline
label-1 & label-2 & label-3 & label-4 & label-5 \\
\hline
\end{tabular}
\end{table}
\bigskip

\begin{table}[h]
\centering
\caption{表格跨行、跨列示範2}\label{e9}
\extrarowheight=3pt
\smallskip
\setlength{\extrarowheight}{2mm}
 \begin{tabular}{ccccccc}
    \toprule
    \multirow{2}{*}{Method}&
    \multicolumn{3}{>{\columncolor{carolinablue}}c}{ A}&\multicolumn{3}{>{\columncolor{lightsalmonpink}}c}{ G}\cr
    \cmidrule(lr){2-4} \cmidrule(lr){5-7}
    &Precision&Recall&F1-Measure&Precision&Recall&F1-Measure\cr
    \midrule
    kNN&0.7324&0.7388&0.7301&0.6371&0.6462&0.6568\cr
    F&0.7321&0.7385&0.7323&0.6363&0.6462&0.6559\cr
    E&0.7321&0.7222&0.7311&0.6243&0.6227&0.6570\cr
    D&0.7654&0.7716&0.7699&0.6695&0.6684&0.6642\cr
    C&0.7435&0.7317&0.7343&0.6386&0.6488&0.6435\cr
    B&0.7667&0.7644&0.7646&0.6609&0.6687&0.6574\cr
   \rowcolor[gray]{0.8}
    A&{\bf 0.8189}&{\bf 0.8139}&{\bf 0.8146}&{\bf 0.6971}&{\bf 0.6904}&{\bf 0.6935}\cr
    \bottomrule
    \end{tabular}
     \end{table}
\subsection{小數點對齊}
原來的 tabular 環境的作法是去增加一個欄位,那個欄位使用 @$\lbrace$.$\rbrace$ 來專門排小數點,這樣一來兩欄的間距會消掉,看起來就像連在一起的數字了,但是如果使用 dcolumn 巨集的話,就可以很有規律的去對齊小數點或逗點。dcolumn 的用法,主要是去取代 tabular 參數中的 lrc 這些參數。\\
如表\ref{ex02}我們再怎麼去排,小數點總是無法對齊。我們只要使用dcolumn 巨集把 tabular 的後面參數改掉就可以讓小數點對齊。如表\ref{ex03}
\begin{table}[h]
    \centering
    \caption{小數點無法對齊}\label{ex02}
    \smallskip
\begin{tabular}{lllll}
\toprule
      & headA & headB & headC & headD \\
\midrule
test1 & 7.879  & 921.661 & 1382.81 & 998.98 \\
test2 & 1.97   & 35.21   & 321.3   & 4791112.11 \\
test3 & 211.97 & 5.2     & 213.629 & 748261594.106 \\
\bottomrule
\end{tabular}
\end{table}\\

\bigskip
\begin{table}[h]
    \centering
    \caption{使用 dcolumn 巨集讓小數點對齊}\label{ex03}
    \smallskip
\begin{tabular}{lD{.}{.}{3}D{.}{.}{3}D{.}{.}{3}D{.}{.}{3}}
\toprule
      & headA & headB & headC & headD \\
\midrule
test1 & 7.879  & 921.661 & 1382.81 & 998.98 \\
test2 & 1.97   & 35.21   & 321.3   & 4791112.11 \\
test3 & 211.97 & 5.2     & 213.629 & 748261594.106 \\
\bottomrule
\end{tabular}
\end{table}

這裡要特別注意的是,在 dcolumn 的效力範圍裡頭,他會自動進入數學模式,裡頭要表現數學式的話,前後不必再加 \$,否則會跳出數學模式。例如 headA 會變成斜體字,這是因為進入了數學模式,要讓他正常的話,就要寫成 \$headA\$ 這樣來跳出數學模式。

  
\subsection{斜線表頭} 
可以使用$\backslash$usepackage$\lbrace$diagbox$\rbrace$套件來製做表頭上的斜線。以便清楚的呈現各個欄位的意思。如表\ref{ex8}

\begin{table}[!htbp]
\centering
\caption{示範斜線表頭}\label{ex8}
\smallskip
\begin{tabular}{|c|c|c|c|}
\hline
\diagbox{甲}{$\alpha_{i,j}$}{乙}&$\beta_1$&$\beta_2$&$\beta_3$\\ 
\hline
$\alpha_1$&-4&0&-8\\
\hline
$\alpha_2$&3&2&4\\
\hline
$\alpha_3$&16&1&-9\\
\hline
$\alpha_4$&-1&1&7\\
\hline
\end{tabular}
\end{table}

  
  
  
\subsection{表格編號與標題}

一般放在文章裡的表格都需要編號(Label)與標題(Caption),方便與內文相呼應。為了讓表格能夠移動,常常會把就是把表格置於 table 環境當中。在裡頭有 $\backslash$caption$\lbrace$ $\rbrace$ 指令可以指定表格的標頭,$\backslash $label$\lbrace$ $\rbrace$指令可以標號。
\bigskip

一般國際上較正式的文件,caption 置放的位置慣例是「表上圖下」,也就是說表格的標題是置於表格上方,圖形則在下方。其中標題前的「表」字,是重新經過定義的。\LaTeX 原來定義的文字是英文 Table,在中文的環境當然不妥,利用$\backslash$renewcommand$\lbrace$ $\backslash$tablename$\rbrace$ $\lbrace$表 $\rbrace$ 可以定義作者自己喜歡的字眼。

\bigskip
\fbox{\color{lightcarminepink}LaTeX 的浮動環境(table環境)} 
\begin{itemize}
\item h(here)	置於下指令處位置
\item t(top)	置於一頁的頂端
\item b(bottom)	置於本頁底部,如空間不夠會置於次頁
\item p(page)	單獨佔一頁,此頁沒有內文的部份
\item $\backslash$suppressfloats[位置]抑制浮動物件置放於本頁的某處,他會出現在次頁
\item ! 置於以上選頁之前,會更強烈要求達到此選項的作用。但對 p 則無作用
\end{itemize}

\begin{table}[H]
    \centering
    \caption{表格編號示範}\label{dou}
    \smallskip
    \extrarowheight=2pt
    \begin{tabular}{lcc}
    \rowcolor{babyblue}
    \hline
    姓名       & 座號    & 成績 \\\hline
    A 同學    & 15       & 80 \\
    B 同學      & 27     & 72 \\
    C 同學     & 35      & 81 \\\hline
    \end{tabular}
\end{table}
\bigskip
表 \ref{dou}中的標題部分也可以加上顏色,甚至是調整行高。

\subsection{表格並排}

常常我們會需要有2個表格並排的情況,所以在\LaTeX 中使用$\backslash$begin $\lbrace$ minipage$\rbrace$的指令來完成,如表\ref{ex5}和表\ref{ex6}。其實這個方法除了能讓表格並排之外還能讓圖片和表格並排。

\bigskip
\begin{minipage}{\textwidth}
        \begin{minipage}[t]{0.45\textwidth}
            \centering
            \makeatletter\def\@captype{table}\makeatother\caption{表格並排示範1}\label{ex5}
            \begin{tabular}{|ccc|}
\hline
\rowcolor{lightcornflowerblue}
n  & L  & $L+n$ \\ \hline
0  & 1  & 1 \\
1  & 3  & 4 \\
2  & 5  & 7 \\
3  & 7  & 10\\
4  & 9  & 13 \\
5  & 11 & 16 \\
\hline
\end{tabular}
        \end{minipage}
        \begin{minipage}[t]{0.45\textwidth}
        \centering
        \makeatletter\def\@captype{table}\makeatother\caption{表格並排示範2}\label{ex6}
            \begin{tabular}{|ccc|}
\hline
\rowcolor{lightcornflowerblue}
n  & L  & $L+n$ \\ \hline
0  & 1  & 1 \\
1  & 3  & 4 \\
2  & 5  & 7 \\
3  & 7  & 10\\
4  & 9  & 13 \\
5  & 11 & 16 \\
\hline
\end{tabular}
\end{minipage}
\end{minipage}


\subsection{大型表格(longtable)}

這可能有兩種情形。一種是很寬的表格,另一種是很長的表格。太寬的表格可考慮旋轉一下,讓他橫放,至於長的表格可以使用 longtable巨集 讓他可以跨頁連續。

\bigskip

用\LaTeX 排版,如果要旋轉文字,圖片,表格等,可以使用rotating巨集來完成。\\

\bigskip
\fbox{\color{lightcarminepink}rotating 巨集套件指令功能} 
\begin{itemize}
\item $\backslash$begin$\lbrace$sideways$\rbrace$將内容旋轉90度$\backslash$end$\lbrace$sideways$\rbrace$
\item $\backslash$begin$\lbrace$turn$\rbrace$ $\lbrace$45$\rbrace$將内容旋轉自定義角度$\backslash$end$\lbrace$turn$\rbrace$
\item $\backslash$begin$\lbrace$rotate$\lbrace$120$\rbrace$將内容旋轉自定義角度,但是旋轉結果並不能保證所需要的空間$\backslash$end$\lbrace$rotate$\rbrace$\\
\end{itemize}

表 \ref{ex07} 是將表格旋轉(採用 rotating 套件),方便做寬型表格時使用。\\
表 \ref{basic_4} 是將表格視為圖片做旋轉(採用  {\A graphicx} 套件),方便做寬型表格時使用。\\
\begin{minipage}{\textwidth}
        \begin{minipage}[t]{0.45\textwidth}
\begin{table}[H]
 \centering
 \caption{寬表格旋轉}\label{ex07}
 \smallskip
 \extrarowheight=2pt
\begin{sideways}
\begin{tabular}{cccccc} \toprule
Models  &  $\hat c$  &  $\hat\alpha$  &  $\hat\beta_0$  &  $\hat\beta_1$  &  $\hat\beta_2$  \\ \hline
model  & 30.6302  & 0.4127  & 9.4257  & - & -  \\
model  & 12.4089  & 0.5169  & 18.6986  & -6.6157  & - \\
model  & 14.8586  & 0.4991  & 19.5421  & -7.0717  & 0.2183 \\
model  & 3.06302  & 0.41266  & 0.11725  & - & - \\
model  & 1.24089  & 0.51691  & 0.83605  & -0.66157  & -  \\
\bottomrule
\end{tabular}
\end{sideways}
\end{table}
\end{minipage}
\begin{minipage}[t]{0.45\textwidth}
\begin{table}[H]
\begin{center}
\caption{旋轉表格}\label{basic_4}
\bigskip
\extrarowheight=2pt
\rotatebox[origin=c]{90}{
\begin{tabular}{|l|ccccc|}
\hline
Source	& Df	& SS			& MS		& F value	& Pr$>$ F \\\hline
model	& 2 	& 543.6 	& 271.8 	& 16.08 	& 0.0004 \\
Error		& 12 & 202.8 	& 16.9 		&{}  			&{} \\\hline
Total		& 14 & 746.4 	&{}  			&{}  			&{} \\
\hline
\end{tabular}}\hspace{10pt}
\end{center}
\end{table}
\end{minipage}
\end{minipage}


\bigskip
如果表格長度超過版面高度,可以使用longtable巨集套件,原來之表格自動拆為兩部分以上,分別排版於兩頁或是多頁之中,如表   \ref{grid_mlmmh} 所示。請注意不同頁面在表格斷續處的文字處理。
\newpage


\begin{center}
\begin{longtable}{|l|l|l|}
\caption[Feasible triples for a highly variable Grid]{Feasible triples for
highly variable Grid, MLMMH .} \label{grid_mlmmh} \\
\hline \multicolumn{1}{|c|}{\textbf{Time (s)}} & \multicolumn{1}{c|}{\textbf{Triple chosen}} & \multicolumn{1}{c|}{\textbf{Other feasible triples}} \\ \hline
\endfirsthead
\multicolumn{3}{c}%
{{\bfseries \tablename\ \thetable{} -- continued from previous page}} \\
\hline \multicolumn{1}{|c|}{\textbf{Time (s)}} &
\multicolumn{1}{c|}{\textbf{Triple chosen}} &
\multicolumn{1}{c|}{\textbf{Other feasible triples}} \\ \hline
\endhead
\hline \multicolumn{3}{|r|}{{Continued on next page}} \\ \hline
\endfoot
\hline \hline
\endlastfoot
0 & (1, 11, 13725) & (1, 12, 10980), (1, 13, 8235), (2, 2, 0), (3, 1, 0) \\
2745 & (1, 12, 10980) & (1, 13, 8235), (2, 2, 0), (2, 3, 0), (3, 1, 0) \\
5490 & (1, 12, 13725) & (2, 2, 2745), (2, 3, 0), (3, 1, 0) \\
8235 & (1, 12, 16470) & (1, 13, 13725), (2, 2, 2745), (2, 3, 0), (3, 1, 0) \\
10980 & (1,12, 16470) & (1, 13, 13725), (2, 2, 2745), (2, 3, 0), (3, 1, 0) \\
13725 & (1, 12, 16470) & (1, 13, 13725), (2, 2, 2745), (2, 3, 0), (3, 1, 0) \\
16470 & (1, 13, 16470) & (2, 2, 2745), (2, 3, 0), (3, 1, 0) \\
19215 & (1, 12, 16470) & (1, 13, 13725), (2, 2, 2745), (2, 3, 0), (3, 1, 0) \\
21960 & (1, 12, 16470) & (1, 13, 13725), (2, 2, 2745), (2, 3, 0), (3, 1, 0) \\
24705 & (1, 12, 16470) & (1, 13, 13725), (2, 2, 2745), (2, 3, 0), (3, 1, 0) \\
27450 & (1, 12, 16470) & (1, 13, 13725), (2, 2, 2745), (2, 3, 0), (3, 1, 0) \\
30195 & (2, 2, 2745) & (2, 3, 0), (3, 1, 0) \\
32940 & (1, 13, 16470) & (2, 2, 2745), (2, 3, 0), (3, 1, 0) \\
35685 & (1, 13, 13725) & (2, 2, 2745), (2, 3, 0), (3, 1, 0) \\
38430 & (1, 13, 10980) & (2, 2, 2745), (2, 3, 0), (3, 1, 0) \\
41175 & (1, 12, 13725) & (1, 13, 10980), (2, 2, 2745), (2, 3, 0), (3, 1, 0) \\
43920 & (1, 13, 10980) & (2, 2, 2745), (2, 3, 0), (3, 1, 0) \\
46665 & (2, 2, 2745) & (2, 3, 0), (3, 1, 0) \\
49410 & (2, 2, 2745) & (2, 3, 0), (3, 1, 0) \\
52155 & (1, 12, 16470) & (1, 13, 13725), (2, 2, 2745), (2, 3, 0), (3, 1, 0) \\
54900 & (1, 13, 13725) & (2, 2, 2745), (2, 3, 0), (3, 1, 0) \\
57645 & (1, 13, 13725) & (2, 2, 2745), (2, 3, 0), (3, 1, 0) \\
60390 & (1, 12, 13725) & (2, 2, 2745), (2, 3, 0), (3, 1, 0) \\
63135 & (1, 13, 16470) & (2, 2, 2745), (2, 3, 0), (3, 1, 0) \\
65880 & (1, 13, 16470) & (2, 2, 2745), (2, 3, 0), (3, 1, 0) \\
68625 & (2, 2, 2745) & (2, 3, 0), (3, 1, 0) \\
71370 & (1, 13, 13725) & (2, 2, 2745), (2, 3, 0), (3, 1, 0) \\
74115 & (1, 12, 13725) & (2, 2, 2745), (2, 3, 0), (3, 1, 0) \\
76860 & (1, 13, 13725) & (2, 2, 2745), (2, 3, 0), (3, 1, 0) \\
79605 & (1, 13, 13725)& (2, 2, 2745), (2, 3, 0), (3, 1, 0) \\
82350 & (1, 12, 13725) & (2, 2, 2745), (2, 3, 0), (3, 1, 0) \\
104310 & (1, 13, 16470) & (2, 2, 2745), (2, 3, 0), (3, 1, 0) \\
107055 & (1, 13, 13725) & (2, 2, 2745), (2, 3, 0), (3, 1, 0) \\
109800 & (1, 13, 13725) & (2, 2, 2745), (2, 3, 0), (3, 1, 0) \\
112545 & (1, 12, 16470) & (1, 13, 13725), (2, 2, 2745), (2, 3, 0), (3, 1, 0) \\
115290 & (1, 13, 16470) & (2, 2, 2745), (2, 3, 0), (3, 1, 0) \\
118035 & (1, 13, 13725) & (2, 2, 2745), (2, 3, 0), (3, 1, 0) \\
120780 & (1, 13, 16470) & (2, 2, 2745), (2, 3, 0), (3, 1, 0) \\
123525 & (1, 13, 13725) & (2, 2, 2745), (2, 3, 0), (3, 1, 0) \\
126270 & (1, 12, 16470) & (1, 13, 13725), (2, 2, 2745), (2, 3, 0), (3, 1, 0) \\
129015 & (2, 2, 2745) & (2, 3, 0), (3, 1, 0) \\
131760 & (2, 2, 2745) & (2, 3, 0), (3, 1, 0) \\
134505 & (1, 13, 16470) & (2, 2, 2745), (2, 3, 0), (3, 1, 0) \\
137250 & (1, 13, 13725) & (2, 2, 2745), (2, 3, 0), (3, 1, 0) \\
139995 & (2, 2, 2745) & (2, 3, 0), (3, 1, 0) \\
142740 & (2, 2, 2745) & (2, 3, 0), (3, 1, 0) \\
145485 & (1, 12, 16470) & (1, 13, 13725), (2, 2, 2745), (2, 3, 0), (3, 1, 0) \\
148230 & (2, 2, 2745) & (2, 3, 0), (3, 1, 0) \\
150975 & (1, 13, 16470) & (2, 2, 2745), (2, 3, 0), (3, 1, 0) \\
153720 & (1, 12, 13725) & (2, 2, 2745), (2, 3, 0), (3, 1, 0) \\
156465 & (1, 13, 13725) & (2, 2, 2745), (2, 3, 0), (3, 1, 0) \\
159210 & (1, 13, 13725) & (2, 2, 2745), (2, 3, 0), (3, 1, 0) \\
161955 & (1, 13, 16470) & (2, 2, 2745), (2, 3, 0), (3, 1, 0) \\
164700 & (1, 13, 13725) & (2, 2, 2745), (2, 3, 0), (3, 1, 0) \\
\end{longtable}
\end{center}
\section*{\MB{結論}}
藉由這次練習之後,已經大致熟悉製作表格的過程和概念,其實如何做出一個美觀又易懂的表格真的不容易。需要多次的練習與嘗試才能掌握技巧。

%\end{document}