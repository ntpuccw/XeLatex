%% These two lines must be incldued to open file under UTF-8
% !TEX TS-program = xelatex								
% !TEX encoding = UTF-8

\documentclass[12pt, a4paper]{article} 		% use larger type; default would be 10pt
\usepackage{fontspec} 				% Font selection for XeLaTeX; see fontspec.pdf for documentation. 
%\usepackage[BoldFont, SlantFont]{xeCJK}% 中文使用 XeCJK,並模擬粗體與斜體(即可以用 \textbf{ } \textit{ })
\usepackage{xeCJK}							% 中文使用 XeCJK,但利用 \setCJKmainfont 定義粗體與斜體的字型
\defaultfontfeatures{Mapping=tex-text} 		% to support TeX conventions like ``---''
\usepackage{xunicode} 		% Unicode support for LaTeX character names (accents, European chars, etc)
\usepackage{xltxtra} 						% Extra customizations for XeLaTeX
\usepackage{amsmath, amssymb}
\usepackage{enumerate}
\usepackage{graphicx, subfig, float} 		% support the \includegraphics command and options
\usepackage{array, booktabs}
\usepackage{color, xcolor}
\usepackage{longtable}
\usepackage{colortbl}   
\usepackage{multirow} 
\usepackage{multicol} 
\usepackage{arydshln}  
\usepackage{dcolumn} 
\usepackage{rotating} 
\usepackage{diagbox} 
\usepackage{wrapfig} %文繞圖
\usepackage{overpic}%圖片上加文字
%\usepackage{natbib}
\usepackage[sort&compress,square,comma,authoryear]{natbib}
%.............................................表格標題註解之巨集套件
\usepackage[parfill]{parskip} % Activate to begin paragraphs with an empty line rather than an indent
%\usepackage{geometry} % See geometry.pdf to learn the layout options. There are lots.
\usepackage[left=1.5in,right=1in,top=1in,bottom=1in]{geometry} 

%-----------------------------------------------------------------------------------------------------------------------
%  主字型設定
\setCJKmainfont							% 設定中文內文字型
	[
		BoldFont=微軟正黑體				% 定義粗體的字型(依使用的電腦安裝的字型而定)
	]
	{新細明體}							% 設定中文內文字型
\setmainfont{Times New Roman}			% 設定英文內文字型
\setsansfont{Arial}						% used with {\sffamily ...}
%\setsansfont[Scale=MatchLowercase,Mapping=tex-text]{Gill Sans}
\setmonofont{Courier New}				% used with {\ttfamily ...}
%\setmonofont[Scale=MatchLowercase]{Andale Mono}
% 其他字型(隨使用的電腦安裝的字型不同,用註解的方式調整(打開或關閉))
% 英文字型
\newfontfamily{\E}{Cambria}				
\newfontfamily{\A}{Arial}
\newfontfamily{\C}[Scale=0.9]{Cambria}
%\newfontfamily{\T}{Times New Roman}
\newfontfamily{\TT}[Scale=0.8]{Times New Roman}

% 中文字型
\newCJKfontfamily{\MB}{微軟正黑體}			% 適用在 Mac 與 Win
\newCJKfontfamily{\SM}[Scale=0.8]{新細明體}	% 縮小版新細明體
\newCJKfontfamily{\K}{標楷體} 
               % Windows 下的標楷體
% 以下為自行安裝的字型:CwTex 組合
\newCJKfontfamily{\CF}{cwTeX Q Fangsong Medium}	% CwTex 仿宋體
\newCJKfontfamily{\CB}{cwTeX Q Hei Bold}			% CwTex 粗黑體
\newCJKfontfamily{\CK}{cwTeX Q Kai Medium}   		% CwTex 楷體
\newCJKfontfamily{\CM}{cwTeX Q Ming Medium}		% CwTex 明體
\newCJKfontfamily{\CR}{cwTeX Q Yuan Medium}		% CwTex 圓體
%-----------------------------------------------------------------------------------------------------------------------
\XeTeXlinebreaklocale "zh"             %這兩行一定要加,中文才能自動換行
\XeTeXlinebreakskip = 0pt plus 1pt     %這兩行一定要加,中文才能自動換行
%-----------------------------------------------------------------------------------------------------------------------
\newcommand{\cw}{\texttt{cw}\kern-.6pt\TeX}	% 這是 cwTex 的 logo 文字
\newcommand{\imgdir}{../images/}				% 設定圖檔的位置
\renewcommand{\tablename}{表}				% 改變表格標號文字為中文的「表」(預設為 Table)
\renewcommand{\figurename}{圖}				% 改變圖片標號文字為中文的「圖」(預設為 Figure)


%------------------------------------------------------------------------------------------------------
%  計數器
\usepackage{amsthm}							% theroemstyle 需要使用的套件
\theoremstyle{plain}                       %樣式
\newtheorem{de}{Definition定義}[section]    %definition獨立編號 %跟section走憶起編號
\newtheorem{thm}{{\MB 定理}}[section]		%theorem 獨立編號,取中文名稱並給予不同字型
\newtheorem{lemma}[thm]{Lemma}				%lemma 與 theorem 共用編號 %跟thm走憶起編號
\newtheorem{ex}{{\E Example}}			 %example 獨立編號,不編入小節數字,走流水號。也換個字型。
\newcounter{e}
\newtheorem{cor}{Corollary}[section]		%not used here
\newtheorem{exercise}{EXERCISE}				%not used here
\newtheorem{re}{\emph{Result}}[section]	%not used here
\newtheorem{axiom}{AXIOM}					%not used here
\renewcommand{\proofname}{\bf{Proof}}		%not used here
\bibliographystyle{plain}
%-----------------------------------------------------------------------------------------------------------------------
% 設定顏色
\definecolor{slight}{gray}{0.9}				
\definecolor{airforceblue}{rgb}{0.36, 0.54, 0.66} % color Table: http://latexcolor.com
\definecolor{arylideyellow}{rgb}{0.91, 0.84, 0.42}
\definecolor{babyblue}{rgb}{0.54, 0.81, 0.94}
\definecolor{cadmiumred}{rgb}{0.89, 0.0, 0.13}
\definecolor{coolblack}{rgb}{0.0, 0.18, 0.39}
\definecolor{cottoncandy}{rgb}{1.0, 0.74, 0.85}
\definecolor{desertsand}{rgb}{0.93, 0.79, 0.69}
\definecolor{electriclavender}{rgb}{0.96, 0.73, 1.0}
\definecolor{lightsalmonpink}{rgb}{1.0, 0.6, 0.6}
\definecolor{amaranth}{rgb}{0.9, 0.17, 0.31}
\definecolor{amethyst}{rgb}{0.6, 0.4, 0.8}
\definecolor{atomictangerine}{rgb}{1.0, 0.6, 0.4}
\definecolor{babyblueeyes}{rgb}{0.63, 0.79, 0.95}
\definecolor{gray(x11gray)}{rgb}{0.75, 0.75, 0.75}
\definecolor{bananamania}{rgb}{0.98, 0.91, 0.71}	
\definecolor{ballblue}{rgb}{0.13, 0.67, 0.8}
\definecolor{azure(colorwheel)}{rgb}{0.0, 0.5, 1.0}	
\definecolor{ceruleanblue}{rgb}{0.16, 0.32, 0.75}
\definecolor{lightsalmonpink}{rgb}{1.0, 0.6, 0.6}
\definecolor{palepink}{rgb}{0.98, 0.85, 0.87}
\definecolor{lightcarminepink}{rgb}{0.9, 0.4, 0.38}
\definecolor{lightcornflowerblue}{rgb}{0.6, 0.81, 0.93}
\definecolor{carolinablue}{rgb}{0.6, 0.73, 0.89}
   % 使用自己維護的定義檔
%-----------------------------------------------------------------------------------------------------------------------
% 文章開始
%\title{ \LaTeX {\MB 外製圖形({\C EPS/JPG/PNG/PDF})的引入與計數器之應用}}
%\author{{\SM 游筑鈞}}
%\date{{\TT \today}} 	        				
%\begin{document}
%\maketitle
%\fontsize{12}{22pt}\selectfont
\chapter{\CB{圖片呈現}}
圖片經常是文件中最生動、最精采的部分,但是用過 Word排 版文件的人都知道,圖片是最不聽話的物件,常常無法依你的要求擺放在你要的位置。因此這裡將會介紹如何用 \LaTeX 來插入圖片並且指定圖片的擺放位置。除了基本的圖片排版之外, \LaTeX 還可以指定幾張圖並排、圖片旋轉等等。圖形檔的型態最常見的有 {\C EPS/JPG/PNG/PDF} 等格式,在這裡將會分別介紹如何插入不同型態的圖檔。在介紹圖片使用的同時搭配 \LaTeX 的計數器和專業的參考文獻功能,一步一步的介紹如何排版一份專業的正式的報告。\LaTeX 提供了許多不同的套件可以使用,接著我們就來一起熟悉如何排版圖片吧 !

\section{\MB{插入圖片}}
插入圖片的方式有好幾種,目前最常用{\A graphicx} 套件。

\subsection{圖片路徑設定}
圖形環境指令$\backslash$begin$\lbrace$figure$\rbrace$可以用來控制圖形開始與結束的位置。預設的圖形檔路徑與文章相同,圖形若不是放置於此,必須指定完整的路徑。

\begin{center}\colorbox{slight}{\begin{tabular}{p{0.9\textwidth}}
	{\A $\backslash$includegraphics\{檔案位置/檔案名稱.jpg\}}
\end{tabular}}\end{center}
\bigskip
如果圖片檔案和此份資料在同一個目錄位置的話,路徑可以省略。
\begin{center}\colorbox{slight}{\begin{tabular}{p{0.9\textwidth}}
	{\A $\backslash$includegraphics\{檔案名稱.jpg\}}
\end{tabular}}\end{center}
當然如果一份文件中引入許多分散在不同目錄的圖檔,相當麻煩,因此將所有檔案都集中到預設的目錄。為避免在指令中放在冗長得完整路徑,一般會在定義區設定一個路徑命令,用來縮短指令所需的長度。本文在定義區設定以下的新命令:
\smallskip
\begin{center}\colorbox{slight}{\begin{tabular}{p{0.9\textwidth}}
	{\A $\backslash$newcommand\{$\backslash$imgdir\}\{images/\}}
\end{tabular}}\end{center}
\smallskip

這個新命令自訂為 {\A $\backslash$imgdir} ,定義了一個與編譯文章路徑相同的子目錄:images,也就是所有圖形檔案放置的目錄。譬如:

\smallskip
\begin{center}\colorbox{slight}{\begin{tabular}{p{0.9\textwidth}}
	{\A $\backslash$includegraphics\{$\backslash$imgdir\{圖片名稱.eps\}\}}
\smallskip\end{tabular}}\end{center}

\subsection{圖片位置與大小}

\textbf{$\backslash$begin$\lbrace$figure$\rbrace$ $[$ H $]$}\\
位置選項變數:h$($here$)$置於現在的位置,t$($top$)$置於本頁上端,b$($bottom$)$置於本頁下端,p$($page$)$自成一頁。H強制要在現在的位置。如果不加選項, 內定值為 $[$tbp$]$。下指令時, 位置選項變數之順序無關緊要, 永遠依照 htbp 之順序尋找適當位置。 \\
\textbf{$\backslash$centering} \\
圖片置中\\
\textbf{$\backslash$includegraphics$[$width=0.7$\backslash$textwidth,angle$=$270$]$ $\lbrace$檔案名稱$\rbrace$}\\
調整圖形之大小: height 圖形高度,totalheight 圖形全高,width 圖形的寬度,angle 圖形旋轉 $($反時鐘方向$)$ 角度,scale 圖形放大 $($或縮小$)$ 之倍數。其中$\backslash$textwidth是以整頁的寬度為調整基準。寬度調整時, 高度也同比例調整。\\
\textbf{$\backslash$ caption$\lbrace$給圖一個名字$\rbrace$} \\
\textbf{$\backslash$label$\lbrace$標籤$\rbrace$} \\
常常在寫文件,應該都需要在載入圖片,除了載入圖片以外,還需要在底下加入標號,好在閱讀的時候可以在文章中指定要用來解釋得圖片,也可以在圖片下方加入簡單的敘述。\\
\textbf{$\backslash$vskip 10pt} \\
指定圖片與下方文字的距離\\
\textbf{$\backslash$end$\lbrace$figure$\rbrace$}

\subsection{插入不同類型的圖檔}
圖(\ref{fig:type1})和圖(\ref{fig:type2})分別是四種不同類型的圖檔,其中可以發現eps檔圖片比較清晰,適合數學圖形的表現。jpg和png檔比較模糊。

\begin{figure}[H]
    \centering
        \subfloat[png圖檔]{
        \includegraphics[width=0.5\textwidth]{\imgdir{p2.PNG}}}
        \subfloat[jpg圖檔]{
        \includegraphics[width=0.5\textwidth]{\imgdir{p2.jpg}}}
    \caption{不同類型的圖檔}
    \label{fig:type1}
\end{figure}
\begin{figure}[H]
    \centering
        \subfloat[eps圖檔]{
        \includegraphics[width=0.5\textwidth]{\imgdir{p2.eps}}}
        \subfloat[pdf圖檔]{
        \includegraphics[width=0.5\textwidth]{\imgdir{p2.pdf}}}
    \caption{不同類型的圖檔}
    \label{fig:type2}
\end{figure}

\subsection{文繞圖}
用wrapfigure套件可以呈現文繞圖的排版方式。\\
\textbf{$\backslash$begin$\lbrace$wrapfigure$\rbrace$ $\lbrace$ l $\rbrace$ $\lbrace$6cm$\rbrace$}\\
設定圖片的位置靠左,並和文字距離6cm,如下圖(\ref{ex01}):
\begin{wrapfigure}{l}{6cm}
\centering
\includegraphics[width=0.4\textwidth]{\imgdir{XeTex.png}}
\label{ex01}
\caption{文繞圖示範}
\end{wrapfigure}

\LaTeX 遵循呈現與內容分離的設計理念,以便作者可以專注於他們正在編寫的內容,而不必同時注視其外觀。在準備\LaTeX 文件時,作者使用章、節、表、圖等簡單的概念指定文件的邏輯結構,並讓LaTeX系統負責這些結構的格式和布局。因此,它鼓勵從內容中分離布局,同時仍然允許在需要時進行手動排版調整。這個概念類似於許多文書處理器允許全域定義整個文件的樣式的機制,或使用層疊樣式表來規定HTML的樣式。\LaTeX 系統是一種可以處理排版和彩現的標記式語言。
\subsection{多圖並排}
圖片並排是常用於排版的方法,方便讀者將多張圖一起比較參照。這裡介紹不同的多圖並排方式。通常會使用subfloat套件\\
兩張圖並排:\\
\begin{figure}[H]
    \centering
        \subfloat[立體圖]{
        \includegraphics[width=0.5\textwidth]{\imgdir{p3.eps}}}
        \subfloat[立體圖]{
        \includegraphics[width=0.5\textwidth]{\imgdir{p4.eps}}}
    \caption{兩張圖並排}
    \label{fig:parallel}
\end{figure}

三張圖並排:\\
\begin{figure}[H]
    \centering
        \subfloat[$\alpha$ $>$ $\beta$]{
        \includegraphics[width=0.33\textwidth]{\imgdir{p5.eps}}}
        \subfloat[$\alpha$ $<$ $\beta$]{
        \includegraphics[width=0.33\textwidth]{\imgdir{p6.eps}}}
        \subfloat[$\alpha$ $=$ $\beta$]{
        \includegraphics[width=0.33\textwidth]{\imgdir{P7.eps}}}
    \caption{三張圖並排}
    \label{fig:parallel}
\end{figure}

兩列兩行:\\
\begin{figure}[H]
    \centering
        \subfloat[$\alpha$ $>$ $\beta$]{
        \includegraphics[width=0.5\textwidth]{\imgdir{p5.eps}}}
        \subfloat[$\alpha$ $<$ $\beta$]{
        \includegraphics[width=0.5\textwidth]{\imgdir{p6.eps}}}
        \quad
        \subfloat[$\alpha$ $=$ $\beta$]{
        \includegraphics[width=0.5\textwidth]{\imgdir{P7.eps}}}
        \subfloat[all]{
        \includegraphics[width=0.5\textwidth]{\imgdir{p8.eps}}}
    \caption{四張圖並排}
    \label{fig:parallel}
\end{figure}
\subsection{圖片上加文字}
用 overpic套件可以在圖片上加入文字或公式。\\
\begin{overpic}[scale=0.8]{\imgdir{p8.eps}}
    \centering
    %\caption{圖片加入文字}
    \label{addtext}
    \put(40,52){\huge \color{slight}{\bf \LaTeX}}
    \put(40,42){\huge \color{slight}{\bf Graphics}}
\end{overpic}
\chapter{\CB{計數器}}
\section{ 計數器的應用}
數理方面的文章或書籍常會使用到定理定義,或類似需要給予編號的一段文字。這些編號的管理有些是順序排列,有些則隨章節排列,有些一起編號,有些分開。這類的編號都由指令 {\C $\backslash$newtheorem} 處理。接著我們就由數統課本裡的例子來示範計數器的使用。\\  

$\backslash$newtheorem 的語法如下:\\
\begin{center}\colorbox{slight}{\begin{tabular}{l}
$\backslash$newtheorem\{env\_name\}\{caption\}[within]\\
$\backslash$newtheorem\{env\_name\}[numbered\_like]\{caption\}\\
\end{tabular}}\end{center}
以下示範說明:\\
\begin{center}\colorbox{slight}{\begin{tabular}{l}

\textbf{$\backslash$theoremstyle\{plain\}} 選擇樣式\\
\textbf{$\backslash$newtheorem\{de\}\{Definition定義\}[section]}\\
definition隨section走,一起編號\\
\textbf{$\backslash$newtheorem\{thm\}\{\{$\backslash$MB theorem定理\}\}[Definition定義]}\\
theorem定理隨section走,一起編號\\
\textbf{$\backslash$newtheorem\{ex\}\{\{$\backslash$E Example\}\}}\\
獨立編號,不編入小節數字,走流水號。\\
\end{tabular}}\end{center}
\section{ \MB{圖片與計數器之實際應用--Random Variables}}
藉由上學期MATLAB作業中製作的圖片搭配數統課本,進行以下得練習。

\subsection{Introduction}

{\color{slight}\rule{\textwidth}{0.2pt}}
\begin{de} 
A \textbf{random variable} is a mappimg\\
 \begin{center}
 X:$\Omega$ $\longleftrightarrow$ $\Re$\\
that assigns a real number X($\omega$)to each outcome.\\
\end{center} 
{\color{slight}\rule{\textwidth}{0.2pt}}
\end{de}

\begin{ex} %
Flip a coin ten times. Let X($\omega$)be the number of heads in the sequence $\omega$.For example, if $\omega$=HHTHHTHHTT,then X($\omega$)=6.
\end{ex}

{\color{slight}\rule{\textwidth}{0.2pt}}
\begin{de} 
The \textbf{cumulative distribution function},or CDF,is the function \\
$F_X$:$\Re$ $\longleftrightarrow$[0,1]defined by\\
 \begin{center}
 $F_X(x)$= : P(X$\leqslant$x)\\
\end{center} 
{\color{slight}\rule{\textwidth}{0.2pt}}
\end{de}

\begin{thm}
Let X have CDF F and Y have CDF G.If F(x)=G(x) for all x,then P(X$\in$A)=P(Y$\in$A) for all A.
\end{thm}

\begin{thm}
A function F mapping the real line to [0,1] is a CDF for some probability P if and only if F satisfies the following three conditions:
\begin{enumerate}
\item F is non-decreasing : $x_1$ < $x_2$ implies that F($x_1$)<F($x_2$).
\item F is normalized:\\
$$\lim_{x \rightarrow -\infty}F(x)=0 \qquad \lim_{x \rightarrow \infty}F(x)=1$$
\item F is right-continuous: F(x) = F($x^+$) for all x.
\end{enumerate}
\end{thm}


{\color{slight}\rule{\textwidth}{0.2pt}}
\begin{de} 
X is \textbf{discrete} if it takes countably many values $\lbrace$ $x_1$,$x_2$,...$\rbrace$. We define the \textbf{probability function} or \textbf{probability mass function} for X by $f_X(x)$=P(X=x) \\
{\color{slight}\rule{\textwidth}{0.2pt}}
\end{de}

\begin{de} 
A random variable X is \textbf{continous} if there exists a function $f_X$ such that $f_X(x)$ $\ge$ 0  for all x,$\int_{-\infty}^{\infty}$ $f_X(x)$dx=1 and for every a $\le$b ,
\begin{center}
$$ P(a<X<b)=\int_{a}^{b}f_X(x)dx$$
\end{center}
The function $f_X$ is called the \textbf{probability density function}(PDF). We have that 
\begin{center}
$$ F_X(x)=\int_{-\infty}^{x}f_X(t)dt$$
\end{center}
and  $f_X(x)$=$F_X(x)$ at all points x and which $F_X$ is differentiable.\\
{\color{slight}\rule{\textwidth}{0.2pt}}
\end{de}


\begin{ex} %
Suppose that X has PDF\\
$$
f_X(x) = \left\{\begin{array}{ll}
                 1, & \mbox{for 0 $\le$ x $\le$ 1} \\  
                 0, & \mbox{otherwise.} \\  
                \end{array} \right.
$$
The CDF is given by
$$
F_X(x) = \left\{\begin{array}{ll}
                 0, & \mbox{  $x<$ 0} \\  
                 x, & \mbox{ 0 $\le$ x $\le$ 1} \\ 
                 1, & \mbox{  $x>$ 1}\\  
                \end{array} \right.
$$
\end{ex}

\subsection{Some Important Discete Random Variables}
\setcounter{ex}{1}
{\color{slight}\rule{\textwidth}{0.2pt}}
\theex\;\; THE DISCRETE UNIFORM DISTRIBUTION.\\
Let k>1 be a given integer.Suppose that X has probability mass function given by\\
$$
f_X(x) = \left\{\begin{array}{ll}
                 1/k & \mbox{for $x$=1,...,k} \\  
                 0 & \mbox{otherwise.} \\  
                \end{array} \right.
$$
We say that X has a uniform distribution on $\lbrace$1,2,.....,k$\rbrace$.\\
{\color{slight}\rule{\textwidth}{0.2pt}}
\addtocounter{ex}{1}
\theex\;\; THE BERNOULLI DISTRIBUTION.\\
Let X represent a binary coin flip. Then P(X=1)=p and P(X=0)=1-p for some p $\in$ [0,1].We say that X has a Bernoulli distribution written X$\sim$Bernoulli(p).The probability function is f(x)=$p^{x}$ ${1-p}^{1-x}$ for x$\in$ $\lbrace$ 0,1 $\rbrace$.
\bigskip
\begin{figure}[H]
    \centering
    \includegraphics[width=0.8\textwidth]{\imgdir{ber01.png}}
    \caption{BERNOULLI DISTRIBUTION}
    \label{ber}
\end{figure}

{\color{slight}\rule{\textwidth}{0.2pt}}
\addtocounter{ex}{1}
\theex\;\; THE BINOMIAL DISTRIBUTION.\\
Suppose we have a coin which falls heads up with probability p for some 0$\le$p$\le$1.Flip the cion n times and let X be the number of heads.Assume that the tosses are independent. Let f(x)=P(X=x)be the mass function.It can be shown that\\
$$
f_X(x) = \left\{\begin{array}{ll}
                 \tbinom{n}{x}p^{x}(1-p)^{n-x} & \mbox{for $x$=0,...,n} \\  
                 0 & \mbox{otherwise.} \\  
                \end{array} \right.
$$

\bigskip
\begin{figure}[H]
    \centering
    \includegraphics[width=0.8\textwidth]{\imgdir{bin01.png}}
    \caption{BINOMIAL DISTRIBUTION}
    \label{bin}
\end{figure}

\begin{thm}
A random variable with this mass function is called a Binomial random variable and we write X $\sim$ Binomial(n , p). If $X_1$ $\sim$ Binomial($n_1$ , p) and $X_2$ $\sim$ Binomial($n_2$ , p) then $X_1$+$X_2$ $\sim$ Binomial($n_1+n_2$ , p).
\end{thm}

\bigskip
\begin{figure}[H]
    \centering
    \includegraphics[width=0.8\textwidth]{\imgdir{ber02.png}}
    \caption{BINOMIAL MASS DISTRIBUTION}
    \label{bin2}
\end{figure}


%{\color{slight}\rule{\textwidth}{0.2pt}}
\addtocounter{ex}{1}
\newpage
\theex\;\; THE GEOMETRIC DISTRIBUTION.\\
X has a geometric distribution with parameter p $\in$(0,1),written X $\sim$ Geom(p),if
$$ 
P(X=k)=p(1-p)^{k-1},k \geqslant 1.
$$
We have that 
$$
\sum_{k=1}^{\infty}P(X=k)=p\sum_{k=1}^{\infty}=\frac{p}{1-(1-p)}=1
$$
Think of X as the number of flips needed until the first head when flipping a coin.

\begin{figure}[H]
    \centering
        \subfloat[PDF]{
        \includegraphics[width=0.5\textwidth]{\imgdir{p13.eps}}}
        \subfloat[CDF]{
        \includegraphics[width=0.5\textwidth]{\imgdir{p14.eps}}}
    \caption{GEOMETRIC DISTRIBUTION}
    \label{fig:geo}
\end{figure}

%1112 exp 1516poi 910 chi2
{\color{slight}\rule{\textwidth}{0.2pt}}
\addtocounter{ex}{1}
\theex\;\; THE POISSON DISTRIBUTION.\\
X has a Poisson distribution with parameter $\lambda$,written X$\sim$Poisson($\lambda$)if
$$
f(x)=e^{-\lambda}\frac{\lambda^{x}}{x!}\qquad x \ge 0.
$$
Note that
$$
\sum_{x=0}^{\infty}=e^{-\lambda}\sum_{x=0}^{\infty}\frac{\lambda^{x}}{x!}=1
$$
\begin{figure}[H]
    \centering
        \subfloat[PDF]{
        \includegraphics[width=0.5\textwidth]{\imgdir{p15.eps}}}
        \subfloat[CDF]{
        \includegraphics[width=0.5\textwidth]{\imgdir{p16.eps}}}
    \caption{Poisson distribution}
    \label{fig:poi}
\end{figure}

\begin{thm}
The Poisson is often used as a model for counts of rare events like radioactive decay and traffic accidents.If $X_1$ $\sim$ Poisson($\lambda_1$) and $X_2$ $\sim$ Poisson($\lambda_2$)then $X_1+X_2$ $\sim$ Poisson($\lambda_1+\lambda_2$)
\end{thm}

\bigskip
\begin{figure}[H]
    \centering
    \includegraphics[width=0.8\textwidth]{\imgdir{POISS02.png}}
    \caption{POISSON DISTRIBUTION}
    \label{poi2}
\end{figure}
\newpage
\subsection{Some Important Continuous Random Variables}

{\color{slight}\rule{\textwidth}{0.2pt}}
\addtocounter{ex}{1}
\theex\;\; THE UNIFORM DISTRIBUTION.\\
X has a Uniform(a,b) distribution, written X $\sim$ Uniform(a,b),if

$$
f_X(x) = \left\{\begin{array}{ll}
                 \frac{1}{b-a} & \mbox{for $x$=1,...,k} \\  
                 0 & \mbox{otherwise.} \\  
                \end{array} \right.
$$

Where a < b. The distribution function if

$$
F_X(x) = \left\{\begin{array}{ll}
                 0, & \mbox{  $x<$ a} \\  
   \frac{x-a}{b-a}, & \mbox{ $x$ $\in$ [a,b]} \\ 
                 1, & \mbox{  $x>$ b}\\  
                \end{array} \right.
$$

 
\begin{figure}[H]
    \centering
        \subfloat[PDF]{
        \includegraphics[width=0.5\textwidth]{\imgdir{p17.eps}}}
        \subfloat[CDF]{
        \includegraphics[width=0.5\textwidth]{\imgdir{p18.eps}}}
    \caption{uniform distribution}
    \label{fig:UNI}
\end{figure}

%{\color{slight}\rule{\textwidth}{0.2pt}}
\addtocounter{ex}{1}
\theex\;\; THE NORMAL DISTRIBUTION.\\
X has a Normal(or Gaussian) distribution with parameters $\mu$ and $\sigma$,denoted by X $\sim$ N($\mu$,$\sigma$),if
$$
f(x)=\frac{1}{\sigma\sqrt{2\pi}}e^{-\frac{(x-\mu)^2}{2\sigma^2}}, \;\;  -\infty < x < \infty 
$$

\bigskip
\begin{figure}[H]
    \centering
    \includegraphics[width=0.8\textwidth]{\imgdir{norm01.png}}
    \caption{NORMAL DISTRIBUTION}
    \label{NORM}
\end{figure}

{\color{slight}\rule{\textwidth}{0.2pt}}
\addtocounter{ex}{1}
\theex\;\; THE EXPONENTIAL DISTRIBUTION.\\
X has an Exponential distribution with parameter $\beta$, denoted by X $\sim$ Exp($\beta$),if
$$
f(x)=\frac{1}{\beta}e^{\frac{-x}{\beta}},\qquad x > 0
$$
Where $\beta$ >0. The exponential distribution is used to model the lifetimes of electronic components and the waiting times between rare events.

\begin{figure}[H]
    \centering
        \subfloat[PDF]{
        \includegraphics[width=0.5\textwidth]{\imgdir{p11.eps}}}
        \subfloat[CDF]{
        \includegraphics[width=0.5\textwidth]{\imgdir{p12.eps}}}
    \caption{Exponential distribution}
    \label{fig:exp}
\end{figure}
{\color{slight}\rule{\textwidth}{0.2pt}}
\addtocounter{ex}{1}
\theex\;\; THE GAMMA DISTRIBUTION.\\
For $\alpha$ > 0 , the \textbf{Gamma function} is defined by $\Gamma(\alpha)$=$\int_{0}^{\infty}$ $y^{\alpha-1}$ $e^{-y}$dy.X has a Gamma distribution with parameters $\alpha$ and $\beta$, denoted by X $\sim $ Gamma($\alpha$,$\beta$),if
$$f(x)=\frac{1}{\Gamma(\alpha)\beta^\alpha}x^{\alpha-1}e^{-\frac{x}{\beta}}, \;\; x\geq 0
$$
where $\alpha$,$\beta$ > 0.The exponential distribution is just a Gamma(1,$\beta$) distribution. If $X_i$ $\sim$ Gamma($\alpha_i$,$\beta$)are independent,then $\sum_{i-1}^{n}$ $X_i$ $\sim$ Gamma($\sum_{i=1}^{n}$,$\beta$).

\begin{figure}[H]
    \centering
        \subfloat[$\alpha$ $>$ $\beta$]{
        \includegraphics[width=0.5\textwidth]{\imgdir{p5.eps}}}
        \subfloat[$\alpha$ $<$ $\beta$]{
        \includegraphics[width=0.5\textwidth]{\imgdir{p6.eps}}}
        \quad
        \subfloat[$\alpha$ $=$ $\beta$]{
        \includegraphics[width=0.5\textwidth]{\imgdir{P7.eps}}}
        \subfloat[all]{
        \includegraphics[width=0.5\textwidth]{\imgdir{p8.eps}}}
    \caption{GAMMA DISTRIBUTION}
    \label{fig:GAM}
\end{figure}
\newpage
%{\color{slight}\rule{\textwidth}{0.2pt}}
\addtocounter{ex}{1}
\theex\;\; THE BETA DISTRIBUTION.\\
X has a Beta distribution with parameters $\alpha$ > 0 and $\beta$ > 0,denoted by X $\sim$ Beta($\alpha$,$\beta$),if
$$
\begin{aligned}
f(x;\alpha,\beta)&=\frac{x^{\alpha-1}(1-x)^{\beta-1}}{\int^1_0u^{\alpha-1}(1-u)^{\beta-1}du}\\[4mm]
&=\frac{\Gamma(\alpha+\beta)}{\Gamma(\alpha)\Gamma(\beta)}x^{\alpha-1}(1-x)^{\beta-1}\\[4mm]
&=\frac{1}{B(\alpha,\beta)}x^{\alpha-1}(1-x)^{\beta-1}
\end{aligned}
$$

\bigskip
\begin{figure}[H]
    \centering
    \includegraphics[width=0.8\textwidth]{\imgdir{beta01.png}}
    \caption{BETA DISTRIBUTION}
    \label{beta}
\end{figure}
\newpage
%{\color{slight}\rule{\textwidth}{0.2pt}}
\addtocounter{ex}{1}
\theex\;\; THE t AND CAUCHY DISTRIBUTION.\\
X has a t distribution with $\nu$ degrees of freedom-written X $\sim$ $t_{\nu}$-if
$$
f(x)=\frac{\Gamma(\frac{\nu+1}{2})}{\Gamma(\frac{\nu}{2})}\frac{1}{(1+\frac{x^{2}}{\nu})^{(\nu+1)/2}}
$$

\bigskip
\begin{figure}[H]
    \centering
    \includegraphics[width=0.8\textwidth]{\imgdir{t2.png}}
    \caption{t DISTRIBUTION}
    \label{t}
\end{figure}

{\color{slight}\rule{\textwidth}{0.2pt}}
\addtocounter{ex}{1}
\theex\;\; THE $\chi^{2}$ DISTRIBUTION.\\
X has a $\chi^{2}$ distribution with p degrees of freedom--written X $\sim$ $\chi_{p}^{2}$-if

$$
f(x)=\frac{1}{\Gamma(p/2)2^{p/2}}x^{(p/2)^{-1}e^{-x/2}},\qquad x>0
$$

\begin{figure}[H]
    \centering
        \subfloat[PDF]{
        \includegraphics[width=0.5\textwidth]{\imgdir{p9.eps}}}
        \subfloat[CDF]{
        \includegraphics[width=0.5\textwidth]{\imgdir{p10.eps}}}
    \caption{$\chi^{2}$ distribution}
    \label{fig:chi}
\end{figure}

\chapter{\CB{參考文獻(Bibliography)}}


\section{\MB{thebibliography 環境}}
在進入 thebibliography,編譯後他會自成一個獨立的章節,如果是 article 類別的文稿,他會自動印出 Referrences 的字樣為標題,如果是 report 或 book 類別的文稿,他會印出 Bibliography 的字樣為標題。 

\section{\MB{BibTeX 簡介}}
如果常常有寫論文的機會,整理出自己的一份參考文獻資料庫可以節省許多時間,正常情況下,使用 bibtex 來處理外部文獻檔案的情形,只有引用到的文獻才會印出來,這樣也就不必擔心印出一堆不相關的文獻了。另外一個好處是,這個參考文獻資料庫可以另外獨立維護,所有的文章都用這一份資料庫,這在維護上會很方便,也減少錯誤的機會。BibTeX 本身提供一個外部的 bibtex 工具程式,在 latex 編譯過文稿後,再利用 bibtex 編譯一次文稿,最後再使用 latex 重編譯過。而參考文獻資料庫是按一定的格式寫於bib 檔案裡頭,在文稿中則以bibliogrphy 指令來引入,編譯過程中自然會去參考這個外部考文獻資料庫。

%\section*{section{參考文獻} }
%本文參考自數統課本\cite{LW} 和\cite{CW}
\subsection{製作方式}

由於使用了另一個檔案( bib 檔),編輯的過程要經過幾道程序。編譯前先準備好 bib 檔。對本檔案編譯四次,程序如下:

\begin{enumerate}
\item XeLatex
\item Bib Tex
\item XeLatex
\item XeLatex
\end{enumerate}
%\bibliography{CHUN_ref}
\section*{\MB{結論}}
在使用latex 的過程中,需要花一些時間摸索,雖然在撰寫時較為複雜,但在經過多次的操作後,便會逐漸熟悉其各個功能的使用。無論是在表格的製作、圖片的安排、計數器、參考文獻等等功能,都有更正式更專業的表現。因此多花一些時間來練習latex排版是值得的。最近第一次嘗試將所學到的latex排版技巧用在競賽的資料分析報告書上,可以發現真的和word排版很不一樣。藉由幾次的練習與應用盡量減少錯誤的出現,讓未來在進行文書處理的工作時, 能更加的得心應手。 

%\end{document}

 