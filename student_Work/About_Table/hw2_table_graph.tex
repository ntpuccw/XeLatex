\input{preamble_CJKapple}
%-----------------------------------------------------------------------------------------------------------------------
% 文章開始
\title{ \LaTeX  {\CB 的表格與圖片}}
\author{{\CF 統計四}\\{\A 410178007} 	{\A Lin Peng-Wen}}
\date{{\C \today}} 	% Activate to display a given date or no date (if empty),
         						% otherwise the current date is printed 
\begin{document}
\maketitle
\fontsize{12}{22pt}\selectfont

%\newpage
本篇內容介紹攥寫文章時插入表格與圖片的使用,通常一篇正常的文章或報告都會希望是圖文並茂的呈現,表格的應用也是相當常見,可以讓呈現的東西更精簡明瞭,在統計上更是不可或缺,而在 \LaTeX 的學習上一般都是知道某些指令的用法後,將它成為一個範本展現出來,日後用到時就可以直接利用節複製貼上的方法改寫內容即可,因此本文除了介紹表格及圖片使用上的基本語法外,這些指令在未來都可以當做查詢使用,甚至依照想要的格式直接複製改寫,但是更重要的是,當碰到沒用過的格式或想打造怎樣呈現的樣貌時,要試著去摸索指令的用法,或藉由搜尋資料或詢問老師,學會新的使用方法後更要將其記錄下來方便日後翻找查看,不僅可以減少編輯上的時間,多學了一樣東西就像口袋裡有多了一道秘笈。

\section{{\CB 表格的製作}}
表格指令的編輯與矩陣的語法有一點類似,都是使用 \& 換欄,用$\backslash \backslash$換列,以一行一行的編輯方式創造表格的樣貌,基本指令是用$\backslash$begin$\lbrace$tabular$\rbrace$ $\lbrace$l|c|c|c$\rbrace$,lccc代表四欄,第一欄靠左,二三四欄置中,兩旁的直線代表欄與欄間劃上直線,換行後每一列後方加上$\backslash$hline 代表加上一條橫線,相反的不想要橫線就不要加,接著就從第一節的基本表格來開始練習看看!\\

\subsection{{\CR 基本表格}}
下方可以看到一個基本原始表格\footnote{資料來源:http://bsal.ym.edu.tw/ja/index.php/blog/44-software/indesign/149-manually-typeset-a-table} 的呈現樣貌,可以清楚發現是完全沒有進行過任何調味的味道,樣貌顯得有點呆版。\\

\begin{tabular}{|l|c|c|c|}%lcc代表三欄,第一欄靠左,二三欄置中,兩旁的直線代表欄與欄間劃上直線
\hline  %劃上一條橫線
  GDP  & 4.4  & 10.9 & 8.8		\\\hline  % &代表換欄 \\代表換下一列
  Domestic demand     & 6.5     & 16.9 & 10.4		\\\hline
  Next export  & 2.1     & 6.0	& 1.5		\\\hline
\end{tabular}\\

表格的呈現是一種藝術,基本上會有一些改變表格樣式的小技巧,而表格在頁面的呈現上也盡量會以置中的方式來讓整個畫面達到平衡的感覺:
\begin{itemize}
\item 去除垂直線
\item 可去除部分水平線
\item 適當的文字對其方式
\item 適當的欄寬
\item 表頭網底
\end{itemize}
第一點去除垂直線相當簡單,只需要將$\lbrace$ $\rbrace$內的||去除即可,然後使用$\backslash$begin$\lbrace$center$\rbrace$ ... $\backslash$end$\lbrace$center$\rbrace$或$\backslash$centering指令\footnote{使用$\backslash$centering指令而沒有用$\backslash$begin$\lbrace$table$\rbrace$時要加在$\backslash$begin$\lbrace$tabular$\rbrace$前方}將之置中,可以看到下面的表格變得比較順眼且不那麼索然無味的感覺,這裡在表格的前後都加上了$\backslash$bigskip指令讓表格跟文字排列在呈現上不會顯得那麼擁擠。\\
\begin{center} 
\begin{tabular}{lccc}
\hline  %劃上一條橫線
  GDP  & 4.4  & 10.9 & 8.8		\\\hline  % &代表換欄 \\代表換下一列
  Domestic demand     & 6.5     & 16.9 & 10.4		\\\hline
  Next export  & 2.1     & 6.0	& 1.5		\\\hline
\end{tabular}
\end{center}
\begin{table}[h] %加入環境指令 table 以控制表格的位置、編號與標題,[h]代表將表格置於 here,其他位置的標示請參考手冊   
    \centering %置中的另一個指令
    \caption{最基本的表格}\label{basic_1}  %加入標題與標號參照的文字
    \bigskip
	\begin{tabular}{|l|ccc|}%lcc代表三欄,第一欄靠左,二三欄置中,兩旁的直線代表欄與欄間劃上直線
	\hline  %劃上一條橫線
  GDP  & 4.4  & 10.9 & 8.8		\\\hline  % &代表換欄 \\代表換下一列
  Domestic demand     & 6.5     & 16.9 & 10.4		\\\hline
  Next export  & 2.1     & 6.0	& 1.5		\\\hline
	\end{tabular}
\end{table}

但一般比較常用來呈現表格的方式通常是利用$\backslash$begin$\lbrace$table$\rbrace$ 指令,在後面加上[h]代表將表格置於 here,更多表達方式可以參考表\ref{location},若是用H則是強制性的要他出現在指定的位置,但一般盡量不要太常使用,否則容易弄巧成拙,而使用table指令就可以在後面加上$\backslash$caption 、$\backslash$label或前面提到的$\backslash$centering等指令,在使用標號時也可以利用$\backslash$bigskip指令在標號與表格間製造一個較大的空間讓畫面看起來更舒適,在讓畫面較寬闊來看不失為一個好用的指令,如表\ref{basic_1}所示。\\
\begin{table}[h] 
    \centering 
    \caption{表格位置}\label{location}  %加入標題與標號參照的文字
    \bigskip
    \begin{tabular}{p{2.5cm}l}
    位置指令 & 意義 \\\toprule[0.8pt]
    h(here) & 置於(here)下指令的地方\\
    t(top) & 一頁的頂端\\
    b(bottom) & 一頁的底端,空間不夠會放在下一頁\\
    p(page) & 獨佔一頁,沒有其他內文\\\bottomrule[1pt]
    \end{tabular}
\end{table}
但是可以看到表\ref{basic_1} 的樣子其實並不是很令人滿意,可以用怎樣的方法來改善呢?\\
可以發現最基本的問題是行高太小,讓整個表格變的壅擠,這裡使用了一個package,記得要在定義檔(preamble)裡加上:
\begin{center}
	\colorbox{pink}{\begin{tabular}{p{0.9\textwidth}}
	{\G $\backslash$usepackage\{array\}}
\end{tabular}}
\end{center}
\bigskip
利用這個package的指令:$\backslash$extrarowheight=4pt來改變行高,至於要多少 pt ,可以藉由多次編譯做調整,依編輯者喜歡的呈現,有些人喜歡空間大一點就可以將數字在調大一點看看,如表\ref{basic_2}就是調整過後的樣子。\\
\begin{table}[h] 
	\centering 
	\caption{修改後的基本表格}\label{basic_2}  
	\extrarowheight=4pt   %加高行高
	\begin{tabular}{|l|ccc|}
	\hline  %劃上一條橫線
  GDP  & 4.4  & 10.9 & 8.8		\\\hline  % &代表換欄 \\代表換下一列
  Domestic demand     & 6.5     & 16.9 & 10.4		\\\hline
  Next export  & 2.1     & 6.0	& 1.5		\\\hline
	\end{tabular}
\end{table}

\newpage
\subsection{{\CR 個人化設定表格}}
 \begin{table}[h] 
	\centering 
	\caption{加底色與換字型的基本表格}\label{basic_3}  
	\extrarowheight=4pt   %加高行高
	\colorbox{slight}{\begin{tabular}{lccc}
	\hline
	{\G GDP}  &{\C 4.4}  &{\A 10.9} &{\A 8.8}		\\\hline  % &代表換欄 \\代表換下一列
	{\G Domestic demand }    & {\A 6.5}     &{\A 16.9 }&{\A 10.4}		\\\hline
	{\G Next export}  & {\C 2.1}     & {\A 6.0}	&{\C 1.5}		\\\hline  
	\end{tabular}}
\end{table}
其實表格的呈現是需要很多個人化設定的,可以依編輯者的喜好去做修改,表\ref{basic_3} 示範的是將表格加了底色還有改變表格裡的字型,改變字型的部分就不加贅述,而底色是利用$\backslash$colorbox$\lbrace$slight$\rbrace\lbrace\backslash$begin$\lbrace$tabular$\rbrace\rbrace$指令,其中slight是在定義檔裡已經定義過的顏色代號,名稱為slight,灰色,深淺度0.8,要在定義檔(preamble)裡加上:
\begin{center}
	\colorbox{slight}{\begin{tabular}{p{0.9\textwidth}}
	{\G $\backslash$definecolor\{slight\}\{gray\}\{0.8\}}
\end{tabular}}
\end{center}
\bigskip
而因為color巨集裡只對了一些顏色做定義,所以其他顏色幾乎都需要自己用red、green、blue去定義。
這提供了編輯著可以針對表格做個人化的設定,好處是可以找到自己看得順眼的表格樣貌,當漸漸找到一個型了之後,就有點像是找到自己愛用的指令模式,當然牽涉到個人對於美感觀念的不同,通過不斷修改以及學習他人呈現的樣式可以更容易做到合大家眼的狀態。\\
這時候就會出現一個問題,我不想要全部的表格都變成一個顏色啊,前面提到的改變表格小技巧最後一點是『表頭網底』,會讓表格看起來比較好看也是常用的技巧,在此使用colortbl套件就可以達到這樣的效果囉!同樣需要再定義檔裡加上:
\begin{center}
	\colorbox{pink}{\begin{tabular}{p{0.9\textwidth}}
	{\G $\backslash$usepackage\{colortbl\}}
\end{tabular}}
\end{center}
用表\ref{sin}	做一個範例,指令為在你想加上網底的地方使用$\backslash$rowcolor[gray]$\lbrace$.9$\rbrace$,$\lbrace$ $\rbrace$裡的數字同樣也是調整深淺的,格子內改變顏色則是使用$\backslash$cellcolor$\lbrace$slight$\rbrace$的指令。\\
\begin{table}[H] 
	\centering 
	\caption{三角函數微分與積分}\label{sin}  
	\extrarowheight=3pt 
	\begin{tabular}{cc}
	\rowcolor[gray]{.9}
	微分前 & 微分後 \\\toprule[1.5pt]
	$\sin x$ & $\cos x$  \\\hline	
	$\cos x$ & $-\sin x$  \\\hline	
	\cellcolor{slight}$\tan x$ & $\sec^2 x$  \\\hline	
	$\cot x$ & $-\csc^2 x$  \\\hline	
	$\sec x$ & $\tan x\sec x$  \\\hline	
	$\csc x$ & \cellcolor{slight}$-\cot x\cos x$  \\\bottomrule	
	\end{tabular}\hspace{20pt}
	\begin{tabular}{cc}	
	\rowcolor[gray]{.9}	
	積分前 & 積分後 \\\toprule[1.5pt]
	$\sin x$ & $-\cos x +C$  \\\hline
	\rowcolor[gray]{.9}	
	$\cos x$ & $\sin x +C$  \\\hline	
	$\tan x$ & $\ln\mid \sec x\mid +C$  \\\hline	
	\rowcolor[gray]{.9}	
	$\cot x$ & $\ln\mid \sin x\mid +C$  \\\hline	
	$\sec x$ & $\ln\mid \sec x+\tan x\mid +C$  \\\hline	
	\rowcolor[gray]{.9}	
	$\csc x$ & $\ln\mid \csc x-\cot x\mid +C$  \\\bottomrule	
	\end{tabular}
\end{table}

這裡同時可以看到一個表格並排的做法,做法很簡單,只要第一個表格結束指令的$\backslash$end$\lbrace$tabular$\rbrace$與下一個表格開始的$\backslash$begin$\lbrace$tabular$\rbrace$之間不要空行,但是可以用$\backslash$hspace$\lbrace$ pt$\rbrace$指令在兩個表格間保留一點空間,讓他們不會黏在一起。\\

除了前面提到的一些個人化設定來改變表格型態之外,這邊再介紹一個好用的package:
\begin{center}
	\colorbox{pink}{\begin{tabular}{p{0.9\textwidth}}
	{\G $\backslash$usepackage\{booktabs\}}
\end{tabular}}
\end{center}
\bigskip
\begin{table}[h]
	\centering
	\caption{booktabs 指令}\label{booktabs}
	\begin{tabular}{ll}	
	指令 & 功能 \\\toprule[1pt]
	$\backslash$topule[pt] & 表格頂端橫線\\
	$\backslash$midrule[pt] & 表格內部橫線\\
	$\backslash$bottomrule[pt] & 表格底部橫線\\
	$\backslash$cmidrule[pt] & 某欄位橫線,取代原$\backslash$cline\\\bottomrule[1pt]
	\end{tabular}\par\smallskip 
	\parbox{7cm}{[pt]自由設定粗細}
\end{table}
\newpage
booktabs可以針對線條的粗細做調整,可以參考表\ref{booktabs}的指令意義做使用,[pt]的部分是可以自己設定線的粗細,我覺得是相當好用的指令,利用這個方法製作了\ref{anova} ANOVA Table,透過不同線條粗細的表達讓整個表格顯的好看而且容易閱讀。這個套件還有另外還有一個功能是設定欄寬,當我們要表達的內容很多時就可以透過蘭寬的設定讓他達到自動換行的效果,參考表\ref{right},第一欄置左,而第二欄的部分設定欄寬為7cm,指令:{\C \{lp\{7cm\}\}},想要個人化調整是利用p這個指令(paragraph)。

\begin{table}[h]
	\centering
	\caption{ANOVA Table}\label{anova}
	\begin{tabular}{ccccc}
	\toprule[1pt]
	source & SS & df & MS & F \\\toprule[2pt]
	A	&	110.164	& 2	& 55.082	& 1179.49\\\hline
	B	&	65.738	&	2 & 32.869	& 703.83 \\\hline
	AB	& 15.382	& 4 	& 3.846 & 82.355 \\\hline
	Error	& 0.42	& 9 & 0.0467	& \\\bottomrule[1.5pt]
	Total & 191.704 & 17 & & \\
	\end{tabular}
\end{table}
\begin{table}[h]
	\centering
	\extrarowheight=4pt 
	\caption{定義欄寬示範}\label{right}
	\begin{tabular}{lp{7cm}}
	\hline
	支配權 & 權利人得直接支配權利客體之權利;如民法中的物權均為支配權。\\\hline
	請求權 & 即請求他人為一定的作為或不作為的權利;如債權人對債務人得行使債之關係上之權利。\\\hline
	形成權 & 即由當事人一方的意思表示,使已成立的法律關係發生得喪變更的權利。\\\hline
	抗辯權 & 在法律上將對抗他人權利行使的一種權利。\\\hline
	\end{tabular}
\end{table}

\newpage
\subsection{{\CR 跨欄表格}}
本小節介紹跨欄表格的使用方法,許多表格在呈現上會以跨欄的方式出現,使用$\backslash$multicolumn指令配合cline畫直線來創造跨欄的表格,編寫格式如下:\\
\begin{table}[h] 
	\centering
	\begin{tabular}{p{0.2\textwidth}cccc}
	\rowcolor{slight}
	{\C $\backslash$multicolumn} & \{3\} & \{c\} & \{年份\}  & $\backslash\backslash\backslash$cline\{2-4\} \\\toprule[1.2pt]
				&	  跨的欄數 & 置中(左右)& 欄標籤 & 跨2-4欄畫直線 \\
	\end{tabular}
\end{table}

利用這樣的指令製作表\ref{mul},可以看到年份欄標籤跨三個欄位置中擺放,練習這樣的表格可以參考表格製作練習1.。
\begin{table}[h] 
	\centering 
	\caption{跨欄表格}\label{mul}  
	\extrarowheight=3pt   
	\begin{tabular}{lccc}
     \hline  
     		& \multicolumn{3}{c}{年份}\\\cline{2-4}
			& 2001 & 2002 & 2003  \\\hline
  GDP  & 4.4  & 10.9 & 8.8		\\\hline  
  Domestic demand     & 6.5     & 16.9 & 10.4		\\\hline
  Next export  & 2.1     & 6.0	& 1.5		\\\hline
	\end{tabular}
\end{table}

\newpage
\subsection{{\CR 旋轉表格}}
有時候遇到比較長的表格,就可利用將表格旋轉方式擺放,這裡示範旋轉表格的寫法,採用graphicx套件,是利用$\backslash$rotatebox[origin=c]\{90\}\{ tabular \} 將整個表格的指令包住,也就是整個向左轉90度,表\ref{rota} 是將表格依序旋轉90度、180度、270度和360度的結果。
\begin{center}
	\colorbox{pink}{\begin{tabular}{p{0.9\textwidth}}
	{\G $\backslash$usepackage\{graphicx\}}
\end{tabular}}
\end{center}
\begin{table}[h] 
	\centering %置中的另一個指令
    \caption{旋轉表格}\label{rota}  %加入標題與標號參照的文字
    \bigskip
    \rotatebox[origin=c]{90}{
	\begin{tabular}{|l|ccc|}
	\hline 
  GDP  & 4.4  & 10.9 & 8.8		\\\hline  % &代表換欄 \\代表換下一列
  Domestic demand     & 6.5     & 16.9 & 10.4		\\\hline
  Next export  & 2.1     & 6.0	& 1.5		\\\hline
	\end{tabular}}\hspace{20pt}
	\rotatebox[origin=c]{180}{
	\begin{tabular}{|l|ccc|}
	\hline 
  GDP  & 4.4  & 10.9 & 8.8		\\\hline  % &代表換欄 \\代表換下一列
  Domestic demand     & 6.5     & 16.9 & 10.4		\\\hline
  Next export  & 2.1     & 6.0	& 1.5		\\\hline
	\end{tabular}}\hspace{20pt}
		\rotatebox[origin=c]{270}{
	\begin{tabular}{|l|ccc|}
	\hline 
  GDP  & 4.4  & 10.9 & 8.8		\\\hline  % &代表換欄 \\代表換下一列
  Domestic demand     & 6.5     & 16.9 & 10.4		\\\hline
  Next export  & 2.1     & 6.0	& 1.5		\\\hline
	\end{tabular}}\hspace{20pt}
		\rotatebox[origin=c]{360}{
	\begin{tabular}{|l|ccc|}
	\hline 
  GDP  & 4.4  & 10.9 & 8.8		\\\hline  % &代表換欄 \\代表換下一列
  Domestic demand     & 6.5     & 16.9 & 10.4		\\\hline
  Next export  & 2.1     & 6.0	& 1.5		\\\hline
	\end{tabular}}
\end{table}

\newpage
\subsection{{\CR 長型表格}}
當遇到一筆大資料,可能跨或整個頁面,針對這樣的長型表格有他的表達方式,使用tabular表格又想要跨頁的話可以用 longtable套件:
\begin{center}
	\colorbox{pink}{\begin{tabular}{p{0.9\textwidth}}
	{\G $\backslash$usepackage\{longtable\}}
\end{tabular}}
\end{center}
\bigskip
範例使用圖\ref{bike} 的Bike data擷取部份資料制程表\ref{longtable},我們從新的一頁開始來看整個表格的樣貌,從前面的$\backslash$begin\{table\}指令改為$\backslash$begin\{longtable\},在輸入長型表格的內容資料之前要先定義每一頁的表頭,還有換頁時的欄位,指令表達方式如下:要先設定好第一頁的結束與下耶頁開始前的表達。
\begin{enumerate}[1]
\item $\backslash$begin\{longtable\}\{ \}
\item $\backslash$toprule
\item 表頭的欄位標題名稱(Day \& Season  \& Temp \& Hum \&  Windspeed $\backslash\backslash$)
\item $\backslash$midrule
\item $\backslash$endfirsthead
\item $\backslash$multicolumn\{ \}\{l\}\{承接上頁\}$\backslash\backslash$
\item $\backslash$toprule
\item  再重複一遍欄標題名稱(Day \& Season  \& Temp \& Hum \&  Windspeed $\backslash\backslash$
\item $\backslash$midrule
\item $\backslash$multicolumn\{ \}\{r\}\{續接下頁\}
\item $\backslash$endfoot
\item $\backslash$endlastfoot
\end{enumerate}

\begin{figure}[h]
    \centering
    \includegraphics[scale=0.3]{\imgdir{bike.jpeg}} 
	\caption{自行車租借資料截圖}
    \label{bike}
\end{figure}

\begin{longtable}{rrrrr}
\caption{長型表格}\label{longtable}\\
\toprule
Day & Season  & Temp & Hum &  Windspeed\\
\midrule
\endfirsthead
\multicolumn{5}{l}{承接上頁}\\
\toprule
Day & Season  & Temp & Hum &  Windspeed\\
\midrule
\endhead
\midrule
\multicolumn{5}{r}{續接下頁}
\endfoot
\endlastfoot
2011/1/1&1&0.24&0.81&0\\
2011/1/1&1&0.22&0.8&0\\
2011/1/1&1&0.22&0.8&0\\
2011/1/1&1&0.22&0.81&0\\
2011/1/1&1&0.24&0.8&0\\
2011/1/1&1&0.24&0.81&0\\
2011/1/1&1&0.24&0.75&0\\
2011/1/1&1&0.22&0.75&0.89\\
2011/1/1&1&0.2&0.81&0\\
2011/1/1&1&0.24&0.81&0\\
2011/1/1&1&0.32&0.86&0\\
2011/1/1&1&0.38&0.81&0\\
2011/1/1&1&0.36&0.72&0\\
2011/1/1&1&0.42&0.77&0.2357\\
2011/1/1&1&0.46&0.88&0.3458\\
2011/1/1&1&0.46&0.81&0.3357\\
2011/1/1&1&0.48&0.76&0.2887\\
2011/1/1&1&0.44&0.77&0.3357\\
2011/1/1&1&0.45&0.81&0.2985\\
2011/1/1&1&0.44&0.8&0.2357\\
2011/1/2&1&0.44&0.81&0.2836\\
2011/1/2&1&0.4&0.87&0.2985\\
2011/1/2&1&0.44&0.87&0.2239\\
2011/1/2&1&0.44&0.88&0.2537\\
2011/1/2&1&0.46&0.94&0.2836\\
2011/1/2&1&0.43&0.94&0.3890\\
2011/1/2&1&0.44&0.81&0.194\\
2011/1/2&1&0.44&0.81&0.178\\
2011/1/2&1&0.4&0.71&0.2339\\
2011/1/2&1&0.42&0.76&0.2230\\
\bottomrule
\end{longtable}

\newpage
\section*{{\CB 表格製作練習}}
\begin{enumerate}
\item 課堂上老師出了一個小挑戰看怎麼樣製造出這樣的表格呢?

\begin{figure}[h]
    \centering
    \includegraphics[scale=0.38]{\imgdir{Table_quiz.jpg.png}} 
	\caption{ Parameter estimates for joint models based on 500 simulation samples.}
    \label{quiz}
\end{figure}

\bigskip
此處在兩個跨欄位的交接處使用了p\{0.001cm\}來製造中間的空隙
\begin{table}[h]
    \centering 
    \caption{Parameter estimates for joint models based on 500 simulation samples.}\label{P}
    \bigskip
    \extrarowheight=2pt
    \begin{tabular}{lrccccp{0.001cm}cccc}
    \toprule
           & 				&\multicolumn{9}{c}{samplesize}\\\cline{3-11}
           	&				&\multicolumn{4}{c}{n=100}		&		&\multicolumn{4}{c}{n=200}\\\cline{3-6}	\cline{8-11}
           &				&Estimate		&$\mbox{SEE}^{a)}$	&$\mbox{SEE}^{b)}$  &$\mbox{CP}^{c)}$	& &Estimate	&SEE  &SSE  &CP \\\hline
    $\beta_1$	&-0.05     & -0.065		& 0.025	&0.024   &88.98  & &-0.056 	&0.010 &0.011 &89.81 \\\hline
    $\beta_1$	& 0.05     & 0.065		&0.024 	&0.023   &90.85  & &0.056	&0.009 &0.010 &89.81 \\\hline
    $\gamma_0$	& 0.5     & 0.785		&1.395 	&1.355   &93.73  & &0.599	&0.750 &0.772 &93.96 \\\hline
    $\gamma_1$	& 0.5     & 0.612		&0.880 	&0.887   &94.24  & &0.561	&0.466 &0.487 &94.91 \\\bottomrule
    \end{tabular}
\end{table}

\newpage
\item 引用汪群超老師期中報告的評分標準檔,試試看做出跟他一樣的如表\ref{comm}!%參考資料網址:http://web.ntpu.edu.tw/~ccw/statmath/word_score.pdf 
\begin{table}[h]
	\centering
	\caption{評分標準}\label{comm}
	\extrarowheight=4pt 
	\begin{tabular}{llr}
	\rowcolor[gray]{.9}
	類別 & 項目 & 分數(\%)\\\toprule[1.2pt]
	外在結構 & 裝訂、序文、目錄(章節、圖、表)*內容、參考書目 & 20\\\hline
	內在結構 & 章節分明 & 5\\
				&樣式精簡明確* & 10\\
				&頁首頁尾應用* & 10 \\
				&版面設計的痕跡(企圖心)& 10\\
				&基本技巧的應用(自動編號、圖表編號、交互參照、功能變數...)*& 25\\\hline
	整體表現& 用心程度、藝術美感、文字的運用... & 20\\\bottomrule[1pt]
	\end{tabular}\par\smallskip 
  \parbox{10cm}{ *:參考電子檔}
\end{table}

\item 練習看看製作這張圖片\footnote{資料來源:UCLA education, www.ats.ucla.edu}的表格\\
 \begin{figure}[h]
    \centering
    \includegraphics[scale=0.55]{\imgdir{spss_output.jpg}}  % scale等比例放大或縮小 {路徑} imagdir是自己定義的路徑
    \caption{ SPSS Annotated Output Regression Analysis}
    \label{spss}
\end{figure}

因為橫向擺放表格的話會超出頁面,所以這邊使用橫向擺放,表\ref{pratice_3} 可以看到其實呈現的樣子並不是太好看,還有地方尚需改善,此部分會再詢問較佳的做法。
\newpage
\begin{table}[H] 
	\centering 
	\caption{Coefficients}\label{pratice_3}
	\rotatebox[origin=c]{90}{
	\begin{tabular}{|ll|cc|c|c|c|cc|}
	\toprule
		&	& \multicolumn{2}{c}{Unstandardlizerd} & Standarlized & & & & \\
		&	& \multicolumn{2}{c}{ Coefficients} & Coefficients & & &\multicolumn{2}{c}{95\% Confidence interval for B }\\\cline{3-5} \cline{8-9}
	\multicolumn{2}{l}{Model} & B & Std.Error & Beta & t & Sig & Lower Bound & Upper Bound\\\midrule
	1	& (Constant) & 12.325 & 3.194 & & 3.859 & .000 & 6.027 & 18.624 \\
		& math source & .389 & .074 & .368 & 5.252 & .000 & .243 & .535\\
		& female & -2.010 & 1.023 & -1.01 & -1.965 & .051 & -4.027 &.007\\
		& social studies source & .050 & .062 & .054 & .801 & .424 & -.073 & .173\\
		& reading source & .335 & .073 & .347 & 4.607 & .000 & .192 & .479\\\bottomrule
		 & \multicolumn{8}{l}{a.Dependent variable:science score} 
    \end{tabular}}
\end{table}

\end{enumerate}

\newpage
\section{{\CB 圖片的匯入與放置}}
圖文的放置在文書報告裡是很重要的一環,怎麼樣讓圖片放置得宜也是一門藝術,\LaTeX 利用指令的放置圖片,主要採用graphicx套件,一般常見的圖片檔有JPG/PNG/PDF/EPS等,而其中EPS檔是呈現起來最好看的圖檔,但因為時代演進壓縮檔的出現,JPG檔開始廣為流行,成為算是大家最常使用的圖檔類型,但也因為其有壓縮過的原因,所以在一些顏色轉換的地方會顯得沒那麼美麗,而針對這麼多種檔案\LaTeX 還有一個厲害之處,就是可以讓這些圖檔同時引用與存在,但是當你把所有的圖都放在現有目錄裡的話會顯得很雜亂,所以通常都會使用一個資料夾將所有圖片集中管理,引用的時候也比較方便,因此在一般的做法上會先定義一個指令:
\begin{center}
	\colorbox{pink}{\begin{tabular}{p{0.9\textwidth}}
	{\G $\backslash$newcommand$\lbrace\backslash$imgdir$\rbrace\lbrace\backslash$images/$\rbrace$}
\end{tabular}}
\end{center}
意思是我自己定義了imgdir這個指令表示在現在的目錄去找到images這個資料夾,目的是在輸入圖檔時從資料夾裡找到要用的圖,另外如果圖檔資料夾裡的圖片是很多tex檔都會重複用到的話,可以用\{..$\backslash$imgdir$\backslash$\}這樣的路徑來指定回上一層,再去找到images的資料夾。

\subsection{{\CR 基本圖片放置}}
\begin{figure}[h]
    \centering
    \includegraphics[scale=0.7]{\imgdir{ntpubird.jpg}}
    \caption{北大鳥}
    \label{ntpubird}
\end{figure}
基本使用的指令是$\backslash$begin\{figure\}來擺放圖片位置,由於前面已經定義過指令,可以直接使用  \colorbox{slight}{{\G $\backslash$includegraphics[scale=0.5]\{$\backslash$imgdir\{name.jpg\}\}}}來放置圖片,指定路徑找到要輸入的圖檔位置,圖\ref{ntpubird} 示範插入一張JPG檔,用scale=0.7表示將圖等比例縮放成0.7的大小,而圖\ref{cap} 是螢幕截圖PNG檔。
\bigskip
\begin{figure}[h]
    \centering
        \includegraphics[scale=0.3]{\imgdir{cap.PNG}}
    \caption{螢幕截圖檔}
    \label{cap}
\end{figure}

\newpage
\subsection{{\CR 圖片並排}}
不只可以有整個的圖標籤,也可以有各自圖片的標籤,然後針對每張圖的縮放大小都可以自由做調整,使用語法如下:用subfloat指令可以並排多張圖片,當並排圖片過多需要換行時,只需在後面加上$\backslash\backslash$即可。
\begin{center}
	\colorbox{pink}{\begin{tabular}{p{0.9\textwidth}}
	{\G $\backslash$subfloat[此圖標籤]\{$\backslash$includegraphics[ ]\{ .eps\}\}}
\end{tabular}}
\end{center}

\begin{figure}[h]
    \centering
        \subfloat[縱向分群]{
        \includegraphics[scale=0.3]{\imgdir{1.eps}}}
        \subfloat[橫向分群]{
        \includegraphics[scale=0.3]{\imgdir{2.eps}}}\\
          \subfloat[是否失敗分組]{
        \includegraphics[scale=0.4]{\imgdir{a.eps}}}
        \subfloat[手術方式分組]{
        \includegraphics[scale=0.4]{\imgdir{b.eps}}}
        \subfloat[Density plot]{
        \includegraphics[scale=0.4]{\imgdir{c.eps}}}
    \caption{圖形並排的作法}
    \label{parallel}
\end{figure}

\newpage
\subsection{{\CR 圖片旋轉}}
在圖片旋轉的部分與前面的表格一樣都是逆時針的旋轉,因此在設定旋轉角度的時候要注意一下,原本編寫scale指令的地方,可以再加上angle=30,或是width=10cm, height=7cm 等多種設定,
\begin{figure}[h]
    \centering
     \includegraphics[angle=30,width=10cm, height=7cm]{\imgdir{2.jpg}}
    \caption{旋轉圖形}
    \label{angle}
\end{figure}
\begin{figure}[H]
    \centering
     \includegraphics[angle=287,scale=0.4]{\imgdir{d.jpeg}}
    \caption{旋轉圖形}
    \label{angle}
\end{figure}


\begin{center}
{\CB 這邊綜合總結一下圖片可以做的個人化改變,參考下表}
\end{center}
\bigskip
\begin{table}[h]
	\centering
	\caption{圖片設定參照}
	\begin{tabular}{p{2.5cm}l}
	指令 & 意義\\\toprule
	scale & 按一定比例縮放,指的是縮的倍數。\\
	angle & 逆時針方向旋轉的角度。\\
	orgin & 旋轉的中心點。\\
	width & 指圖形的寬度,會自動伸縮調整,長度亦會等比例調整。\\
	height & 指圖形的高度,會自動伸縮調整,寬度亦會等比例調整。\\
	totalheight & 指圖形的總高度。\\\bottomrule[1pt]
	\end{tabular}
\end{table}

\end{document}
