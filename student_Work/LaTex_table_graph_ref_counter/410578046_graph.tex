\documentclass[12pt, oneside, a4paper]{book}
%\documentclass[12pt, a4paper]{book}
%----- 定義使用的 packages ----------------------

\usepackage{fontspec} 										% Font selection for XeLaTeX; see fontspec.pdf for documentation. 
\usepackage{xeCJK}											% 中文使用 XeCJK,但利用 \setCJKmainfont 定義粗體與斜體的字型
\defaultfontfeatures{Mapping=tex-text} 				% to support TeX conventions like ``---''
\usepackage{xunicode} 										% Unicode support for LaTeX character names (accents, European chars, etc)
\usepackage{xltxtra} 											% Extra customizations for XeLaTeX
\usepackage[sf,small]{titlesec}
\usepackage{amsmath, amssymb}
\usepackage{amsthm}										% theroemstyle 需要使用的套件
\usepackage{bm}                                                 % 排版粗體數學符號
\usepackage{enumerate}
\usepackage{graphicx, subfig, float} 					% support the \includegraphics command and options
\usepackage{array}
\usepackage{color, xcolor}
\usepackage{longtable, lscape}                                   % 跨頁的超長表格;lscape是旋轉此類表格的
\usepackage{threeparttable}                                     % 巨集,使表格加註解更容易(手冊p169)
\usepackage{multirow, booktabs}                                   % 讓表格編起來更美的套件(手冊p166),編輯跨列標題重覆的表格(手冊p182)
\usepackage{colortbl}                          				%.............................................表格標題註解之巨集套件
\usepackage{natbib}											% for Reference
\usepackage{makeidx}										% for Indexing
\usepackage[parfill]{parskip} % Activate to begin paragraphs with an empty line rather than an indent
%\usepackage{geometry} % See geometry.pdf to learn the layout options. There are lots.
%\usepackage[left=1.5in,right=1in,top=1in,bottom=1in]{geometry} 
\usepackage{url}                                                % 文稿內徵引網址
    \def\UrlFont{\rm}                                           % 網頁
\usepackage{fancyhdr}
	\pagestyle{fancy}
	\fancyhf{}                                     % 清除所有頁眉頁足
	\renewcommand{\headrulewidth}{0pt}                              % 頁眉下方的橫線    
%-----------------------------------------------------------------------------------------------------------------------
%  主字型設定
\setCJKmainfont
	[
		BoldFont=Heiti TC Medium								% 定義粗體的字型(依使用的電腦安裝的字型而定)
	]
	{cwTeX Q Ming Medium} 										% 設定中文內文字型
%	{新細明體}	
\setmainfont{Times New Roman}								% 設定英文內文字型
\setsansfont{Arial}														% used with {\sffamily ...}
%\setsansfont[Scale=MatchLowercase,Mapping=tex-text]{Gill Sans}
\setmonofont{Courier New}										% used with {\ttfamily ...}
%\setmonofont[Scale=MatchLowercase]{Andale Mono}
% 其他字型(隨使用的電腦安裝的字型不同,用註解的方式調整(打開或關閉))
% 英文字型
\newfontfamily{\E}{Cambria}										% 套用在內文中所有的英文字母
\newfontfamily{\A}{Arial}
\newfontfamily{\C}[Scale=0.9]{Cambria}
\newfontfamily{\T}{Times New Roman}
\newfontfamily{\TT}[Scale=0.8]{Times New Roman}
% 中文字型
\newCJKfontfamily{\MB}{微軟正黑體}							% 適用在 Mac 與 Win
\newCJKfontfamily{\SM}[Scale=0.8]{新細明體}				% 縮小版
%\newCJKfontfamily{\K}{標楷體}                        			% Windows 下的標楷體
\newCJKfontfamily{\K}{Kaiti TC Regular}         			% Mac OS 下的標楷體
\newCJKfontfamily{\BM}{Heiti TC Medium}					% Mac OS 下的黑體(粗體)
\newCJKfontfamily{\SR}{Songti TC Regular}				% Mac OS 下的宋體
\newCJKfontfamily{\SB}{Songti TC Bold}					% Mac OS 下的宋體(粗體)
\newCJKfontfamily{\CF}{cwTeX Q Fangsong Medium}	% CwTex 仿宋體
\newCJKfontfamily{\CB}{cwTeX Q Hei Bold}				% CwTex 粗黑體
\newCJKfontfamily{\CK}{cwTeX Q Kai Medium}   		% CwTex 楷體
\newCJKfontfamily{\CM}{cwTeX Q Ming Medium}		% CwTex 明體
\newCJKfontfamily{\CR}{cwTeX Q Yuan Medium}		% CwTex 圓體
%-----------------------------------------------------------------------------------------------------------------------
\XeTeXlinebreaklocale "zh"                  				%這兩行一定要加,中文才能自動換行
\XeTeXlinebreakskip = 0pt plus 1pt     %這兩行一定要加,中文才能自動換行
%-----------------------------------------------------------------------------------------------------------------------
%----- 重新定義的指令 ---------------------------
\newcommand{\cw}{\texttt{cw}\kern-.6pt\TeX}	% 這是 cwTex 的 logo 文字
\newcommand{\imgdir}{graph/}							% 設定圖檔的位置
\renewcommand{\tablename}{表}						% 改變表格標號文字為中文的「表」(預設為 Table)
\renewcommand{\figurename}{圖}						% 改變圖片標號文字為中文的「圖」(預設為 Figure)
\renewcommand{\contentsname}{目~錄}
\renewcommand\listfigurename{圖目錄}
\renewcommand\listtablename{表目錄}
\renewcommand{\appendixname}{附~錄}                  
\renewcommand{\indexname}{索引}
\renewcommand{\bibname}{參考文獻}
%-----------------------------------------------------------------------------------------------------------------------

\theoremstyle{plain}
\newtheorem{de}{Definition}[section]				%definition獨立編號
\newtheorem{thm}{定理}[section]			%theorem 獨立編號,取中文名稱並給予不同字型
\newtheorem{lemma}[thm]{引理}				%lemma 與 theorem 共用編號
\newtheorem{ex}{{\E Example}}						%example 獨立編號,不編入小節數字,走流水號。也換個字型。
\newtheorem{cor}{Corollary}[section]				%not used here
\newtheorem{exercise}{EXERCISE}					%not used here
\newtheorem{re}{\emph{Result}}[section]		%not used here
\newtheorem{axiom}{AXIOM}							%not used here
\renewcommand{\proofname}{\textbf{Proof}}		%not used here

\newcommand{\loflabel}{圖} % 圖目錄出現 圖 x.x 的「圖」字
\newcommand{\lotlabel}{表}  % 表目錄出現 表 x.x 的「表」字

\parindent=0pt

%--- 其他定義 ----------------------------------
% 定義章節標題的字型、大小
\titleformat{\chapter}[display]{\raggedleft\LARGE\bfseries\CF}		% 定義章抬頭靠右(\reggedleft)
 { 第\ \thechapter\ 章}{0.2cm}{}
%\titleformat{\chapter}[hang]{\centering\LARGE\sf}{\MB 第~\thesection~章}{0.2cm}{}%控制章的字體
%\titleformat{\section}[hang]{\Large\sf}{\MB 第~\thesection~節}{0.2cm}{}%控制章的字體
%\titleformat{\subsection}[hang]{\centering\Large\sf}{\MB 第~\thesubsection~節}{0.2cm}{}%控制節的字體
%\titleformat*{\section}{\normalfont\Large\bfseries\MB}
%\titleformat*{\subsection}{\normalfont\large\bfseries\MB}
%\titleformat*{\subsubsection}{\normalfont\large\bfseries\MB}


% 顏色定義
\definecolor{heavy}{gray}{.9}								% 0.9深淺度之灰色
\definecolor{light}{gray}{.8}
\definecolor{pink}{rgb}{0.99,0.91,0.95}               % 定義pink顏色
  


\title{ \LaTeX{\UD 圖片應用}}
\author{{\NC 張馨云}\;\; {\JF 410578046}}
\date{{\BR \today}} 	

\begin{document}
\maketitle
\fontsize{12}{22 pt}\selectfont

\centerline{{\BCF 引言}}
\setlength{\parindent}{2em} 
	除了表格之外,圖形也是筆者傳達訊息的有效工具之一,無論是一般文稿還是學術論文,都經常使用到圖形。\LaTeX 雖然可以繪製簡單的線條圖形,但複雜的圖形就需要引入外製圖形來表達,本文將介紹 \LaTeX 是如何引用這些外製圖形,以及這些圖形是如何依靠指令與參數做排版上的變化。

\section{{\CB 圖形檔案規格}}
	圖形檔案規格基本上分成兩大類:一為點陣圖(bit-mapped),一為向量圖(vector-based):
	\begin{enumerate}[i. ]
	\item 點陣圖(bit-mapped) \\
	點陣圖的使用相當廣泛,一般我們使用的圖形檔案大多為點陣圖。點陣圖與向量圖之間最大的差異就在,點陣圖會因為縮方而失真:點陣圖使用自然的方式來儲存數位資料,圖形所佔的頁面由許多細小的方格所組成,每一個小方格代表一個圖素(pixel),表示不同顏色。單位的小方格數越多,解析度越高。當點陣圖放大時,圖素固定,視覺上會看到一個個的小格子,影像因此失真。\\
	常見的點陣圖圖檔格式有:jpeg, jpg, gif, bmp, ico, xpm, png, psd, tiff...etc.
	\item 向量圖(vector-based) \\
	與點陣圖不同的是,向量圖僅儲存數學運算的基本描述,顯像時再計算出結果來顯示。以圓形圖為例,圖檔僅儲存圓心所在、圓半徑、顏色索引值等資料,要顯示時馬上計算顯像。由於每次顯像都會重新計算,無論放大或縮小皆不會造成圖形失真。 \\
	常見的向量圖圖檔格式有:eps, ps, pdf, ai, svg...etc.
	\end{enumerate}



\section{{\CB 引入圖形}}
	引用外製圖形,將會使用到graphix套件,向量圖以及點陣圖皆可引入。 \LaTeX 處理圖形的方法是透過預覽/列印程式,故而使用graphix套件時,必須指定驅動程式,常見驅動程式有 dvips, pdftex, 以及 dvipdfm等等,若使用前兩者,\LaTeX 會自動判斷,故使用套件時不須以選項設定;若使用 dvipdfm 或 dvipdfmx 程式,使用套件時須加入 dvipdfm 選項。 \\
	\indent 與表格使用的table浮動環境相似的是,$\backslash$begin$\{\backslash$figure$\}$ 進入figure浮動環境,用以控制圖形位置。graphix套件中,引入圖片的指令為: \\
	\indent $\backslash$includegraphics[圖形選項參數]$\{$路徑名稱/檔案名稱.檔案類型$\}$  \\	
	\indent 預設的圖形檔路徑與文章相同,若圖形非至於此,須指定完整的\footnote{在\LaTeX 中,所有指令都以反斜線「$\backslash$」表示,而路徑皆以斜線「/」表示。} 路徑以引入圖形。在引用圖形時,若使用圖形的絕對路徑,可能會造成日後在不同電腦編譯上的問題,為解決此問題,我們創建一個位於文章當前所在位置的子目錄,將圖形置於其中,其檔案之相對路徑為「子目錄名稱/檔案名稱.檔案類型」。而為避免在編寫文件時,子目錄的名稱更動造成編譯問題,我們在定義區設定一個新的路徑命令「$\backslash$imgdir」,以\footnote{$\backslash$newcommand$\{$imgdir$\}\{$images/$\}$ \\}$\backslash$newcommand定義了與編譯文章路徑相同的子目錄(images),用以更彈性的更動檔案所在的子目錄: \\ 
 	\indent 引入圖形的指令變為: \\
 	\indent $\backslash$includegraphics$\{$[圖形選項參數]$\backslash$imgdir$\{$檔案名稱.檔案類型$\}\}$
 	
\subsection{{\CB 圖形選項參數}}\label{fig_option}
	graphix套件之中,圖形選項的參數眾多,可依需求調整圖形的各項數值。置入的選項參數可以有多個,以逗點來分隔,設定值則是使用等號,以下屆紹graphix套件之中的圖形選項參數:
	\begin{enumerate}[i. ]
	\item	bb:Bounding Box \\
			設定圖形檔案的邊界,由圖形左下角座標與圖形右上角座標組成,每個值以空白值隔開。參考標準是可被印出紙張的左下角為 (0,0)。請注意,如果沒有指定單位的話,那預設是 bp。而且,這個設定在 pdflatex 會不被接受,此時請改使用 trim 選項參數。
一般如果是 eps/ps 檔,可以使用編輯器去修改他的 Bounding Box 值,無需用到這 些不好控制的參數,如何抓到座標值呢?使用 gv 或 gsview 把圖檔載入後,將游標置於圖中所要的位置,這些軟體就會顯示所在處的座標,然後就可以依自己需要去修改他了。
	\item	clip:修剪圖的四周指定的邊緣。\\
			作用同bb,由四個要去除的部分長度值組成(預設單位bp)。可用於pdflatex。
	\item	angle:旋轉的角度。 \\
			方向為逆時針,使用負數角度時,方向為順時針。
	\item 	orgin:旋轉的中心點。
	\item 	width:圖形的寬度。\\
			自動伸縮調整,長度亦會等比例調整。如若要將原圖調整為內文行寬的X倍,則以[width=X$backslash$textwidth]設置。
	\item 	height:圖形的高度。 \\
			自動伸縮調整,高度亦會等比例調整。
	\item 	totalheight:這是指圖形的總高度。\\
			height 再加上 depth 的值。會自動伸縮調整,寬度亦會等比例調整。
	\item  	scale:縮放倍數(等比例)。	
	\end{enumerate}
 	

\subsection{{\CB 圖形排版}}
\subsubsection{{\CB 子圖並列}}	
	多圖並列時,我們會使用到subfig套件,相關指令用法如下: \\
	\indent $\backslash$subfloat[欲並列的子圖的名稱]{引入欲並列的子圖} \\
	\indent 在subfloat之中加入$\backslash$label,定義參照名稱,可為並列的子圖加上圖標號以便於內文參照。而一個subfloat對應圖中的一列,若一行放2個圖形,則可將圖形調整至0.4-0.5倍,一行內多圖時,為避免圖形超出版面,切記調整圖形縮放倍數。	\\
	\indent 以R語言內建數據lung畫出的圖為例,我們製作出包含圖\ref{ex_para_1} 以及圖\ref{ex_para_2} 的圖\ref{ex_para}。	
		\begin{figure}[H]
		\centering
		\subfloat[lung的KM survival]{
		\label{ex_para_1}
		\includegraphics[scale=0.4]{\imgdir{survlung.eps}}
		}
		\subfloat[lung的cumulative events]{
		\label{ex_para_2}
		\includegraphics[scale=0.4]{\imgdir{cumlung.eps}}
		}
		\caption{兩圖形並列} \label{ex_para}
		\end{figure}

	
\subsubsection{{\CB  2 by 2 圖形排列}}				
	同樣以R語言內建數據lung畫出的圖為例,製作圖\ref{ex_2by2} :
		\begin{figure}[H]
		\centering
			\subfloat[lung的KM survival]{
			\label{ex_2by2_1}
			\includegraphics[scale=0.4]{\imgdir{survlung.eps}} 
			} 
			\hspace{0.01\linewidth} 
			\subfloat[lung的Cumulative events]{
			\label{ex_2by2_2}
			\includegraphics[scale=0.4]{\imgdir{cumlung.eps}}
			} 
			\vfill
			\subfloat[lung的log(SurvProb)]{
			\label{ex_2by2_3}
			\includegraphics[scale=0.4]{\imgdir{logSurvlung.eps}}
			} 
			\hspace{0.01\linewidth}
			\subfloat[lung的Cumulative hazard]{
			\label{ex_2by2_4}
			\includegraphics[scale=0.4]{\imgdir{CunHarzlung.eps}}
			}
		\caption{2 by 2圖形編排} \label{ex_2by2}
		\end{figure}
\subsubsection{{\CB 多圖排列}}		
	開啟figure浮動環境之後,利用表格的方式來實現多圖形的排列,製作出圖\ref{ex_mutifig}:
		\begin{figure}[H]
		\centering
			\begin{tabular}{cccc}
			\subfloat[A]{\includegraphics{\imgdir{LaTeX.png}}} & 
			\subfloat[B]{\includegraphics{\imgdir{LaTeX.png}}} & 
			\multirow{-3}[2]{*}{\subfloat[D]{\includegraphics[width=2cm,height=4cm]{\imgdir{LaTeX.png}}}} &
			\multirow{-3}[-3]{*}{\subfloat[E]{\includegraphics[width=2cm,height=3cm]{\imgdir{LaTeX.png}}}} \\
			\subfloat[E]{\includegraphics{\imgdir{LaTeX.png}}} & 
			\subfloat[F]{\includegraphics{\imgdir{LaTeX.png}}}\\
			\subfloat[G]{\includegraphics{\imgdir{LaTeX.png}}}&
			\subfloat[H]{\includegraphics{\imgdir{LaTeX.png}}} \\
			\end{tabular}
			\caption{多圖編排} \label{ex_mutifig}
		\end{figure}
\subsubsection{{\CB 文繞圖}}	
	當有文繞圖需求時,我們會使用到wrapfig套件提供的wrapfigure浮動環境,相關指令用法如下: \\
	\indent $\backslash$begin$\{$wrapfigure$\}\{$行數$\}\{$位置$\}$[超出長度]$\{$宽度$\}$\\
	\indent 引用圖形	 \\
	\indent	$\backslash$begin$\{$wrapfigure$\}$ \\
	\begin{wrapfigure}{r}{0.2\textwidth}
		\centering
		\includegraphics[width=0.2\textwidth]{\imgdir{wolf.jpg}}
		\caption{狼圖騰}\label{ex_aroundtext}
	\end{wrapfigure}
	\indent 其中,行數指的是引用圖形的高度所占文章的行數,若無設置則\LaTeX 將會自動計算。位置則又由rlio控制,r與l分別表示right與left,表示圖形位於文章的右/左側,而i表示圖形位於頁面靠裏的一邊,o表示圖形位於頁面靠外的一邊。超出長度則是指圖形超出文章邊界的長度,預設值為0pt。\\	
	使用wrapfigure浮動環境有幾點事項需要注意:
	\begin{itemize}
		\item[•] 在 wrapfigure環境之後必須緊接著輸入段落文字,否則可能會出現錯誤。
		\item[•] 不能在任何列表的環境之中使用wrapfigure環境,亦不能在列表環境前後使用,除非兩者之間有空一行或分段指令($\backslash$par)
		\item[•] 在雙欄版面模式不可使用
	\end{itemize}
\subsubsection{{\CB 加入圖標號}}
	與表標號不同的是,一般圖標號在排版上都是置於圖形下方。在figure環境之中,將命名圖片的$\backslash$caption指令置於引入圖片之後即可。而若要更改subfloat之中,子圖標號的編號方式,可於引用圖形之前,參考表\ref{CounterStyle},以$\backslash$renewcommand重新定義:\\
	\indent	$\backslash$renewcommand$\{\backslash$thesubfigure$\}\{\backslash$計數形式$\{$subfigure$\}\}$
	\begin{table}[H]
 	\centering
 	\extrarowheight=3pt
    \caption{計數形式}\label{CounterStyle}
		\begin{tabular}{lcc} 
		\hline
		Counter Style \quad & \quad Code & \quad Example \\
		\hline
		數字 & $\backslash$arabic$\{$counter$\}$ & 	1, 2 \\
		小寫英文字母 & $\backslash$alph$\{$counter$\}$ & 	a, b \\
		大寫英文字母 & $\backslash$Alph$\{$counter$\}$ & 	A, B \\
		小寫羅馬數字 & $\backslash$roman$\{$counter$\}$ & i, ii \\
		大寫羅馬數字 & $\backslash$roman$\{$counter$\}$ & I, II \\
		\hline
		\end{tabular}
	\end{table}
	\indent 以圖\ref{ex_para}中的圖形為例,更改子圖以羅馬數字編號,製作圖\ref{ex_label_roman}:
	\renewcommand{\thesubfigure}{\roman{subfigure}}
	\begin{figure}[H]
		\centering
		\subfloat[lung的KM survival]{
		\label{ex_label_roman_1}
		\includegraphics[scale=0.4]{\imgdir{survlung.eps}}
		}
		\subfloat[lung的cumulative events]{
		\label{ex_label_roman_2}
		\includegraphics[scale=0.4]{\imgdir{cumlung.eps}}
		}
		\caption{以羅馬數字編號子圖} \label{ex_label_roman}
		\end{figure}
		
	\indent 以各檔案類型的圖形為例,更改子圖以阿拉伯數字編號,製作圖\ref{ex_format}:
		\renewcommand{\thesubfigure}{\arabic{subfigure}}
		\begin{figure}[H]
		\centering
			\subfloat[eps檔案 ]{
			\label{ex_eps}
			\includegraphics[width=0.4\textwidth]{\imgdir{surv.eps}} 
			} 
			\hspace{0.01\linewidth} 
			\subfloat[bmp檔案]{
			\label{ex_bmp}
			\includegraphics[width=0.4\textwidth]{\imgdir{surv1.bmp}}
			} 
			\\
			\subfloat[pdf檔案]{
			\label{ex_pdf}
			\includegraphics[width=0.4\textwidth]{\imgdir{surv2.pdf}}
			} 
			\hspace{0.01\linewidth}
			\subfloat[png檔案]{
			\label{ex_png}
			\includegraphics[width=0.4\textwidth]{\imgdir{chi_F.png}}
			}
		\caption{各個檔案類型的圖形} \label{ex_format}
		\end{figure}
\subsubsection{{\CB 旋轉圖片}}
	選轉圖片時,會利用\ref{fig_option}提到過的angle選項:
	\begin{figure}[H]
    \centering
        \includegraphics[angle=45, scale=0.4]{\imgdir{3group_knn.eps}}
    \caption{逆時鐘旋轉圖片}\label{ex_angle}
	\end{figure}
\subsubsection{{\CB 變更圖片樣式}}
	別於wrapfig套件提供的wrapfigure環境,\footnote{\LaTeX 需要手動安裝此套件,否則在載入此套件時,會出現'File "picins.sty" not Found.'的錯誤訊息。至\href{https://ctan.org/tex-archive/macros/latex209/contrib/picins/}{picins載點}下載完整zip檔案,將解壓出來的整個picins目錄置於C:/Program Files/MiKTeX 2.9/tex/latex 之下,再打開MikTeX Console,點選左上角Tasks,選擇"Refresh file name database",手動更新TeX檔案資料庫。} picins套件之中提供的parpic指令,不只能夠將圖形穿插於文章之中,若設定適當的參數,可以改變圖片的樣式,使用方法如表\ref{usage_parpic} 所示:\\
	\begin{table}[!]
	\centering
 	\extrarowheight=4pt
    \caption{parpic參數說明}\label{usage_parpic}
		\begin{tabular}{cl}
		\hline
		\multicolumn{2}{l}{\multirow{2}{*}{$\backslash$parpic(寬度, 高度)(水平位移, 垂直位移)[樣式][位置]{圖形}}} \\
		\\		
		\rowcolor{paleblue}
		\multicolumn{2}{l}{位置選項(僅兩種)} \\
		\quad l & \quad 將圖形置於文章左側(預設值) \\
		\quad r & \quad 將圖形置於文章右側 \\
		\rowcolor{paleblue}
		\multicolumn{2}{l}{樣式選項} \\
		\quad f & \quad 將圖形置於一個實框中 \\
		\quad d & \quad 將圖形置於一個虛框中 \\
		\quad o & \quad 將圖形置於一個園角框中 \\
		\quad s & \quad 將圖形置於一個具有陰影效果的框中 \\
		\quad x & \quad 將圖形置於一個具有立體效果的框中 \\
		\rowcolor{paleblue}
		\multicolumn{2}{l}{\footnote{位置僅當給定的寬度和高度與圖形的實際大小相差很多時才起作用。若水平位移或垂直位移已定義,則此項不起作用。預設為將圖形置於方框中央}位置選項} \\
		\quad l & \quad 將圖形置於框左方 \\
		\quad r & \quad 將圖形置於框右方 \\
		\quad t & \quad 將圖形置於框上方 \\
		\quad b & \quad 將圖形置於框下方 \\
		\hline
		\end{tabular}
	\end{table}
	以圖\ref{ex_aroundtext}的圖為例,改變其樣式,穿插於一段介紹《狼的智慧》的文章中:\\

\parpic{
  \includegraphics[width=2cm]
                  {\imgdir{wolf.jpg}}}
經過萬年的演化,和人一起居住的狼就成了後來的家犬。在過去幾百年裡,我們把家犬在遺傳上改造得連牠們的野狼祖先也都認不出來了。人類成功馴化的哺乳動物其實屈指可數,因為絕大多數動物可以說是不具有可被馴化的天性吧。論馴化的歷史悠久和親密程度,狼的後代可是穩居榜首的位置。
\boxlength{3pt}
\parpic[s]{
  \includegraphics[width=2cm]
                  {\imgdir{wolf.jpg}}}
近年有愈來愈多動物行為學的實驗顯示,狗可能比我們所有靈長動物的表親都更能夠理解人類的語言及非語言訊息,是最能夠有效和人類溝通的非人哺乳動物。或許這都可能是因為牠們的祖先——野狼,在某程度上,和人類真是有幾分相似的。
\boxlength{3pt}
\parpic(2cm,1cm)[xr]{
  \includegraphics[width=2cm]
                  {\imgdir{wolf.jpg}}}
《狼的智慧》作者愛莉•瑞丁格(Elli H. Radinger)是來自德國的野狼觀察員,她原本是學法律的,儘管媒體有時會用狼來比喻一些律師的貪婪和凶狠,但是瑞丁格寧可和野狼相處也不願意在辦公室裡和同行為伍,於是遠赴美國的田野去當志工觀察野狼。在十幾年在田野長期保持距離觀察狼的生活中,她一點也不覺得辛苦和乏味,甚至還從狼身上體悟到不少待人接物的道理。 
\boxlength{10pt}
\parpic(3.5cm,4cm)[or]{
  \includegraphics[width=3cm]
                  {\imgdir{wolf.jpg}}}
過去,狼曾經是是地球上分布地區最廣的哺乳動物,包括北美和歐亞大陸都有狼跡。然而在不斷的獵殺以及棲地大量破壞後,狼在許多地區已絕跡,包括著名的美國黃石國家公園。狼的銷聲匿跡,讓馬鹿等草食動物族群不受控地大幅增加,對植被造成意料之外的破壞,於是黃石國家公園又再從東加拿大引進了野狼,而瑞丁格的主要工作就是觀察牠們,累積了超過十萬次看狼經驗。




%		\center
%		\begin{figure}[H]
%		\caption{tiff檔案}\label{ex_tiff}
%		\includegraphics[scale=0.5]{\imgdir{3group_knn.tiff}}		
%		\end{figure}	
			
%		\center
%		\begin{figure}[H]
%		\caption{tif檔案}\label{ex_tif}
%		\includegraphics{\imgdir{B_afixed.tif}}		
%		\end{figure}			

%		\center
%		\begin{figure}[H]
%		\caption{svg檔案}\label{ex_svg}
%		\includegraphics{\imgdir{3group_knn.svg}}		
%		\end{figure}

%		\center
%		\begin{figure}[H]
%		\caption{tif檔案}\label{ex_tif}
%		\includegraphics{\imgdir{norm.tif}}		
%		\end{figure}

%		\center
%		\begin{figure}[H]
%		\caption{fig檔案}\label{ex_fig}
%		\includegraphics{\imgdir{figure.fig}}		
%		\end{figure}


\newpage
\centerline{{\BCF 結語與問題}}
\setlength{\parindent}{2em} 
	只要圖形的位置與選項指令設定得當,利用\LaTeX 對圖形做排版其實相較WORD更為精準與美觀,所有選項設置都是由參數組成,多圖並列時,不需要擔心圖形高度及位置參差不齊。\\
	\indent 因不同編譯器能夠編譯的檔案類型不同,故而在引入.tiff, .fig, .svg等檔案時,嘗試過各類型編譯器,但皆未果,屬本次作品未解之遺憾。
	

\newpage
\nocite{*}
\bibliographystyle{amsplain}
\bibliography{reference}









\end{document}