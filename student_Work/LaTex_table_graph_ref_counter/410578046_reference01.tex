\documentclass[12pt, oneside, a4paper]{book}
%\documentclass[12pt, a4paper]{book}
%----- 定義使用的 packages ----------------------

\usepackage{fontspec} 										% Font selection for XeLaTeX; see fontspec.pdf for documentation. 
\usepackage{xeCJK}											% 中文使用 XeCJK,但利用 \setCJKmainfont 定義粗體與斜體的字型
\defaultfontfeatures{Mapping=tex-text} 				% to support TeX conventions like ``---''
\usepackage{xunicode} 										% Unicode support for LaTeX character names (accents, European chars, etc)
\usepackage{xltxtra} 											% Extra customizations for XeLaTeX
\usepackage[sf,small]{titlesec}
\usepackage{amsmath, amssymb}
\usepackage{amsthm}										% theroemstyle 需要使用的套件
\usepackage{bm}                                                 % 排版粗體數學符號
\usepackage{enumerate}
\usepackage{graphicx, subfig, float} 					% support the \includegraphics command and options
\usepackage{array}
\usepackage{color, xcolor}
\usepackage{longtable, lscape}                                   % 跨頁的超長表格;lscape是旋轉此類表格的
\usepackage{threeparttable}                                     % 巨集,使表格加註解更容易(手冊p169)
\usepackage{multirow, booktabs}                                   % 讓表格編起來更美的套件(手冊p166),編輯跨列標題重覆的表格(手冊p182)
\usepackage{colortbl}                          				%.............................................表格標題註解之巨集套件
\usepackage{natbib}											% for Reference
\usepackage{makeidx}										% for Indexing
\usepackage[parfill]{parskip} % Activate to begin paragraphs with an empty line rather than an indent
%\usepackage{geometry} % See geometry.pdf to learn the layout options. There are lots.
%\usepackage[left=1.5in,right=1in,top=1in,bottom=1in]{geometry} 
\usepackage{url}                                                % 文稿內徵引網址
    \def\UrlFont{\rm}                                           % 網頁
\usepackage{fancyhdr}
	\pagestyle{fancy}
	\fancyhf{}                                     % 清除所有頁眉頁足
	\renewcommand{\headrulewidth}{0pt}                              % 頁眉下方的橫線    
%-----------------------------------------------------------------------------------------------------------------------
%  主字型設定
\setCJKmainfont
	[
		BoldFont=Heiti TC Medium								% 定義粗體的字型(依使用的電腦安裝的字型而定)
	]
	{cwTeX Q Ming Medium} 										% 設定中文內文字型
%	{新細明體}	
\setmainfont{Times New Roman}								% 設定英文內文字型
\setsansfont{Arial}														% used with {\sffamily ...}
%\setsansfont[Scale=MatchLowercase,Mapping=tex-text]{Gill Sans}
\setmonofont{Courier New}										% used with {\ttfamily ...}
%\setmonofont[Scale=MatchLowercase]{Andale Mono}
% 其他字型(隨使用的電腦安裝的字型不同,用註解的方式調整(打開或關閉))
% 英文字型
\newfontfamily{\E}{Cambria}										% 套用在內文中所有的英文字母
\newfontfamily{\A}{Arial}
\newfontfamily{\C}[Scale=0.9]{Cambria}
\newfontfamily{\T}{Times New Roman}
\newfontfamily{\TT}[Scale=0.8]{Times New Roman}
% 中文字型
\newCJKfontfamily{\MB}{微軟正黑體}							% 適用在 Mac 與 Win
\newCJKfontfamily{\SM}[Scale=0.8]{新細明體}				% 縮小版
%\newCJKfontfamily{\K}{標楷體}                        			% Windows 下的標楷體
\newCJKfontfamily{\K}{Kaiti TC Regular}         			% Mac OS 下的標楷體
\newCJKfontfamily{\BM}{Heiti TC Medium}					% Mac OS 下的黑體(粗體)
\newCJKfontfamily{\SR}{Songti TC Regular}				% Mac OS 下的宋體
\newCJKfontfamily{\SB}{Songti TC Bold}					% Mac OS 下的宋體(粗體)
\newCJKfontfamily{\CF}{cwTeX Q Fangsong Medium}	% CwTex 仿宋體
\newCJKfontfamily{\CB}{cwTeX Q Hei Bold}				% CwTex 粗黑體
\newCJKfontfamily{\CK}{cwTeX Q Kai Medium}   		% CwTex 楷體
\newCJKfontfamily{\CM}{cwTeX Q Ming Medium}		% CwTex 明體
\newCJKfontfamily{\CR}{cwTeX Q Yuan Medium}		% CwTex 圓體
%-----------------------------------------------------------------------------------------------------------------------
\XeTeXlinebreaklocale "zh"                  				%這兩行一定要加,中文才能自動換行
\XeTeXlinebreakskip = 0pt plus 1pt     %這兩行一定要加,中文才能自動換行
%-----------------------------------------------------------------------------------------------------------------------
%----- 重新定義的指令 ---------------------------
\newcommand{\cw}{\texttt{cw}\kern-.6pt\TeX}	% 這是 cwTex 的 logo 文字
\newcommand{\imgdir}{graph/}							% 設定圖檔的位置
\renewcommand{\tablename}{表}						% 改變表格標號文字為中文的「表」(預設為 Table)
\renewcommand{\figurename}{圖}						% 改變圖片標號文字為中文的「圖」(預設為 Figure)
\renewcommand{\contentsname}{目~錄}
\renewcommand\listfigurename{圖目錄}
\renewcommand\listtablename{表目錄}
\renewcommand{\appendixname}{附~錄}                  
\renewcommand{\indexname}{索引}
\renewcommand{\bibname}{參考文獻}
%-----------------------------------------------------------------------------------------------------------------------

\theoremstyle{plain}
\newtheorem{de}{Definition}[section]				%definition獨立編號
\newtheorem{thm}{定理}[section]			%theorem 獨立編號,取中文名稱並給予不同字型
\newtheorem{lemma}[thm]{引理}				%lemma 與 theorem 共用編號
\newtheorem{ex}{{\E Example}}						%example 獨立編號,不編入小節數字,走流水號。也換個字型。
\newtheorem{cor}{Corollary}[section]				%not used here
\newtheorem{exercise}{EXERCISE}					%not used here
\newtheorem{re}{\emph{Result}}[section]		%not used here
\newtheorem{axiom}{AXIOM}							%not used here
\renewcommand{\proofname}{\textbf{Proof}}		%not used here

\newcommand{\loflabel}{圖} % 圖目錄出現 圖 x.x 的「圖」字
\newcommand{\lotlabel}{表}  % 表目錄出現 表 x.x 的「表」字

\parindent=0pt

%--- 其他定義 ----------------------------------
% 定義章節標題的字型、大小
\titleformat{\chapter}[display]{\raggedleft\LARGE\bfseries\CF}		% 定義章抬頭靠右(\reggedleft)
 { 第\ \thechapter\ 章}{0.2cm}{}
%\titleformat{\chapter}[hang]{\centering\LARGE\sf}{\MB 第~\thesection~章}{0.2cm}{}%控制章的字體
%\titleformat{\section}[hang]{\Large\sf}{\MB 第~\thesection~節}{0.2cm}{}%控制章的字體
%\titleformat{\subsection}[hang]{\centering\Large\sf}{\MB 第~\thesubsection~節}{0.2cm}{}%控制節的字體
%\titleformat*{\section}{\normalfont\Large\bfseries\MB}
%\titleformat*{\subsection}{\normalfont\large\bfseries\MB}
%\titleformat*{\subsubsection}{\normalfont\large\bfseries\MB}


% 顏色定義
\definecolor{heavy}{gray}{.9}								% 0.9深淺度之灰色
\definecolor{light}{gray}{.8}
\definecolor{pink}{rgb}{0.99,0.91,0.95}               % 定義pink顏色
  

\title{ \LaTeX{\UD 參考文獻的引用1}}
\author{{\NC 張馨云}\;\; {\JF 410578046}}
\date{{\BR \today}} 	

\begin{document}
\maketitle
\fontsize{12}{22 pt}\selectfont

\centerline{{\BCF 引言}}
\setlength{\parindent}{2em}   
	寫作文章,報告,論文等等,常常會參考並引用各個文獻中的資料,此時必須標明參考資料之來源。參考文獻眾多,本文旨在介紹參考文獻的使用以期正確並有效率的引用文獻。以下將介紹BibTEX的使用。\\
\section{{\UD BibTEX的使用}}	
	透過BibTeX 引入外部文獻檔案:建立一個文獻資料庫,檔案類型為bib,每一筆文獻資料都佔據一個欄位,每一個欄位依其文獻性質而有不同的必要標籤規定,詳見表\ref{ref_rule}。 \\
	\indent 以$\backslash$bibliography$\{$資料庫的名稱$\}$引用文獻資料庫,以natbib套件所提供的各個樣式編排,常用的參考文獻樣式列於表\ref{ref_style},使用方法為:\\
	\indent 以$\backslash$bibliographystyle$\{$參考文獻樣式$\}$ \\
	\indent 使用BibTEX編排時,不可直接使用「快速編譯器」直接編排,應透過不同編譯器編譯\footnote{第一次引用 bib 檔或是更新 bib 檔時才需要四道程序,如果只是修改文章內容,只需進行一般的編譯。}4次,順序依次為:「XeLaTeX」$\longrightarrow$「BibTeX」$\longrightarrow$「XeLaTeX」$\longrightarrow$「XeLaTeX」。若無法編譯,可刪除資料夾內.aux及.bbl檔案,重新編譯,若仍出現錯誤訊息,則逐檢視並一排除log之中顯示的錯誤。
	\indent 一般以bibliography引用參考文獻時,reference的標題置左對齊,若希望標題置中,可藉由\footnote{$\backslash$renewcommand$\{\backslash$refname$\}\{\backslash$centering References$\}$} $\backslash$renewcommand指令重新定義其位置。
	
	
	\begin{table}[H]
	\centering
 	\extrarowheight=3pt
    \caption{標籤與其敘述}\label{ref_lab}
		\begin{tabular}{ll}
		\hline
		標籤 & 敘述 \\
		\hline
		address & 出版商的地址(城市或完整地址) \\
		\rowcolor{lightpink}
		annote & 註解,用於註解型參考書目樣式的註解 \\
		author & 作者姓名,英文姓名以"and";中文姓名以"•"分隔 \\
		\rowcolor{lightpink}
		booktitle & 書籍名稱(當所引用的僅僅是其部分內容時) \\
		chapter & 章號 \\
		\rowcolor{lightpink}
		crossref & 交叉引用條目 \\
		date & 出版日期;導出至BibTeX時為"month"和"year" \\
		\rowcolor{lightpink}	
		edition & 書籍的版次;長格式(如"first"或"second") \\
		editor  & 編者姓名 \\
		\rowcolor{lightpink}
		howpublished & 出版方式(當出版方法不標準) \\
		journal & 出版當前著作的期刊或雜誌 \\
		\rowcolor{lightpink}
		key & \tabincell{l}{當缺少"author"和"editor"時,\\用於指定或者覆蓋條目字母順序的隱藏字段}\\
		month & 出版月份 \\
		\rowcolor{lightpink}
		note & 備註,額外的雜項信息 \\
		number & 編號,指期刊、雜誌或技術報告的"編號" \\
		\rowcolor{lightpink}
		pages & 頁碼,採用逗號或雙連字符來分隔\\
		publisher & 出版商名稱 \\
		\rowcolor{lightpink}
		school & 編寫此學位論文時所在的院校 \\
		title & 著作名稱 \\ 
		\rowcolor{lightpink}
		url &  網址 \\
		volume &  卷,期刊或多卷本書籍的捲號 \\
		\rowcolor{lightpink}
		year & 出版年份 \\
		\hline
		\end{tabular}
	\end{table}
	
	
	\begin{table}[H]
	\centering
 	\extrarowheight=3pt
    \caption{各類型參考文獻標籤規定}\label{ref_rule}
		\begin{tabular}{lll}
		\hline
		類型 & 敘述 & 必要標籤 \\
		\hline
		article & 期刊雜誌論文 & author, title, journal, year \\
		\rowcolor{lightpink}
		book &  公開出版的圖書 & author/editor, title, publisher, year \\
		booklet & 無出版商或作者的圖書 & title \\
		\rowcolor{lightpink}
		inbook & 書籍的一部分章節 & \tabincell{l}{author/editor, title, chapter
									 \\ and/or pages, publisher, year} \\
		incollection & 書籍中帶獨立標題的章節 & author, title, booktitle, publisher, year \\
		\rowcolor{lightpink}
		inproceedings & 會議論文集中的一篇 & author, title, booktitle, year \\
		manual & 技術文檔 & title \\
		\rowcolor{lightpink}
		mastersthesis & 碩士論文 & author, title, school, year \\
 		misc & 其他 & none \\
 		\rowcolor{lightpink}
		phdthesis & 博士論文 & author, title, year, school \\
		techreport & 教育,商業機構的技術報告 & author, title, institution, year\\
		\rowcolor{lightpink}
		unpublished & 未出版的論文,圖書 & author, title, note \\
		\hline
		\end{tabular}
	\end{table}
	
	
	\begin{table}[H]
	\centering
 	\extrarowheight=3pt
    \caption{常用參考文獻樣式}\label{ref_style}
		\begin{tabular}{ll}
		\hline
		樣式 & 敘述 \\
		\hline
		plain & 按字母的順序排列,比較次序為作者、年度和標題 \\
		\rowcolor{lightpink}
		unsrt & 樣式同plain,按照引用的先後排序 \\
		alpha & 用作者名首字母+年份後兩位作標號,以字母順序排序 \\		
		\rowcolor{lightpink}
		abbrv & 似plain,將月份全拼改為縮寫 \\ 	
		\hline
		\end{tabular}
	\end{table}

	
\section{{\UD 可能遇到的問題}}	 
	在撰寫本文,研究參考文獻的使用時,曾遇到過許多問題,在此列出幾個可能遇到的問題與其解決方法:
\begin{enumerate}[i. ]
\item I found no $\backslash$citation commands bibtex while reading file article.aux \\
	可以在bibliography指令之前,置入$\backslash$nocite$\{*\}$
\item Warning--to sort, need author or key \\
	可能因為bib資料庫中,參考文獻的標籤不齊全,無法排列順序,這時只要加上key這個標籤即可。
\item Package natbib Error: Bibliography not compatible with author-year citations. \\
	修改引用natbib套件為:$\backslash$usepackage[square,sort,comma,numbers]$\{$natbib$\}$,若出現新錯誤:"option clash for package natbib.",可能是因為某些期刊的document class已经預設加載了natbib套件,當再次引用套件時,將會導致option衝突。
\item thebibliography與BibTeX混用可能會造成以下問題: \\
	The top-level auxiliary file: reference.aux
	\begin{itemize}
	\item[•]  I found no $\backslash$citation commands---while reading file reference.aux 
    \item[•]  I found no $\backslash$bibdata command---while reading file reference.aux 
    \item[•]  I found no $\backslash$bibstyle command---while reading file reference.aux 
	\end{itemize}
 	(There were 3 error messages)
\end{enumerate}


\nocite{*}
\bibliographystyle{amsplain}
\bibliography{practice}
	
\end{document}
