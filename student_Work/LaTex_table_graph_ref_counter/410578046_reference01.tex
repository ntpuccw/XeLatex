% !TEX TS-program = xelatex								
% !TEX encoding = UTF-8

\documentclass[12pt, a4paper]{article} 		
\usepackage{fontspec} 				% Font selection for XeLaTeX; see fontspec.pdf for documentation. 
%\usepackage[BoldFont, SlantFont]{xeCJK}% 中文使用 XeCJK,並模擬粗體與斜體(即可以用 \textbf{ } \textit{ })
\usepackage{xeCJK}							% 中文使用 XeCJK,但利用 \setCJKmainfont 定義粗體與斜體的字型
\defaultfontfeatures{Mapping=tex-text} 		% to support TeX conventions like ``---''
\usepackage{xunicode} 						% Unicode support for LaTeX character names (accents, European chars, etc)
\usepackage{xltxtra} 						% Extra customizations for XeLaTeX


\usepackage[parfill]{parskip} % Activate to begin paragraphs with an empty line rather than an indent
%\usepackage{geometry} % See geometry.pdf to learn the layout options. There are lots.
\usepackage[left=1.5in,right=1in,top=1in,bottom=1in]{geometry} %設定頁面的四個邊界
%---------------------------------------------------
 %防止Package reports “command already defined”出現
%\usepackage{amsmath, amssymb}
\usepackage{savesym}    
\usepackage{amsmath}
\savesymbol{iint}
\usepackage{txfonts}
\restoresymbol{TXF}{iint}  
%--------------------------------------------------
\usepackage{enumerate}

\usepackage{bm} %對希臘字母加粗

%-----------------------------------------------------------------------------------------------------------------------
%  主字型設定
\setCJKmainfont								% 設定中文內文字型
	[
		BoldFont=cwTeX Q Hei Bold			% 定義粗體的字型(依使用的電腦安裝的字型而定)
	]
%	{cwTeX Q Ming Medium} 					% 設定中文內文字型
	{新細明體}	
\setmainfont{Times New Roman}				% 設定英文內文字型
\setsansfont{Arial}							% used with {\sffamily ...}
%\setsansfont[Scale=MatchLowercase,Mapping=tex-text]{Gill Sans}
\setmonofont{Courier New}					% used with {\ttfamily ...}
%\setmonofont[Scale=MatchLowercase]{Andale Mono}
% 其他字型(隨使用的電腦安裝的字型不同,用註解的方式調整(打開或關閉))
% 英文字型
\newfontfamily{\C}{Cambria}					% 套用在內文中所有的英文字母
\newfontfamily{\A}{Arial}
\newfontfamily{\sC}[Scale=0.9]{Cambria}
\newfontfamily{\TNR}{Times New Roman}
\newfontfamily{\TN}[Scale=0.8]{Times New Roman}
\newfontfamily{\F}{Forte}
\newfontfamily{\JF}{JackeyFont}
\newfontfamily{\BR}{Bunny Rabbits}
\newfontfamily{\SF}{SanafonMaru}
% 中文字型
\newCJKfontfamily{\MJH}{微軟正黑體}			
\newCJKfontfamily{\sMLU}[Scale=0.8]{新細明體}	  
\newCJKfontfamily{\BK}{標楷體}    
\newCJKfontfamily{\UD}{UD Digi Kyokasho NP-B} 
\newCJKfontfamily{\BM}{BugMaruGothic} 
\newCJKfontfamily{\pig}{pigmo-00}
\newCJKfontfamily{\HC}{華康兒風體W4}
\newCJKfontfamily{\NC}{Nagurigaki Crayon}
\newCJKfontfamily{\OY}{onryou}
% 以下為自行安裝的字型:CwTex 組合
\newCJKfontfamily{\CF}{cwTeX Q Fangsong Medium}		% CwTex 仿宋體
\newCJKfontfamily{\BCF}[Scale=2.0]{cwTeX Q Fangsong Medium}
\newCJKfontfamily{\CB}{cwTeX Q Hei Bold}			% CwTex 粗黑體
\newCJKfontfamily{\CK}{cwTeX Q Kai Medium}   		% CwTex 楷體
\newCJKfontfamily{\CM}{cwTeX Q Ming Medium}			% CwTex 明體
\newCJKfontfamily{\CR}{cwTeX Q Yuan Medium}			% CwTex 圓體
%-----------------------------------------------------------------------------------------------------------------------
\XeTeXlinebreaklocale "zh"                  		%這兩行一定要加,中文才能自動換行
\XeTeXlinebreakskip = 0pt plus 1pt     				%這兩行一定要加,中文才能自動換行
%-----------------------------------------------------------------------------------------------------------------------
\newcommand{\cw}{\texttt{cw}\kern-.6pt\TeX}			% 這是 cwTex 的 logo 文字
\renewcommand{\tablename}{表}						% 改變表格標號文字為中文的「表」(預設為 Table)
\renewcommand{\figurename}{圖}						% 改變圖片標號文字為中文的「圖」(預設為 Figure)



%計數器---------------------------------------------------------------------------------------------------------
\let\openbox\relax   %避免! LaTeX Error: Command \openbox already defined. 
\usepackage{amsthm} % theroemstyle 需要使用的套件

\newtheorem{Def}{Definition}[section]		%definition獨立編號
\newtheorem{thm}{{\HC 定理}}[section]		%theorem 獨立編號,取中文名稱並給予不同字型
\newtheorem{lemma}[thm]{Lemma}				%lemma 與 theorem 共用編號
\newtheorem{ex}{{\F Example}}				%example 獨立編號,不編入小節數字,走流水號。也換個字型。
\newtheorem{EX}[ex]{{\HC 範例}} 				%定義與example共用編號的範例


%設定表格-------------------------------------------------------------------------------------------------------
\usepackage{array} 
\usepackage{booktabs}
\usepackage{multirow}
\usepackage{longtable}
\usepackage{dcolumn}   %用以對齊小數點 
\usepackage{graphicx}  %用以旋轉
\usepackage{diagbox} %製作斜線表頭 
%縮放
\newcommand{\bpara}[4]{ % #1 x; #2 y; #3 angle; #4 height
\begin{picture}(0,0)%
\setlength{\unitlength}{1pt}%
\put(#1,#2){\rotatebox{#3}{\raisebox{0mm}[0mm][0mm]{%
\makebox[0mm]{$\left.\rule{0mm}{#4pt}\right\}$}}}}%
\end{picture}}		

%表格內折行
\newcommand{\tabincell}[2]{\begin{tabular}{@{}#1@{}}#2\end{tabular}}  
		%用法:\tabincell{clr}{第一行\\第二行} 

%設定圖片-------------------------------------------------------------------------------------------------------
\usepackage{graphicx}		 	%插入圖片的套件
\usepackage{float} 				%設置圖片浮動位置
\usepackage{subfig} 			%插入多圖時用子圖顯示
\usepackage{wrapfig}			%文繞圖
%\usepackage{subfigure}
%\usepackage{graphicx, subfig, float} 		% support the \includegraphics command and options

	
\newcommand{\imgdir}{images/}		%設定圖形所在子目錄

%圖形樣式設計
\usepackage{picins}
%要把圖標號與圖形一起旋轉
\usepackage{blindtext}
\usepackage{adjustbox}



%設定顏色-------------------------------------------------------------------------------------------------------
\usepackage{color, xcolor}
\usepackage{colortbl}
\definecolor{slight}{gray}{0.6}			
\definecolor{lightpink}{rgb}{1.0, 0.71, 0.76}
\definecolor{lightskyblue}{rgb}{0.53, 0.81, 0.98}
\definecolor{lightsalmon}{rgb}{1.0, 0.63, 0.48}
\definecolor{champagne}{rgb}{0.97, 0.91, 0.81}
\definecolor{paleblue}{rgb}{0.69, 0.93, 0.93}
\definecolor{bananamania}{rgb}{0.98, 0.91, 0.71}
\definecolor{lavendergray}{rgb}{0.77, 0.76, 0.82}
\definecolor{lightyellow}{rgb}{1.0, 1.0, 0.88}
%-----------------------------------------------------------------------------------------------------------------------
%設定縮格(section或subsection底下默認不縮排)
\usepackage{indentfirst}
\setlength{\parindent}{2em}

%置入網頁連結---------------------------------------------------------------------
\usepackage[colorlinks,linkcolor=black, urlcolor=blue]{hyperref}
		%用法:\href{網頁連結},若連結為電子郵件:\href{mailto;電子郵件}
		
%參考資料樣式----------------------------------------------------------------------------------------------------
%\usepackage{natbib}
%\usepackage[sort&compress,square,comma,authoryear]{natbib}		
		
%-----------------------------------------------------------------------------------------------------------------------
%更改章節編號字體
%\usepackage{titlesec}
%\newfontfamily\sectionNC{Nagurigaki Crayon}
%\newfontfamily\subsectionLBP{Local BaseBall Park}
%\titleformat*{\section}{sectionNC}
%\titleformat*{\subsection}{sectionLBP}  

\title{ \LaTeX{\UD 參考文獻的引用1}}
\author{{\NC 張馨云}\;\; {\JF 410578046}}
\date{{\BR \today}} 	

\begin{document}
\maketitle
\fontsize{12}{22 pt}\selectfont

\centerline{{\BCF 引言}}
\setlength{\parindent}{2em}   
	寫作文章,報告,論文等等,常常會參考並引用各個文獻中的資料,此時必須標明參考資料之來源。參考文獻眾多,本文旨在介紹參考文獻的使用以期正確並有效率的引用文獻。以下將介紹BibTEX的使用。\\
\section{{\UD BibTEX的使用}}	
	透過BibTeX 引入外部文獻檔案:建立一個文獻資料庫,檔案類型為bib,每一筆文獻資料都佔據一個欄位,每一個欄位依其文獻性質而有不同的必要標籤規定,詳見表\ref{ref_rule}。 \\
	\indent 以$\backslash$bibliography$\{$資料庫的名稱$\}$引用文獻資料庫,以natbib套件所提供的各個樣式編排,常用的參考文獻樣式列於表\ref{ref_style},使用方法為:\\
	\indent 以$\backslash$bibliographystyle$\{$參考文獻樣式$\}$ \\
	\indent 使用BibTEX編排時,不可直接使用「快速編譯器」直接編排,應透過不同編譯器編譯\footnote{第一次引用 bib 檔或是更新 bib 檔時才需要四道程序,如果只是修改文章內容,只需進行一般的編譯。}4次,順序依次為:「XeLaTeX」$\longrightarrow$「BibTeX」$\longrightarrow$「XeLaTeX」$\longrightarrow$「XeLaTeX」。若無法編譯,可刪除資料夾內.aux及.bbl檔案,重新編譯,若仍出現錯誤訊息,則逐檢視並一排除log之中顯示的錯誤。
	\indent 一般以bibliography引用參考文獻時,reference的標題置左對齊,若希望標題置中,可藉由\footnote{$\backslash$renewcommand$\{\backslash$refname$\}\{\backslash$centering References$\}$} $\backslash$renewcommand指令重新定義其位置。
	
	
	\begin{table}[H]
	\centering
 	\extrarowheight=3pt
    \caption{標籤與其敘述}\label{ref_lab}
		\begin{tabular}{ll}
		\hline
		標籤 & 敘述 \\
		\hline
		address & 出版商的地址(城市或完整地址) \\
		\rowcolor{lightpink}
		annote & 註解,用於註解型參考書目樣式的註解 \\
		author & 作者姓名,英文姓名以"and";中文姓名以"•"分隔 \\
		\rowcolor{lightpink}
		booktitle & 書籍名稱(當所引用的僅僅是其部分內容時) \\
		chapter & 章號 \\
		\rowcolor{lightpink}
		crossref & 交叉引用條目 \\
		date & 出版日期;導出至BibTeX時為"month"和"year" \\
		\rowcolor{lightpink}	
		edition & 書籍的版次;長格式(如"first"或"second") \\
		editor  & 編者姓名 \\
		\rowcolor{lightpink}
		howpublished & 出版方式(當出版方法不標準) \\
		journal & 出版當前著作的期刊或雜誌 \\
		\rowcolor{lightpink}
		key & \tabincell{l}{當缺少"author"和"editor"時,\\用於指定或者覆蓋條目字母順序的隱藏字段}\\
		month & 出版月份 \\
		\rowcolor{lightpink}
		note & 備註,額外的雜項信息 \\
		number & 編號,指期刊、雜誌或技術報告的"編號" \\
		\rowcolor{lightpink}
		pages & 頁碼,採用逗號或雙連字符來分隔\\
		publisher & 出版商名稱 \\
		\rowcolor{lightpink}
		school & 編寫此學位論文時所在的院校 \\
		title & 著作名稱 \\ 
		\rowcolor{lightpink}
		url &  網址 \\
		volume &  卷,期刊或多卷本書籍的捲號 \\
		\rowcolor{lightpink}
		year & 出版年份 \\
		\hline
		\end{tabular}
	\end{table}
	
	
	\begin{table}[H]
	\centering
 	\extrarowheight=3pt
    \caption{各類型參考文獻標籤規定}\label{ref_rule}
		\begin{tabular}{lll}
		\hline
		類型 & 敘述 & 必要標籤 \\
		\hline
		article & 期刊雜誌論文 & author, title, journal, year \\
		\rowcolor{lightpink}
		book &  公開出版的圖書 & author/editor, title, publisher, year \\
		booklet & 無出版商或作者的圖書 & title \\
		\rowcolor{lightpink}
		inbook & 書籍的一部分章節 & \tabincell{l}{author/editor, title, chapter
									 \\ and/or pages, publisher, year} \\
		incollection & 書籍中帶獨立標題的章節 & author, title, booktitle, publisher, year \\
		\rowcolor{lightpink}
		inproceedings & 會議論文集中的一篇 & author, title, booktitle, year \\
		manual & 技術文檔 & title \\
		\rowcolor{lightpink}
		mastersthesis & 碩士論文 & author, title, school, year \\
 		misc & 其他 & none \\
 		\rowcolor{lightpink}
		phdthesis & 博士論文 & author, title, year, school \\
		techreport & 教育,商業機構的技術報告 & author, title, institution, year\\
		\rowcolor{lightpink}
		unpublished & 未出版的論文,圖書 & author, title, note \\
		\hline
		\end{tabular}
	\end{table}
	
	
	\begin{table}[H]
	\centering
 	\extrarowheight=3pt
    \caption{常用參考文獻樣式}\label{ref_style}
		\begin{tabular}{ll}
		\hline
		樣式 & 敘述 \\
		\hline
		plain & 按字母的順序排列,比較次序為作者、年度和標題 \\
		\rowcolor{lightpink}
		unsrt & 樣式同plain,按照引用的先後排序 \\
		alpha & 用作者名首字母+年份後兩位作標號,以字母順序排序 \\		
		\rowcolor{lightpink}
		abbrv & 似plain,將月份全拼改為縮寫 \\ 	
		\hline
		\end{tabular}
	\end{table}

	
\section{{\UD 可能遇到的問題}}	 
	在撰寫本文,研究參考文獻的使用時,曾遇到過許多問題,在此列出幾個可能遇到的問題與其解決方法:
\begin{enumerate}[i. ]
\item I found no $\backslash$citation commands bibtex while reading file article.aux \\
	可以在bibliography指令之前,置入$\backslash$nocite$\{*\}$
\item Warning--to sort, need author or key \\
	可能因為bib資料庫中,參考文獻的標籤不齊全,無法排列順序,這時只要加上key這個標籤即可。
\item Package natbib Error: Bibliography not compatible with author-year citations. \\
	修改引用natbib套件為:$\backslash$usepackage[square,sort,comma,numbers]$\{$natbib$\}$,若出現新錯誤:"option clash for package natbib.",可能是因為某些期刊的document class已经預設加載了natbib套件,當再次引用套件時,將會導致option衝突。
\item thebibliography與BibTeX混用可能會造成以下問題: \\
	The top-level auxiliary file: reference.aux
	\begin{itemize}
	\item[•]  I found no $\backslash$citation commands---while reading file reference.aux 
    \item[•]  I found no $\backslash$bibdata command---while reading file reference.aux 
    \item[•]  I found no $\backslash$bibstyle command---while reading file reference.aux 
	\end{itemize}
 	(There were 3 error messages)
\end{enumerate}


\nocite{*}
\bibliographystyle{amsplain}
\bibliography{practice}
	
\end{document}
