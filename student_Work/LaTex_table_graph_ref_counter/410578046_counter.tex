\documentclass[12pt, oneside, a4paper]{book}
%\documentclass[12pt, a4paper]{book}
%----- 定義使用的 packages ----------------------

\usepackage{fontspec} 										% Font selection for XeLaTeX; see fontspec.pdf for documentation. 
\usepackage{xeCJK}											% 中文使用 XeCJK,但利用 \setCJKmainfont 定義粗體與斜體的字型
\defaultfontfeatures{Mapping=tex-text} 				% to support TeX conventions like ``---''
\usepackage{xunicode} 										% Unicode support for LaTeX character names (accents, European chars, etc)
\usepackage{xltxtra} 											% Extra customizations for XeLaTeX
\usepackage[sf,small]{titlesec}
\usepackage{amsmath, amssymb}
\usepackage{amsthm}										% theroemstyle 需要使用的套件
\usepackage{bm}                                                 % 排版粗體數學符號
\usepackage{enumerate}
\usepackage{graphicx, subfig, float} 					% support the \includegraphics command and options
\usepackage{array}
\usepackage{color, xcolor}
\usepackage{longtable, lscape}                                   % 跨頁的超長表格;lscape是旋轉此類表格的
\usepackage{threeparttable}                                     % 巨集,使表格加註解更容易(手冊p169)
\usepackage{multirow, booktabs}                                   % 讓表格編起來更美的套件(手冊p166),編輯跨列標題重覆的表格(手冊p182)
\usepackage{colortbl}                          				%.............................................表格標題註解之巨集套件
\usepackage{natbib}											% for Reference
\usepackage{makeidx}										% for Indexing
\usepackage[parfill]{parskip} % Activate to begin paragraphs with an empty line rather than an indent
%\usepackage{geometry} % See geometry.pdf to learn the layout options. There are lots.
%\usepackage[left=1.5in,right=1in,top=1in,bottom=1in]{geometry} 
\usepackage{url}                                                % 文稿內徵引網址
    \def\UrlFont{\rm}                                           % 網頁
\usepackage{fancyhdr}
	\pagestyle{fancy}
	\fancyhf{}                                     % 清除所有頁眉頁足
	\renewcommand{\headrulewidth}{0pt}                              % 頁眉下方的橫線    
%-----------------------------------------------------------------------------------------------------------------------
%  主字型設定
\setCJKmainfont
	[
		BoldFont=Heiti TC Medium								% 定義粗體的字型(依使用的電腦安裝的字型而定)
	]
	{cwTeX Q Ming Medium} 										% 設定中文內文字型
%	{新細明體}	
\setmainfont{Times New Roman}								% 設定英文內文字型
\setsansfont{Arial}														% used with {\sffamily ...}
%\setsansfont[Scale=MatchLowercase,Mapping=tex-text]{Gill Sans}
\setmonofont{Courier New}										% used with {\ttfamily ...}
%\setmonofont[Scale=MatchLowercase]{Andale Mono}
% 其他字型(隨使用的電腦安裝的字型不同,用註解的方式調整(打開或關閉))
% 英文字型
\newfontfamily{\E}{Cambria}										% 套用在內文中所有的英文字母
\newfontfamily{\A}{Arial}
\newfontfamily{\C}[Scale=0.9]{Cambria}
\newfontfamily{\T}{Times New Roman}
\newfontfamily{\TT}[Scale=0.8]{Times New Roman}
% 中文字型
\newCJKfontfamily{\MB}{微軟正黑體}							% 適用在 Mac 與 Win
\newCJKfontfamily{\SM}[Scale=0.8]{新細明體}				% 縮小版
%\newCJKfontfamily{\K}{標楷體}                        			% Windows 下的標楷體
\newCJKfontfamily{\K}{Kaiti TC Regular}         			% Mac OS 下的標楷體
\newCJKfontfamily{\BM}{Heiti TC Medium}					% Mac OS 下的黑體(粗體)
\newCJKfontfamily{\SR}{Songti TC Regular}				% Mac OS 下的宋體
\newCJKfontfamily{\SB}{Songti TC Bold}					% Mac OS 下的宋體(粗體)
\newCJKfontfamily{\CF}{cwTeX Q Fangsong Medium}	% CwTex 仿宋體
\newCJKfontfamily{\CB}{cwTeX Q Hei Bold}				% CwTex 粗黑體
\newCJKfontfamily{\CK}{cwTeX Q Kai Medium}   		% CwTex 楷體
\newCJKfontfamily{\CM}{cwTeX Q Ming Medium}		% CwTex 明體
\newCJKfontfamily{\CR}{cwTeX Q Yuan Medium}		% CwTex 圓體
%-----------------------------------------------------------------------------------------------------------------------
\XeTeXlinebreaklocale "zh"                  				%這兩行一定要加,中文才能自動換行
\XeTeXlinebreakskip = 0pt plus 1pt     %這兩行一定要加,中文才能自動換行
%-----------------------------------------------------------------------------------------------------------------------
%----- 重新定義的指令 ---------------------------
\newcommand{\cw}{\texttt{cw}\kern-.6pt\TeX}	% 這是 cwTex 的 logo 文字
\newcommand{\imgdir}{graph/}							% 設定圖檔的位置
\renewcommand{\tablename}{表}						% 改變表格標號文字為中文的「表」(預設為 Table)
\renewcommand{\figurename}{圖}						% 改變圖片標號文字為中文的「圖」(預設為 Figure)
\renewcommand{\contentsname}{目~錄}
\renewcommand\listfigurename{圖目錄}
\renewcommand\listtablename{表目錄}
\renewcommand{\appendixname}{附~錄}                  
\renewcommand{\indexname}{索引}
\renewcommand{\bibname}{參考文獻}
%-----------------------------------------------------------------------------------------------------------------------

\theoremstyle{plain}
\newtheorem{de}{Definition}[section]				%definition獨立編號
\newtheorem{thm}{定理}[section]			%theorem 獨立編號,取中文名稱並給予不同字型
\newtheorem{lemma}[thm]{引理}				%lemma 與 theorem 共用編號
\newtheorem{ex}{{\E Example}}						%example 獨立編號,不編入小節數字,走流水號。也換個字型。
\newtheorem{cor}{Corollary}[section]				%not used here
\newtheorem{exercise}{EXERCISE}					%not used here
\newtheorem{re}{\emph{Result}}[section]		%not used here
\newtheorem{axiom}{AXIOM}							%not used here
\renewcommand{\proofname}{\textbf{Proof}}		%not used here

\newcommand{\loflabel}{圖} % 圖目錄出現 圖 x.x 的「圖」字
\newcommand{\lotlabel}{表}  % 表目錄出現 表 x.x 的「表」字

\parindent=0pt

%--- 其他定義 ----------------------------------
% 定義章節標題的字型、大小
\titleformat{\chapter}[display]{\raggedleft\LARGE\bfseries\CF}		% 定義章抬頭靠右(\reggedleft)
 { 第\ \thechapter\ 章}{0.2cm}{}
%\titleformat{\chapter}[hang]{\centering\LARGE\sf}{\MB 第~\thesection~章}{0.2cm}{}%控制章的字體
%\titleformat{\section}[hang]{\Large\sf}{\MB 第~\thesection~節}{0.2cm}{}%控制章的字體
%\titleformat{\subsection}[hang]{\centering\Large\sf}{\MB 第~\thesubsection~節}{0.2cm}{}%控制節的字體
%\titleformat*{\section}{\normalfont\Large\bfseries\MB}
%\titleformat*{\subsection}{\normalfont\large\bfseries\MB}
%\titleformat*{\subsubsection}{\normalfont\large\bfseries\MB}


% 顏色定義
\definecolor{heavy}{gray}{.9}								% 0.9深淺度之灰色
\definecolor{light}{gray}{.8}
\definecolor{pink}{rgb}{0.99,0.91,0.95}               % 定義pink顏色
  


\title{ \LaTeX{\UD 計數器的應用}}
\author{{\NC 張馨云}\;\; {\JF 410578046}}
\date{{\BR \today}} 	

\begin{document}
\maketitle
\fontsize{12}{22 pt}\selectfont

\centerline{{\BCF 引言}}
\setlength{\parindent}{2em} 
	在閱讀數理相關的文章、書籍、講義,甚或作答考試卷時,常常會看到一些定義公式、範例,題號等等皆以相同的格式呈現,並因不同章節而有計數之分別,為避免手動輸入及維護造成錯誤與麻煩,本文將介紹如何使用\LaTeX 指令計數。

\newpage
\section{\HC 計數器基本指令}
\subsection{{\HC \LaTeX 內建計數器}}
	\LaTeX 有內建幾種計數器以便使用者使用,指令如表\ref{counterx_latex}所示:

\begin{table}[H]
\centering
\extrarowheight=4pt
\caption{\LaTeX 內建計數器}\label{counterx_latex}
	\begin{tabular}{ll}
	\hline
	指令(加上$\backslash$) \quad &  \quad 用途  \\
	\hline
	part & 部序號計數器 \\
	\rowcolor{lavendergray}
	equation & 公式序號計數器 \\
	chapter & 章序號計數器 \\
	\rowcolor{lavendergray}
	page & 頁碼計數器 \\
	section & 節序號計數器 \\
	\rowcolor{lavendergray}
	footnote & 腳註序號計數器 \\
	subsection & 小節序號計數器 \\
	\rowcolor{lavendergray}
	mpfootnote & 小頁環境中的腳註序號計數器 \\
	subsubsection & 小小節序號計數器 \\
	\rowcolor{lavendergray}
	enumi & 排序列表第1層序號計數器 \\
	enumii & 排序列表第2層序號計數器\\
	\rowcolor{lavendergray}
	enumiii & 排序列表第3層序號計數器 \\
	enumiv & 排序列表第4層序號計數器 \\
	\rowcolor{lavendergray}
	paragraph & 段序號計數器 \\
	subparagraph & 小段序號計數器 \\
	\rowcolor{lavendergray}
	figure & 插圖序號計數器 \\
	table & 表格序號計數器  \\
	\hline
	\end{tabular}
\end{table}


\subsection{{\HC 定義新的計數器}}\label{counter_def}
	定義新的計數器,需要用到amsthm套件所提中的$\backslash$theroemstyle指令,使用方法如下,詳見表\ref{counter_thmstyle}:\\
      \indent $\{\backslash$newtheorem$\{$name$\}\{$Printed output$\}$[numberby]\\
      \indent $\{\backslash$newtheorem$\{$name$\}$[counter]$\{$Printed output$\}$\\
	\begin{table}[H]
	\centering
	\extrarowheight=4pt
	\caption{theoremstyle說明}\label{counter_thmstyle}
		\begin{tabular}{ll}
		\hline
		選項設置 & 說明 \\
		\hline
		\textbf{name} & 定義新計數器的指令。 \\
		\rowcolor{lavendergray}
		\textbf{Printed output} & 計數器最終呈現於文章的字樣 \\
		\textbf{numberby} & 新定義的計數器將以原先已定義的計數器為計數範圍 \\
		\rowcolor{lavendergray}
		\textbf{counter} & \tabincell{ll}{counter為已定義之計數器。代表新定義之計數器將\\與其共用相同的計數器,計算數字時會同步累計。} \\
		\tabincell{ll}{\textbf{$\backslash$begin$\{$name$\}$}\\ \textbf{$\backslash$end$\{$name$\}$}} & 開啟新定義之計數器環境 \\
		\hline
		\end{tabular}
	\end{table}

\subsection{{\HC 直接定義計數之數}}
	別於\ref{counter_def}所介紹之事先定義再以指令去呈現的計數器,另一種計數器使用方法如表\ref{counter_numb}所示:
	\begin{table}[H]
	\centering
	\extrarowheight=4pt
	\caption{counter使用說明}\label{counter_numb}
		\begin{tabular}{ll}
		\hline
		指令  & 說明 \\
		\hline
		$\backslash$newcounter$\{$name$\}$ & 定義計數器名稱 \\				\rowcolor{lavendergray}
		$\backslash$setcounter$\{$name$\}\{$數字$\}$ & 設置計數器起始數字,預設為0 \\
		$\backslash$addtocounter$\{$name$\}\{1\}$ & 第二次使用往上加1 \\
		\rowcolor{lavendergray}
		$\backslash$thename & 顯示該計數器的值 \\
		$\backslash$value$\{$name$\}$ & 調用該計數器之值 \\
		\hline
		\end{tabular}
	\end{table}
	
	\indent 使用此方法計數,每一次計數都需要使用$\backslash$addtocounter,相比之下,使用事先定義之計數器指令計數更為簡便。

\section{\HC 計數器實際操作}
	表\ref{counter_new}定義了幾個新的計數器:
	\begin{table}[H]
	\centering
	\extrarowheight=4pt
	\caption{定義新的計數器}\label{counter_new}
		\begin{tabular}{ll}
		\hline
		定義 & 說明 \\
		\hline
		$\{\backslash$newtheorem$\{$Def$\}\{$Definition$\}$[section]  & 定義Definition \\ 
		\rowcolor{lavendergray}
		$\{\backslash$newtheorem$\{$thm$\}\{\{\backslash$HC 定理$\}\}$[section] & 定義 theorem為定理 \\
		$\{\backslash$newtheorem$\{$lemma$\}$[thm]$\{$Lemma$\}$ & 定義與theorem共用編號的lemma \\
		\rowcolor{lavendergray}
		$\{\backslash$newtheorem$\{$ex$\}\{\{\backslash$F Example$\}\}$ & 定義獨立編號的example \\
		$\{\backslash$newtheorem$\{$EX$\}$[ex]$\{\{\backslash$HC 範例$\}\}$ & 定義與example共用編號的範例 \\
		\hline
		\end{tabular} 
	\end{table}

	\indent 或以有底色之表格,或以\footnote{$\backslash$rule$\{\backslash$textwidth$\}\{$0.2pt$\}$} 上下限之線條,我們常看見,在文章、書籍中,定理、範例等將定義區隔開來,以較明顯的方式呈現給讀者,不妨在以下練習試試:

	\begin{center}
	\colorbox{lightpink}{\begin{tabular}{p{0.9\textwidth}}
		\begin{thm}\label{ex_theorem}
		Let $X$ have the cdf $F(x)$ of the continuous type that is strictly increasing on the support$ a < x < b $.  Then the random variable  $Y$, defined by $Y = F(X)$, has a distribution that is $\mathbf{U}(0,1)$. 
		\end{thm}
		\end{tabular}}
	\end{center}


	\noindent \rule{\textwidth}{0.2pt}
	\begin{Def}\label{ex_Def}
(Factorization Theorem) \\
Let $X_1,X_2,\cdots,X_n$ denote random variables with joint pdf or pmf $f(x_1,x_2,\cdots,x_n;\theta)$, which depends on the parameter $\theta$. The statistic $Y =u(X_1,X_2,\cdots,X_n)$ is sufficient for $\theta$ if and only if $f(x_1,x_2,\cdots,x_n;\theta)$= $\varphi[u(x_1,x_2,\cdots,x_n);\theta]h(x_1,x_2,\cdots,x_n)$, where $\varphi$ depends on $x_1,x_2,\cdots,x_n$ only through $u(x_1,\cdots,x_n)$ and $h(x_1,\cdots,x_n)$ does not depend on $\theta$.
	\end{Def}
	\noindent \rule{\textwidth}{0.2pt}
	
	\begin{center}
	\colorbox{bananamania}{\begin{tabular}{p{0.9\textwidth}}
		\begin{ex}\label{ex_ex}
	Let $X_1,X_2,\cdots, X_n$ be a random sample from an exponential distribution with pdf
	$$f(x;\theta)=\frac{1}{\theta}e^{-x/\theta}=exp\left[ x(-\frac{1}{\theta})-ln\theta\right], \quad 0 < x < \infty, $$
	provided that $ 0 < x < \infty$. Here, $K(x)=x$.Thus, $\sum_{i=1}^nX_i$ is sufficient for $\theta$; of courfe,$\bar{X}=\sum_{i=1}^nX_i/n$ is also sufficient.
		\end{ex}
		\end{tabular}}
	\end{center}

	\noindent \rule{\textwidth}{0.2pt}
	\begin{lemma}\label{ex_lem}
	(Neyman–Pearson Lemma)\\
	 Let $X_1,X_2,\cdots,X_n$ be a random sample of size n from a distribution with pdf or pmf $f(x;\theta)$, where $\theta_0$ and $\theta_1$ are two possible values of $\theta$. Denote the joint pdf or pmf of $X_1,X_2,\cdots,X_n$ by the likelihood function $$L(\theta)=L(\theta;x_1,x_2,\cdots,x_n)=f(x_1;\theta)f(x_2;\theta)\cdots f(x_n;\theta).$$ 
	 If there exist a positive constant $k$ and $a$ subset $C$ of the sample space such that 
	 	\begin{enumerate}[(a)]
	 	\item $P[(X_1,X_2,\cdots,X_n) \in C;\theta_0]= \alpha$
	 	\item $\frac{L(\theta_0)}{L(\theta_1)} \leq k for (X_1,X_2,\cdots,X_n)\in C$, and 
	 	\item $\frac{L(\theta_0)}{L(\theta_1)}  \geq k for (x_1,x_2,\cdots,x_n)\in C',$ 
	 	\end{enumerate}
	then $C$ is a best critical region of size $\alpha$ for testing the simple null hypothesis $H_0: \theta = \theta_0$ against the simple alternative hypothesis $H_1: \theta = \theta_1$.
	\end{lemma}
	\noindent \rule{\textwidth}{0.2pt}

		
	\begin{center}
	\colorbox{bananamania}{\begin{tabular}{p{0.9\textwidth}}
		\begin{EX}\label{ex_EX}
		Let $\bar{X_n}$ denote the mean of a random sample of size n from a distribution that has pdf $f(x)=e^{-x}, 0 < x < \infty$, zero elsewhere.
		Use $\Delta$ -method to find the limiting distribution of $\sqrt{n}(\sqrt{\bar{X_n}}-1)$.
		\end{EX}
		\end{tabular}}
	\end{center}


	因為我們曾經定義Lemma與定理共用計數編號;範例與Example共用計數編號,可以看到Lemma\ref{ex_lem}接續定理\ref{ex_theorem}標號;範例\ref{ex_EX}接續Example\ref{ex_ex}編號。

\section{{\HC 結語與問題}}
	利用\LaTeX 之計數器能使文章書籍在編輯時更有效率且錯誤率更低,只要事先在preamble檔案中設計好計數器樣式、字型等等運作規則,操作起來比WORD更加方便快速!
	在引用amsthm套件定義計數器時,可能會因為套件衝突,而產生"! LaTeX Error: Command $\backslash$openbox already defined."之錯誤信息,此時,只要在引用amsthm套件之前,使用$\backslash$let$\backslash$openbox$\backslash$relax,忽略$\backslash$openbox這個指令,就可以成功的引用amsthm套件囉!


\nocite{*}
\bibliographystyle{amsplain}
\bibliography{reference}




\end{document}