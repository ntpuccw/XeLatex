% !TEX TS-program = xelatex								
% !TEX encoding = UTF-8

\documentclass[12pt, a4paper]{article} 		
\usepackage{fontspec} 				% Font selection for XeLaTeX; see fontspec.pdf for documentation. 
%\usepackage[BoldFont, SlantFont]{xeCJK}% 中文使用 XeCJK,並模擬粗體與斜體(即可以用 \textbf{ } \textit{ })
\usepackage{xeCJK}							% 中文使用 XeCJK,但利用 \setCJKmainfont 定義粗體與斜體的字型
\defaultfontfeatures{Mapping=tex-text} 		% to support TeX conventions like ``---''
\usepackage{xunicode} 						% Unicode support for LaTeX character names (accents, European chars, etc)
\usepackage{xltxtra} 						% Extra customizations for XeLaTeX


\usepackage[parfill]{parskip} % Activate to begin paragraphs with an empty line rather than an indent
%\usepackage{geometry} % See geometry.pdf to learn the layout options. There are lots.
\usepackage[left=1.5in,right=1in,top=1in,bottom=1in]{geometry} %設定頁面的四個邊界
%---------------------------------------------------
 %防止Package reports “command already defined”出現
%\usepackage{amsmath, amssymb}
\usepackage{savesym}    
\usepackage{amsmath}
\savesymbol{iint}
\usepackage{txfonts}
\restoresymbol{TXF}{iint}  
%--------------------------------------------------
\usepackage{enumerate}

\usepackage{bm} %對希臘字母加粗

%-----------------------------------------------------------------------------------------------------------------------
%  主字型設定
\setCJKmainfont								% 設定中文內文字型
	[
		BoldFont=cwTeX Q Hei Bold			% 定義粗體的字型(依使用的電腦安裝的字型而定)
	]
%	{cwTeX Q Ming Medium} 					% 設定中文內文字型
	{新細明體}	
\setmainfont{Times New Roman}				% 設定英文內文字型
\setsansfont{Arial}							% used with {\sffamily ...}
%\setsansfont[Scale=MatchLowercase,Mapping=tex-text]{Gill Sans}
\setmonofont{Courier New}					% used with {\ttfamily ...}
%\setmonofont[Scale=MatchLowercase]{Andale Mono}
% 其他字型(隨使用的電腦安裝的字型不同,用註解的方式調整(打開或關閉))
% 英文字型
\newfontfamily{\C}{Cambria}					% 套用在內文中所有的英文字母
\newfontfamily{\A}{Arial}
\newfontfamily{\sC}[Scale=0.9]{Cambria}
\newfontfamily{\TNR}{Times New Roman}
\newfontfamily{\TN}[Scale=0.8]{Times New Roman}
\newfontfamily{\F}{Forte}
\newfontfamily{\JF}{JackeyFont}
\newfontfamily{\BR}{Bunny Rabbits}
\newfontfamily{\SF}{SanafonMaru}
% 中文字型
\newCJKfontfamily{\MJH}{微軟正黑體}			
\newCJKfontfamily{\sMLU}[Scale=0.8]{新細明體}	  
\newCJKfontfamily{\BK}{標楷體}    
\newCJKfontfamily{\UD}{UD Digi Kyokasho NP-B} 
\newCJKfontfamily{\BM}{BugMaruGothic} 
\newCJKfontfamily{\pig}{pigmo-00}
\newCJKfontfamily{\HC}{華康兒風體W4}
\newCJKfontfamily{\NC}{Nagurigaki Crayon}
\newCJKfontfamily{\OY}{onryou}
% 以下為自行安裝的字型:CwTex 組合
\newCJKfontfamily{\CF}{cwTeX Q Fangsong Medium}		% CwTex 仿宋體
\newCJKfontfamily{\BCF}[Scale=2.0]{cwTeX Q Fangsong Medium}
\newCJKfontfamily{\CB}{cwTeX Q Hei Bold}			% CwTex 粗黑體
\newCJKfontfamily{\CK}{cwTeX Q Kai Medium}   		% CwTex 楷體
\newCJKfontfamily{\CM}{cwTeX Q Ming Medium}			% CwTex 明體
\newCJKfontfamily{\CR}{cwTeX Q Yuan Medium}			% CwTex 圓體
%-----------------------------------------------------------------------------------------------------------------------
\XeTeXlinebreaklocale "zh"                  		%這兩行一定要加,中文才能自動換行
\XeTeXlinebreakskip = 0pt plus 1pt     				%這兩行一定要加,中文才能自動換行
%-----------------------------------------------------------------------------------------------------------------------
\newcommand{\cw}{\texttt{cw}\kern-.6pt\TeX}			% 這是 cwTex 的 logo 文字
\renewcommand{\tablename}{表}						% 改變表格標號文字為中文的「表」(預設為 Table)
\renewcommand{\figurename}{圖}						% 改變圖片標號文字為中文的「圖」(預設為 Figure)



%計數器---------------------------------------------------------------------------------------------------------
\let\openbox\relax   %避免! LaTeX Error: Command \openbox already defined. 
\usepackage{amsthm} % theroemstyle 需要使用的套件

\newtheorem{Def}{Definition}[section]		%definition獨立編號
\newtheorem{thm}{{\HC 定理}}[section]		%theorem 獨立編號,取中文名稱並給予不同字型
\newtheorem{lemma}[thm]{Lemma}				%lemma 與 theorem 共用編號
\newtheorem{ex}{{\F Example}}				%example 獨立編號,不編入小節數字,走流水號。也換個字型。
\newtheorem{EX}[ex]{{\HC 範例}} 				%定義與example共用編號的範例


%設定表格-------------------------------------------------------------------------------------------------------
\usepackage{array} 
\usepackage{booktabs}
\usepackage{multirow}
\usepackage{longtable}
\usepackage{dcolumn}   %用以對齊小數點 
\usepackage{graphicx}  %用以旋轉
\usepackage{diagbox} %製作斜線表頭 
%縮放
\newcommand{\bpara}[4]{ % #1 x; #2 y; #3 angle; #4 height
\begin{picture}(0,0)%
\setlength{\unitlength}{1pt}%
\put(#1,#2){\rotatebox{#3}{\raisebox{0mm}[0mm][0mm]{%
\makebox[0mm]{$\left.\rule{0mm}{#4pt}\right\}$}}}}%
\end{picture}}		

%表格內折行
\newcommand{\tabincell}[2]{\begin{tabular}{@{}#1@{}}#2\end{tabular}}  
		%用法:\tabincell{clr}{第一行\\第二行} 

%設定圖片-------------------------------------------------------------------------------------------------------
\usepackage{graphicx}		 	%插入圖片的套件
\usepackage{float} 				%設置圖片浮動位置
\usepackage{subfig} 			%插入多圖時用子圖顯示
\usepackage{wrapfig}			%文繞圖
%\usepackage{subfigure}
%\usepackage{graphicx, subfig, float} 		% support the \includegraphics command and options

	
\newcommand{\imgdir}{images/}		%設定圖形所在子目錄

%圖形樣式設計
\usepackage{picins}
%要把圖標號與圖形一起旋轉
\usepackage{blindtext}
\usepackage{adjustbox}



%設定顏色-------------------------------------------------------------------------------------------------------
\usepackage{color, xcolor}
\usepackage{colortbl}
\definecolor{slight}{gray}{0.6}			
\definecolor{lightpink}{rgb}{1.0, 0.71, 0.76}
\definecolor{lightskyblue}{rgb}{0.53, 0.81, 0.98}
\definecolor{lightsalmon}{rgb}{1.0, 0.63, 0.48}
\definecolor{champagne}{rgb}{0.97, 0.91, 0.81}
\definecolor{paleblue}{rgb}{0.69, 0.93, 0.93}
\definecolor{bananamania}{rgb}{0.98, 0.91, 0.71}
\definecolor{lavendergray}{rgb}{0.77, 0.76, 0.82}
\definecolor{lightyellow}{rgb}{1.0, 1.0, 0.88}
%-----------------------------------------------------------------------------------------------------------------------
%設定縮格(section或subsection底下默認不縮排)
\usepackage{indentfirst}
\setlength{\parindent}{2em}

%置入網頁連結---------------------------------------------------------------------
\usepackage[colorlinks,linkcolor=black, urlcolor=blue]{hyperref}
		%用法:\href{網頁連結},若連結為電子郵件:\href{mailto;電子郵件}
		
%參考資料樣式----------------------------------------------------------------------------------------------------
%\usepackage{natbib}
%\usepackage[sort&compress,square,comma,authoryear]{natbib}		
		
%-----------------------------------------------------------------------------------------------------------------------
%更改章節編號字體
%\usepackage{titlesec}
%\newfontfamily\sectionNC{Nagurigaki Crayon}
%\newfontfamily\subsectionLBP{Local BaseBall Park}
%\titleformat*{\section}{sectionNC}
%\titleformat*{\subsection}{sectionLBP}  


\title{ \LaTeX{\UD 計數器的應用}}
\author{{\NC 張馨云}\;\; {\JF 410578046}}
\date{{\BR \today}} 	

\begin{document}
\maketitle
\fontsize{12}{22 pt}\selectfont

\centerline{{\BCF 引言}}
\setlength{\parindent}{2em} 
	在閱讀數理相關的文章、書籍、講義,甚或作答考試卷時,常常會看到一些定義公式、範例,題號等等皆以相同的格式呈現,並因不同章節而有計數之分別,為避免手動輸入及維護造成錯誤與麻煩,本文將介紹如何使用\LaTeX 指令計數。

\newpage
\section{\HC 計數器基本指令}
\subsection{{\HC \LaTeX 內建計數器}}
	\LaTeX 有內建幾種計數器以便使用者使用,指令如表\ref{counterx_latex}所示:

\begin{table}[H]
\centering
\extrarowheight=4pt
\caption{\LaTeX 內建計數器}\label{counterx_latex}
	\begin{tabular}{ll}
	\hline
	指令(加上$\backslash$) \quad &  \quad 用途  \\
	\hline
	part & 部序號計數器 \\
	\rowcolor{lavendergray}
	equation & 公式序號計數器 \\
	chapter & 章序號計數器 \\
	\rowcolor{lavendergray}
	page & 頁碼計數器 \\
	section & 節序號計數器 \\
	\rowcolor{lavendergray}
	footnote & 腳註序號計數器 \\
	subsection & 小節序號計數器 \\
	\rowcolor{lavendergray}
	mpfootnote & 小頁環境中的腳註序號計數器 \\
	subsubsection & 小小節序號計數器 \\
	\rowcolor{lavendergray}
	enumi & 排序列表第1層序號計數器 \\
	enumii & 排序列表第2層序號計數器\\
	\rowcolor{lavendergray}
	enumiii & 排序列表第3層序號計數器 \\
	enumiv & 排序列表第4層序號計數器 \\
	\rowcolor{lavendergray}
	paragraph & 段序號計數器 \\
	subparagraph & 小段序號計數器 \\
	\rowcolor{lavendergray}
	figure & 插圖序號計數器 \\
	table & 表格序號計數器  \\
	\hline
	\end{tabular}
\end{table}


\subsection{{\HC 定義新的計數器}}\label{counter_def}
	定義新的計數器,需要用到amsthm套件所提中的$\backslash$theroemstyle指令,使用方法如下,詳見表\ref{counter_thmstyle}:\\
      \indent $\{\backslash$newtheorem$\{$name$\}\{$Printed output$\}$[numberby]\\
      \indent $\{\backslash$newtheorem$\{$name$\}$[counter]$\{$Printed output$\}$\\
	\begin{table}[H]
	\centering
	\extrarowheight=4pt
	\caption{theoremstyle說明}\label{counter_thmstyle}
		\begin{tabular}{ll}
		\hline
		選項設置 & 說明 \\
		\hline
		\textbf{name} & 定義新計數器的指令。 \\
		\rowcolor{lavendergray}
		\textbf{Printed output} & 計數器最終呈現於文章的字樣 \\
		\textbf{numberby} & 新定義的計數器將以原先已定義的計數器為計數範圍 \\
		\rowcolor{lavendergray}
		\textbf{counter} & \tabincell{ll}{counter為已定義之計數器。代表新定義之計數器將\\與其共用相同的計數器,計算數字時會同步累計。} \\
		\tabincell{ll}{\textbf{$\backslash$begin$\{$name$\}$}\\ \textbf{$\backslash$end$\{$name$\}$}} & 開啟新定義之計數器環境 \\
		\hline
		\end{tabular}
	\end{table}

\subsection{{\HC 直接定義計數之數}}
	別於\ref{counter_def}所介紹之事先定義再以指令去呈現的計數器,另一種計數器使用方法如表\ref{counter_numb}所示:
	\begin{table}[H]
	\centering
	\extrarowheight=4pt
	\caption{counter使用說明}\label{counter_numb}
		\begin{tabular}{ll}
		\hline
		指令  & 說明 \\
		\hline
		$\backslash$newcounter$\{$name$\}$ & 定義計數器名稱 \\				\rowcolor{lavendergray}
		$\backslash$setcounter$\{$name$\}\{$數字$\}$ & 設置計數器起始數字,預設為0 \\
		$\backslash$addtocounter$\{$name$\}\{1\}$ & 第二次使用往上加1 \\
		\rowcolor{lavendergray}
		$\backslash$thename & 顯示該計數器的值 \\
		$\backslash$value$\{$name$\}$ & 調用該計數器之值 \\
		\hline
		\end{tabular}
	\end{table}
	
	\indent 使用此方法計數,每一次計數都需要使用$\backslash$addtocounter,相比之下,使用事先定義之計數器指令計數更為簡便。

\section{\HC 計數器實際操作}
	表\ref{counter_new}定義了幾個新的計數器:
	\begin{table}[H]
	\centering
	\extrarowheight=4pt
	\caption{定義新的計數器}\label{counter_new}
		\begin{tabular}{ll}
		\hline
		定義 & 說明 \\
		\hline
		$\{\backslash$newtheorem$\{$Def$\}\{$Definition$\}$[section]  & 定義Definition \\ 
		\rowcolor{lavendergray}
		$\{\backslash$newtheorem$\{$thm$\}\{\{\backslash$HC 定理$\}\}$[section] & 定義 theorem為定理 \\
		$\{\backslash$newtheorem$\{$lemma$\}$[thm]$\{$Lemma$\}$ & 定義與theorem共用編號的lemma \\
		\rowcolor{lavendergray}
		$\{\backslash$newtheorem$\{$ex$\}\{\{\backslash$F Example$\}\}$ & 定義獨立編號的example \\
		$\{\backslash$newtheorem$\{$EX$\}$[ex]$\{\{\backslash$HC 範例$\}\}$ & 定義與example共用編號的範例 \\
		\hline
		\end{tabular} 
	\end{table}

	\indent 或以有底色之表格,或以\footnote{$\backslash$rule$\{\backslash$textwidth$\}\{$0.2pt$\}$} 上下限之線條,我們常看見,在文章、書籍中,定理、範例等將定義區隔開來,以較明顯的方式呈現給讀者,不妨在以下練習試試:

	\begin{center}
	\colorbox{lightpink}{\begin{tabular}{p{0.9\textwidth}}
		\begin{thm}\label{ex_theorem}
		Let $X$ have the cdf $F(x)$ of the continuous type that is strictly increasing on the support$ a < x < b $.  Then the random variable  $Y$, defined by $Y = F(X)$, has a distribution that is $\mathbf{U}(0,1)$. 
		\end{thm}
		\end{tabular}}
	\end{center}


	\noindent \rule{\textwidth}{0.2pt}
	\begin{Def}\label{ex_Def}
(Factorization Theorem) \\
Let $X_1,X_2,\cdots,X_n$ denote random variables with joint pdf or pmf $f(x_1,x_2,\cdots,x_n;\theta)$, which depends on the parameter $\theta$. The statistic $Y =u(X_1,X_2,\cdots,X_n)$ is sufficient for $\theta$ if and only if $f(x_1,x_2,\cdots,x_n;\theta)$= $\varphi[u(x_1,x_2,\cdots,x_n);\theta]h(x_1,x_2,\cdots,x_n)$, where $\varphi$ depends on $x_1,x_2,\cdots,x_n$ only through $u(x_1,\cdots,x_n)$ and $h(x_1,\cdots,x_n)$ does not depend on $\theta$.
	\end{Def}
	\noindent \rule{\textwidth}{0.2pt}
	
	\begin{center}
	\colorbox{bananamania}{\begin{tabular}{p{0.9\textwidth}}
		\begin{ex}\label{ex_ex}
	Let $X_1,X_2,\cdots, X_n$ be a random sample from an exponential distribution with pdf
	$$f(x;\theta)=\frac{1}{\theta}e^{-x/\theta}=exp\left[ x(-\frac{1}{\theta})-ln\theta\right], \quad 0 < x < \infty, $$
	provided that $ 0 < x < \infty$. Here, $K(x)=x$.Thus, $\sum_{i=1}^nX_i$ is sufficient for $\theta$; of courfe,$\bar{X}=\sum_{i=1}^nX_i/n$ is also sufficient.
		\end{ex}
		\end{tabular}}
	\end{center}

	\noindent \rule{\textwidth}{0.2pt}
	\begin{lemma}\label{ex_lem}
	(Neyman–Pearson Lemma)\\
	 Let $X_1,X_2,\cdots,X_n$ be a random sample of size n from a distribution with pdf or pmf $f(x;\theta)$, where $\theta_0$ and $\theta_1$ are two possible values of $\theta$. Denote the joint pdf or pmf of $X_1,X_2,\cdots,X_n$ by the likelihood function $$L(\theta)=L(\theta;x_1,x_2,\cdots,x_n)=f(x_1;\theta)f(x_2;\theta)\cdots f(x_n;\theta).$$ 
	 If there exist a positive constant $k$ and $a$ subset $C$ of the sample space such that 
	 	\begin{enumerate}[(a)]
	 	\item $P[(X_1,X_2,\cdots,X_n) \in C;\theta_0]= \alpha$
	 	\item $\frac{L(\theta_0)}{L(\theta_1)} \leq k for (X_1,X_2,\cdots,X_n)\in C$, and 
	 	\item $\frac{L(\theta_0)}{L(\theta_1)}  \geq k for (x_1,x_2,\cdots,x_n)\in C',$ 
	 	\end{enumerate}
	then $C$ is a best critical region of size $\alpha$ for testing the simple null hypothesis $H_0: \theta = \theta_0$ against the simple alternative hypothesis $H_1: \theta = \theta_1$.
	\end{lemma}
	\noindent \rule{\textwidth}{0.2pt}

		
	\begin{center}
	\colorbox{bananamania}{\begin{tabular}{p{0.9\textwidth}}
		\begin{EX}\label{ex_EX}
		Let $\bar{X_n}$ denote the mean of a random sample of size n from a distribution that has pdf $f(x)=e^{-x}, 0 < x < \infty$, zero elsewhere.
		Use $\Delta$ -method to find the limiting distribution of $\sqrt{n}(\sqrt{\bar{X_n}}-1)$.
		\end{EX}
		\end{tabular}}
	\end{center}


	因為我們曾經定義Lemma與定理共用計數編號;範例與Example共用計數編號,可以看到Lemma\ref{ex_lem}接續定理\ref{ex_theorem}標號;範例\ref{ex_EX}接續Example\ref{ex_ex}編號。

\section{{\HC 結語與問題}}
	利用\LaTeX 之計數器能使文章書籍在編輯時更有效率且錯誤率更低,只要事先在preamble檔案中設計好計數器樣式、字型等等運作規則,操作起來比WORD更加方便快速!
	在引用amsthm套件定義計數器時,可能會因為套件衝突,而產生"! LaTeX Error: Command $\backslash$openbox already defined."之錯誤信息,此時,只要在引用amsthm套件之前,使用$\backslash$let$\backslash$openbox$\backslash$relax,忽略$\backslash$openbox這個指令,就可以成功的引用amsthm套件囉!


\nocite{*}
\bibliographystyle{amsplain}
\bibliography{reference}




\end{document}