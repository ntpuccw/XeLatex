%%\documentclass[12pt, a4paper]{article} 

\usepackage{fontspec} % Font selection for XeLaTeX; see fontspec.pdf. 
\usepackage{xeCJK}	% 中文使用 XeCJK,利用 \setCJKmainfont 定義中文內文、粗體與斜體的字型
\defaultfontfeatures{Mapping=tex-text} % to support TeX conventions like ``---''
\usepackage{xunicode} % Unicode support for LaTeX character names(accents, European chars, etc)
\usepackage{xltxtra} 				% Extra customizations for XeLaTeX
\usepackage{amsmath, amssymb}
\usepackage{enumerate}
\usepackage{graphicx,subfig,float,wrapfig} % support the \includegraphics command and options
\usepackage[outercaption]{sidecap} %[options]=[outercaption], [innercaption], [leftcaption], [rightcaption]
\usepackage{array, booktabs}
\usepackage{color, xcolor}
\usepackage{longtable}
\usepackage{colortbl}                          				
\usepackage{listings}						% 直接將 latex 碼轉換成顯示文字
\usepackage[parfill]{parskip} 				% 新段落前加一空行,不使用縮排
\usepackage[left=1.5in,right=1in,top=1in,bottom=1in]{geometry} 
\usepackage{url}

%-----------------------------------------------------------------
%  中英文內文字型設定
\setCJKmainfont							% 設定中文內文字型
	[
		BoldFont=Microsoft YaHei	    %定義粗體的字型(Win)
%		BoldFont=蘋果儷中黑	    		%定義粗體的字型(Mac)
	]
	{新細明體}						% 設定中文內文字型(Win)
%	{宋體-繁}							% 設定中文內文字型(Mac)	
\setmainfont{Times New Roman}		% 設定英文內文字型
\setsansfont{Arial}					% 無襯字字型 used with {\sffamily ...}
%\setsansfont[Scale=MatchLowercase,Mapping=tex-text]{Gill Sans}
\setmonofont{Courier New}			% 等寬字型 used with {\ttfamily ...}
%\setmonofont[Scale=MatchLowercase]{Andale Mono}
% 其他字型(隨使用的電腦安裝的字型不同,用註解的方式調整(打開或關閉))
% 英文字型
\newfontfamily{\E}{Calibri}				
\newfontfamily{\A}{Arial}
\newfontfamily{\C}[Scale=0.9]{Arial}
\newfontfamily{\R}{Times New Roman}
\newfontfamily{\TT}[Scale=0.8]{Times New Roman}
% 中文字型
\newCJKfontfamily{\MB}{微軟正黑體}				% 等寬及無襯線字體 Win
%\newCJKfontfamily{\MB}{黑體-繁}				% 等寬及無襯線字體 Mac
\newCJKfontfamily{\SM}[Scale=0.8]{新細明體}	% 縮小版(Win)
%\newCJKfontfamily{\SM}[Scale=0.8]{宋體-繁}	% 縮小版(Mac)
\newCJKfontfamily{\K}{標楷體}                	% Windows下的標楷體
%\newCJKfontfamily{\K}{楷體-繁}               	% Mac下的標楷體
\newCJKfontfamily{\BB}{Microsoft YaHei}		% 粗體 Win
%\newCJKfontfamily{\BB}{蘋果儷中黑}		% 粗體 Mac
% 以下為自行安裝的字型:CwTex 組合
%\newCJKfontfamily{\CF}{cwTeX Q Fangsong Medium}	% CwTex 仿宋體
%\newCJKfontfamily{\CB}{cwTeX Q Hei Bold}			% CwTex 粗黑體
%\newCJKfontfamily{\CK}{cwTeX Q Kai Medium}   	% CwTex 楷體
%\newCJKfontfamily{\CM}{cwTeX Q Ming Medium}		% CwTex 明體
%\newCJKfontfamily{\CR}{cwTeX Q Yuan Medium}		% CwTex 圓體
%-----------------------------------------------------------------------------------------------------------------------
\XeTeXlinebreaklocale "zh"             		%這兩行一定要加,中文才能自動換行
\XeTeXlinebreakskip = 0pt plus 1pt     		%這兩行一定要加,中文才能自動換行
%-----------------------------------------------------------------------------------------------------------------------
\newcommand{\cw}{\texttt{cw}\kern-.6pt\TeX}	% 這是 cwTex 的 logo 文字
\newcommand{\imgdir}{images/}				% 設定圖檔的目錄位置
\renewcommand{\tablename}{表}	% 改變表格標號文字為中文的「表」(預設為 Table)
\renewcommand{\figurename}{圖}% 改變圖片標號文字為中文的「圖」(預設為 Figure)

% 設定顏色 see color Table: http://latexcolor.com
\definecolor{slight}{gray}{0.9}				
\definecolor{airforceblue}{rgb}{0.36, 0.54, 0.66} 
\definecolor{arylideyellow}{rgb}{0.91, 0.84, 0.42}
\definecolor{babyblue}{rgb}{0.54, 0.81, 0.94}
\definecolor{cadmiumred}{rgb}{0.89, 0.0, 0.13}
\definecolor{coolblack}{rgb}{0.0, 0.18, 0.39}
\definecolor{beaublue}{rgb}{0.74, 0.83, 0.9}
\definecolor{beige}{rgb}{0.96, 0.96, 0.86}
\definecolor{bisque}{rgb}{1.0, 0.89, 0.77}
\definecolor{gray(x11gray)}{rgb}{0.75, 0.75, 0.75}
\definecolor{limegreen}{rgb}{0.2, 0.8, 0.2}
\definecolor{splashedwhite}{rgb}{1.0, 0.99, 1.0}

%---------------------------------------------------------------------
% 映出程式碼 \begin{lstlisting} 的內部設定
\lstset
{	language=[LaTeX]TeX,
    breaklines=true,
    %basicstyle=\tt\scriptsize,
    basicstyle=\tt\normalsize,
    keywordstyle=\color{blue},
    identifierstyle=\color{black},
    commentstyle=\color{limegreen}\itshape,
    stringstyle=\rmfamily,
    showstringspaces=false,
    %backgroundcolor=\color{splashedwhite},
    backgroundcolor=\color{slight},
    frame=single,							%default frame=none 
    rulecolor=\color{gray(x11gray)},
    framerule=0.4pt,							%expand outward 
    framesep=3pt,							%expand outward
    xleftmargin=3.4pt,		%to make the frame fits in the text area. 
    xrightmargin=3.4pt,		%to make the frame fits in the text area. 
    tabsize=2				%default :8 only influence the lstlisting and lstinline.
}

% 映出程式碼 \begin{lstlisting} 的內部設定 for Python codes
%\lstset{language=Python}
%\lstset{frame=lines}
%\lstset{basicstyle=\SCP\normalsize}
%\lstset{keywordstyle=\color{blue}}
%\lstset{commentstyle=\color{airforceblue}\itshape}
%\lstset{backgroundcolor=\color{beige}}   % 使用自己維護的定義檔
%
%% 以下的設定在測試成功後,可以轉入前置檔(preamble_CJK.tex )一起管理。
%\usepackage{amsthm}						% theroemstyle 需要使用的套件
%\theoremstyle{plain}
%
%\newtheorem{de}{Definition}[section]		%definition獨立編號
%\newtheorem{thm}{{\MB 定理}}[section]		%theorem 獨立編號,取中文名稱並給予不同字型
%\newtheorem{lemma}[thm]{Lemma}			%lemma 與 theorem 共用編號
%\newtheorem{ex}{{\E Example}}				%example 獨立編號,不編入小節數字,走流水號。也換個字型。
%
%\newtheorem{cor}{Corollary}[section]		%not used here
%\newtheorem{exercise}{EXERCISE}			%not used here
%\newtheorem{re}{\emph{Result}}[section]	%not used here
%\newtheorem{axiom}{AXIOM}				%not used here
%\renewcommand{\proofname}{證明}			%not used here
%
%\newcounter{quiz}						% start a simple and new counter
%\setcounter{quiz}{1}						% start to count from 1
%
%%---------------------------------------------------------
%% 文章開始
%\title{ \LaTeX {\MB 定理計數器的使用}}
%\author{{\MB 汪群超}}
%\date{{\TT \today}} 			 
%\begin{document}
%\maketitle
%\fontsize{12}{22pt}\selectfont
\chapter{ \LaTeX {\MB 定理計數器的使用}}
數理方面的文章或書籍常會使用到定理、定義,或類似需要給予編號的一段文字。這些編號的管理有些是順序排列,有些則隨章節排列,有些一起編號,有些分開。這類的編號都由指令 {\A $\backslash$newtheorem} 處理。  

\section{{\MB 語法}}
$\backslash$newtheorem 的語法如下

 \begin{center}{\begin{tabular}{l}
 $\backslash$newtheorem\{env\_name\}\{caption\}[within]\\
 $\backslash$newtheorem\{env\_name\}[numbered\_like]\{caption\}\\
 \end{tabular}}\end{center}
其中
\begin{itemize}
\item env\_name:新計數器名稱,通常以簡短文字代表將呈現的文字。譬如, thm 代表 Theorem 字樣。
\item caption:表示將呈現的文字,一般如 Theorem, Definition, Lemma... 等或使用中文的「定理」「定義」等。
\item within:代表一個已經存在的計數器,譬如,章 (chapter) 或節 (section),表示目前的計數器將以該存在的計數器為計數範圍。以章為例,在第二章出現的第一個編號將是 2.1,以節為例,第三章第二節出現的第三個編號將是 3.2.3。
\item number\_like:一個已經被定義過的計數器名稱,譬如,thm。代表目前定義的計數器將共用相同的計數器。沒有這項參數定義的都是為獨立編號,不予其他計數器共用。
\end{itemize}
以下範例舉 Definition, Example, Theorem, Lemma 為例,其中 Definition, Example 獨立編號,而 Theorem 與 Lemma 共同編號。定義方式如下:

 \begin{center}{\begin{tabular}{l}
 $\backslash$newtheorem\{de\}\{Definition\}[section]\\
 $\backslash$newtheorem\{ex\}\{$\backslash$emph\{Example\}\}[section]\\
 $\backslash$newtheorem\{th\}\{Theorem\}[section]\\
 $\backslash$newtheorem\{lemma\}[th]\{Lemma\}\\
\end{tabular}}\end{center}


\section{{\MB 隨機變數的定義}}
 \rule{\textwidth}{0.2pt}
 \begin{de}\footnote{摘自 Casella and Berger 2002, Definition 1.1.1}  %def 1.1.1(Casella & Berger (2002))
The set, $\mathcal{S}$, of all possible outcomes of a particular experiment is called the \textbf{sample space} for the experiment.\\
 \rule{\textwidth}{0.2pt}
\end{de}
\noindent 通常定義、定理會用特別的方式呈現出來,讓讀者容易一眼看到。本章將定義前後個加上一條橫線來突顯它的位置。另外一種常見的方式請參考本章最後的「定理」與「 lemma」,用表格加上底色做出明顯的框架,裡面採用的表格與底色技術請參考講義「表格製作參考」。

\noindent \rule{\textwidth}{0.2pt}
\begin{de}\footnote{摘自 Casella and Berger 2002, Definition 1.5.1} %def 1.1.2
The \textbf{cumulative distribution function}  or CDF of a random variable X, denoted by $F_X(x)$,
is defined by
\[F_X(x)=P_X(X \leq x). \mbox{ for all x}, x\in \mathcal{S}.\]
 \rule{\textwidth}{0.2pt}
\end{de}
\noindent 以下的 Example 編號不隨小節計數,係按流水號順序。
\begin{ex}[指數分配隨機變數的呈現] %
假設  X 服從指數分配,其 CDF 為 $y=F_X(x)=1-e^{-x}, \forall x>0$,記為 $X\sim F_X(x)=1-e^{-x}$。
\end{ex}

\begin{ex}[幾何分配隨機變數的呈現] %
假設 X 服從幾何分配,其 CDF 為 $y=F_X(x)=1-(1-p)^k$,其中 $k=[x]\in \mathcal{N}$,記為 $X\sim F_X(x)=1-(1-p)^x$。此函數又稱為階梯函數(step function)。
\end{ex}


\section{{\MB 離散型隨機變數}}
\noindent \rule{\textwidth}{0.2pt}
\begin{de}\footnote{摘自 Casella and Berger 2002, Definition 1.6.1} %def 1.6.1
The \textbf{probability massfunction} (\textbf{pmf}) of a discrete random variable $X$ is given by
\[f_X(x)=P(X=x) \mbox{ for all } x.\]
\noindent \rule{\textwidth}{0.2pt}
\end{de}
\bigskip
 
\begin{ex}[Geometric probabilities] %ex 1.6.2
For the \textbf{geometric distributio}n of Example 1.1.2, we have the \textbf{pmf}
\[f_X(x)=P(X=x)=\left\{\begin{array}{ll} p(1-p)^{x-1}  & \mbox{ for } x=1, 2, \cdots \\
                                              0        & \mbox{ otherwise } \\ \end{array}\right.\]
\end{ex}
\bigskip
\noindent \rule{\textwidth}{0.2pt}
\begin{de}\footnote{摘自 Casella and Berger 2002, Definition 1.6.1}  %def 1.6.3
The \textbf{probability density function} or PDF, $f_X(x)$, of a continuous random variable $X$ is the function that
satisfies
\[F_X(x)=\int_{-\infty}^x f_X(t)dt \mbox{ for all } x.\]
\noindent \rule{\textwidth}{0.2pt}
\end{de}
\bigskip

\begin{ex}[Exponential probabilities] %ex 1.6.4
For the exponential distribution of the previous Example we have
\[F_X(x)=1-e^{-x}\]
and, hence,
\[f_X(x)=\frac{d}{dx}F_X(x)=e^{-x}.\]
\end{ex}
\bigskip

\noindent \rule{\textwidth}{0.2pt}
\begin{de}\footnote{摘自 Casella and Berger 2002, Definition 4.5.10}   %def 4.5.10
Let $-\infty<\mu_X<\infty$, $-\infty<\mu_Y<\infty$, $0<\sigma_X$, $0<\sigma_Y$, and
$-1<\rho<1$ be five real numbers.  The bivariate normal pdf with means $\mu_X$ and $\mu_Y$,
variances $\sigma_X^2$ and $\sigma_Y^2$, and correlation $\rho$ is the bivariate pdf given by
\begin{eqnarray*}
f(x,y)&=&\left( 2\pi\sigma_X\sigma_Y\sqrt{1-\rho^2}\right)^{-1}\\
      && \times\exp\left(-\frac{1}{2(1-\rho^2)}\left(\left(\frac{x-\mu_x}{\sigma_X}\right)^2\right.\right.\\
      && \left.\left.-2\rho\left(\frac{x-\mu_x}{\sigma_X}\right)\left(\frac{y-\mu_y}{\sigma_Y}\right)
      +\left(\frac{y-\mu_y}{\sigma_Y}\right)^2 \right)\right)
\end{eqnarray*}
for $-\infty<x<\infty$ and $-\infty<y<\infty$.\\
\noindent \rule{\textwidth}{0.2pt}
\end{de}

\begin{ex}[{\C Bivariate Normal}] %ex 1.6.2
二維常態參數 $\mu_X=10, \mu_Y=20, \sigma_X=1, \sigma_Y=2, \rho=0.6$ 其分配函數為
\begin{equation*}
f(x,y)=\frac{1}{3.2\pi}\exp\left[-\frac{1}{1.28}\left((\frac{x-10}{1})^2
       -1.2(\frac{x-10}{1})(\frac{y-20}{2})+(\frac{y-20}{2})^2 \right)\right]
\end{equation*}
\end{ex}
\bigskip
\begin{center}\colorbox{slight}{\begin{tabular}{p{0.9\textwidth}}
\begin{thm}\label{demo_ref}\footnote{摘自 Casella and Berger 2002, Theorem 5.3.1} %theorem 5.3.1
Let $X_1, \cdots, X_n$ be a random sample from a $N(\mu, \sigma^2)$ distribution, and let
$\bar{X}=\frac{1}{n}\sum_{i=1}^n X_i$ and $S^2=\frac{1}{n-1}\sum_{i=1}^n (X_i-\bar{X})^2$.  Then
\begin{itemize}
\item[a.] $\bar{X}$ and $S^2$ are independent random variables,
\item[b.] $\bar{X}$ has a $N(\mu, \sigma^2/n)$ distribution,
\item[c.] $(n-1)S^2/\sigma^2$ has a chi squared distribution with $n-1$ degrees of freedom.
\end{itemize}
\end{thm}
 \end{tabular}}\end{center}
\bigskip

定理 \ref{demo_ref} 展示兩件事,其一是加入標號的引用($\backslash$label)與此處的參照對應,其二是自訂的項目符號(\textit{a. b. c.})。接著是個 lemma ,其編號隨著定理續編。
\begin{center}\colorbox{slight}{\begin{tabular}{p{0.9\textwidth}}
\begin{lemma}. Let $a_1,a_2,\cdots$ be a sequence of numbers converging to $a$, that is, $\lim_{n\rightarrow \infty} a_n=a$. Then
$$\lim_{n\rightarrow \infty} (1+\frac{a_n}{n})^n=e^n.$$
\end{lemma}
 \end{tabular}}\end{center}

\section{練習題}
利用上課時間完成以下兩個練習,順便展示簡易型計數器(newcounter)的使用。

\rule{\textwidth}{0.2pt}
\thequiz \;\;對本文所使用到的計數器,分別再多加一個(或以上)範例,方能確定可以掌握這個技術。\\
\rule{\textwidth}{0.2pt}
\addtocounter{quiz}{1} 		% 指定計數器 quiz 加 1
\thequiz \;\;新增一個計數器,並在適當位置加入計數的範例。\\
\rule{\textwidth}{0.2pt}
\addtocounter{quiz}{1} 		% 指定計數器 quiz 加 1
\thequiz \;\;為前一節的定理與 Lemma 的方框換顏色。\\
\rule{\textwidth}{0.2pt}
%\end{document}
