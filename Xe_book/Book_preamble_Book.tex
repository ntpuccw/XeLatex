%--- BOOK 定義檔
\documentclass[12pt, a4paper]{book}
\setlength{\textwidth}{13cm} %設定書面文字寬度
%--- 定義頁眉頁足 -------------------------------
\usepackage{fancyhdr}
\pagestyle{fancy}
\fancyhf{}
\renewcommand{\chaptermark}[1]{\markboth{\thechapter .\ #1}{}} %去除章編號前後的字
\fancyhead[LE]{\thepage \;\; \XeLaTeX 一本書}
\fancyhead[RO]{\leftmark \;\; \thepage }
%\rhead{\ctxff 數學與電腦}
%\cfoot{\thepage}
\renewcommand{\headrulewidth}{1pt} %頁眉下方的橫線

%----- 定義使用的 packages ----------------------
\usepackage{fontspec} 										 
%\usepackage[BoldFont, SlantFont]{xeCJK}		
\usepackage{xeCJK}											
\defaultfontfeatures{Mapping=tex-text} 				% to support TeX conventions like ``---''
\usepackage{xunicode} 								% Unicode support for LaTeX character names (accents, European chars, etc)
\usepackage{xltxtra} 								% Extra customizations for XeLaTeX
\usepackage{amsmath, amssymb}
\usepackage{amsthm}									% theroemstyle 需要使用的套件
\usepackage[sf,small]{titlesec}
\usepackage{enumerate}
\usepackage{graphicx,subfig,float,wrapfig} 		
\usepackage[outercaption]{sidecap} %[options]=[outercaption], [innercaption], [leftcaption], [rightcaption]
\usepackage{array, booktabs}
\usepackage{color, xcolor}
\usepackage{longtable}
\usepackage{colortbl}                          				
\usepackage{natbib}									% for Reference
\usepackage{makeidx}									% for Indexing
\usepackage{listings}								%直接從 latex 碼轉換成顯示文字
\usepackage[parfill]{parskip} 						% Activate to begin paragraphs with an empty line rather than an indent
%\usepackage{geometry} 			% See geometry.pdf to learn the layout options. There are lots.
%\usepackage[left=1.5in,right=1in,top=1in,bottom=1in]{geometry} 

%-----------------------------------------------------------------------------------------------------------------------
\setCJKmainfont
	[
		BoldFont=MS Gothic	% 定義粗體的字型(依使用的電腦安裝的字型而定)
	]
%	{cwTeX Q Ming Medium} 							% 設定中文內文字型
	{新細明體}	
\setmainfont{Times New Roman}							% 設定英文內文字型
\setsansfont{Arial}									% used with {\sffamily ...}
%\setsansfont[Scale=MatchLowercase,Mapping=tex-text]{Gill Sans}
\setmonofont{Courier New}							% used with {\ttfamily ...}
%\setmonofont[Scale=MatchLowercase]{Andale Mono}
% 其他字型(隨使用的電腦安裝的字型不同,用註解的方式調整(打開或關閉))
% 英文字型
\newfontfamily{\E}{Calibri}							% 套用在內文中所有的英文字母
\newfontfamily{\A}{Arial}
\newfontfamily{\C}[Scale=0.9]{Calibri}
\newfontfamily{\TT}[Scale=0.8]{Times New Roman}
% 中文字型
\newCJKfontfamily{\MB}{Microsoft JhengHei UI}	% 適用在 Mac 與 Win
\newCJKfontfamily{\SM}[Scale=0.8]{新細明體}			% 縮小版新細明體
\newCJKfontfamily{\K}{標楷體}                        	% Windows 下的標楷體
%--------------------------------------------------------------------------------------
\XeTeXlinebreaklocale "zh"             %這兩行一定要加,中文才能自動換行
\XeTeXlinebreakskip = 0pt plus 1pt     %這兩行一定要加,中文才能自動換行
%---------------------------------------------------------------------------------------
%----- 重新定義的指令 ---------------------------
\newcommand{\cw}{\texttt{cw}\kern-.6pt\TeX}	% 這是 cwTex 的 logo 文字
\newcommand{\imgdir}{images/}					% 設定圖檔的位置
\renewcommand{\tablename}{表}					% 改變表格標號文字為中文的「表」(預設為 Table)
\renewcommand{\figurename}{圖}				% 改變圖片標號文字為中文的「圖」(預設為 Figure)
%\renewcommand{\figurename}{圖\hspace*{-.5mm}}
%  for Long Report or Book
\renewcommand{\contentsname}{{\MB 目錄}}
\renewcommand\listfigurename{{\MB 圖目錄}}
\renewcommand\listtablename{{\MB 表目錄}}
\renewcommand{\indexname}{{\MB 索引}}
\renewcommand{\bibname}{{\MB 參考文獻}}
%----------COUNTER -----------------------------------------------------------------------------------------------

\theoremstyle{plain}
\newtheorem{de}{Definition}[section]		%definition獨立編號
\newtheorem{thm}{{\MB 定理}}[section]		%theorem 獨立編號,取中文名稱並給予不同字型
\newtheorem{lemma}[thm]{Lemma}			%lemma 與 theorem 共用編號
\newtheorem{ex}{{\E Example}}				%example 獨立編號,不編入小節數字,走流水號。也換個字型。
\newtheorem{cor}{Corollary}[section]		%not used here
\newtheorem{exercise}{EXERCISE}			%not used here
\newtheorem{re}{\emph{Result}}[section]	%not used here
\newtheorem{axiom}{AXIOM}				%not used here
\renewcommand{\proofname}{\bf{Proof}}		%not used here
\newcounter{quiz}						% start a simple and new counter
\setcounter{quiz}{1}						% start to count from 1
%------------------------------------------------------------------------------------------------------------------

\newcommand{\loflabel}{圖} % 圖目錄出現 圖 x.x 的「圖」字
\newcommand{\lotlabel}{表}  % 表目錄出現 表 x.x 的「表」字

\parindent=0pt
\setcounter{tocdepth}{0}

%--- 其他定義 ----------------------------------
% 定義章節標題的字型、大小
\titleformat{\chapter}[display]{\raggedright\LARGE\MB}{\MB 第\ \thechapter\ 章}{0.2cm}{}
% \titleformat{\chapter}{\center\LARGE\MB}{\MB 第\ \thechapter\ 章、}{0.2cm}{}
% \titleformat{\chapter}{\centering\Huge\bfseries}{第\,\thechapter\,章}{1em}{}
%\titleformat{\chapter}[hang]{\centering\LARGE\sf}{\MB 第~\thesection~章}{0.2cm}{}%控制章的字體
%\titleformat{\section}[hang]{\Large\sf}{\MB 第~\thesection~節}{0.2cm}{}%控制章的字體
%\titleformat{\subsection}[hang]{\centering\Large\sf}{\MB 第~\thesubsection~節}{0.2cm}{}%控制節的字體
\titleformat*{\section}{\normalfont\Large\bfseries\MB}
\titleformat*{\subsection}{\normalfont\large\bfseries\MB}
\titleformat*{\subsubsection}{\normalfont\large\bfseries\MB}


\definecolor{slight}{gray}{0.9}				% 設定顏色
\definecolor{airforceblue}{rgb}{0.36, 0.54, 0.66} % color Table: http://latexcolor.com
\definecolor{arylideyellow}{rgb}{0.91, 0.84, 0.42}
\definecolor{babyblue}{rgb}{0.54, 0.81, 0.94}
\definecolor{cadmiumred}{rgb}{0.89, 0.0, 0.13}
\definecolor{coolblack}{rgb}{0.0, 0.18, 0.39}
\definecolor{beaublue}{rgb}{0.74, 0.83, 0.9}
\definecolor{beige}{rgb}{0.96, 0.96, 0.86}
\definecolor{bisque}{rgb}{1.0, 0.89, 0.77}
\definecolor{gray(x11gray)}{rgb}{0.75, 0.75, 0.75}
\definecolor{limegreen}{rgb}{0.2, 0.8, 0.2}
\definecolor{splashedwhite}{rgb}{1.0, 0.99, 1.0}

%--------------------------------------------------------------------------------------------------------------------------------------------
% 映出程式碼 \begin{lstlisting}
\lstset
{	language=[LaTeX]TeX,
    breaklines=true,
    %basicstyle=\tt\scriptsize,
    basicstyle=\tt\normalsize,
    keywordstyle=\color{blue},
    identifierstyle=\color{black},
    commentstyle=\color{limegreen}\itshape,
    stringstyle=\rmfamily,
    showstringspaces=false,
    %backgroundcolor=\color{splashedwhite},
    backgroundcolor=\color{slight},
    frame=single,							%default frame=none 
    rulecolor=\color{gray(x11gray)},
    framerule=0.4pt,						%expand outward 
    framesep=3pt,							%expand outward
    xleftmargin=3.4pt,						%to make the frame fits in the text area. 
    xrightmargin=3.4pt,						%to make the frame fits in the text area. 
    tabsize=2								%default :8 only influence the lstlisting and lstinline.
}